\chapter{Structured Task Creation Architecture}
\label{sec:taskarchitecture}

\sect{Overview}

PEBL provides a comprehensive architecture for creating structured, reusable, and maintainable psychological tests. This optional layer sits atop the core PEBL language and provides standardized systems for:

\begin{itemize}
\item \textbf{Layout \& Response System} - Unified screen layouts with platform-aware responses (Chapter~\ref{sec:layout})
\item \textbf{Parameter Management} - JSON/CSV configuration with schemas and presets
\item \textbf{Translation System} - Multi-language support with structured translation files
\item \textbf{Data Upload} - Centralized data collection via web servers
\end{itemize}

These systems work together to create battery tasks that are:
\begin{itemize}
\item \textbf{Configurable} - Parameters control all aspects without code changes
\item \textbf{International} - Full translation support for any language
\item \textbf{Platform-aware} - Automatic adaptation to native/web environments
\item \textbf{Data-ready} - Built-in upload to servers or PEBLHub
\end{itemize}

\textbf{Important:} All of these systems are optional. Core PEBL functionality works independently, but these systems significantly reduce development time and improve maintainability for battery tasks.

\sect{Parameter Management System}

The parameter system allows experiments to be configured without modifying code. Parameters are defined with defaults, loaded from files or URLs, and accessed as properties throughout the experiment.

\subsection{Basic Parameter Usage}

Define parameters with defaults in your PEBL script:

\begin{verbatim}
define Start(p)
{
  ## Define parameters with defaults
  parpairs <- [
    ["numTrials", 20],
    ["responsemode", "keyboardShift"],
    ["showFeedback", 1],
    ["stimulusDuration", 1000]
  ]

  ## Load parameters (from file or use defaults)
  gParams <- CreateParameters(parpairs, gParamFile)

  ## Access parameters as properties
  loop(trial, Sequence(1, gParams.numTrials, 1))
  {
    ShowStimulus(gParams.stimulusDuration)

    if(gParams.showFeedback)
    {
      ShowFeedback()
    }
  }
}
\end{verbatim}

The \texttt{CreateParameters()} function:
\begin{itemize}
\item Returns a property list with all parameters
\item Merges file values over defaults
\item Supports both JSON and CSV formats
\item Uses defaults if file doesn't exist or is empty
\end{itemize}

\subsection{Parameter Files}

PEBL supports two parameter file formats:

\subsubsection{JSON Format (Recommended)}

Modern format, ideal for web integration:

\textbf{File:} \texttt{params/taskname.pbl.par.json}
\begin{verbatim}
{
  "numTrials": 30,
  "responsemode": "mousetarget",
  "showFeedback": 1,
  "stimulusDuration": 1500
}
\end{verbatim}

\textbf{Advantages:}
\begin{itemize}
\item Web-friendly (easy JavaScript parsing/generation)
\item Supports complex types (arrays, nested objects)
\item Human-readable and editable
\item Can be fetched from URLs
\end{itemize}

\subsubsection{CSV Format (Legacy)}

Traditional format, still fully supported:

\textbf{File:} \texttt{params/taskname.pbl.par}
\begin{verbatim}
numTrials,30
responsemode,mousetarget
showFeedback,1
stimulusDuration,1500
\end{verbatim}

\textbf{Note:} No code changes needed to support both formats. \texttt{CreateParameters()} auto-detects the format.

\subsection{Parameter Schemas}

Schema files provide rich metadata about parameters for documentation and web UIs. They are entirely optional but highly recommended for battery tasks.

\textbf{File:} \texttt{params/taskname.pbl.schema.json}
\begin{verbatim}
{
  "test": "taskname",
  "version": "1.0",
  "description": "A psychological test measuring...",
  "citation": "Smith et al. (2020)",
  "parameters": [
    {
      "name": "numTrials",
      "type": "integer",
      "default": 20,
      "min": 1,
      "max": 100,
      "step": 1,
      "label": "Number of Trials",
      "description": "Total number of test trials to present"
    },
    {
      "name": "responsemode",
      "type": "string",
      "default": "keyboardShift",
      "options": ["keyboardShift", "keyboardSafe",
                  "mousetarget", "mousebutton"],
      "label": "Response Mode",
      "description": "How participants respond (keyboard/mouse)"
    },
    {
      "name": "showFeedback",
      "type": "boolean",
      "default": 1,
      "label": "Show Feedback",
      "description": "Display feedback after each trial"
    },
    {
      "name": "stimulusDuration",
      "type": "integer",
      "default": 1000,
      "min": 100,
      "max": 5000,
      "step": 100,
      "unit": "ms",
      "label": "Stimulus Duration",
      "description": "How long each stimulus is displayed"
    }
  ]
}
\end{verbatim}

\textbf{Schema fields:}
\begin{itemize}
\item \texttt{name} - Parameter identifier (used in code)
\item \texttt{type} - Data type (integer, string, boolean, number, array)
\item \texttt{default} - Default value
\item \texttt{label} - Human-readable name
\item \texttt{description} - Detailed explanation
\item \texttt{min/max/step} - Numeric constraints (optional)
\item \texttt{unit} - Display unit (ms, sec, px, etc.) (optional)
\item \texttt{options} - Valid choices for dropdowns (optional)
\end{itemize}

\subsection{Parameter Presets}

Create named presets for common configurations:

\textbf{File:} \texttt{params/short-version.json}
\begin{verbatim}
{
  "name": "Short Version",
  "description": "Quick testing with reduced trials",
  "parameters": {
    "numTrials": 10,
    "responsemode": "keyboardShift",
    "showFeedback": 1,
    "stimulusDuration": 800
  }
}
\end{verbatim}

\textbf{File:} \texttt{params/tablet-version.json}
\begin{verbatim}
{
  "name": "Tablet Version",
  "description": "Touch-screen optimized configuration",
  "parameters": {
    "numTrials": 20,
    "responsemode": "mousetarget",
    "showFeedback": 1,
    "stimulusDuration": 1500
  }
}
\end{verbatim}

\textbf{Usage:}
\begin{verbatim}
## Command line:
pebl2 taskname.pbl -v jsonparam=params/tablet-version.json

## Or via launcher preset dropdown
\end{verbatim}

\subsection{Loading Parameters from URLs}

For web-based studies, parameters can be fetched from remote servers:

\begin{verbatim}
pebl2 taskname.pbl --pfile https://study.example.com/api/params?token=ABC123
\end{verbatim}

The server can:
\begin{enumerate}
\item Read the task's \texttt{.pbl.schema.json} to understand parameters
\item Generate custom parameter JSON based on study configuration
\item Serve it to PEBL with appropriate authentication
\item Track which parameters were used for each participant
\end{enumerate}

\textbf{Example workflow:}
\begin{enumerate}
\item Researcher configures study on web platform
\item Platform generates URL with authentication token
\item PEBL fetches JSON parameters at runtime
\item Experiment runs with server-specified configuration
\end{enumerate}

\subsection{Best Practices}

\textbf{Parameter Naming:}
\begin{itemize}
\item Use descriptive names (\texttt{stimulusDuration}, not \texttt{sd})
\item Use camelCase for multi-word names
\item Group related parameters with common prefixes
\end{itemize}

\textbf{Defaults:}
\begin{itemize}
\item Always provide sensible defaults
\item Default to the most common use case
\item Document defaults in schema descriptions
\end{itemize}

\textbf{Validation:}
\begin{itemize}
\item Validate parameters at script start
\item Check ranges, types, and valid options
\item Provide clear error messages for invalid values
\end{itemize}

\begin{verbatim}
## Validate parameters
if(gParams.numTrials < 1 or gParams.numTrials > 100)
{
  MessageBox("Error: numTrials must be between 1 and 100", gWin)
  QuitPEBL()
}

if(not IsMember(gParams.responsemode,
   ["keyboardShift", "keyboardSafe", "mousetarget", "mousebutton"]))
{
  MessageBox("Error: Invalid responsemode", gWin)
  QuitPEBL()
}
\end{verbatim}

\sect{Translation System}

The translation system enables complete internationalization of battery tasks. All participant-visible text is stored in JSON files, allowing tasks to run in any language without code changes.

\subsection{Basic Translation Usage}

Store translations in JSON files and load them at runtime:

\textbf{File:} \texttt{translations/taskname.pbl-en.json}
\begin{verbatim}
{
  "INSTRUCTIONS": "Press the key corresponding to the stimulus color.",
  "PRACTICE_BEGIN": "Practice trials will begin. Press any key.",
  "PRACTICE_COMPLETE": "Practice complete! Press any key for test.",
  "TEST_BEGIN": "Test trials will begin. Press any key.",
  "TRIAL_HEADER": "Trial",
  "CORRECT_FEEDBACK": "Correct!",
  "INCORRECT_FEEDBACK": "Incorrect.",
  "TOO_SLOW": "Too slow!",
  "DEBRIEF": "Thank you for participating!"
}
\end{verbatim}

\textbf{Load and use in PEBL:}
\begin{verbatim}
define Start(p)
{
  ## Load translations for current language
  gStrings <- GetTranslations("taskname", gLanguage)

  ## Use throughout experiment
  MessageBox(gStrings.INSTRUCTIONS, gWin)

  gLayout.header.text <- gStrings.TRIAL_HEADER + " " + trialNum
  gLayout.footer.text <- gStrings.INSTRUCTIONS

  ## Dynamic text with substitution
  feedback <- gStrings.CORRECT_FEEDBACK
  ShowMessage(feedback)
}
\end{verbatim}

\subsection{Translation File Structure}

\textbf{Naming convention:} \texttt{translations/taskname.pbl-LANG.json}

Where \texttt{LANG} is the two-letter ISO 639-1 language code:
\begin{itemize}
\item \texttt{en} - English
\item \texttt{es} - Spanish
\item \texttt{fr} - French
\item \texttt{de} - German
\item \texttt{pt} - Portuguese
\item \texttt{cn} - Chinese
\item \texttt{he} - Hebrew
\item \texttt{ru} - Russian
\item \texttt{it} - Italian
\item etc.
\end{itemize}

\textbf{Key conventions:}
\begin{itemize}
\item Use UPPERCASE for all keys (recommended)
\item Use underscores to separate words (\texttt{TRIAL\_HEADER})
\item Group related strings with common prefixes
\item Keys are case-insensitive in PEBL (can access as \texttt{.trial\_header})
\end{itemize}

\subsection{Creating Translations}

\textbf{Step 1:} Create English baseline
\begin{verbatim}
translations/taskname.pbl-en.json
\end{verbatim}

\textbf{Step 2:} Copy to new language
\begin{verbatim}
cp translations/taskname.pbl-en.json \
   translations/taskname.pbl-es.json
\end{verbatim}

\textbf{Step 3:} Translate all values (keep keys unchanged)

\textbf{Before (English):}
\begin{verbatim}
{
  "INSTRUCTIONS": "Press the key corresponding to the color.",
  "PRACTICE_BEGIN": "Practice will begin. Press any key."
}
\end{verbatim}

\textbf{After (Spanish):}
\begin{verbatim}
{
  "INSTRUCTIONS": "Presione la tecla correspondiente al color.",
  "PRACTICE_BEGIN": "La práctica comenzará. Presione cualquier tecla."
}
\end{verbatim}

\textbf{Step 4:} Test with new language
\begin{verbatim}
pebl2 taskname.pbl -v language=es
\end{verbatim}

\subsection{Special Characters and Formatting}

\textbf{Line breaks:} Use \texttt{CR(1)} function in PEBL, not \texttt{\textbackslash n}
\begin{verbatim}
## In translation file - single line
"MULTI_LINE": "First line. Second line. Third line."

## In PEBL code - split with CR(1)
lines <- SplitString(gStrings.MULTI_LINE, ".")
text <- lines[1] + CR(1) + lines[2] + CR(1) + lines[3]
\end{verbatim}

\textbf{Unicode characters:} JSON supports full Unicode for international languages.

Translation files can contain any UTF-8 text including:
\begin{itemize}
\item Hebrew, Arabic (right-to-left scripts)
\item Chinese, Japanese, Korean characters
\item Accented Latin characters (Spanish, French, etc.)
\item Special symbols and emoji
\end{itemize}

\textbf{Dynamic substitution:} Use placeholders and string replacement
\begin{verbatim}
## Translation file
"TRIAL_PROGRESS": "Trial <NUM> of <TOTAL>"

## PEBL code
text <- SubstituteStrings(gStrings.TRIAL_PROGRESS,
                          [["<NUM>", trial + ""],
                           ["<TOTAL>", gParams.numTrials + ""]])
## Result: "Trial 5 of 20"
\end{verbatim}

\subsection{Fallback Strategy}

The translation system uses a sophisticated three-tier fallback mechanism to ensure translations are always available:

\subsubsection{Three-Tier Fallback}

\textbf{Tier 1: Explicit Language} (if gLanguage is non-empty)
\begin{verbatim}
## Try language specified via --language or -v language=XX
pebl2 taskname.pbl --language es
## Looks for: translations/taskname.pbl-es.json
\end{verbatim}

\textbf{Tier 2: System Locale} (if gLanguage is empty or Tier 1 failed)

The system automatically detects the operating system locale using \texttt{GetSystemLocale()}:
\begin{verbatim}
## On a Spanish-language system (locale: es_ES)
pebl2 taskname.pbl
## Automatically looks for: translations/taskname.pbl-es.json

## On a Chinese system (locale: zh_CN)
pebl2 taskname.pbl
## Automatically looks for: translations/taskname.pbl-zh.json
\end{verbatim}

The system extracts the two-letter language code from the full locale string (e.g., ``es\_ES'' becomes ``es'', ``zh\_CN'' becomes ``zh'').

\textbf{Tier 3: English Fallback} (if both tiers fail)
\begin{verbatim}
## If neither explicit language nor system locale translation exists
## Falls back to: translations/taskname.pbl-en.json
\end{verbatim}

\subsubsection{Fallback Examples}

\textbf{Example 1: Explicit language overrides system}
\begin{verbatim}
## System locale: de_DE (German)
## Command: pebl2 taskname.pbl --language fr
## Result: Uses French translation (tier 1 explicit override)
\end{verbatim}

\textbf{Example 2: System locale detection}
\begin{verbatim}
## System locale: ja_JP (Japanese)
## Command: pebl2 taskname.pbl
## gLanguage is empty, so system locale is detected
## Result: Uses Japanese translation (tier 2 system locale)
\end{verbatim}

\textbf{Example 3: English fallback}
\begin{verbatim}
## System locale: ko_KR (Korean)
## Command: pebl2 taskname.pbl
## But translations/taskname.pbl-ko.json doesn't exist
## Result: Falls back to English (tier 3)
\end{verbatim}

\textbf{Example 4: Custom language identifiers}

The \texttt{--language} parameter accepts any string, not just ISO 639-1 codes. This enables custom language variants:
\begin{verbatim}
pebl2 taskname.pbl --language es-mx
## Looks for: translations/taskname.pbl-es-mx.json
## (Mexican Spanish variant)

pebl2 taskname.pbl --language en-uk
## Looks for: translations/taskname.pbl-en-uk.json
## (British English variant)
\end{verbatim}

\subsubsection{Checking Translation Availability}

Always check if strings are available for error handling:
\begin{verbatim}
gStrings <- GetTranslations("taskname", gLanguage)

if(Length(GetPropertyList(gStrings)) == 0)
{
  MessageBox("Warning: No translation file found. Using defaults.", gWin)
  ## Provide hardcoded fallback strings
}
\end{verbatim}

However, with the three-tier fallback system, this should only occur if no translation files exist at all (including English).

\subsection{Translation Best Practices}

\textbf{What to translate:}
\begin{itemize}
\item All instructions and debriefing text
\item Header and footer text
\item Feedback messages
\item Response labels ("LEFT", "RIGHT", etc.)
\item Error messages
\item Any participant-visible text
\end{itemize}

\textbf{What NOT to translate:}
\begin{itemize}
\item Stimulus content (if language-specific by design)
\item Data file headers and output
\item Internal debug messages
\item Parameter names
\end{itemize}

\textbf{Testing translations:}
\begin{itemize}
\item Test with at least two languages
\item Verify text fits in layout zones
\item Check right-to-left languages (Hebrew, Arabic)
\item Ensure special characters display correctly
\item Have native speakers review translations
\end{itemize}

\textbf{Organizing large translation sets:}
For tasks with many strings, group by function:
\begin{verbatim}
{
  "INST_1": "First instruction...",
  "INST_2": "Second instruction...",
  "PRACTICE_HEADER": "Practice",
  "PRACTICE_BEGIN": "Practice will begin...",
  "PRACTICE_COMPLETE": "Practice complete...",
  "TEST_HEADER": "Test",
  "TEST_BEGIN": "Test will begin...",
  "FEEDBACK_CORRECT": "Correct!",
  "FEEDBACK_INCORRECT": "Incorrect.",
  "DEBRIEF_THANKS": "Thank you..."
}
\end{verbatim}

\sect{Data Upload System}

The data upload system enables centralized collection of experimental data to web servers. PEBL supports uploading to two server types:

\begin{itemize}
\item \textbf{PEBL Simple Data Server} - Included with PEBL (\texttt{dataserver/})
\item \textbf{PEBLHub Online} - Cloud platform at \href{https://peblhub.online}{peblhub.online}
\end{itemize}

\subsection{Upload Configuration}

Create an \texttt{upload.json} file specifying server details:

\textbf{File:} \texttt{upload.json}
\begin{verbatim}
{
  "server": "your-domain.com",
  "page": "/pebl-data/upload.php",
  "port": "80",
  "username": "admin",
  "upload_password": "yourpassword",
  "taskname": "stroop"
}
\end{verbatim}

\textbf{For HTTPS (recommended):}
\begin{verbatim}
{
  "server": "your-domain.com",
  "page": "/pebl-data/upload.php",
  "port": "443",
  "username": "admin",
  "upload_password": "yourpassword",
  "taskname": "stroop"
}
\end{verbatim}

\textbf{File location:} Place \texttt{upload.json} in the same directory as your \texttt{.pbl} script.

\subsection{Using Upload in PEBL Scripts}

\subsubsection{Basic Upload}

Upload data files after experiment completes:

\begin{verbatim}
define Start(p)
{
  ## ... run experiment ...

  ## Close data file
  FileClose(gFileOut)

  ## Upload to server
  if(FileExists("upload.json"))
  {
    success <- UploadFile(gSubNum, gFileOut)

    if(success)
    {
      MessageBox("Data uploaded successfully!", gWin)
    } else {
      MessageBox("Upload failed. Data saved locally.", gWin)
    }
  }
}
\end{verbatim}

\subsubsection{Parameter-Controlled Upload}

Allow upload to be enabled/disabled via parameters:

\begin{verbatim}
define Start(p)
{
  ## Define parameters
  parpairs <- [
    ["uploadData", 0],
    ["uploadURL", ""],
    ["numTrials", 20]
  ]

  gParams <- CreateParameters(parpairs, gParamFile)

  ## ... run experiment ...

  FileClose(gFileOut)

  ## Upload if enabled
  if(gParams.uploadData)
  {
    ## Show upload screen
    MessageBox("Uploading data, please wait...", gWin)

    ## Upload
    success <- UploadFile(gSubNum, gFileOut)

    ## Report result
    if(success)
    {
      MessageBox("Upload complete!", gWin)
    } else {
      MessageBox("Upload failed. Data saved locally at: " +
                 gFileOut, gWin)
    }
  }
}
\end{verbatim}

\textbf{Schema entry for upload parameters:}
\begin{verbatim}
{
  "name": "uploadData",
  "type": "boolean",
  "default": 0,
  "label": "Upload Data",
  "description": "Upload results to server after completion"
},
{
  "name": "uploadURL",
  "type": "string",
  "default": "",
  "label": "Upload Server URL",
  "description": "Server URL for data upload (if enabled)"
}
\end{verbatim}

\subsection{PEBL Simple Data Server}

The Simple Data Server is a lightweight PHP server included in \texttt{pebl/dataserver/}. It requires no database and runs on any PHP 5.5+ web server.

\subsubsection{Server Features}

\begin{itemize}
\item Single admin account (username/password)
\item Automatic subject number generation
\item Organized file structure: \texttt{uploads/taskname/subnum/}
\item File validation and security
\item Works on shared hosting (cPanel, etc.)
\item No database required
\end{itemize}

\subsubsection{Quick Server Setup}

\textbf{Step 1:} Upload files to web server
\begin{verbatim}
your-domain.com/pebl-data/
  config.php
  upload.php
  getnewsubnum.php
  generate_password.php
  .htaccess
  counter.txt (auto-created)
  uploads/ (auto-created)
\end{verbatim}

\textbf{Step 2:} Set permissions
\begin{verbatim}
chmod 755 uploads/
chmod 644 config.php upload.php getnewsubnum.php
\end{verbatim}

\textbf{Step 3:} Change default password

Generate password hash:
\begin{verbatim}
php generate_password.php yourNewPassword
\end{verbatim}

Edit \texttt{config.php}:
\begin{verbatim}
define('ADMIN_PASSWORD_HASH', 'your-hash-here');
\end{verbatim}

\textbf{Step 4:} Test upload
\begin{verbatim}
curl -X POST \
  -F "user_name=admin" \
  -F "upload_password=yourpassword" \
  -F "taskname=test" \
  -F "subnum=001" \
  -F "fileToUpload=@test.csv" \
  http://your-domain.com/pebl-data/upload.php
\end{verbatim}

\subsubsection{Getting Subject Numbers}

The server can generate sequential subject numbers:

\textbf{PEBL function:}
\begin{verbatim}
gSubNum <- GetNewSubNum("your-domain.com",
                        "/pebl-data/getnewsubnum.php",
                        "80",
                        "admin",
                        "yourpassword")

Print("Assigned subject number: " + gSubNum)
\end{verbatim}

This ensures unique subject numbers across multiple testing stations.

\subsubsection{Server File Structure}

Data is organized hierarchically:
\begin{verbatim}
uploads/
  stroop/               # Task name
    001/                # Subject number
      stroop.csv
      stroop_1.csv      # Auto-numbered if duplicate
    002/
    003/
  ant/                  # Another task
    001/
    002/
\end{verbatim}

\subsubsection{Server Configuration}

Edit \texttt{config.php} to customize:

\begin{verbatim}
// Admin credentials
define('ADMIN_USERNAME', 'admin');
define('ADMIN_PASSWORD_HASH', 'your-hash');

// Upload limits
define('MAX_FILE_SIZE', 10485760);  // 10MB
define('ALLOWED_EXTENSIONS', ['txt', 'csv', 'tsv', 'dat', 'log']);

// Timezone
date_default_timezone_set('America/New_York');

// Debug mode (for troubleshooting)
define('DEBUG_MODE', false);
\end{verbatim}

\subsubsection{Security Best Practices}

\begin{itemize}
\item \textbf{Use HTTPS} - Always use SSL/TLS encryption (port 443)
\item \textbf{Strong passwords} - Use 12+ character passwords
\item \textbf{Regular backups} - Download \texttt{uploads/} directory regularly
\item \textbf{Monitor uploads} - Check for unusual files periodically
\item \textbf{IP restrictions} - Limit access to known networks (optional)
\item \textbf{Update PHP} - Use PHP 7.4+ for security patches
\end{itemize}

\subsection{PEBLHub Online}

PEBLHub (\href{https://peblhub.online}{peblhub.online}) is a cloud-based platform for PEBL data collection. It provides:

\begin{itemize}
\item Web-based experiment launching
\item Centralized data management
\item Study creation and configuration
\item Participant tracking
\item Data visualization and export
\item No server setup required
\end{itemize}

\subsubsection{Using PEBLHub}

\textbf{Step 1:} Create account at \href{https://peblhub.online}{peblhub.online}

\textbf{Step 2:} Create a study and configure tests

\textbf{Step 3:} Download test package from PEBLHub

PEBLHub allows you to download a complete test package that includes the test script, parameters, translations, and a pre-configured \texttt{upload.json} file. Simply download the package from your study dashboard and run it locally---upload settings work out-of-the-box without any additional configuration.

\textbf{Step 4:} (Alternative) Manually create \texttt{upload.json} with PEBLHub credentials

If you prefer to configure upload manually, get credentials from study settings and create:
\begin{verbatim}
{
  "server": "peblhub.online",
  "page": "/api/upload",
  "port": "443",
  "username": "your-study-id",
  "upload_password": "your-study-token",
  "taskname": "stroop"
}
\end{verbatim}

\textbf{Step 5:} Upload is automatic when experiment completes

\subsubsection{PEBLHub Features}

\begin{itemize}
\item \textbf{Study Management} - Create and configure multi-test batteries
\item \textbf{Participant Portal} - Secure links for participant access
\item \textbf{Data Dashboard} - Real-time collection monitoring
\item \textbf{Bulk Export} - Download all data as CSV/JSON
\item \textbf{Test Library} - Browse and deploy battery tests
\item \textbf{Version Control} - Track parameter changes over time
\end{itemize}

\subsection{Upload Best Practices}

\textbf{Error Handling:}
\begin{verbatim}
## Always handle upload failures gracefully
success <- UploadFile(gSubNum, gFileOut)

if(not success)
{
  ## Inform participant
  MessageBox("Data saved locally. Please contact experimenter.", gWin)

  ## Log failure
  Print("WARNING: Upload failed for subject " + gSubNum)

  ## Data is still saved locally - never lost!
}
\end{verbatim}

\textbf{Network Testing:}
\begin{itemize}
\item Test uploads on actual network before study
\item Check firewall and proxy settings
\item Verify HTTPS certificates are valid
\item Test with slow/unreliable connections
\end{itemize}

\textbf{Participant Privacy:}
\begin{itemize}
\item Use subject numbers, not names
\item Avoid collecting identifying information in data files
\item Ensure server has appropriate security
\item Follow institutional IRB requirements
\end{itemize}

\textbf{Backup Strategy:}
\begin{itemize}
\item Local data files are always saved (even if upload fails)
\item Download server data regularly
\item Keep local backups during data collection
\item Verify data integrity after upload
\end{itemize}

\sect{Complete Example Task}

This example demonstrates all systems working together:

\subsection{Directory Structure}

\begin{verbatim}
battery/mytask/
  mytask.pbl                      # Main script
  mytask.pbl.about.txt            # Task description
  params/
    mytask.pbl.schema.json        # Parameter schema
    mytask.pbl.par.json           # Default parameters
    short-version.json            # Preset: short version
    tablet-version.json           # Preset: tablet mode
  translations/
    mytask.pbl-en.json            # English strings
    mytask.pbl-es.json            # Spanish strings
    mytask.pbl-fr.json            # French strings
  upload.json                     # Upload configuration
  data/                           # Output (auto-created)
\end{verbatim}

\subsection{Parameter Schema}

\textbf{File:} \texttt{params/mytask.pbl.schema.json}
\begin{verbatim}
{
  "test": "mytask",
  "version": "1.0",
  "description": "Example task demonstrating full architecture",
  "parameters": [
    {
      "name": "numPracticeTrials",
      "type": "integer",
      "default": 5,
      "min": 0,
      "max": 20,
      "label": "Practice Trials",
      "description": "Number of practice trials"
    },
    {
      "name": "numTestTrials",
      "type": "integer",
      "default": 20,
      "min": 1,
      "max": 100,
      "label": "Test Trials",
      "description": "Number of test trials"
    },
    {
      "name": "responsemode",
      "type": "string",
      "default": "auto",
      "options": ["auto", "keyboardShift", "keyboardSafe",
                  "mousetarget", "mousebutton"],
      "label": "Response Mode",
      "description": "Response method (auto = platform-aware)"
    },
    {
      "name": "showFeedback",
      "type": "boolean",
      "default": 1,
      "label": "Show Feedback",
      "description": "Display feedback after each trial"
    },
    {
      "name": "uploadData",
      "type": "boolean",
      "default": 0,
      "label": "Upload Data",
      "description": "Upload data to server after completion"
    }
  ]
}
\end{verbatim}

\subsection{Translation Files}

\textbf{File:} \texttt{translations/mytask.pbl-en.json}
\begin{verbatim}
{
  "TITLE": "My Task",
  "INSTRUCTIONS": "Press the key corresponding to the stimulus.",
  "PRACTICE_BEGIN": "Practice will begin. Press any key.",
  "PRACTICE_COMPLETE": "Practice complete! Press any key for test.",
  "TEST_BEGIN": "Test will begin. Press any key.",
  "TRIAL_HEADER": "Trial",
  "FEEDBACK_CORRECT": "Correct!",
  "FEEDBACK_INCORRECT": "Incorrect.",
  "FOOTER_PROMPT": "Respond as quickly and accurately as possible.",
  "UPLOADING": "Uploading data...",
  "UPLOAD_SUCCESS": "Upload complete!",
  "UPLOAD_FAILED": "Upload failed. Data saved locally.",
  "DEBRIEF": "Thank you for participating!"
}
\end{verbatim}

\textbf{File:} \texttt{translations/mytask.pbl-es.json}
\begin{verbatim}
{
  "TITLE": "Mi Tarea",
  "INSTRUCTIONS": "Presione la tecla correspondiente al estímulo.",
  "PRACTICE_BEGIN": "La práctica comenzará. Presione cualquier tecla.",
  "PRACTICE_COMPLETE": "¡Práctica completa! Presione para la prueba.",
  "TEST_BEGIN": "La prueba comenzará. Presione cualquier tecla.",
  "TRIAL_HEADER": "Ensayo",
  "FEEDBACK_CORRECT": "¡Correcto!",
  "FEEDBACK_INCORRECT": "Incorrecto.",
  "FOOTER_PROMPT": "Responda lo más rápido y preciso posible.",
  "UPLOADING": "Subiendo datos...",
  "UPLOAD_SUCCESS": "¡Carga completa!",
  "UPLOAD_FAILED": "Fallo de carga. Datos guardados localmente.",
  "DEBRIEF": "¡Gracias por participar!"
}
\end{verbatim}

\subsection{Main Script}

\textbf{File:} \texttt{mytask.pbl}
\begin{verbatim}
define Start(p)
{
  ##=================================================================
  ## SETUP
  ##=================================================================
  gScriptname <- "My Task"

  ## Define parameters with defaults
  parpairs <- [
    ["numPracticeTrials", 5],
    ["numTestTrials", 20],
    ["responsemode", "auto"],
    ["showFeedback", 1],
    ["uploadData", 0]
  ]

  ## Load parameters (from file, URL, or defaults)
  gParams <- CreateParameters(parpairs, gParamFile)

  ## Create window
  gWin <- MakeWindow("black")

  ## Get subject number
  if(gSubNum+"" == "0")
  {
    gSubNum <- GetSubNum(gWin)
  }

  ## Load translations
  gStrings <- GetTranslations("mytask", gLanguage)

  ##=================================================================
  ## CREATE LAYOUT & RESPONSE SYSTEM
  ##=================================================================
  gLayout <- CreateLayout("mytask", gWin, gParams)

  ## Configure zones
  gLayout.header.text <- gStrings.TITLE
  gLayout.footer.text <- gStrings.INSTRUCTIONS

  ##=================================================================
  ## DATA FILE
  ##=================================================================
  gFileOut <- GetNewDataFile(gSubNum, gWin, "mytask", "csv",
    "subnum,trial,phase,stimulus,response,correct,rt,timestamp")

  ##=================================================================
  ## INSTRUCTIONS
  ##=================================================================
  MessageBox(gStrings.INSTRUCTIONS, gWin)

  ##=================================================================
  ## PRACTICE TRIALS
  ##=================================================================
  if(gParams.numPracticeTrials > 0)
  {
    gLayout.subheader.text <- "PRACTICE"
    gLayout.footer.text <- gStrings.PRACTICE_BEGIN
    Draw()
    WaitForAnyKeyPress()

    loop(trial, Sequence(1, gParams.numPracticeTrials, 1))
    {
      result <- RunTrial("practice", trial)
      FilePrint(gFileOut, result)
    }

    ## Practice complete
    gLayout.footer.text <- gStrings.PRACTICE_COMPLETE
    Draw()
    WaitForAnyKeyPress()
  }

  ##=================================================================
  ## TEST TRIALS
  ##=================================================================
  gLayout.subheader.text <- "TEST"
  gLayout.footer.text <- gStrings.TEST_BEGIN
  Draw()
  WaitForAnyKeyPress()

  loop(trial, Sequence(1, gParams.numTestTrials, 1))
  {
    result <- RunTrial("test", trial)
    FilePrint(gFileOut, result)
  }

  ##=================================================================
  ## DATA UPLOAD
  ##=================================================================
  FileClose(gFileOut)

  if(gParams.uploadData)
  {
    ## Show upload screen
    gLayout.header.text <- gStrings.UPLOADING
    gLayout.footer.text <- ""
    Draw()

    ## Upload
    success <- UploadFile(gSubNum, gFileOut)

    ## Report result
    if(success)
    {
      gLayout.footer.text <- gStrings.UPLOAD_SUCCESS
    } else {
      gLayout.footer.text <- gStrings.UPLOAD_FAILED
    }

    Draw()
    Wait(2000)
  }

  ##=================================================================
  ## DEBRIEF
  ##=================================================================
  gLayout.header.text <- gStrings.TITLE
  gLayout.subheader.visible <- 0
  gLayout.footer.text <- gStrings.DEBRIEF
  Draw()
  WaitForAnyKeyPress()
}

define RunTrial(phase, trialNum)
{
  ## Update header
  gLayout.header.text <- gStrings.TITLE + " - " +
                         gStrings.TRIAL_HEADER + " " + trialNum

  ## Generate stimulus (example: random color)
  colors <- ["red", "blue", "green", "yellow"]
  stimulus <- SampleN(colors, 1)[1]
  correctResp <- First(["left", "right"])

  ## Show stimulus
  stim <- Circle(gLayout.stimulusRegion.centerX,
                 gLayout.stimulusRegion.centerY,
                 50, MakeColor(stimulus), 1)
  AddObject(stim, gWin)
  Draw()

  ## Collect response
  startTime <- GetTime()
  response <- WaitForLayoutResponse(gLayout, 5000)
  rt <- GetTime() - startTime

  ## Determine correctness
  correct <- (response == correctResp)

  ## Show feedback (if enabled)
  if(gParams.showFeedback)
  {
    if(response == "<timeout>")
    {
      feedback <- "TOO SLOW"
    } else {
      if(correct)
      {
        feedback <- gStrings.FEEDBACK_CORRECT
      } else {
        feedback <- gStrings.FEEDBACK_INCORRECT
      }
    }

    ShowFeedback(feedback, 500)
  }

  ## Clean up
  RemoveObject(stim, gWin)

  ## Return data row
  return(gSubNum + "," + trialNum + "," + phase + "," +
         stimulus + "," + response + "," + correct + "," +
         rt + "," + TimeStamp())
}

define ShowFeedback(message, duration)
{
  ## Show feedback in subheader
  gLayout.subheader.text <- message
  gLayout.subheader.visible <- 1
  Draw()
  Wait(duration)
  gLayout.subheader.visible <- 0
  Draw()
}
\end{verbatim}

\subsection{Running the Task}

\textbf{Default configuration:}
\begin{verbatim}
pebl2 battery/mytask/mytask.pbl
\end{verbatim}

\textbf{With Spanish translation:}
\begin{verbatim}
pebl2 battery/mytask/mytask.pbl --language es
\end{verbatim}

\textbf{With preset configuration:}
\begin{verbatim}
pebl2 battery/mytask/mytask.pbl --pfile params/tablet-version.json
\end{verbatim}

\textbf{With specific subject number:}
\begin{verbatim}
pebl2 battery/mytask/mytask.pbl -s 042
\end{verbatim}

\textbf{With remote parameters:}
\begin{verbatim}
pebl2 battery/mytask/mytask.pbl \
  --pfile https://study.com/api/params?token=ABC123
\end{verbatim}

\sect{Migration Checklist}

To add structured architecture to existing battery tasks:

\subsection{Parameters}

\begin{itemize}
\item[\textbullet] Create \texttt{params/taskname.pbl.schema.json} with all parameters
\item[\textbullet] Create \texttt{params/taskname.pbl.par.json} with defaults
\item[\textbullet] Replace hardcoded values with \texttt{gParams.paramname}
\item[\textbullet] Add parameter validation at script start
\item[\textbullet] Create preset files for common configurations
\item[\textbullet] Test with various parameter combinations
\end{itemize}

\subsection{Translations}

\begin{itemize}
\item[\textbullet] Create \texttt{translations/} directory
\item[\textbullet] Extract all participant-visible text to \texttt{-en.json}
\item[\textbullet] Replace hardcoded strings with \texttt{gStrings.KEY}
\item[\textbullet] Load translations with \texttt{GetTranslations()}
\item[\textbullet] Test with English to verify extraction
\item[\textbullet] Create additional language translations
\item[\textbullet] Test with multiple languages
\end{itemize}

\subsection{Layout \& Response System}

\begin{itemize}
\item[\textbullet] Add \texttt{responsemode} parameter
\item[\textbullet] Replace hardcoded layouts with \texttt{CreateLayout()}
\item[\textbullet] Use \texttt{WaitForLayoutResponse()} for input
\item[\textbullet] Update instructions based on response mode
\item[\textbullet] Test with each response mode
\item[\textbullet] Test on native and Emscripten platforms
\end{itemize}

\subsection{Data Upload}

\begin{itemize}
\item[\textbullet] Add \texttt{uploadData} parameter
\item[\textbullet] Add upload logic after file close
\item[\textbullet] Handle upload failures gracefully
\item[\textbullet] Create \texttt{upload.json} example file
\item[\textbullet] Test upload with local server
\item[\textbullet] Document upload setup in \texttt{.about.txt}
\end{itemize}

\sect{Summary}

The structured task creation architecture provides:

\begin{itemize}
\item \textbf{Configurability} - Parameters control all aspects without code changes
\item \textbf{Internationalization} - Full translation support for any language
\item \textbf{Platform awareness} - Automatic adaptation to native/web environments
\item \textbf{Data collection} - Built-in upload to self-hosted or cloud servers
\item \textbf{Maintainability} - Standardized structure across all battery tasks
\item \textbf{Documentation} - Self-describing schemas and translation files
\end{itemize}

All systems are optional and can be adopted incrementally. Core PEBL functionality remains independent, ensuring backward compatibility with existing experiments.

For detailed information on the Layout \& Response System, see Chapter~\ref{sec:layout}.
