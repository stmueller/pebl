\chapter{The PEBL Layout \& Response System}
\label{sec:layout}

\sect{Overview}

The PEBL Layout \& Response System provides a unified interface for creating structured experimental layouts with automatic screen scaling and platform-aware response handling. This system simplifies the creation of battery tasks by providing:

\begin{itemize}
\item \textbf{Zone-based layouts} with automatic positioning and scaling
\item \textbf{Platform-aware response modes} (keyboard, mouse, touch)
\item \textbf{Single-function API} with intelligent defaults
\item \textbf{Theme support} and accessibility features
\item \textbf{Integration} with parameters, translations, and data upload
\end{itemize}

\sect{Quick Start}

The simplest way to use the Layout \& Response System is with two function calls:

\begin{verbatim}
define Start(p)
{
  gWin <- MakeWindow("black")
  gParams <- CreateParameters(parpairs, gParamFile)

  ## Create layout with response system
  gLayout <- CreateLayout("mytask", gWin, gParams)

  ## Set zone content
  gLayout.header.text <- "My Task - Trial 1"
  gLayout.footer.text <- "Press a response key"

  Draw()

  ## Wait for response (returns "left" or "right")
  response <- WaitForLayoutResponse(gLayout, 5000)

  Print("Response: " + response)
}
\end{verbatim}

This creates a full-screen layout with five zones (header, subheader, stimulus, response, footer) and handles both keyboard and mouse responses automatically based on the platform.

\sect{Layout Zones}

The layout system divides the screen into five configurable zones:

\subsection{Zone Structure}

\begin{itemize}
\item \textbf{Header Zone} - Title and primary information (default: 50px height, 28pt font)
\item \textbf{Subheader Zone} - Trial counter, status (default: 25px height, 18pt font, optional)
\item \textbf{Stimulus Zone} - Main content area (flexible height, takes remaining space)
\item \textbf{Response Zone} - Response key/target labels (default: 50px height, 20pt font)
\item \textbf{Footer Zone} - Instructions, prompts (default: 50px height, 16pt font)
\end{itemize}

All zones scale proportionally based on screen resolution. The default baseline is 800x600, with all dimensions scaling up or down to match the actual screen size.

\subsection{Accessing Zones}

Zone labels are accessed as properties of the layout object:

\begin{verbatim}
gLayout.header.text <- "Trial " + trial
gLayout.subheader.text <- "Block " + block
gLayout.footer.text <- "Press LEFT or RIGHT"

## Control visibility
gLayout.header.visible <- 1  ## show
gLayout.subheader.visible <- 0  ## hide
\end{verbatim}

\subsection{Stimulus Region}

The stimulus region is the flexible area where experimental stimuli are presented:

\begin{verbatim}
## Center of stimulus region
centerX <- gLayout.stimulusRegion.centerX
centerY <- gLayout.stimulusRegion.centerY

## Add stimuli
stim <- Circle(centerX, centerY, 50, MakeColor("red"), 1)
AddObject(stim, gWin)
\end{verbatim}

The stimulus region automatically adjusts its size based on which zones are visible and the current screen dimensions.

\sect{Response Modes}

The Layout \& Response System supports multiple response modes that can be selected via parameters:

\subsection{Keyboard Modes}

\textbf{keyboardShift} - Left/right shift keys (native platforms only)
\begin{itemize}
\item \textbf{Keys}: Left Shift, Right Shift
\item \textbf{Labels}: "LEFT-SHIFT", "RIGHT-SHIFT"
\item \textbf{Platform}: Native only (avoids Windows Sticky Keys dialog)
\item \textbf{Use case}: Traditional two-alternative tasks on desktop
\end{itemize}

\textbf{keyboardSafe} - Browser-safe letter keys
\begin{itemize}
\item \textbf{Keys}: Z, / (slash)
\item \textbf{Labels}: "Z", "/"
\item \textbf{Platform}: All (default for Emscripten/browser)
\item \textbf{Use case}: Web-based experiments, avoids modifier key issues
\end{itemize}

\subsection{Mouse Modes}

\textbf{mousetarget} - Click on labeled targets
\begin{itemize}
\item \textbf{Target}: Clickable response zone labels
\item \textbf{Labels}: "Click LEFT", "Click RIGHT"
\item \textbf{Platform}: All
\item \textbf{Use case}: Touch screens, tablets, mouse-based interfaces
\end{itemize}

\textbf{mousebutton} - Left/right mouse buttons
\begin{itemize}
\item \textbf{Buttons}: Left button (1), Right button (3)
\item \textbf{Labels}: "LEFT-CLICK", "RIGHT-CLICK"
\item \textbf{Platform}: All
\item \textbf{Use case}: Tasks requiring fast mouse-based responses
\end{itemize}

\subsection{Platform-Aware Selection}

When \texttt{responsemode} is set to \texttt{"auto"} (the default), the system automatically selects:
\begin{itemize}
\item \textbf{Native platforms} (Linux, macOS, Windows): keyboardShift
\item \textbf{Emscripten/Browser}: keyboardSafe
\end{itemize}

This ensures optimal performance and user experience across platforms.

\sect{Parameter Integration}

The Layout \& Response System integrates with PEBL's parameter management system. The most important parameter is \texttt{responsemode}, which determines how participants respond.

\subsection{Basic Usage}

Pass parameters to the layout to control response mode:

\begin{verbatim}
define Start(p)
{
  ## Define parameters including responsemode
  parpairs <- [
    ["responsemode", "auto"],  ## Platform-aware by default
    ["numTrials", 20]
  ]

  ## Load parameters
  gParams <- CreateParameters(parpairs, gParamFile)

  ## Pass to layout (reads gParams.responsemode)
  gLayout <- CreateLayout("taskname", gWin, gParams)
}
\end{verbatim}

\subsection{Response Mode Parameter}

The \texttt{responsemode} parameter accepts:
\begin{itemize}
\item \texttt{"auto"} - Platform-aware (keyboardShift on native, keyboardSafe on web)
\item \texttt{"keyboardShift"} - Left/right shift keys
\item \texttt{"keyboardSafe"} - Z and / keys (browser-safe)
\item \texttt{"mousetarget"} - Click on labeled targets
\item \texttt{"mousebutton"} - Left/right mouse buttons
\end{itemize}

\textbf{For complete details on parameters, schemas, presets, and remote loading, see Chapter~\ref{sec:taskarchitecture}.}

\sect{Translation Support}

The Layout \& Response System integrates with PEBL's translation system for multilingual support.

\subsection{Basic Usage}

Load translations and use them in layout zones:

\begin{verbatim}
define Start(p)
{
  ## Load translations for current language
  gStrings <- GetTranslations("taskname", gLanguage)

  ## Use translated strings in layout zones
  gLayout.header.text <- gStrings.HEADER_TITLE
  gLayout.footer.text <- gStrings.INSTRUCTIONS

  Draw()
}
\end{verbatim}

\subsection{Running in Different Languages}

\begin{verbatim}
## English (default)
pebl2 taskname.pbl

## Spanish
pebl2 taskname.pbl -v language=es

## French
pebl2 taskname.pbl -v language=fr
\end{verbatim}

\textbf{For complete details on translation file structure, special characters, creating translations, and best practices, see Chapter~\ref{sec:taskarchitecture}.}

\sect{Data Upload Integration}

Tasks using the Layout \& Response System can integrate with PEBL's data upload functionality to provide feedback during upload.

\subsection{Basic Usage}

Use layout zones to show upload status:

\begin{verbatim}
define Start(p)
{
  ## ... run experiment ...

  FileClose(gFileOut)

  ## Upload with status display
  if(gParams.uploadData)
  {
    ## Show upload screen
    gLayout.header.text <- "Uploading Data"
    gLayout.footer.text <- "Please wait..."
    Draw()

    ## Upload file
    success <- UploadFile(gSubNum, gFileOut)

    ## Show result
    if(success)
    {
      gLayout.footer.text <- "Upload complete!"
    } else {
      gLayout.footer.text <- "Upload failed. Data saved locally."
    }

    Draw()
    Wait(2000)
  }
}
\end{verbatim}

The layout system provides a clean way to show upload progress and results without requiring popup dialogs.

\textbf{For complete details on data upload configuration, server setup (PEBL Simple Data Server and PEBLHub), and best practices, see Chapter~\ref{sec:taskarchitecture}.}

\sect{Layout Customization}

The Layout \& Response System can be customized per-task or globally.

\subsection{Custom Layout Configuration}

Create a custom layout file \texttt{battery/taskname/layout/taskname.pbl.layout.json}:

\begin{verbatim}
{
  "name": "Custom Task Layout",
  "version": "1.0",
  "baseline": {
    "width": 800,
    "height": 600
  },
  "margins": {
    "x": 25,
    "y": 25,
    "bottomReserve": 25
  },
  "zones": {
    "header": {
      "height": 60,
      "fontSize": 32,
      "visible": true
    },
    "subheader": {
      "height": 30,
      "fontSize": 20,
      "visible": true
    },
    "stimulus": {
      "flexGrow": true,
      "fontSize": 24
    },
    "response": {
      "height": 70,
      "fontSize": 24,
      "visible": true
    },
    "footer": {
      "height": 60,
      "fontSize": 18,
      "visible": true
    }
  },
  "responseDefaults": {
    "type": "auto",
    "numAlternatives": 2
  }
}
\end{verbatim}

\subsection{Modifying Global Defaults}

To change the default layout for all tasks, edit:
\begin{verbatim}
media/settings/default-layout.json
\end{verbatim}

Changes to the default layout affect all tasks that don't have a custom layout configuration.

\sect{Complete Example}

Here is a complete example demonstrating all features:

\begin{verbatim}
define Start(p)
{
  ##--------------------------------------------------------------------------
  ## 1. SETUP
  ##--------------------------------------------------------------------------
  gScriptname <- "My Task"

  ## Parameters
  parpairs <- [
    ["numPracticeTrials", 5],
    ["numTestTrials", 20],
    ["responsemode", "keyboardShift"],
    ["uploadData", 0],
    ["uploadURL", ""]
  ]
  gParams <- CreateParameters(parpairs, gParamFile)

  ## Window and subject
  gWin <- MakeWindow("black")
  if(gSubNum+"" == "0") {
    gSubNum <- GetSubNum(gWin)
  }

  ## Load translations
  gStrings <- GetTranslations("mytask", gLanguage)

  ##--------------------------------------------------------------------------
  ## 2. CREATE LAYOUT
  ##--------------------------------------------------------------------------
  gLayout <- CreateLayout("mytask", gWin, gParams)

  ## Configure zones
  gLayout.header.text <- gStrings.TITLE
  gLayout.footer.text <- gStrings.INSTRUCTIONS

  ##--------------------------------------------------------------------------
  ## 3. DATA FILE
  ##--------------------------------------------------------------------------
  gFileOut <- GetNewDataFile(gSubNum, gWin, "mytask", "csv",
    "subnum,trial,phase,stimulus,response,correct,rt,timestamp")

  ##--------------------------------------------------------------------------
  ## 4. PRACTICE TRIALS
  ##--------------------------------------------------------------------------
  gLayout.subheader.text <- "PRACTICE"
  Draw()
  WaitForAnyKeyPress()

  loop(trial, Sequence(1, gParams.numPracticeTrials, 1))
  {
    result <- RunTrial("practice", trial)
    FilePrint(gFileOut, result)
  }

  ##--------------------------------------------------------------------------
  ## 5. TEST TRIALS
  ##--------------------------------------------------------------------------
  gLayout.subheader.text <- "TEST"
  gLayout.footer.text <- gStrings.TEST_INSTRUCTIONS
  Draw()
  WaitForAnyKeyPress()

  loop(trial, Sequence(1, gParams.numTestTrials, 1))
  {
    result <- RunTrial("test", trial)
    FilePrint(gFileOut, result)
  }

  ##--------------------------------------------------------------------------
  ## 6. UPLOAD & DEBRIEF
  ##--------------------------------------------------------------------------
  FileClose(gFileOut)

  if(gParams.uploadData and gParams.uploadURL != "")
  {
    UploadFile(gFileOut, gParams.uploadURL)
  }

  gLayout.header.text <- gStrings.COMPLETE
  gLayout.subheader.visible <- 0
  gLayout.footer.text <- gStrings.DEBRIEF
  Draw()
  Wait(3000)
}

define RunTrial(phase, trialNum)
{
  ## Update display
  gLayout.header.text <- gStrings.TITLE + " - Trial " + trialNum

  ## Show stimulus
  stim <- Circle(gLayout.stimulusRegion.centerX,
                 gLayout.stimulusRegion.centerY,
                 50, MakeColor("red"), 1)
  AddObject(stim, gWin)
  Draw()

  ## Collect response
  startTime <- GetTime()
  response <- WaitForLayoutResponse(gLayout, 5000)
  rt <- GetTime() - startTime

  ## Clean up
  RemoveObject(stim, gWin)

  ## Return data
  return(gSubNum + "," + trialNum + "," + phase +
         ",red_circle," + response + ",,+ rt + "," + TimeStamp())
}
\end{verbatim}

\sect{Best Practices}

\subsection{Layout Usage}
\begin{itemize}
\item Use \texttt{CreateLayout()} once at the start of your experiment
\item Store the result in \texttt{gLayout} for global access
\item Update zone text as needed, don't recreate the layout
\item Use \texttt{Draw()} after updating zone content
\end{itemize}

\subsection{Response Handling}
\begin{itemize}
\item Always pass \texttt{gParams} to \texttt{CreateLayout()} for response mode support
\item Use semantic responses ("left", "right") not raw keys in your logic
\item Handle timeout cases (response == "<timeout>")
\item Provide clear instructions that match the response mode
\end{itemize}

\subsection{Parameters}
\begin{itemize}
\item Always provide a schema file with descriptions and defaults
\item Use descriptive parameter names (not single letters)
\item Create preset JSON files for common configurations
\item Document parameter options in the schema
\end{itemize}

\subsection{Translations}
\begin{itemize}
\item Use UPPERCASE keys by convention
\item Include all participant-visible text
\item Test with at least two languages
\item Provide English (\texttt{-en.json}) as the default
\end{itemize}

\sect{Reference}

\subsection{Core Functions}

\textbf{CreateLayout(testName, win, params)}

Creates a layout with zones, response labels, and scaling.
\begin{itemize}
\item \textbf{testName}: Name of your task (string)
\item \textbf{win}: Window object from MakeWindow()
\item \textbf{params}: Parameter object with .responsemode (optional)
\item \textbf{Returns}: Layout object (also sets gLayout global)
\end{itemize}

\textbf{WaitForLayoutResponse(layout, timeout)}

Waits for a response according to the layout's response mode.
\begin{itemize}
\item \textbf{layout}: Layout object from CreateLayout()
\item \textbf{timeout}: Maximum wait in milliseconds (default: 60000)
\item \textbf{Returns}: Semantic response ("left", "right", etc.) or "<timeout>"
\end{itemize}

\subsection{Layout Properties}

\begin{itemize}
\item \texttt{layout.header.text} - Header text
\item \texttt{layout.header.visible} - Header visibility (0/1)
\item \texttt{layout.subheader.text} - Subheader text
\item \texttt{layout.subheader.visible} - Subheader visibility (0/1)
\item \texttt{layout.footer.text} - Footer text
\item \texttt{layout.footer.visible} - Footer visibility (0/1)
\item \texttt{layout.stimulusRegion.centerX} - Stimulus area center X
\item \texttt{layout.stimulusRegion.centerY} - Stimulus area center Y
\item \texttt{layout.stimulusRegion.width} - Stimulus area width
\item \texttt{layout.stimulusRegion.height} - Stimulus area height
\item \texttt{layout.responseLabels} - List of response label objects
\item \texttt{layout.responseMode.type} - Current response mode type
\end{itemize}

\subsection{Default Configuration}

The default layout configuration (800x600 baseline) includes:
\begin{itemize}
\item Margins: 25px (x and y)
\item Bottom reserve: 25px (no-go zone for windowed mode)
\item Header: 50px height, 28pt font
\item Subheader: 25px height, 18pt font (visible by default)
\item Response: 50px height, 20pt font
\item Footer: 50px height, 16pt font
\item Stimulus: Flexible (takes remaining space, typically ~70\% of screen)
\end{itemize}
