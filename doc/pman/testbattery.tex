%\usepackage[ascii]{inputenc}
%\usepackage[T1]{fontenc}
%\usepackage{amsmath}
%\usepackage{amssymb,amsfonts,textcomp}
%\usepackage{color}
%\usepackage{array}
%\usepackage{hhline}
%\usepackage[pdftex]{graphicx}
% Outline numbering
%\setcounter{secnumdepth}{0}
%\makeatletter
%\newcommand\arraybslash{\let\\\@arraycr}
%\makeatother
% Page layout (geometry)
%\setlength\voffset{-1in}
%\setlength\hoffset{-1in}
%\setlength\topmargin{1in}
%\setlength\oddsidemargin{1.25in}
%\setlength\textheight{9.0in}
%\setlength\textwidth{6.0in}
%\setlength\footskip{0.0cm}
%\setlength\headheight{0cm}
%\setlength\headsep{0cm}
% Footnote rule
\setlength{\skip\footins}{0.0469in}
\renewcommand\footnoterule{\vspace*{-0.0071in}\setlength\leftskip{0pt}\setlength\rightskip{0pt plus 1fil}\noindent\textcolor{black}{\rule{0.25\columnwidth}{0.0071in}}\vspace*{0.0398in}}
% Pages styles
%\makeatletter
%\newcommand\ps@Standard{
%  \renewcommand\@oddhead{}
%  \renewcommand\@evenhead{}
%  \renewcommand\@oddfoot{}
%  \renewcommand\@evenfoot{}
%  \renewcommand\thepage{\arabic{page}}
%}
%\makeatother
%\pagestyle{Standard}
%\setlength\tabcolsep{1mm}
%\renewcommand\arraystretch{1.3}
%\includegraphics[width=6.2291in,height=0.9374in]{images/PEBL20List20of20TestsSTV-img1.jpg}
\chapter{The PEBL Psychological Test Battery}
\label{sec:testbattery} 
\textbf{This chapter contributed by Bryan Rowley in collaboration with Shane Mueller} 

\sect{About the PEBL Test Battery}



This site is for a battery of psychological tests implemented in PEBL
and distributed (and redistributeable) freely. They are designed to be
easily used on multiple computing platforms, running natively under
Win32, Linux, and OSX Operating Systems. The tests are designed to
implement a wide range of computer-administered psychological tests and
experiments of interest to neuropsychological, cognitive, clinical
communities. 

{The current version of the battery is designed to work with PEBL
version 2.0 and was released in 2016. It is distributed with PEBL
2.0, and is automatically installed in My
Documents{\textbackslash}pebl.2.0{\textbackslash}battery on windows.} 


These tests are designed to implement a wide range of tests that are
used throughout the psychological, neuropsychological research and
clinical communities. Some are reimplementations of tests that are only
available on limited computing platforms or cost hundreds of dollars.
Each experiment saves the complete data set for later analysis, and
many compute basic analyses that it writes in report format. 

\sect{Setting Parameters of Battery Tests}
More details of parameter-setting are available within the next chapter that covers the launcher.  

The tests within the battery typically expose the most important instrumentation variables that control important aspects of the test.  These often include the number of trials, the make-up of stimuli, etc. For example, the following shows parameters for the ANT test, which is opened when you hit the 'edit' button near the parameters pulldown when you have a parameter-enabled test selected in the file window.  In this test, the leftmost column indicates the name of the parameter; the next column indicates its current value, and the final column describes the value along with its default value.   
\includegraphics[width=\textwidth]{images/params.png}

If you want to create a custom parameter set, edit these values and click 'Save file and exit'.  This will create a default parameter file that will be used.  You can also type a a new name, and save it, and then select the new name in the parameters pulldown. You can then create an experiment chain and select one parameter set or another to make setup easier and error-free.



\sect{Translating or changing test instructions}
Most of the tests within the test battery permit translating any participant-visible text.  This usually includes instructions, debriefing, and headers/stimuli/labels. Each test needs to be designed to permit this, but most of the tests in the battery have (most that don't involve primarily English stimuli/materials, such as memory tests).

To translate a test, first be sure the 'language' entry box is named according to the two-letter code associated with your language of choice.  By default, it will choose en for English. Then, select the test within the file section scrollbox, and click on the 'translate test' button on the lower right of the window. This will bring up the following screen:

\includegraphics[width=\textwidth]{images/translatedialog.png}

If the language selection is correct, you are fine; otherwise edit the language to be whatever two-letter code you want to use.  If you choose en, you will edit the default instructions, and if you make an error you may have to re-copy the translation file from the main PEBL\\battery\\ directory (i.e., in Program Files(x86)\\PEBL).

In this dialog,  each critical piece of text has a name that is referred to within the testing script.  The next column indicates the original text, and the third column is the translated text (which will probably be in English if no translations have been made previously).  Select the name on the left, and edit the text on the right.  If you want, you can right-click on the window to clear the text or copy in text created elsewhere.  

For some languages, this translation dialog may not work--we are still working on improving international keyboard input.  In reality, this just edits a .csv file that is stored within the test\\translations\\ directory.  For a test called test.pbl, the English file will be called test\\translations\\test.pbl-en.csv.  You can also edit this with a normal text editor or spreadsheet.  To edit by hand, copy the English file to one with a name associated with your chosen language, replacing -en with your language code.  Then, edit using either a text editor like notepad++, or a spreadsheet program.  Edit only the words within the second column.  If you want to add a line break, use \verb|\n|.


\sect{The Tests}
{

The following table describes the basic tests currently implemented in
the PEBL Test Battery. Many of them represent the only Free version of
proprietory tests available anywhere. They include a free Iowa Gambling
Task, a free version of the TOVA{\textregistered}, a free Wisconsin
Card Sort Test{\textregistered}, a free version of Conners Continuous
Performance task, and a number of other useful tasks, with more to
come. All screenshots found on this page are released into the public
domain, and can be used for whatever purpose without copyright
assignment, including in academic papers. More information on tests is
found in the


\href{http://pebl.sourceforge.net/wiki/index.php?title=PEBL_Test_Battery}{\textstyleInternetlink{PEBL WIKI}}} 


\newpage

\begin{longtable}{p{5cm}p{7cm}} 
\caption{Test Battery}  \label{tab:testbattery} \\ 
\toprule 
 \textbf{PEBL Test/Version of:} &
 \textbf{Description} \\\hline


\midrule 
\addlinespace[0.2cm] 
\endfirsthead 

\midrule 
 \textbf{PEBL Test/Version of:} &
 \textbf{Description} \\
\midrule 
\addlinespace[0.2cm] 
\endhead 
\hline

\href{http://pebl.sourceforge.net/wiki/index.php?title=Bechara's_Gambling_Task}{\textstyleInternetlink{
Bechera{\textquotesingle}s Gambling Task}}\newline  
(version of Bechara{\textquotesingle}s Iowa Gambling Task
{\textregistered} \newline
\href{http://en.wikipedia.org/wiki/Iowa_gambling_task}{\textstyleInternetlink{wikipedia}})\newline

\includegraphics[width=2in]{images/PEBL20List20of20TestsSTV-img2.jpg}
 &
 
 Choose from four decks, each choice with a cost
and each providing reward. Used for tests of executive control.\newline
\textbf{Key Skills used:} Decision Making, Strategy and Problem
Solving, Risk Assessment. \newline
\textbf{Note:} the task requires individuals
to decide on which deck to choose from, with the chance of loosing in
the process. Test can be modified to ask individual to achieve a
certain amount of money. 
\\\hline

%%%%%%%%%%%%%%%%%%%%%%%%%%%%%%%%%%%%%%%%%%%%%%%%
\href{http://pebl.sourceforge.net/wiki/index.php?title=The_Hungry_Donkey_Task}{\textstyleInternetlink{The {\textquotedbl}Hungry Donkey{\textquotedbl} Task}} \newline
 A version of Bechera{\textquotesingle}s
Gambling Task for children\newline
\includegraphics[width=2in]{images/PEBL20List20of20TestsSTV-img3.png}
 &
 The donkey chooses from four doors, each door
has a cost and reward in apples. Used for tests of executive control.\newline
\textbf{Key Skills used:} Fine-motor skills, Visual processing.\newline
\textbf{Note:} Test can be modified to ask
individual to reach a certain number of apples (i.e. 10 apples) in a
certain amount of time.  
\\\hline


%%%%%%%%%%%%%%%%%%%%%%%%%%%%%%%%%%%
\href{http://pebl.sourceforge.net/wiki/index.php?title=TOAV:_Test_of_Attentional_Vigilance}{\textstyleInternetlink{TOAV: Test of Attentional Vigilance}}\newline
A Version of TOVA{\textregistered}: Test of Variables of
Attention \newline
\href{http://en.wikipedia.org/wiki/Tova}{\textstyleInternetlink{wikipedia}}\newline

\includegraphics[width=2in]{images/PEBL20List20of20TestsSTV-img4.png}
&
 22-minute test requiring subject to detect a
rare visual stimulus (top or bottom). Used to diagnose ADD, ADHD, etc.\newline

\textbf{Key Skills used:} Concentration, Reaction Time, Attention\newline
\textbf{Note:} This task requires the
individual to concentrate for an extended period of time. Thus, the
extent to which their reaction time scores alter through the duration
of this test can be indicative of how their attention levels have been
affected.
 \\ \hline

%%%%%%%%%%%%%%%%%%%%%%%%%%%%%%%%%%%
\href{http://pebl.sourceforge.net/wiki/index.php?title=PEBL_Continuous_Performance_Test}{\textstyleInternetlink{PEBL Continuous Performance Test}} \newline
Version of Conners CPT \href{http://en.wikipedia.org/wiki/Continuous_Performance_Task}{\textstyleInternetlink{wikipedia}}\newline
\includegraphics[width=2in]{images/PEBL20List20of20TestsSTV-img5.png}
 &
 14-minute vigilance test requiring subject
respond to non-matches. Used to diagnose ADD, ADHD, etc.\newline
 \textbf{Key Skills Used:} Reaction Time, Attention, Concentration.\newline
 \textbf{Note:} The test length allows for observation of how their results change overtime (i.e. attention levels altering).  
\\\hline

%%%%%%%%%%%%%%%%%%%%%%%%%%%%%%%%%%%%%%%%%%
\href{http://pebl.sourceforge.net/wiki/index.php?title=PEBL_Perceptual_Vigilance_Task}{\textstyleInternetlink{PEBL Perceptual Vigilance Task (PPVT)}} \newline
 Wilkinson \& Houghton{\textquotesingle}s
Psychomotor Vigilance Task \newline
\href{http://en.wikipedia.org/wiki/Psychomotor_vigilance_task}{\textstyleInternetlink{wikipedia}}\newline
\includegraphics[width=2in]{images/PEBL20List20of20TestsSTV-img6.png}
&

 A vigilance task used to detect vigilance and
sleep lapses. \newline
\textbf{Key Skills Used:} Reaction Time, Attention,
Concentration. \newline
\textbf{Note:} The individual{\textquoteright}s results
can be viewed in data section, and we can observe how their performance
declines or improves throughout test duration. 

\\\hline
%%%%%%%%%%%%%%%%%%%%%%%%%%%%%%%%%%
\href{http://pebl.sourceforge.net/wiki/index.php?title=Berg's_Card_Sorting_Test}{\textstyleInternetlink{Berg{\textquotesingle}s Card Sorting Test}} \newline
version of Berg{\textquotesingle}s (1948) Wisconsin Card
Sorting Test\newline
\href{http://en.wikipedia.org/wiki/Wisconsin_card_sort}{\textstyleInternetlink{wikipedia}}\newline
\includegraphics[width=2in]{images/PEBL20List20of20TestsSTV-img7.png}
 
 &
 Sort multi-attribute cards into piles according
to an unknown and changing rule.\newline
 \textbf{Key Skills used:} Strategy and
Problem Solving, Decision Making, Inhibition, Working Memory.\newline

\textbf{Note:} The results from the data section provide an indication
of which rule (shape, color or number) is easiest for the individual
via reaction time. We are able to see how the
individual{\textquoteright}s working memory is operating by their
ability to recall which rule is active (via correct responses). 
\\\hline

%%%%%%%%%%%%%%%%%%%%%%%%%%%%%%%%%%%%%%%%%%%%%%%%%%
\href{http://pebl.sourceforge.net/wiki/index.php?title=Simple_Response_Time}{\textstyleInternetlink{Simple Response Time}} \newline
\href{http://en.wikipedia.org/wiki/Reaction_time}{\textstyleInternetlink{wikipedia}}\newline
\includegraphics[width=2in]{images/PEBL20List20of20TestsSTV-img8.png}

 &
 Detect the presence of a visual stimulus, as
quickly and accurately as possible.\newline
 \newline\textbf{Key Skills Used:} Reaction
Time, Attention, Fine Motor Skills. \newline
\newline\textbf{Note:} The task allows for
observation of how their attention and reactivity alter throughout the
test{\textquoteright}s duration. The individual can also work on their
executive control and fine motor ability. 
\\\hline

%%%%%%%%%%%%%%%%%%%%%%%%%%%%%%%%%%%%%%
\href{http://pebl.sourceforge.net/wiki/index.php?title=Digit_Span}{\textstyleInternetlink{Digit Span}} \newline
 A component of many intelligence tests \newline
\href{http://en.wikipedia.org/wiki/Digit_span}{\textstyleInternetlink{wikipedia}} \newline
\includegraphics[width=2in]{images/PEBL20List20of20TestsSTV-img9.png}
 &
 Remember a sequence of digits.\newline
  \textbf{Key Skills Used:} Working Memory, Numerical Processing, Short Term Memory.\newline

\textbf{Note:} Primacy, Recency effects can be observed in
this task (i.e. which numbers in the set are being remembered, first
numbers or last numbers). \\\hline

%%%%%%%%%%%%%%%%%%%%%%%%%%%%%%%%%%%%%%%%%%%%%%%%%%%%%%
\href{http://pebl.sourceforge.net/wiki/index.php?title=Partial_Report_Procedure}{\textstyleInternetlink{Partial Report Procedure}} \newline
 Lu et al.{\textquotesingle}s (2005) update of Sperling{\textquotesingle}s iconic memory procedure. \href{http://en.wikipedia.org/wiki/Iconic_memory}{\textstyleInternetlink{wikipedia}}\newline
\includegraphics[width=2in]{images/PEBL20List20of20TestsSTV-img10.png}
 &

  May provide an early-warning sign for
Alzheimer{\textquotesingle}s. \newline
\textbf{Key Skills used:} Reaction Time,
Decision Making, Working Memory. \newline
\textbf{Note:} Individuals are
required to make quick decision based on a brief stimulus shown. Not
recommended for people with slow reaction times. 
\\\hline

%%%%%%%%%%%%%%%%%%%%%%%%%%%%%%%%%%%%%%%%%%%%%%%%%%%%%%
\href{http://pebl.sourceforge.net/wiki/index.php?title=Implicit_Association_Test}{\textstyleInternetlink{Implicit Association Test}}  \newline
 A test of automatic associations between memory
representations. \href{http://en.wikipedia.org/wiki/Implicit_Association_Test}{\textstyleInternetlink{wikipedia}}\newline
\includegraphics[width=2in]{images/PEBL20List20of20TestsSTV-img11.png}
 &
  Tests association between two sets of binary
classifications. \newline

\\\hline

%%%%%%%%%%%%%%%%%%%%%%%%%%%%%%%

\href{http://pebl.sourceforge.net/wiki/index.php?title=Tower_Of_London}{\textstyleInternetlink{Tower of London}} \newline
 Traditional problem solving/planning task\newline
\href{http://en.wikipedia.org/wiki/Tower_of_London_Test}{\textstyleInternetlink{wikipedia}}\newline

\includegraphics[width=2in]{images/PEBL20List20of20TestsSTV-img12.png}
& 

Tests ability to make and follow plans in
problem solving task. \newline

 \textbf{Key Skills Used:} Strategy and Problem
Solving,\newline
\textbf{} Color Processing, Hand-eye coordination, Fine Motor
Skills.\newline
\textbf{ Note:} Test cannot be completed successfully for
color-blind individuals. Task is great for individuals trying to
improve on executive control, and requires both strategy and problem
solving skills to complete successfully. 
 
\\\hline


%%%%%%%%%%%%%%%%%%%%%%%%%%%%%%%%%%%
\textbf{Symbol Counter Task}\newline 
 Garavan (2000) counter task \newline
\includegraphics[width=2in]{images/PEBL20List20of20TestsSTV-img13.png}\newline
 &
 Useful indicator of executive control.\newline
\textbf{Key Skills used:} Reaction Time, Working Memory, Selective
Attention. \newline
\textbf{Note:} We can view if the individual will be able to
recall which symbols are associated with which shift tab (i.e. a
measure of working memory via correct responses). 

\\\hline

%%%%%%%%%%%%%%%%%%%%%%%%%%%%%%%
\href{http://pebl.sourceforge.net/wiki/index.php?title=Four_Choice_Response_Time}{\textstyleInternetlink{Four choice response time}} \newline
 Wilkinson \& Houghton{\textquotesingle}s
4-choice response time\newline
\href{http://en.wikipedia.org/wiki/Reaction_time}{\textstyleInternetlink{wikipedia}}\newline
\includegraphics[width=2in]{images/PEBL20List20of20TestsSTV-img14.png}

 &
 Respond to a plus sign that appears in one of
four corners of the screen.\newline
 \textbf{Key Skills used:} Reaction Time,
Selective Attention, Visual Processing. \newline
\textbf{Note:} the task
measures how quickly the individual{\textquoteright}s attention leads
them to the correct location, combining visual processing abilities
with reaction time.  
\\\hline

%%%%%%%%%%%%%%%%%%%%%%%%%%%%%%%%%%
\href{http://pebl.sourceforge.net/wiki/index.php?title=Time_Wall}{\textstyleInternetlink{Time Wall}} \newline
 UTCPAB{\textquotesingle}s Time wall  \newline

\includegraphics[width=2in]{images/PEBL20List20of20TestsSTV-img15.png}

&
 Estimate the time when a moving target will
reach a location behind a wall.\newline
 \textbf{Key Skills Used:} Reasoning,
Calculating, Reaction Time, Strategy and Problem Solving. \newline
\textbf{Note:
}this task requires tracking of an object after its disappearance. It
requires the individual to in a sense to imagine the location of this
object using precise calculating (of object{\textquoteright}s speed). 
\\\hline

%%%%%%%%%%%%%%%%%%%%%%%%%%%%%%%
\href{http://pebl.sourceforge.net/wiki/index.php?title=PEBL_Compensatory_Tracker}{\textstyleInternetlink{PEBL Compensatory Tracker}}  
\newline
 Similar to Makeig \& Jolley{\textquotesingle}s
 \href{http://sccn.ucsd.edu/~scott/}{\textstyleInternetlink{CompTrack}}
\newline
\includegraphics[width=2in]{images/PEBL20List20of20TestsSTV-img16.png}
&
 Use mouse/trackball to keep a randomly moving
target inside a bullseye.\newline
 \textbf{Key Skills used:} Fine Motor Skills,
Strategy and Problem Solving, Hand Eye Coordination.\newline
 \textbf{Note:} this task can be helpful for individuals wanting to get better with
using a mouse for the computer.  
 
\\\hline

%%%%%%%%%%%%%%%%%%%%%%%%%%%%%%%
\href{http://pebl.sourceforge.net/wiki/index.php?title=Lexical_Decision_Task}{\textstyleInternetlink{Lexical Decision}}\newline
Meyer \& Schvaneveldt{\textquotesingle}s LDT\newline
\href{http://en.wikipedia.org/wiki/Lexical_decision_task}{\textstyleInternetlink{wikipedia}}\newline
\includegraphics[width=2in]{images/PEBL20List20of20TestsSTV-img17.png}
&
Determine whether a stimulus is a word or nonword.\newline
\textbf{Key skills used:}  Linguistic Processing, Language Processing.\newline
\textbf{Note:} the words are able to be changed for the
test. They can be changed to fit closely to an
individual{\textquoteright}s expertise (ex. If individual is aphasic
but has an interest in bands, the words can be altered to include words
of bands they listen to frequently).  
\\\hline

\href{http://pebl.sourceforge.net/wiki/index.php?title=Mental_Rotation}{\textstyleInternetlink{Mental Rotation}} \newline
 Shepard{\textquotesingle}s mental rotation task\newline
\href{http://en.wikipedia.org/wiki/Mental_rotation}{\textstyleInternetlink{wikipedia}}\newline

\includegraphics[width=2in]{images/PEBL20List20of20TestsSTV-img18.png}\newline
 &
 Determine whether two figures are identical,
subject to rotation. \newline
\textbf{Key Skills used:} Reasoning, Visual
Processing, Decision Making. \newline
\textbf{Note:} while observing both
objects, the individual is required to make a decision of whether the
objects are similar, and requires precise reasoning due to their
similarities (i.e. be able to reason that object on left looks
identical to the object on the right, only inverted from the object on
the right) 
\\\hline

%%%%%%%%%%%%%%%%%%%%%%%%%%%%%%%%
\href{http://pebl.sourceforge.net/wiki/index.php?title=Matrix_Rotation}{\textstyleInternetlink{Matrix Rotation}} \newline
Version of UTC test battery matrix rotation  \newline
\includegraphics[width=2in]{images/PEBL20List20of20TestsSTV-img19.png}
 &
 Determine whether a 6x6 matrix is the same
(with rotation) as another. \newline
\textbf{Key skills used:} Selective
Attention, Working Memory, Visual Processing.\newline
 \textbf{Note:} Working
Memory is being tested, we can see how individual{\textquoteright}s
object manipulation or {\textquoteleft}visuo-spatial
sketchpad{\textquoteright} is operating (I.e. correct responses being a
measure of working memory, and the
{\textquoteleft}sketchpad{\textquoteright} the specific component being
measured). 

 \\\hline
 
%%%%%%%%%%%%%%%%%%%%%%%
\href{http://pebl.sourceforge.net/wiki/index.php?title=Spatial_Cueing}{\textstyleInternetlink{Spatial Cueing}} \newline
Posner{\textquotesingle}s attentional cueing (spotlight) task.\href{http://en.wikipedia.org/wiki/Michael_Posner_(psychologist)}{\textstyleInternetlink{wikipedia}}\newline
\includegraphics[width=2in]{images/PEBL20List20of20TestsSTV-img20.png}\newline
 &
 Given a probabilistic cue of where a stimulus
will appear, respond as fast as possible.\newline
 \textbf{Key Skills used:} Selective Attention, Inhibition. \newline 
\textbf{Note:} this task tests the
individual{\textquoteright}s ability to make the correct response
regardless of the correct cue or the distracter cue. We can view how
the distracter cue affects the individual via correct responses and
reaction time.  
\\\hline

\href{http://pebl.sourceforge.net/wiki/index.php?title=Two_column_addition}{\textstyleInternetlink{Two column addition}} \newline
 UTC test battery{\textquotesingle}s 2-column
addition.\newline
 
\includegraphics[width=2in]{images/PEBL20List20of20TestsSTV-img21.png}\newline
&
 Add three two-digit numbers and respond quickly
and accurately. \newline 
\textbf{Key Skills Used:} Mathematical Processing,
Numerical Processing, Working Memory.\newline
\textbf{Note:} it is important to
distinguish between Mathematical and Numerical, as mathematical
processing in this test refers to the manipulation of numerical
information, whereas numerical processing refers to the knowledge of
numerical information (i.e. the understanding that the number
{\textquoteleft}one{\textquoteright} means
{\textquoteleft}1{\textquoteright}.) 
\\\hline

%%%%%%%%%%%%%%%%%%%%%%%%%%%%%%%%%%%%%
\href{http://pebl.sourceforge.net/wiki/index.php?title=Stroop_task}{\textstyleInternetlink{Stroop task}}  \newline
 Stroop{\textquotesingle}s attention
task\newline
\href{http://en.wikipedia.org/wiki/Stroop_effect}{\textstyleInternetlink{wikipedia}}\newline

\includegraphics[width=2in]{images/PEBL20List20of20TestsSTV-img22.png}

 &
 Respond to either the color or name of stimuli.\newline
\textbf{Key Skills Used:} Inhibition, Selective Attention.\newline
\textbf{Note:} Reaction Time is recorded in the data section, allowing
for analysis of which trails are easiest, and which are most
challenging.
\\\hline

%%%%%%%%%%%%%%%%%%%%%%%%%%%%%%%%

\href{http://pebl.sourceforge.net/wiki/index.php?title=PEBL_Manual_Dexterity}{\textstyleInternetlink{PEBL Manual Dexterity}}  \newline
\includegraphics[width=2in]{images/PEBL20List20of20TestsSTV-img23.png}
&
 Move a noisy cursor to the target. \newline
 \textbf{Key Skills used:} Fine Motor Skills, Strategy and Problem Solving, Hand-eye Coordination.\newline
 \textbf{Note:} This task is helpful for individuals
trying to improve their mouse ability with the computer. 

\\\hline

%%%%%%%%%%%%%%%%%%%%%%%%%%%%%%%%%%%
\href{http://pebl.sourceforge.net/wiki/index.php?title=PEBL_Trail-making_task}{\textstyleInternetlink{PEBL Trail-making test}} \newline
 Version of Reitan{\textquotesingle}s (1958)
Trail-making A and B tests.\newline
\href{http://en.wikipedia.org/wiki/Trail-making_test}{\textstyleInternetlink{wikipedia}}\newline

\includegraphics[width=2in]{images/PEBL20List20of20TestsSTV-img24.png}

 &
  Connect the dots task. \newline
 \textbf{Key Skills used:} Language Processing, Numerical Processing, Hand-eye coordination.\newline
\textbf{Note:}this task tests both linguistic and
numerical processing, and tests the individual{\textquoteright}s
ability to navigate to the correct location (i.e. visual processing).
\\\hline

%%%%%%%%%%%%%%%%%%%%%%%%%%%
\href{http://pebl.sourceforge.net/wiki/index.php?title=Aimed_Movement_Task}{\textstyleInternetlink{Aimed Movement (Fitts{\textquotesingle}s Law) test}}\newline  
\href{http://en.wikipedia.org/wiki/Fitts's_Law}{\textstyleInternetlink{wikipedia}}\newline
\includegraphics[width=2in]{images/PEBL20List20of20TestsSTV-img25.png}
 
&
 Mouse-driven implementation of classic perceptual-motor task. \newline
 \textbf{Key Skills used:} Hand-eye
coordination, Fine Motor Skills, Concentration. \newline
\textbf{Note:} The
number of trails (105) requires continuous concentration on the
participants{\textquoteright} behalf.
\\\hline


%%%%%%%%%%%%%%%%%%%%%%%%%%%%%%%%%
\href{http://pebl.sourceforge.net/wiki/index.php?title=Pursuit_Rotor}{\textstyleInternetlink{Pursuit Rotor task}}  \newline
 Classic mechanical test device  \newline
\includegraphics[width=2in]{images/PEBL20List20of20TestsSTV-img26.png}


& Mouse-driven motor pursuit.\newline
 \textbf{Key Skills
used:} Hand --eye coordination, Fine Motor Skills, Strategy and Problem
Solving. \newline
\textbf{Note:} The task requires the individual to adapt to
the rate at which the circle is moving, thus requiring incorporation of
a calculating strategy to complete successfully.  
\\\hline

\href{http://pebl.sourceforge.net/wiki/index.php?title=Match-to-sample_task}{\textstyleInternetlink{Match to sample task}}  \newline
 Classic non-visual short-memory task  \newline
\includegraphics[width=2in]{images/PEBL20List20of20TestsSTV-img27.png}
 
 &
 Match a matrix pattern to one presented after a
delay. \newline
\textbf{Key Skills used:} Reasoning, Calculating,
Color-processing.\newline
 \textbf{Note:} color-blind individuals will not be
as successful in this task.  
\\\hline

%%%%%%%%%%%%%%%%%%%%%%%%%%%%%%%%%%%%%%%
\href{http://pebl.sourceforge.net/wiki/index.php?title=Corsi_Blocks}{\textstyleInternetlink{Corsi block test}}  \newline
 Version of physical {\textquotedbl}Corsi
block-tapping test{\textquotedbl}  \newline

\includegraphics[width=2in]{images/PEBL20List20of20TestsSTV-img28.png}

&
Measure of visual-spatial working memory.\newline
\textbf{Key Skills used:} Working Memory, Visual Processing.\newline
\textbf{Note:} reaction time can be measured in the trails varying in
length. 
\\\hline

%%%%%%%%%%%%%%%%%%%%%%%%%%%%%%%%%%%
\href{http://pebl.sourceforge.net/wiki/index.php?title=Change_Detection_task}{\textstyleInternetlink{Change Detection test}}\newline  
 Version of numerous change blindness paradigms \newline
\includegraphics[width=2in]{images/PEBL20List20of20TestsSTV-img29.png}
&
 Assess whether participant sees change in a
display of colored circles.\newline
 \textbf{Key Skills used:} Selective
attention, Visual processing, Concentration. \newline
\textbf{Note:} the
changing object may not be so obvious at first, so additional
concentration may be required. 
\\\hline

%%%%%%%%%%%%%%%%%%%%%%%%%%%%%%%%%%%%%%%%%
\href{http://pebl.sourceforge.net/wiki/index.php?title=Clock_Test}{\textstyleInternetlink{Clock Test}} \newline
 Mackworth{\textquotesingle}s Sustained
attention test  \newline

\includegraphics[width=2in]{images/PEBL20List20of20TestsSTV-img30.png}

&
 Watch a clock, and respond whenever it skips a
beat. \newline  
\textbf{Key Skills used:} Selective attention.\newline
\textbf{Note:} Reaction Time is revealed in the data section, indicating the
individual{\textquoteright}s attention levels as the task progresses. 
\\\hline

%%%%%%%%%%%%%%%%%%%%%%%%%%%%%%%%
\href{http://pebl.sourceforge.net/wiki/index.php?title=Device_Mimicry_Task}{\textstyleInternetlink{Device Mimicry Test}}\newline

  
\includegraphics[width=2in]{images/PEBL20List20of20TestsSTV-img31.png}
 
&
 Operate a 4-df etch-a-sketch to recreate paths
produced by the computer.\newline
\textbf{Key Skills used:} Calculating,
Hand-eye coordination, concentration, Fine Motor Skills, Strategy and
Problem Solving.\newline
\textbf{Note:} This task requires precision to
complete successfully. Test can be very helpful for
individual{\textquoteright}s trying to improve their computer skills,
or in cognitive rehabilitation sessions.  

\\\hline


%%%%%%%%%%%%%%%%%%%%%%%%%%%%%%%%
\href{http://pebl.sourceforge.net/wiki/index.php?title=Item-Order_Test}{\textstyleInternetlink{Item-Order Test}} \newline

\includegraphics[width=2in]{images/PEBL20List20of20TestsSTV-img32.png}

&
 Assess two consecutive letter strings, and
determine whether they are the same or different. Different trials are
creating either by changing identity of a letter or the order of two
adjacent letters.\newline
 \textbf{Key Skills used:} Language Processing,
Working Memory. \newline
\textbf{Note:} Does the duration of the test result in
better or poorer performance? This can be measured in the data section.
\\\hline

\href{http://pebl.sourceforge.net/wiki/index.php?title=Letter-Digit_Task}{\textstyleInternetlink{Letter-Digit substitution}}  \newline
 Version of UTCPAB and Wechsler tests  \newline

\includegraphics[width=2in]{images/PEBL20List20of20TestsSTV-img33.png}
  
 &
 Recode stimuli according to a letter-digit code
chart. \newline
\textbf{Key Skills used:} Language Processing, Numerical
Processing.\newline
 \textbf{Note:} great test to use with Aphasiac patients to
see how they map language information with mathematical information.
Reaction time revealed in data section. 
\\\hline

%%%%%%%%%%%%%%%%%%%%%%%%%%
\href{http://pebl.sourceforge.net/wiki/index.php?title=Math_Processing_task}{\textstyleInternetlink{Math Processing}}  \newline

\includegraphics[width=2in]{images/PEBL20List20of20TestsSTV-img34.png}

&
 Do simple arithmetic problems. \newline
 \textbf{Key
Skills used:} Mathematical processing, Numerical processing, Reaction
Time. \newline
\textbf{Note:} Important to distinguish between mathematical and
numerical processes, as the former refers to the manipulation of
numerical information, and the latter refers to basic processing of
numerical information (i.e. that {\textquoteleft}1{\textquoteright}
means {\textquoteleft}one{\textquoteright}).

\\\hline

%%%%%%%%%%%%%%%%%%%%%%%%%%%%%%%%%%%%
\href{http://pebl.sourceforge.net/wiki/index.php?title=Memory_Span}{\textstyleInternetlink{Memory Span (Visual)}}  \newline

 Classic experimental paradigm  \newline
 
\includegraphics[width=2in]{images/PEBL20List20of20TestsSTV-img35.png}
 
 &
 See a sequence of items, then respond using
mouse or touchscreen. \newline
\textbf{Key Skills used:} Working Memory, Short
Term Memory, Visual Processing. \newline
\textbf{Note:} Individuals familiarity
with certain objects may result in better recall for those objects
(i.e. animal lovers). 
\\\hline

%%%%%%%%%%%%%%%%%%%%%%%%%%%%%%%%%%%%%%%%%
\href{http://pebl.sourceforge.net/wiki/index.php?title=Object_Judgment}{\textstyleInternetlink{Object Judgment}}  \newline
 
\includegraphics[width=2in]{images/PEBL20List20of20TestsSTV-img36.png}
 
& 
 Determine whether two polygons are identical,
while manipulating shape, orientation, size. \newline
\textbf{Key Skills used:} Calculating, Reasoning, Visual Processing. \newline
\textbf{Note:} may require concentration due to the duration of task. Task requires visual
manipulation of the stimuli presented. 
\\\hline


%%%%%%%%%%%%%%%%%%%%%%%%%%%%%%%%%%%%%%%%%%%%%
\href{http://pebl.sourceforge.net/wiki/index.php?title=Pattern_Comparison_Task}{\textstyleInternetlink{Pattern Comparison Test}}  \newline

\includegraphics[width=2in]{images/PEBL20List20of20TestsSTV-img37.png}

&
 Examine two grid patterns and determine whether
they are the same. \newline
\textbf{Key Skills used:} Calculating, Visual
Processing. \newline
\textbf{Note:} pattern-samediff.pbl requires reaction time
(found in data section), while pattern-sequential.pbl requires working
memory to function (via correct responses).
\\\hline

%%%%%%%%%%%%%%%%%%%%%%%%%%%%%%%%%%%%%%%%%%%%%%%%%
\href{http://pebl.sourceforge.net/wiki/index.php?title=Probability_Monitor}{\textstyleInternetlink{Probability Monitor}}  \newline

\includegraphics[width=2in]{images/PEBL20List20of20TestsSTV-img38.png}\newline


& Watch a set of gauges to determine when one
gets a hit.\newline
 \textbf{Key Skills used:} Calculating, Inhibition, Visual
Processing, Reasoning.\newline
 \textbf{Note:} while trying to detect a pattern
(calculating and reasoning), the individual is required to inhibit
other random dials on later trails (trails 2 and 3). Reaction time is
measured in data section. 
\\\hline


%%%%%%%%%%%%%%%%%%%%%%%%%%%%%%%%%%%%%%%%%
\href{http://pebl.sourceforge.net/wiki/index.php?title=Situation_Awareness_Test}{\textstyleInternetlink{Situation Awareness Test}}  \newline

\includegraphics[width=2in]{images/PEBL20List20of20TestsSTV-img39.png}

 &
 Watch a set of moving targets and respond to
probes about their locations and identities.\newline
\textbf{ Key Skills used:} Selective Attention, Working Memory, Visual Processing.\newline
 \textbf{Notes:} Test great for combining visual awareness with working memory.  
\\\hline

%%%%%%%%%%%%%%%%%%%%%%%%%%%%%%%%%%%%%%%%%%%%
\href{http://pebl.sourceforge.net/wiki/index.php?title=Comfort_Scales}{\textstyleInternetlink{Comfort scales}}  \newline

\includegraphics[width=2in]{images/PEBL20List20of20TestsSTV-img40.png}

&
 Respond to four visual-analytic scales about
different dimensions of comfort. \newline
\textbf{Key Skills used:} Linguistic
Processing, Calculating. \newline
\textbf{Note:} Allows for extensive self
reflection, and requires linguistic ability for responses (to indicate
how they feel). 
\\\hline

%%%%%%%%%%%%%%%%%%%%%%%%%%%%%%%%%%%%%%%%%%%%%
\href{http://pebl.sourceforge.net/wiki/index.php?title=Tapping}{\textstyleInternetlink{Speed tapping test}}  \newline
 Version of Reitan test battery  \newline
  \includegraphics[width=2in]{images/PEBL20List20of20TestsSTV-img41.png}
&
 
  Tap a key as quickly as possible. \newline
  \textbf{Key Skills used:} Fine Motor Skills. \textbf{Note:} can be used for
individuals in rehabilitation sessions. \newline

\\\hline

%%%%%%%%%%%%%%%%%%%%%%%%%%%%%%%%%%%%%%%%%%%%%%%%
\href{http://pebl.sourceforge.net/wiki/index.php?title=Timetap}{\textstyleInternetlink{Time tapping test}}  \newline

\includegraphics[width=2in]{images/PEBL20List20of20TestsSTV-img42.png}

&
 Tap for a production period at a prespecified
pace. \newline
\textbf{Key Skills used:} Calculating, Working Memory.\newline
\textbf{Note:} requires individual to recall and implement the pace at
which they are required to tap.  

\\\hline


%%%%%%%%%%%%%%%%%%%%%%%%%%%%%%%%%%%%%%%%%%%%%%%%%
\href{http://pebl.sourceforge.net/wiki/index.php?title=Tower_of_Hanoi}{\textstyleInternetlink{Tower of Hanoi test}}  \newline
Classic puzzle and cognitive test of planning \newline
\includegraphics[width=2in]{images/PEBL20List20of20TestsSTV-img43.png}

&
 Solve game with disks.\newline
  \textbf{Key Skills used:} Calculating, Reasoning, Hand-eye coordination, Fine Motor Skills, Working Memory, Visual Processing, Strategy and Problem Solving.\newline
  
\textbf{Note:} Able to track the individual{\textquoteright}s number of
moves. Task is very great for a multitude of cognitive abilities, and
is helpful for patients with cognitive disorders. 
\\\hline


%%%%%%%%%%%%%%%%%%%%%%%%%%%%%%%%%%%%%%%%%%%%
\href{http://pebl.sourceforge.net/wiki/index.php?title=Two_column_addition}{\textstyleInternetlink{Two-column addition}}  \newline

\includegraphics[width=2in]{images/PEBL20List20of20TestsSTV-img44.png}\newline

&
 Do mental arithmetic of at least three
two-digit summands. \newline
\textbf{Key Skills used:} Mathematical Processing,
Working Memory, Calculating.\newline
 \textbf{Note:} Individual can be asked how
they decided to solve the problems (i.e. with what strategy: first
column then the next two columns, or adding all the numbers at once
etc.) 

\\\hline

\href{http://pebl.sourceforge.net/wiki/index.php?title=Visual_Search}{\textstyleInternetlink{Visual Search}}  \newline
\includegraphics[width=2in]{images/PEBL20List20of20TestsSTV-img45.png}


&
 Find the target amidst clutter. \newline
 \textbf{Key
Skills used:} Language Processing, Visual Processing, Selective
Attention, Colour Processing, Inhibition, Concentration. \newline
\textbf{Note:
}X{\textquoteright}s and O{\textquoteright}s are quite distinguishable
letters.\textbf{ O{\textquoteright}s} look more similar to the other
letters than X does, and therefore the trials with X{\textquoteright}s
and O{\textquoteright}s can be compared to see which ones are easier
(via correct response or not) and found quicker (via reaction time). 

\\\hline

\href{http://pebl.sourceforge.net/wiki/index.php?title=PANT}{\textstyleInternetlink{Attentional Network Task}}  \newline
Version of Fan et al.{\textquotesingle}s ANT  \newline
  
\includegraphics[width=2in]{images/PEBL20List20of20TestsSTV-img46.png}

&
 Assess three types of attention.\newline
  \textbf{Key Skills used:} Selective Attention, Reaction Time, Inhibition.\newline
\textbf{Note:} The data section reveals trial and the corresponding
reaction times. Can be viewed is how their attention processes alter
through the test{\textquoteright}s duration.
\\\hline

%%%%%%%%%%%%%%%%%%%%%%%%%%%%%%%%%%%%%%%%%
\href{http://pebl.sourceforge.net/wiki/index.php?title=Balloon_Analog_Risk_Task}{\textstyleInternetlink{PEBL Balloon Analog Risk Task}}  \newline
Version of LeJuez et al{\textquotesingle}s BART \newline

\includegraphics[width=2in]{images/PEBL20List20of20TestsSTV-img47.png}
  &
Assess three types of attention. \newline
\textbf{Key Skills used:} Risk Assessment and risk aversion. \newline
\textbf{Note:} Test can be modified to
ask the participant to reach a certain money value in a set amount of
time.  
\\\hline

%%%%%%%%%%%%%%%%%%%%%%%%%%%%%%%%%%%%%%%%%%%%
\href{http://pebl.sourceforge.net/wiki/index.php?title=Dot_Judgment}{\textstyleInternetlink{Dot Judgment Task}}  \newline
 Determine which field has more dots.\newline

\includegraphics[width=2in]{images/PEBL20List20of20TestsSTV-img48.png}

&
\textbf{Key Skills used:} Calculating, Decision Making\newline
\textbf{ Note:} Threshold provides an opportunity to observe how the individual
performs (with correct judgment) when dot amounts are similar. 
\\\hline


%%%%%%%%%%%%%%%%%%%%%%%%%%%%%%%%%%%
\href{http://pebl.sourceforge.net/wiki/index.php?title=Flanker_Task}{\textstyleInternetlink{Flanker Task}}\newline 
 Eriksen{\textquotesingle}s Flanker Task  \newline
\includegraphics[width=2in]{images/PEBL20List20of20TestsSTV-img49.png}
  
 &
 Make direction response with distraction.\newline
\textbf{Key Skills used:} Selective Attention, Reaction Time,
Inhibition. \newline
\textbf{Note:} The data section reveals trial and the
corresponding reaction times. Can be viewed is how their attention
processes progress through the test{\textquoteright}s duration.
\\\hline

\href{http://pebl.sourceforge.net/wiki/index.php?title=Go/No-go_Task}{\textstyleInternetlink{Go/No-go Task}}  \newline
Version of Bezdjian{\textquotesingle}s 2009 Implementation\newline
\includegraphics[width=2in]{images/PEBL20List20of20TestsSTV-img50.png}

 &
  Classic continuous performance task.\newline
\textbf{Key Skills used:} Inhibition, Reaction Time, Language
Processing, Selective attention.\newline
 \textbf{Note:} The data section allows
for observation of their scores, and to view if their inhibition skills
are increasing or decreasing with time. 
\\\hline

%%%%%%%%%%%%%%%%%%%%%%%%%%%%%%%%%%%%%%%%%%%%%%%%%%
\href{http://pebl.sourceforge.net/wiki/index.php?title=Manikin}{\textstyleInternetlink{Manikin Task}}  \newline

\includegraphics[width=2in]{images/PEBL20List20of20TestsSTV-img51.png}

&Assess mental rotation. \newline

\\\hline

\href{http://pebl.sourceforge.net/wiki/index.php?title=TLX}{\textstyleInternetlink{TLX Workload Assessment}}  \newline

An implementation of NASA{\textquotesingle}s TLX workload assessment
\newline
\href{http://en.wikipedia.org/wiki/NASA-TLX}{\textstyleInternetlink{wikipedia}}\newline

\includegraphics[width=2in]{images/PEBL20List20of20TestsSTV-img52.png}

 &
 Assess workload of task on multiple dimensions.\newline
\textbf{Key Skills used:} Concentration, Linguistic Processing,
Calculating.\newline
 \textbf{Note:} requires the individual to self reflect,
read the information, and calculate their levels according to the scale
provided. 

\\\hline

%%%%%%%%%%%%%%%%%%%%%%%%%%%%%%%%%%%%%%%%%%%%%%
\href{http://pebl.sourceforge.net/wiki/index.php?title=Muller-Lyer_Illusion}{\textstyleInternetlink{Muller-Lyer Illusion}}  \newline
 Classic perceptual illusion\newline
\href{http://en.wikipedia.org/wiki/Muller-Lyer_Illusion}{\textstyleInternetlink{wikipedia}}\newline
 
\includegraphics[width=2in]{images/PEBL20List20of20TestsSTV-img53.png}
 
 &
 Psychometric study of Illusion.\newline
 \textbf{Key Skills used:} Calculating, Reaction Time.\newline
  \textbf{Note:} the task
requires a quick response, thus attention abilities can be hard to
measure in this task. 
\\\hline

\href{http://pebl.sourceforge.net/wiki/index.php?title=Oddball_Task}{\textstyleInternetlink{Oddball Task}}  \newline
 Version of Huettel{\textquotesingle}s implementation\newline  
\includegraphics[width=2in]{images/PEBL20List20of20TestsSTV-img54.png}

&

 Respond to a stimulus dimension overshadowed by
irrelevant dimension. \newline
\textbf{Key Skills used:} Inhibition, Selective
Attention, Visual Processing, Reaction Time. \newline
\textbf{Note:} Inhibition
skills require the individual to ignore the location and instead focus
on the shape differences. 

\\\hline

%%%%%%%%%%%%%%%%%%%%%%%%%%%%%%%%
\href{http://pebl.sourceforge.net/wiki/index.php?title=Simon_Task}{\textstyleInternetlink{Simon Task}}\newline
 Simon{\textquotesingle}s S-R compatibility test\newline

\includegraphics[width=2in]{images/PEBL20List20of20TestsSTV-img55.png}

 &
 Respond to a stimulus dimension, overshadowed
by spatial location.\newline
 \textbf{Key Skills used:}Color Processing,
Inhibition, Visual Processing, Selective Attention, Reaction Time.\newline

\textbf{Note:} those who are color blind will have difficulty in
completing this task. Individual{\textquoteright}s inhibition abilities
can be measured (via correct responses) to see how well they can focus
on the point of the task (color) and not be distracted by its
location. 
\\\hline

\href{http://pebl.sourceforge.net/wiki/index.php?title=Switcher_Task}{\textstyleInternetlink{Switcher Task}}  \newline
\includegraphics[width=2in]{images/PEBL20List20of20TestsSTV-img56.png}

& Respond to a matched and changing stimulus
dimension. \newline
\textbf{Key Skills used:} Visual Processing, Selective
Attention. \newline
\textbf{Note:} reaction time is measured in the data
section, along with trail type. Thus, times associated with color,
shape and letter can be measured to see which is easiest and most
challenging for the individual. 

\\\hline


\end{longtable}

\bigskip

{
\textbf{Norms and Other Uses}\newline
Many of the original versions of the tasks we implement here have been
normed on a large population. Such norms are available in published
articles. Because these implementations are not identical (many of them
use slightly different stimuli, response methods, timing, etc.) one
must be careful when applying the results to the normed data. If you
use PEBL or the PEBL Psychological Test Battery, please reference us!
If you are interested in helping develop norms for PEBL tests, have
access to subject populations and testing facilities, join the
\href{http://lists.sourceforge.net/lists/listinfo/pebl-norms}{\textstyleInternetlink{pebl-norms@lists.sourceforge.net
mailing list}} and tell us what norms you are most interested in.} 


\bigskip

{
\textbf{Support and Contact info}} 

{
If you have any general questions about PEBL or the PEBL Psychological
Test Battery, you can contact us at:
\href{mailto:pebl-list@lists.sourceforge.net}{\textstyleInternetlink{pebl-list@lists.sourceforge.net}}.
Email support is available free-of-charge. You can sign up for this
\href{http://lists.sourceforge.net/lists/listinfo/pebl-list}{\textstyleInternetlink{email
list or browse the archives here}}. More information about the
\href{http://obereed.net/}{\textstyleInternetlink{main author is
available here}}. Enquire on the list if you are interested in paying
someone to write new experiments or modify existing ones for your
needs.} 


\bigskip

\textbf{Obtaining the Battery} The PEBL Test Battery is installed with the main PEBL installation.  The first time you run PEBL, it will be copied into a folder in your Documents\ directory called pebl-exp.2.0 (or similar depending on the version of PEBL you are running).  On Linux, running \texttt{> pebl --install} will copy the battery directory there.  The PEBL launcher will start in that directory, and let you explore and navigate the different tests in the battery. 


\bigskip

\subparagraph[http://pebl.sourceforge.net/battery.html
]{http://pebl.sourceforge.net/battery.html} 

