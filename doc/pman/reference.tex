% This file is auto-generated from Sphinx documentation
% To regenerate: cd sphinx-prototype && make latex-chapter-with-styles
% Source files: source/reference/

% IMPORTANT: This file requires the Sphinx package to be loaded in the preamble.
% Add this line to main.tex BEFORE \begin{document}:
%   \usepackage{sphinx}
%
% The sphinx.sty file and its dependencies must be in the same directory as main.tex
% (21 .sty files are automatically copied by the build process)

\chapter{Detailed Function and Keyword Reference}
\label{sec:reference}

% Suppress section marks in headers for this chapter
% This prevents section titles from appearing in running headers
\renewcommand{\sectionmark}[1]{}

\setlength{\parindent}{0pt}

\phantomsection\label{\detokenize{index::doc}}


\sphinxAtStartPar
Welcome to the PEBL Function Reference. This documentation covers all functions available in the Psychology Experiment Building Language.


\sphinxAtStartPar
PEBL functions are organized by namespace. Click on any namespace below to see its functions:

\sphinxAtStartPar
\sphinxstylestrong{Built\sphinxhyphen{}in Function Namespaces:}
\begin{itemize}
\item {} 
\sphinxAtStartPar
{\hyperref[\detokenize{reference/peblenvironment::doc}]{\sphinxcrossref{\DUrole{doc}{PEBLEnvironment \sphinxhyphen{} System and Environment}}}}

\item {} 
\sphinxAtStartPar
{\hyperref[\detokenize{reference/pebllist::doc}]{\sphinxcrossref{\DUrole{doc}{PEBLList \sphinxhyphen{} List Manipulation}}}}

\item {} 
\sphinxAtStartPar
{\hyperref[\detokenize{reference/peblobjects::doc}]{\sphinxcrossref{\DUrole{doc}{PEBLObjects \sphinxhyphen{} Graphics and Objects}}}}

\item {} 
\sphinxAtStartPar
{\hyperref[\detokenize{reference/peblstream::doc}]{\sphinxcrossref{\DUrole{doc}{PEBLStream \sphinxhyphen{} File and Network I/O}}}}

\item {} 
\sphinxAtStartPar
{\hyperref[\detokenize{reference/peblstring::doc}]{\sphinxcrossref{\DUrole{doc}{PEBLString \sphinxhyphen{} String Manipulation}}}}

\end{itemize}

\sphinxAtStartPar
\sphinxstylestrong{PEBL Library Functions:}
\begin{itemize}
\item {} 
\sphinxAtStartPar
{\hyperref[\detokenize{reference/design::doc}]{\sphinxcrossref{\DUrole{doc}{Design Library \sphinxhyphen{} Experimental Design}}}}

\item {} 
\sphinxAtStartPar
{\hyperref[\detokenize{reference/graphics::doc}]{\sphinxcrossref{\DUrole{doc}{Graphics Library \sphinxhyphen{} Advanced Graphics}}}}

\item {} 
\sphinxAtStartPar
{\hyperref[\detokenize{reference/html::doc}]{\sphinxcrossref{\DUrole{doc}{HTML Library \sphinxhyphen{} HTML Generation}}}}

\item {} 
\sphinxAtStartPar
{\hyperref[\detokenize{reference/math::doc}]{\sphinxcrossref{\DUrole{doc}{Math Library \sphinxhyphen{} Extended Mathematical}}}}

\item {} 
\sphinxAtStartPar
{\hyperref[\detokenize{reference/ui::doc}]{\sphinxcrossref{\DUrole{doc}{UI Library \sphinxhyphen{} User Interface}}}}

\item {} 
\sphinxAtStartPar
{\hyperref[\detokenize{reference/utility::doc}]{\sphinxcrossref{\DUrole{doc}{Utility Library \sphinxhyphen{} Helpers and Utilities}}}}

\end{itemize}

\sphinxstepscope


\section{PEBLEnvironment \sphinxhyphen{} System and Environment}
\label{\detokenize{reference/peblenvironment:peblenvironment-system-and-environment}}\label{\detokenize{reference/peblenvironment::doc}}
\sphinxAtStartPar
This module contains functions for system interaction, timing, input/output, and environment management.

\begin{sphinxShadowBox}
\sphinxstyletopictitle{Function Index}
\begin{itemize}
\item {} 
\sphinxAtStartPar
\phantomsection\label{\detokenize{reference/peblenvironment:id1}}{\hyperref[\detokenize{reference/peblenvironment:cleareventloop}]{\sphinxcrossref{ClearEventLoop()}}}

\item {} 
\sphinxAtStartPar
\phantomsection\label{\detokenize{reference/peblenvironment:id2}}{\hyperref[\detokenize{reference/peblenvironment:copyfromclipboard}]{\sphinxcrossref{CopyFromClipboard()}}}

\item {} 
\sphinxAtStartPar
\phantomsection\label{\detokenize{reference/peblenvironment:id3}}{\hyperref[\detokenize{reference/peblenvironment:callfunction}]{\sphinxcrossref{CallFunction()}}}

\item {} 
\sphinxAtStartPar
\phantomsection\label{\detokenize{reference/peblenvironment:id4}}{\hyperref[\detokenize{reference/peblenvironment:deletefile}]{\sphinxcrossref{DeleteFile()}}}

\item {} 
\sphinxAtStartPar
\phantomsection\label{\detokenize{reference/peblenvironment:id5}}{\hyperref[\detokenize{reference/peblenvironment:exitquietly}]{\sphinxcrossref{ExitQuietly()}}}

\item {} 
\sphinxAtStartPar
\phantomsection\label{\detokenize{reference/peblenvironment:id6}}{\hyperref[\detokenize{reference/peblenvironment:fileexists}]{\sphinxcrossref{FileExists()}}}

\item {} 
\sphinxAtStartPar
\phantomsection\label{\detokenize{reference/peblenvironment:id7}}{\hyperref[\detokenize{reference/peblenvironment:getcurrentscreenresolution}]{\sphinxcrossref{GetCurrentScreenResolution()}}}

\item {} 
\sphinxAtStartPar
\phantomsection\label{\detokenize{reference/peblenvironment:id8}}{\hyperref[\detokenize{reference/peblenvironment:getdirectorylisting}]{\sphinxcrossref{GetDirectoryListing()}}}

\item {} 
\sphinxAtStartPar
\phantomsection\label{\detokenize{reference/peblenvironment:id9}}{\hyperref[\detokenize{reference/peblenvironment:getdrivers}]{\sphinxcrossref{GetDrivers()}}}

\item {} 
\sphinxAtStartPar
\phantomsection\label{\detokenize{reference/peblenvironment:id10}}{\hyperref[\detokenize{reference/peblenvironment:getexecutablename}]{\sphinxcrossref{GetExecutableName()}}}

\item {} 
\sphinxAtStartPar
\phantomsection\label{\detokenize{reference/peblenvironment:id11}}{\hyperref[\detokenize{reference/peblenvironment:gethomedirectory}]{\sphinxcrossref{GetHomeDirectory()}}}

\item {} 
\sphinxAtStartPar
\phantomsection\label{\detokenize{reference/peblenvironment:id12}}{\hyperref[\detokenize{reference/peblenvironment:getjoystickaxisstate}]{\sphinxcrossref{GetJoystickAxisState()}}}

\item {} 
\sphinxAtStartPar
\phantomsection\label{\detokenize{reference/peblenvironment:id13}}{\hyperref[\detokenize{reference/peblenvironment:getjoystickballstate}]{\sphinxcrossref{GetJoystickBallState()}}}

\item {} 
\sphinxAtStartPar
\phantomsection\label{\detokenize{reference/peblenvironment:id14}}{\hyperref[\detokenize{reference/peblenvironment:getjoystickbuttonstate}]{\sphinxcrossref{GetJoystickButtonState()}}}

\item {} 
\sphinxAtStartPar
\phantomsection\label{\detokenize{reference/peblenvironment:id15}}{\hyperref[\detokenize{reference/peblenvironment:getjoystickhatstate}]{\sphinxcrossref{GetJoystickHatState()}}}

\item {} 
\sphinxAtStartPar
\phantomsection\label{\detokenize{reference/peblenvironment:id16}}{\hyperref[\detokenize{reference/peblenvironment:getmousecursorposition}]{\sphinxcrossref{GetMouseCursorPosition()}}}

\item {} 
\sphinxAtStartPar
\phantomsection\label{\detokenize{reference/peblenvironment:id17}}{\hyperref[\detokenize{reference/peblenvironment:getmousestate}]{\sphinxcrossref{GetMouseState()}}}

\item {} 
\sphinxAtStartPar
\phantomsection\label{\detokenize{reference/peblenvironment:id18}}{\hyperref[\detokenize{reference/peblenvironment:getnumjoystickaxes}]{\sphinxcrossref{GetNumJoystickAxes()}}}

\item {} 
\sphinxAtStartPar
\phantomsection\label{\detokenize{reference/peblenvironment:id19}}{\hyperref[\detokenize{reference/peblenvironment:getnumjoystickballs}]{\sphinxcrossref{GetNumJoystickBalls()}}}

\item {} 
\sphinxAtStartPar
\phantomsection\label{\detokenize{reference/peblenvironment:id20}}{\hyperref[\detokenize{reference/peblenvironment:getnumjoystickbuttons}]{\sphinxcrossref{GetNumJoystickButtons()}}}

\item {} 
\sphinxAtStartPar
\phantomsection\label{\detokenize{reference/peblenvironment:id21}}{\hyperref[\detokenize{reference/peblenvironment:getnumjoystickhats}]{\sphinxcrossref{GetNumJoystickHats()}}}

\item {} 
\sphinxAtStartPar
\phantomsection\label{\detokenize{reference/peblenvironment:id22}}{\hyperref[\detokenize{reference/peblenvironment:getnumjoysticks}]{\sphinxcrossref{GetNumJoysticks()}}}

\item {} 
\sphinxAtStartPar
\phantomsection\label{\detokenize{reference/peblenvironment:id23}}{\hyperref[\detokenize{reference/peblenvironment:getpeblversion}]{\sphinxcrossref{GetPEBLVersion()}}}

\item {} 
\sphinxAtStartPar
\phantomsection\label{\detokenize{reference/peblenvironment:id24}}{\hyperref[\detokenize{reference/peblenvironment:getsystemtype}]{\sphinxcrossref{GetSystemType()}}}

\item {} 
\sphinxAtStartPar
\phantomsection\label{\detokenize{reference/peblenvironment:id25}}{\hyperref[\detokenize{reference/peblenvironment:gettextboxcursorfromclick}]{\sphinxcrossref{GetTextBoxCursorFromClick()}}}

\item {} 
\sphinxAtStartPar
\phantomsection\label{\detokenize{reference/peblenvironment:id26}}{\hyperref[\detokenize{reference/peblenvironment:gettime}]{\sphinxcrossref{GetTime()}}}

\item {} 
\sphinxAtStartPar
\phantomsection\label{\detokenize{reference/peblenvironment:id27}}{\hyperref[\detokenize{reference/peblenvironment:gettimeofday}]{\sphinxcrossref{GetTimeOfDay()}}}

\item {} 
\sphinxAtStartPar
\phantomsection\label{\detokenize{reference/peblenvironment:id28}}{\hyperref[\detokenize{reference/peblenvironment:getvideomodes}]{\sphinxcrossref{GetVideoModes()}}}

\item {} 
\sphinxAtStartPar
\phantomsection\label{\detokenize{reference/peblenvironment:id29}}{\hyperref[\detokenize{reference/peblenvironment:getworkingdirectory}]{\sphinxcrossref{GetWorkingDirectory()}}}

\item {} 
\sphinxAtStartPar
\phantomsection\label{\detokenize{reference/peblenvironment:id30}}{\hyperref[\detokenize{reference/peblenvironment:isanykeydown}]{\sphinxcrossref{IsAnyKeyDown()}}}

\item {} 
\sphinxAtStartPar
\phantomsection\label{\detokenize{reference/peblenvironment:id31}}{\hyperref[\detokenize{reference/peblenvironment:isaudioout}]{\sphinxcrossref{IsAudioOut()}}}

\item {} 
\sphinxAtStartPar
\phantomsection\label{\detokenize{reference/peblenvironment:id32}}{\hyperref[\detokenize{reference/peblenvironment:iscanvas}]{\sphinxcrossref{IsCanvas()}}}

\item {} 
\sphinxAtStartPar
\phantomsection\label{\detokenize{reference/peblenvironment:id33}}{\hyperref[\detokenize{reference/peblenvironment:iscolor}]{\sphinxcrossref{IsColor()}}}

\item {} 
\sphinxAtStartPar
\phantomsection\label{\detokenize{reference/peblenvironment:id34}}{\hyperref[\detokenize{reference/peblenvironment:iscustomobject}]{\sphinxcrossref{IsCustomObject()}}}

\item {} 
\sphinxAtStartPar
\phantomsection\label{\detokenize{reference/peblenvironment:id35}}{\hyperref[\detokenize{reference/peblenvironment:isdirectory}]{\sphinxcrossref{IsDirectory()}}}

\item {} 
\sphinxAtStartPar
\phantomsection\label{\detokenize{reference/peblenvironment:id36}}{\hyperref[\detokenize{reference/peblenvironment:isfilestream}]{\sphinxcrossref{IsFileStream()}}}

\item {} 
\sphinxAtStartPar
\phantomsection\label{\detokenize{reference/peblenvironment:id37}}{\hyperref[\detokenize{reference/peblenvironment:isfloat}]{\sphinxcrossref{IsFloat()}}}

\item {} 
\sphinxAtStartPar
\phantomsection\label{\detokenize{reference/peblenvironment:id38}}{\hyperref[\detokenize{reference/peblenvironment:isfont}]{\sphinxcrossref{IsFont()}}}

\item {} 
\sphinxAtStartPar
\phantomsection\label{\detokenize{reference/peblenvironment:id39}}{\hyperref[\detokenize{reference/peblenvironment:isimage}]{\sphinxcrossref{IsImage()}}}

\item {} 
\sphinxAtStartPar
\phantomsection\label{\detokenize{reference/peblenvironment:id40}}{\hyperref[\detokenize{reference/peblenvironment:isinteger}]{\sphinxcrossref{IsInteger()}}}

\item {} 
\sphinxAtStartPar
\phantomsection\label{\detokenize{reference/peblenvironment:id41}}{\hyperref[\detokenize{reference/peblenvironment:iskeydown}]{\sphinxcrossref{IsKeyDown()}}}

\item {} 
\sphinxAtStartPar
\phantomsection\label{\detokenize{reference/peblenvironment:id42}}{\hyperref[\detokenize{reference/peblenvironment:iskeyup}]{\sphinxcrossref{IsKeyUp()}}}

\item {} 
\sphinxAtStartPar
\phantomsection\label{\detokenize{reference/peblenvironment:id43}}{\hyperref[\detokenize{reference/peblenvironment:islabel}]{\sphinxcrossref{IsLabel()}}}

\item {} 
\sphinxAtStartPar
\phantomsection\label{\detokenize{reference/peblenvironment:id44}}{\hyperref[\detokenize{reference/peblenvironment:islist}]{\sphinxcrossref{IsList()}}}

\item {} 
\sphinxAtStartPar
\phantomsection\label{\detokenize{reference/peblenvironment:id45}}{\hyperref[\detokenize{reference/peblenvironment:isnumber}]{\sphinxcrossref{IsNumber()}}}

\item {} 
\sphinxAtStartPar
\phantomsection\label{\detokenize{reference/peblenvironment:id46}}{\hyperref[\detokenize{reference/peblenvironment:isshape}]{\sphinxcrossref{IsShape()}}}

\item {} 
\sphinxAtStartPar
\phantomsection\label{\detokenize{reference/peblenvironment:id47}}{\hyperref[\detokenize{reference/peblenvironment:isstring}]{\sphinxcrossref{IsString()}}}

\item {} 
\sphinxAtStartPar
\phantomsection\label{\detokenize{reference/peblenvironment:id48}}{\hyperref[\detokenize{reference/peblenvironment:istext}]{\sphinxcrossref{IsText()}}}

\item {} 
\sphinxAtStartPar
\phantomsection\label{\detokenize{reference/peblenvironment:id49}}{\hyperref[\detokenize{reference/peblenvironment:istextbox}]{\sphinxcrossref{IsTextBox()}}}

\item {} 
\sphinxAtStartPar
\phantomsection\label{\detokenize{reference/peblenvironment:id50}}{\hyperref[\detokenize{reference/peblenvironment:iswidget}]{\sphinxcrossref{IsWidget()}}}

\item {} 
\sphinxAtStartPar
\phantomsection\label{\detokenize{reference/peblenvironment:id51}}{\hyperref[\detokenize{reference/peblenvironment:iswindow}]{\sphinxcrossref{IsWindow()}}}

\item {} 
\sphinxAtStartPar
\phantomsection\label{\detokenize{reference/peblenvironment:id52}}{\hyperref[\detokenize{reference/peblenvironment:launchfile}]{\sphinxcrossref{LaunchFile()}}}

\item {} 
\sphinxAtStartPar
\phantomsection\label{\detokenize{reference/peblenvironment:id53}}{\hyperref[\detokenize{reference/peblenvironment:makedirectory}]{\sphinxcrossref{MakeDirectory()}}}

\item {} 
\sphinxAtStartPar
\phantomsection\label{\detokenize{reference/peblenvironment:id54}}{\hyperref[\detokenize{reference/peblenvironment:openjoystick}]{\sphinxcrossref{OpenJoystick()}}}

\item {} 
\sphinxAtStartPar
\phantomsection\label{\detokenize{reference/peblenvironment:id55}}{\hyperref[\detokenize{reference/peblenvironment:playmovie}]{\sphinxcrossref{PlayMovie()}}}

\item {} 
\sphinxAtStartPar
\phantomsection\label{\detokenize{reference/peblenvironment:id56}}{\hyperref[\detokenize{reference/peblenvironment:registerevent}]{\sphinxcrossref{RegisterEvent()}}}

\item {} 
\sphinxAtStartPar
\phantomsection\label{\detokenize{reference/peblenvironment:id57}}{\hyperref[\detokenize{reference/peblenvironment:setmousecursorposition}]{\sphinxcrossref{SetMouseCursorPosition()}}}

\item {} 
\sphinxAtStartPar
\phantomsection\label{\detokenize{reference/peblenvironment:id58}}{\hyperref[\detokenize{reference/peblenvironment:setworkingdirectory}]{\sphinxcrossref{SetWorkingDirectory()}}}

\item {} 
\sphinxAtStartPar
\phantomsection\label{\detokenize{reference/peblenvironment:id59}}{\hyperref[\detokenize{reference/peblenvironment:showcursor}]{\sphinxcrossref{ShowCursor()}}}

\item {} 
\sphinxAtStartPar
\phantomsection\label{\detokenize{reference/peblenvironment:id60}}{\hyperref[\detokenize{reference/peblenvironment:signalfatalerror}]{\sphinxcrossref{SignalFatalError()}}}

\item {} 
\sphinxAtStartPar
\phantomsection\label{\detokenize{reference/peblenvironment:id61}}{\hyperref[\detokenize{reference/peblenvironment:starteventloop}]{\sphinxcrossref{StartEventLoop()}}}

\item {} 
\sphinxAtStartPar
\phantomsection\label{\detokenize{reference/peblenvironment:id62}}{\hyperref[\detokenize{reference/peblenvironment:systemcall}]{\sphinxcrossref{SystemCall()}}}

\item {} 
\sphinxAtStartPar
\phantomsection\label{\detokenize{reference/peblenvironment:id63}}{\hyperref[\detokenize{reference/peblenvironment:systemcallupdate}]{\sphinxcrossref{SystemCallUpdate()}}}

\item {} 
\sphinxAtStartPar
\phantomsection\label{\detokenize{reference/peblenvironment:id64}}{\hyperref[\detokenize{reference/peblenvironment:timestamp}]{\sphinxcrossref{TimeStamp()}}}

\item {} 
\sphinxAtStartPar
\phantomsection\label{\detokenize{reference/peblenvironment:id65}}{\hyperref[\detokenize{reference/peblenvironment:translatekeycode}]{\sphinxcrossref{TranslateKeyCode()}}}

\item {} 
\sphinxAtStartPar
\phantomsection\label{\detokenize{reference/peblenvironment:id66}}{\hyperref[\detokenize{reference/peblenvironment:translatestring}]{\sphinxcrossref{TranslateString()}}}

\item {} 
\sphinxAtStartPar
\phantomsection\label{\detokenize{reference/peblenvironment:id67}}{\hyperref[\detokenize{reference/peblenvironment:variableexists}]{\sphinxcrossref{VariableExists()}}}

\item {} 
\sphinxAtStartPar
\phantomsection\label{\detokenize{reference/peblenvironment:id68}}{\hyperref[\detokenize{reference/peblenvironment:wait}]{\sphinxcrossref{Wait()}}}

\item {} 
\sphinxAtStartPar
\phantomsection\label{\detokenize{reference/peblenvironment:id69}}{\hyperref[\detokenize{reference/peblenvironment:waitforallkeysup}]{\sphinxcrossref{WaitForAllKeysUp()}}}

\item {} 
\sphinxAtStartPar
\phantomsection\label{\detokenize{reference/peblenvironment:id70}}{\hyperref[\detokenize{reference/peblenvironment:waitforkeydown}]{\sphinxcrossref{WaitForKeyDown()}}}

\item {} 
\sphinxAtStartPar
\phantomsection\label{\detokenize{reference/peblenvironment:id71}}{\hyperref[\detokenize{reference/peblenvironment:waitforanykeydown}]{\sphinxcrossref{WaitForAnyKeyDown()}}}

\item {} 
\sphinxAtStartPar
\phantomsection\label{\detokenize{reference/peblenvironment:id72}}{\hyperref[\detokenize{reference/peblenvironment:waitforanykeydownwithtimeout}]{\sphinxcrossref{WaitForAnyKeyDownWithTimeout()}}}

\item {} 
\sphinxAtStartPar
\phantomsection\label{\detokenize{reference/peblenvironment:id73}}{\hyperref[\detokenize{reference/peblenvironment:waitforanykeypress}]{\sphinxcrossref{WaitForAnyKeyPress()}}}

\item {} 
\sphinxAtStartPar
\phantomsection\label{\detokenize{reference/peblenvironment:id74}}{\hyperref[\detokenize{reference/peblenvironment:waitforanykeypresswithtimeout}]{\sphinxcrossref{WaitForAnyKeyPressWithTimeout()}}}

\item {} 
\sphinxAtStartPar
\phantomsection\label{\detokenize{reference/peblenvironment:id75}}{\hyperref[\detokenize{reference/peblenvironment:waitforkeylistdown}]{\sphinxcrossref{WaitForKeyListDown()}}}

\item {} 
\sphinxAtStartPar
\phantomsection\label{\detokenize{reference/peblenvironment:id76}}{\hyperref[\detokenize{reference/peblenvironment:waitforkeypress}]{\sphinxcrossref{WaitForKeyPress()}}}

\item {} 
\sphinxAtStartPar
\phantomsection\label{\detokenize{reference/peblenvironment:id77}}{\hyperref[\detokenize{reference/peblenvironment:waitforkeyup}]{\sphinxcrossref{WaitForKeyUp()}}}

\item {} 
\sphinxAtStartPar
\phantomsection\label{\detokenize{reference/peblenvironment:id78}}{\hyperref[\detokenize{reference/peblenvironment:waitforkeyrelease}]{\sphinxcrossref{WaitForKeyRelease()}}}

\item {} 
\sphinxAtStartPar
\phantomsection\label{\detokenize{reference/peblenvironment:id79}}{\hyperref[\detokenize{reference/peblenvironment:waitforlistkeypress}]{\sphinxcrossref{WaitForListKeyPress()}}}

\item {} 
\sphinxAtStartPar
\phantomsection\label{\detokenize{reference/peblenvironment:id80}}{\hyperref[\detokenize{reference/peblenvironment:waitforlistkeypresswithtimeout}]{\sphinxcrossref{WaitForListKeyPressWithTimeout()}}}

\item {} 
\sphinxAtStartPar
\phantomsection\label{\detokenize{reference/peblenvironment:id81}}{\hyperref[\detokenize{reference/peblenvironment:waitformousebutton}]{\sphinxcrossref{WaitForMouseButton()}}}

\item {} 
\sphinxAtStartPar
\phantomsection\label{\detokenize{reference/peblenvironment:id82}}{\hyperref[\detokenize{reference/peblenvironment:waitformousebuttonwithtimeout}]{\sphinxcrossref{WaitForMouseButtonWithTimeout()}}}

\end{itemize}
\end{sphinxShadowBox}

\index{ClearEventLoop@\spxentry{ClearEventLoop}}\ignorespaces 

\subsection{ClearEventLoop()}
\label{\detokenize{reference/peblenvironment:cleareventloop}}\label{\detokenize{reference/peblenvironment:index-0}}
\sphinxAtStartPar
\sphinxstyleemphasis{Clears all trigger events from event loop}

\sphinxAtStartPar
\sphinxstylestrong{Description:}

\sphinxAtStartPar
Clears the event loop.  This function is currently experimental, and its usage may change in future versions of PEBL.

\sphinxAtStartPar
\sphinxstylestrong{Usage:}

\sphinxAtStartPar
\sphinxstylestrong{See Also:}

\sphinxAtStartPar
\sphinxcode{\sphinxupquote{RegisterEvent()}}, \sphinxcode{\sphinxupquote{StartEventLoop()}}

\index{CopyFromClipboard@\spxentry{CopyFromClipboard}}\ignorespaces 

\subsection{CopyFromClipboard()}
\label{\detokenize{reference/peblenvironment:copyfromclipboard}}\label{\detokenize{reference/peblenvironment:index-1}}
\sphinxAtStartPar
\sphinxstyleemphasis{Copies text from system clipboard.}

\sphinxAtStartPar
\sphinxstylestrong{Description:}

\sphinxAtStartPar
This copies text currently living in the system clipboard. Note that (depending on platform), text copied into the clipboard may not remain there after PEBL exits.

\sphinxAtStartPar
\sphinxstylestrong{Example:}

\begin{sphinxVerbatim}[commandchars=\\\{\}]
\PYG{n+nv}{text}\PYG{+w}{ }\PYG{o}{\PYGZlt{}\PYGZhy{}}\PYG{+w}{ }\PYG{n+nf}{CopyFromClipboard}\PYG{p}{(}\PYG{p}{)}
\PYG{+w}{     }\PYG{n+nv}{textbox.text}\PYG{+w}{ }\PYG{o}{\PYGZlt{}\PYGZhy{}}\PYG{+w}{ }\PYG{n+nv}{text}
\end{sphinxVerbatim}

\sphinxAtStartPar
\sphinxstylestrong{See Also:}

\sphinxAtStartPar
\sphinxcode{\sphinxupquote{CopyToClipboard()}}

\index{CallFunction@\spxentry{CallFunction}}\ignorespaces 

\subsection{CallFunction()}
\label{\detokenize{reference/peblenvironment:callfunction}}\label{\detokenize{reference/peblenvironment:index-2}}
\sphinxAtStartPar
\sphinxstyleemphasis{Calls a PEBL function by name with a list of arguments}

\sphinxAtStartPar
\sphinxstylestrong{Description:}

\sphinxAtStartPar
Calls a PEBL function dynamically using its name as a string and a list of arguments. This is useful for implementing callbacks, event handlers, or calling functions whose names are determined at runtime.

\sphinxAtStartPar
\sphinxstylestrong{Usage:}

\begin{sphinxVerbatim}[commandchars=\\\{\}]
\PYG{n+nf}{CallFunction}\PYG{p}{(}\PYG{o}{\PYGZlt{}}\PYG{n+nv}{function\PYGZus{}name}\PYG{o}{\PYGZgt{}}\PYG{p}{,}\PYG{+w}{ }\PYG{o}{\PYGZlt{}}\PYG{n+nv}{argument\PYGZus{}list}\PYG{o}{\PYGZgt{}}\PYG{p}{)}
\end{sphinxVerbatim}

\sphinxAtStartPar
\sphinxstylestrong{Example:}

\begin{sphinxVerbatim}[commandchars=\\\{\}]
\PYG{c+c1}{\PYGZsh{}\PYGZsh{} Call a function by name}
\PYG{n+nv}{result}\PYG{+w}{ }\PYG{o}{\PYGZlt{}\PYGZhy{}}\PYG{+w}{ }\PYG{n+nf}{CallFunction}\PYG{p}{(}\PYG{l+s+s2}{\PYGZdq{}Max\PYGZdq{}}\PYG{p}{,}\PYG{+w}{ }\PYG{p}{[}\PYG{l+m+mi}{1}\PYG{p}{,}\PYG{+w}{ }\PYG{l+m+mi}{5}\PYG{p}{,}\PYG{+w}{ }\PYG{l+m+mi}{3}\PYG{p}{,}\PYG{+w}{ }\PYG{l+m+mi}{2}\PYG{p}{]}\PYG{p}{)}
\PYG{n+nf}{Print}\PYG{p}{(}\PYG{n+nv}{result}\PYG{p}{)}\PYG{+w}{  }\PYG{c+c1}{\PYGZsh{} == 5}

\PYG{c+c1}{\PYGZsh{}\PYGZsh{} Use for callbacks}
\PYG{n+nv}{myCallback}\PYG{+w}{ }\PYG{o}{\PYGZlt{}\PYGZhy{}}\PYG{+w}{ }\PYG{l+s+s2}{\PYGZdq{}ProcessResponse\PYGZdq{}}
\PYG{n+nf}{CallFunction}\PYG{p}{(}\PYG{n+nv}{myCallback}\PYG{p}{,}\PYG{+w}{ }\PYG{p}{[}\PYG{n+nv}{response}\PYG{p}{,}\PYG{+w}{ }\PYG{n+nv}{rt}\PYG{p}{]}\PYG{p}{)}
\end{sphinxVerbatim}

\sphinxAtStartPar
\sphinxstylestrong{See Also:}

\sphinxAtStartPar
\sphinxcode{\sphinxupquote{PropertyExists()}}, \sphinxcode{\sphinxupquote{MakeCustomObject()}}

\index{DeleteFile@\spxentry{DeleteFile}}\ignorespaces 

\subsection{DeleteFile()}
\label{\detokenize{reference/peblenvironment:deletefile}}\label{\detokenize{reference/peblenvironment:index-3}}
\sphinxAtStartPar
\sphinxstyleemphasis{Deletes a file}

\sphinxAtStartPar
\sphinxstylestrong{Description:}

\sphinxAtStartPar
Deletes a file from the file system.

\sphinxAtStartPar
\sphinxstylestrong{Usage:}

\begin{sphinxVerbatim}[commandchars=\\\{\}]
\PYG{n+nf}{DeleteFile}\PYG{p}{(}\PYG{+w}{ }\PYG{o}{\PYGZlt{}}\PYG{n+nv}{filename}\PYG{o}{\PYGZgt{}}\PYG{p}{)}
\end{sphinxVerbatim}

\sphinxAtStartPar
\sphinxstylestrong{Example:}

\begin{sphinxVerbatim}[commandchars=\\\{\}]
\PYG{n+nv}{tmpfile}\PYG{+w}{ }\PYG{o}{\PYGZlt{}\PYGZhy{}}\PYG{+w}{ }\PYG{n+nf}{FileOpenWrite}\PYG{p}{(}\PYG{l+s+s2}{\PYGZdq{}tmp.txt\PYGZdq{}}\PYG{p}{)}
\PYG{n+nf}{FilePrint}\PYG{p}{(}\PYG{n+nv}{tmpfile}\PYG{p}{,}\PYG{n+nf}{Random}\PYG{p}{(}\PYG{p}{)}\PYG{p}{)}
\PYG{n+nf}{FileClose}\PYG{p}{(}\PYG{n+nv}{tmpfile}\PYG{p}{)}
\PYG{n+nv}{text}\PYG{+w}{ }\PYG{o}{\PYGZlt{}\PYGZhy{}}\PYG{+w}{ }\PYG{n+nf}{FileReadText}\PYG{p}{(}\PYG{l+s+s2}{\PYGZdq{}tmp.txt\PYGZdq{}}\PYG{p}{)}
\PYG{n+nf}{DeleteFile}\PYG{p}{(}\PYG{l+s+s2}{\PYGZdq{}tmp.txt\PYGZdq{}}\PYG{p}{)}
\end{sphinxVerbatim}

\sphinxAtStartPar
\sphinxstylestrong{See Also:}

\sphinxAtStartPar
\sphinxcode{\sphinxupquote{GetDirectoryListing()}}, \sphinxcode{\sphinxupquote{FileExists()}},       \sphinxcode{\sphinxupquote{IsDirectory()}},            \sphinxcode{\sphinxupquote{MakeDirectory()}}

\index{ExitQuietly@\spxentry{ExitQuietly}}\ignorespaces 

\subsection{ExitQuietly()}
\label{\detokenize{reference/peblenvironment:exitquietly}}\label{\detokenize{reference/peblenvironment:index-4}}
\sphinxAtStartPar
\sphinxstylestrong{Description:}

\sphinxAtStartPar
Stops PEBL and prints \sphinxcode{\sphinxupquote{\textless{}message\textgreater{}}} to stderr. Unlike SignalFatalError, it will NOT pop\sphinxhyphen{}up a window with the error message. Useful exiting a study or application without causing a popup error message.

\sphinxAtStartPar
\sphinxstylestrong{Usage:}

\begin{sphinxVerbatim}[commandchars=\\\{\}]
\PYG{n+nf}{ExitQuietly}\PYG{p}{(}\PYG{o}{\PYGZlt{}}\PYG{n+nv}{message}\PYG{o}{\PYGZgt{}}\PYG{p}{)}
\end{sphinxVerbatim}

\sphinxAtStartPar
\sphinxstylestrong{Example:}

\begin{sphinxVerbatim}[commandchars=\\\{\}]
\PYG{k}{If}\PYG{p}{(}\PYG{n+nv}{response}\PYG{+w}{ }\PYG{o}{==}\PYG{+w}{ }\PYG{l+s+s2}{\PYGZdq{}exit\PYGZdq{}}\PYG{p}{)}
\PYG{+w}{     }\PYG{p}{\PYGZob{}}
\PYG{+w}{        }\PYG{n+nf}{ExitQuietly}\PYG{p}{(}\PYG{l+s+s2}{\PYGZdq{}Exiting study.\PYGZdq{}}\PYG{p}{)}
\PYG{+w}{     }\PYG{p}{\PYGZcb{}}
\PYG{+w}{     }\PYG{c+c1}{\PYGZsh{}\PYGZsh{}Prints out error message and}
\PYG{+w}{     }\PYG{c+c1}{\PYGZsh{}\PYGZsh{}line/filename of function}
\end{sphinxVerbatim}

\sphinxAtStartPar
\sphinxstylestrong{See Also:}

\sphinxAtStartPar
\sphinxcode{\sphinxupquote{MessageBox}}, \sphinxcode{\sphinxupquote{Print()}}, \sphinxcode{\sphinxupquote{SignalFatalError()}}

\index{FileExists@\spxentry{FileExists}}\ignorespaces 

\subsection{FileExists()}
\label{\detokenize{reference/peblenvironment:fileexists}}\label{\detokenize{reference/peblenvironment:index-5}}
\sphinxAtStartPar
\sphinxstyleemphasis{Checks whether a file exists}

\sphinxAtStartPar
\sphinxstylestrong{Description:}

\sphinxAtStartPar
Checks whether a file exists.  Returns 1 if it exists, 0 otherwise.

\sphinxAtStartPar
\sphinxstylestrong{Usage:}

\begin{sphinxVerbatim}[commandchars=\\\{\}]
\PYG{n+nf}{FileExists}\PYG{p}{(}\PYG{o}{\PYGZlt{}}\PYG{n+nv}{path}\PYG{o}{\PYGZgt{}}\PYG{p}{)}
\end{sphinxVerbatim}

\sphinxAtStartPar
\sphinxstylestrong{Example:}

\begin{sphinxVerbatim}[commandchars=\\\{\}]
\PYG{n+nv}{filename}\PYG{+w}{ }\PYG{o}{\PYGZlt{}\PYGZhy{}}\PYG{+w}{ }\PYG{l+s+s2}{\PYGZdq{}data\PYGZhy{}\PYGZdq{}}\PYG{o}{+}\PYG{n+nv+vg}{gSubNum}\PYG{o}{+}\PYG{l+s+s2}{\PYGZdq{}.csv\PYGZdq{}}
\PYG{+w}{ }\PYG{n+nv}{exists}\PYG{+w}{ }\PYG{o}{\PYGZlt{}\PYGZhy{}}\PYG{+w}{  }\PYG{n+nf}{FileExists}\PYG{p}{(}\PYG{n+nv}{filename}\PYG{p}{)}
\PYG{+w}{  }\PYG{k}{if}\PYG{p}{(}\PYG{n+nv}{exists}\PYG{p}{)}
\PYG{+w}{   }\PYG{p}{\PYGZob{}}
\PYG{+w}{    }\PYG{n+nf}{MessageBox}\PYG{p}{(}\PYG{l+s+s2}{\PYGZdq{}Subject file already exists. \PYGZdq{}}\PYG{o}{+}
\PYG{+w}{    }\PYG{l+s+s2}{\PYGZdq{} Please try a new one.\PYGZdq{}}\PYG{p}{,}\PYG{n+nv+vg}{gWin}\PYG{p}{)}
\PYG{+w}{    }\PYG{n+nf}{SignalFatalError}\PYG{p}{(}\PYG{l+s+s2}{\PYGZdq{}filename already used\PYGZdq{}}\PYG{p}{)}
\PYG{+w}{   }\PYG{p}{\PYGZcb{}}
\end{sphinxVerbatim}

\sphinxAtStartPar
\sphinxstylestrong{See Also:}

\sphinxAtStartPar
\sphinxcode{\sphinxupquote{GetDirectoryListing()}}, \sphinxcode{\sphinxupquote{FileExists()}},       \sphinxcode{\sphinxupquote{IsDirectory()}},            \sphinxcode{\sphinxupquote{MakeDirectory()}}

\index{GetCurrentScreenResolution@\spxentry{GetCurrentScreenResolution}}\ignorespaces 

\subsection{GetCurrentScreenResolution()}
\label{\detokenize{reference/peblenvironment:getcurrentscreenresolution}}\label{\detokenize{reference/peblenvironment:index-6}}
\sphinxAtStartPar
\sphinxstylestrong{Description:}

\sphinxAtStartPar
Returns an list of {[}width,height{]} specifying what the  current computer screen resolution is.  This is used within the pebl launcher in order to use the current resolution to run the experiment.

\sphinxAtStartPar
\sphinxstylestrong{Usage:}

\begin{sphinxVerbatim}[commandchars=\\\{\}]
\PYG{n+nv}{res}\PYG{+w}{ }\PYG{o}{\PYGZlt{}\PYGZhy{}}\PYG{+w}{ }\PYG{n+nf}{GetCurrentScreenResolution}\PYG{p}{(}\PYG{p}{)}
\end{sphinxVerbatim}

\sphinxAtStartPar
\sphinxstylestrong{Example:}

\begin{sphinxVerbatim}[commandchars=\\\{\}]
\PYG{k}{define}\PYG{+w}{ }\PYG{n+nf}{Start}\PYG{p}{(}\PYG{n+nv}{p}\PYG{p}{)}
\PYG{p}{\PYGZob{}}
\PYG{+w}{   }\PYG{c+c1}{\PYGZsh{}\PYGZsh{} For testing, let\PYGZsq{}s make the screen resolution a bit smaller than the}
\PYG{+w}{   }\PYG{c+c1}{\PYGZsh{}\PYGZsh{} current one so that it doesn\PYGZsq{}t get hidden by the bottom task bar}
\PYG{+w}{   }\PYG{c+c1}{\PYGZsh{}\PYGZsh{}}
\PYG{+w}{   }\PYG{n+nv}{res}\PYG{+w}{ }\PYG{o}{\PYGZlt{}\PYGZhy{}}\PYG{+w}{ }\PYG{n+nf}{GetCurrentScreenResolution}\PYG{p}{(}\PYG{p}{)}
\PYG{+w}{   }\PYG{n+nv+vg}{gVideoWidth}\PYG{+w}{ }\PYG{o}{\PYGZlt{}\PYGZhy{}}\PYG{+w}{ }\PYG{n+nf}{First}\PYG{p}{(}\PYG{n+nv}{res}\PYG{p}{)}\PYG{o}{\PYGZhy{}}\PYG{l+m+mi}{100}
\PYG{+w}{   }\PYG{n+nv+vg}{gVideoHeight}\PYG{+w}{ }\PYG{o}{\PYGZlt{}\PYGZhy{}}\PYG{+w}{ }\PYG{n+nf}{Second}\PYG{p}{(}\PYG{n+nv}{res}\PYG{p}{)}\PYG{o}{\PYGZhy{}}\PYG{l+m+mi}{100}
\PYG{+w}{   }\PYG{n+nv+vg}{gWin}\PYG{+w}{ }\PYG{o}{\PYGZlt{}\PYGZhy{}}\PYG{+w}{ }\PYG{n+nf}{MakeWindow}\PYG{p}{(}\PYG{p}{)}
\PYG{+w}{   }\PYG{n+nf}{MessageBox}\PYG{p}{(}\PYG{l+s+s2}{\PYGZdq{}Window slightly smaller than screen\PYGZdq{}}\PYG{p}{,}\PYG{n+nv+vg}{gWin}\PYG{p}{)}
\PYG{p}{\PYGZcb{}}
\end{sphinxVerbatim}

\sphinxAtStartPar
\sphinxstylestrong{See Also:}

\sphinxAtStartPar
\sphinxcode{\sphinxupquote{GetVideoModes()}}

\index{GetDirectoryListing@\spxentry{GetDirectoryListing}}\ignorespaces 

\subsection{GetDirectoryListing()}
\label{\detokenize{reference/peblenvironment:getdirectorylisting}}\label{\detokenize{reference/peblenvironment:index-7}}
\sphinxAtStartPar
\sphinxstyleemphasis{Returns a list of all the files/subdirectories in a path}

\sphinxAtStartPar
\sphinxstylestrong{Description:}

\sphinxAtStartPar
Returns a list of files and directories in a particular directory/folder.

\sphinxAtStartPar
\sphinxstylestrong{Usage:}

\begin{sphinxVerbatim}[commandchars=\\\{\}]
\PYG{n+nv}{list}\PYG{+w}{ }\PYG{o}{\PYGZlt{}\PYGZhy{}}\PYG{+w}{ }\PYG{n+nf}{GetDirectoryListing}\PYG{p}{(}\PYG{o}{\PYGZlt{}}\PYG{n+nv}{path}\PYG{o}{\PYGZgt{}}\PYG{p}{)}
\end{sphinxVerbatim}

\sphinxAtStartPar
\sphinxstylestrong{Example:}

\begin{sphinxVerbatim}[commandchars=\\\{\}]
\PYG{n+nv}{files}\PYG{+w}{ }\PYG{o}{\PYGZlt{}\PYGZhy{}}\PYG{+w}{  }\PYG{n+nf}{GetDirectoryListing}\PYG{p}{(}\PYG{l+s+s2}{\PYGZdq{}./\PYGZdq{}}\PYG{p}{)}
\end{sphinxVerbatim}

\sphinxAtStartPar
\sphinxstylestrong{See Also:}

\sphinxAtStartPar
\sphinxcode{\sphinxupquote{GetDirectoryListing()}}, \sphinxcode{\sphinxupquote{FileExists()}},       \sphinxcode{\sphinxupquote{IsDirectory()}},            \sphinxcode{\sphinxupquote{MakeDirectory()}}

\index{GetDrivers@\spxentry{GetDrivers}}\ignorespaces 

\subsection{GetDrivers()}
\label{\detokenize{reference/peblenvironment:getdrivers}}\label{\detokenize{reference/peblenvironment:index-8}}
\sphinxAtStartPar
\sphinxstyleemphasis{Gets a list of possible video drivers}

\sphinxAtStartPar
\sphinxstylestrong{Description:}

\sphinxAtStartPar
Gets a list of video drivers on the current platform. This is usually one of opengl, opengles, software, and directx, different ones of which are available on different platforms.  This is most useful for building launchers, although it could be used within a script \sphinxstyleemphasis{before} MakeWindow is called to choose the best available driver.

\sphinxAtStartPar
\sphinxstylestrong{Usage:}

\begin{sphinxVerbatim}[commandchars=\\\{\}]
\PYG{n+nv}{drivers}\PYG{+w}{ }\PYG{o}{\PYGZlt{}\PYGZhy{}}\PYG{+w}{ }\PYG{n+nf}{GetDrivers}\PYG{p}{(}\PYG{p}{)}
\end{sphinxVerbatim}

\sphinxAtStartPar
\sphinxstylestrong{See Also:}

\sphinxAtStartPar
\sphinxcode{\sphinxupquote{GetCurrentScreenResolution()}}, \sphinxcode{\sphinxupquote{gVideoWidth()}}, \sphinxcode{\sphinxupquote{gVideoHeight()}},          \sphinxcode{\sphinxupquote{GetVideoModes()}}

\index{GetExecutableName@\spxentry{GetExecutableName}}\ignorespaces 

\subsection{GetExecutableName()}
\label{\detokenize{reference/peblenvironment:getexecutablename}}\label{\detokenize{reference/peblenvironment:index-9}}
\sphinxAtStartPar
\sphinxstyleemphasis{Returns the name/path of the PEBL executable}

\sphinxAtStartPar
\sphinxstylestrong{Description:}

\sphinxAtStartPar
This function signals a fatal error directing users to use the global variable \sphinxcode{\sphinxupquote{gExecutableName}} instead. The executable name is set at program startup and stored in this global variable.

\sphinxAtStartPar
\sphinxstylestrong{Usage:}

\begin{sphinxVerbatim}[commandchars=\\\{\}]
\PYG{n+nv}{name}\PYG{+w}{ }\PYG{o}{\PYGZlt{}\PYGZhy{}}\PYG{+w}{ }\PYG{n+nv+vg}{gExecutableName}\PYG{+w}{  }\PYG{c+c1}{\PYGZsh{}\PYGZsh{}Use this global variable instead}
\end{sphinxVerbatim}

\sphinxAtStartPar
\sphinxstylestrong{See Also:}

\sphinxAtStartPar
\sphinxcode{\sphinxupquote{GetSystemType()}}, \sphinxcode{\sphinxupquote{GetWorkingDirectory()}}

\index{GetHomeDirectory@\spxentry{GetHomeDirectory}}\ignorespaces 

\subsection{GetHomeDirectory()}
\label{\detokenize{reference/peblenvironment:gethomedirectory}}\label{\detokenize{reference/peblenvironment:index-10}}
\sphinxAtStartPar
\sphinxstyleemphasis{Returns the user’s home directory path}

\sphinxAtStartPar
\sphinxstylestrong{Description:}

\sphinxAtStartPar
Returns the path to the current user’s home directory. This is platform\sphinxhyphen{}specific and will return different values on Windows, Linux, and Mac OS.

\sphinxAtStartPar
\sphinxstylestrong{Usage:}

\begin{sphinxVerbatim}[commandchars=\\\{\}]
\PYG{n+nf}{GetHomeDirectory}\PYG{p}{(}\PYG{p}{)}
\end{sphinxVerbatim}

\sphinxAtStartPar
\sphinxstylestrong{Example:}

\begin{sphinxVerbatim}[commandchars=\\\{\}]
\PYG{n+nv}{homedir}\PYG{+w}{ }\PYG{o}{\PYGZlt{}\PYGZhy{}}\PYG{+w}{ }\PYG{n+nf}{GetHomeDirectory}\PYG{p}{(}\PYG{p}{)}
\PYG{n+nf}{Print}\PYG{p}{(}\PYG{l+s+s2}{\PYGZdq{}User home directory: \PYGZdq{}}\PYG{+w}{ }\PYG{o}{+}\PYG{+w}{ }\PYG{n+nv}{homedir}\PYG{p}{)}
\end{sphinxVerbatim}

\sphinxAtStartPar
\sphinxstylestrong{See Also:}

\sphinxAtStartPar
\sphinxcode{\sphinxupquote{GetWorkingDirectory()}}, \sphinxcode{\sphinxupquote{SetWorkingDirectory()}}, \sphinxcode{\sphinxupquote{GetDirectoryListing()}}

\index{GetJoystickAxisState@\spxentry{GetJoystickAxisState}}\ignorespaces 

\subsection{GetJoystickAxisState()}
\label{\detokenize{reference/peblenvironment:getjoystickaxisstate}}\label{\detokenize{reference/peblenvironment:index-11}}
\sphinxAtStartPar
\sphinxstyleemphasis{Gets the state of a joystick axis}

\sphinxAtStartPar
\sphinxstylestrong{Description:}

\sphinxAtStartPar
This gets the state of a particular joystick axis.  You need to specify a joystick object, which is created with OpenJoystick().  You also need to specify the axis.  You can determine how many axes a joystick has with the GetNumJoystickAxes() function.  The function returns  a value between 1 and 32768.

\sphinxAtStartPar
\sphinxstylestrong{See Also:}

\sphinxAtStartPar
GetNumJoysticks(), OpenJoystick(), GetNumJoystickAxes() GetNumJoystickBalls(), GetNumJoystickButtons(), GetNumJoystickHats() GetJoystickAxisState(), GetJoystickHatState(), GetJoystickButtonState()

\index{GetJoystickBallState@\spxentry{GetJoystickBallState}}\ignorespaces 

\subsection{GetJoystickBallState()}
\label{\detokenize{reference/peblenvironment:getjoystickballstate}}\label{\detokenize{reference/peblenvironment:index-12}}
\sphinxAtStartPar
\sphinxstyleemphasis{Gets the state of a joystick ball}

\sphinxAtStartPar
\sphinxstylestrong{Description:}

\sphinxAtStartPar
Not implemented.

\sphinxAtStartPar
\sphinxstylestrong{See Also:}

\sphinxAtStartPar
GetNumJoysticks(), OpenJoystick(), GetNumJoystickAxes() GetNumJoystickBalls(), GetNumJoystickButtons(), GetNumJoystickHats() GetJoystickAxisState(), GetJoystickHatState(), GetJoystickButtonState()

\index{GetJoystickButtonState@\spxentry{GetJoystickButtonState}}\ignorespaces 

\subsection{GetJoystickButtonState()}
\label{\detokenize{reference/peblenvironment:getjoystickbuttonstate}}\label{\detokenize{reference/peblenvironment:index-13}}
\sphinxAtStartPar
\sphinxstylestrong{Description:}

\sphinxAtStartPar
This gets the state of a particular joystick button.  You need to specify a joystick object, which is created with OpenJoystick().  You also need to specify the button.  You can determine how many buttons a joystick has with the GetNumJoystickButtons() function.  The function returns either 0 (for unpressed) or 1 (for pressed).

\sphinxAtStartPar
\sphinxstylestrong{See Also:}

\sphinxAtStartPar
GetNumJoysticks(), OpenJoystick(), GetNumJoystickAxes() GetNumJoystickBalls(), GetNumJoystickButtons(), GetNumJoystickHats() GetJoystickAxisState(), GetJoystickHatState(), GetJoystickButtonState()

\index{GetJoystickHatState@\spxentry{GetJoystickHatState}}\ignorespaces 

\subsection{GetJoystickHatState()}
\label{\detokenize{reference/peblenvironment:getjoystickhatstate}}\label{\detokenize{reference/peblenvironment:index-14}}
\sphinxAtStartPar
\sphinxstyleemphasis{Gets the state of a joystick hat}

\sphinxAtStartPar
\sphinxstylestrong{Description:}

\sphinxAtStartPar
\sphinxcode{\sphinxupquote{GetJoystickHatState(js,1)}}    This gets the state of a particular joystick hat.  You need to specify a joystick object, which is created with OpenJoystick().  You also need to specify the hat id.  You can determine how many hats a joystick has with the GetNumJoystickHats() function.  The function returns a value between 0 and 15, which is the sum of values specifying whether each primary NSEW direction is pressed.  The coding is: 0=no buttons; 1=N, 2=E, 4=S, 8=W.  Thus, if 1 is returned, the north hat button is pressed.  If 3 is returned, NorthEast.  If 12 is returned, SW, and so on.

\sphinxAtStartPar
\sphinxstylestrong{See Also:}

\sphinxAtStartPar
GetNumJoysticks(), OpenJoystick(), GetNumJoystickAxes() GetNumJoystickBalls(), GetNumJoystickButtons(), GetNumJoystickHats() GetJoystickAxisState(), GetJoystickHatState(), GetJoystickButtonState()

\index{GetMouseCursorPosition@\spxentry{GetMouseCursorPosition}}\ignorespaces 

\subsection{GetMouseCursorPosition()}
\label{\detokenize{reference/peblenvironment:getmousecursorposition}}\label{\detokenize{reference/peblenvironment:index-15}}
\sphinxAtStartPar
\sphinxstylestrong{Description:}

\sphinxAtStartPar
Gets the current x,y coordinates of the mouse   pointer.

\sphinxAtStartPar
\sphinxstylestrong{Usage:}

\begin{sphinxVerbatim}[commandchars=\\\{\}]
\PYG{n+nf}{GetMouseCursorPosition}\PYG{p}{(}\PYG{p}{)}
\end{sphinxVerbatim}

\sphinxAtStartPar
\sphinxstylestrong{Example:}

\begin{sphinxVerbatim}[commandchars=\\\{\}]
\PYG{n+nv}{pos}\PYG{+w}{ }\PYG{o}{\PYGZlt{}\PYGZhy{}}\PYG{+w}{ }\PYG{n+nf}{GetMouseCursorPosition}\PYG{p}{(}\PYG{p}{)}
\end{sphinxVerbatim}

\sphinxAtStartPar
\sphinxstylestrong{See Also:}

\sphinxAtStartPar
\sphinxcode{\sphinxupquote{ShowCursor()}}, \sphinxcode{\sphinxupquote{WaitForMouseButton()}},   \sphinxcode{\sphinxupquote{SetMouseCursorPosition()}}, \sphinxcode{\sphinxupquote{GetMouseCursorPosition()}}

\index{GetMouseState@\spxentry{GetMouseState}}\ignorespaces 

\subsection{GetMouseState()}
\label{\detokenize{reference/peblenvironment:getmousestate}}\label{\detokenize{reference/peblenvironment:index-16}}
\sphinxAtStartPar
\sphinxstyleemphasis{Gets {[}x,y,b1,b2,b3{]} list of mouse state, including button states}

\sphinxAtStartPar
\sphinxstylestrong{Description:}

\sphinxAtStartPar
Gets the current x,y coordinates of the mouse   pointer, plus the current state of the buttons.  Returns a 5\sphinxhyphen{}element list, with the first two indicating x,y position, the third is either 0 or 1 depending on if the left mouse is clicked, the fourth 0 or 2 depending on whether the middle mouse is clicked, and the fifth either 0 or 4 depending on whether the right mouse is clicked.

\sphinxAtStartPar
\sphinxstylestrong{Example:}

\begin{sphinxVerbatim}[commandchars=\\\{\}]
\PYG{k}{define}\PYG{+w}{ }\PYG{n+nf}{Start}\PYG{p}{(}\PYG{n+nv}{p}\PYG{p}{)}
\PYG{p}{\PYGZob{}}

\PYG{+w}{  }\PYG{n+nv}{win}\PYG{+w}{ }\PYG{o}{\PYGZlt{}\PYGZhy{}}\PYG{+w}{ }\PYG{n+nf}{MakeWindow}\PYG{p}{(}\PYG{p}{)}
\PYG{+w}{  }\PYG{n+nv}{i}\PYG{+w}{ }\PYG{o}{\PYGZlt{}\PYGZhy{}}\PYG{+w}{ }\PYG{l+m+mi}{1}
\PYG{+w}{  }\PYG{k}{while}\PYG{p}{(}\PYG{n+nv}{i}\PYG{+w}{ }\PYG{o}{\PYGZlt{}}\PYG{+w}{ }\PYG{l+m+mi}{100}\PYG{p}{)}
\PYG{+w}{  }\PYG{p}{\PYGZob{}}
\PYG{+w}{    }\PYG{n+nf}{Draw}\PYG{p}{(}\PYG{p}{)}
\PYG{+w}{    }\PYG{n+nf}{Print}\PYG{p}{(}\PYG{n+nf}{GetMouseState}\PYG{p}{(}\PYG{p}{)}\PYG{p}{)}

\PYG{+w}{    }\PYG{n+nf}{Wait}\PYG{p}{(}\PYG{l+m+mi}{100}\PYG{p}{)}
\PYG{+w}{    }\PYG{n+nv}{i}\PYG{+w}{ }\PYG{o}{\PYGZlt{}\PYGZhy{}}\PYG{+w}{ }\PYG{n+nv}{i}\PYG{+w}{ }\PYG{o}{+}\PYG{+w}{ }\PYG{l+m+mi}{1}

\PYG{+w}{  }\PYG{p}{\PYGZcb{}}
\PYG{c+c1}{\PYGZsh{}\PYGZsh{}Returns look like:}
\PYG{p}{[}\PYG{l+m+mi}{417}\PYG{p}{,}\PYG{+w}{ }\PYG{l+m+mi}{276}\PYG{p}{,}\PYG{+w}{ }\PYG{l+m+mi}{0}\PYG{p}{,}\PYG{+w}{ }\PYG{l+m+mi}{0}\PYG{p}{,}\PYG{+w}{ }\PYG{l+m+mi}{0}\PYG{p}{]}
\PYG{p}{[}\PYG{l+m+mi}{495}\PYG{p}{,}\PYG{+w}{ }\PYG{l+m+mi}{286}\PYG{p}{,}\PYG{+w}{ }\PYG{l+m+mi}{0}\PYG{p}{,}\PYG{+w}{ }\PYG{l+m+mi}{0}\PYG{p}{,}\PYG{+w}{ }\PYG{l+m+mi}{0}\PYG{p}{]}
\PYG{p}{[}\PYG{l+m+mi}{460}\PYG{p}{,}\PYG{+w}{ }\PYG{l+m+mi}{299}\PYG{p}{,}\PYG{+w}{ }\PYG{l+m+mi}{0}\PYG{p}{,}\PYG{+w}{ }\PYG{l+m+mi}{0}\PYG{p}{,}\PYG{+w}{ }\PYG{l+m+mi}{0}\PYG{p}{]}
\PYG{p}{[}\PYG{l+m+mi}{428}\PYG{p}{,}\PYG{+w}{ }\PYG{l+m+mi}{217}\PYG{p}{,}\PYG{+w}{ }\PYG{l+m+mi}{0}\PYG{p}{,}\PYG{+w}{ }\PYG{l+m+mi}{0}\PYG{p}{,}\PYG{+w}{ }\PYG{l+m+mi}{0}\PYG{p}{]}
\PYG{p}{[}\PYG{l+m+mi}{446}\PYG{p}{,}\PYG{+w}{ }\PYG{l+m+mi}{202}\PYG{p}{,}\PYG{+w}{ }\PYG{l+m+mi}{0}\PYG{p}{,}\PYG{+w}{ }\PYG{l+m+mi}{0}\PYG{p}{,}\PYG{+w}{ }\PYG{l+m+mi}{4}\PYG{p}{]}
\PYG{p}{[}\PYG{l+m+mi}{446}\PYG{p}{,}\PYG{+w}{ }\PYG{l+m+mi}{202}\PYG{p}{,}\PYG{+w}{ }\PYG{l+m+mi}{1}\PYG{p}{,}\PYG{+w}{ }\PYG{l+m+mi}{0}\PYG{p}{,}\PYG{+w}{ }\PYG{l+m+mi}{0}\PYG{p}{]}
\PYG{p}{[}\PYG{l+m+mi}{446}\PYG{p}{,}\PYG{+w}{ }\PYG{l+m+mi}{202}\PYG{p}{,}\PYG{+w}{ }\PYG{l+m+mi}{1}\PYG{p}{,}\PYG{+w}{ }\PYG{l+m+mi}{0}\PYG{p}{,}\PYG{+w}{ }\PYG{l+m+mi}{0}\PYG{p}{]}
\PYG{p}{[}\PYG{l+m+mi}{446}\PYG{p}{,}\PYG{+w}{ }\PYG{l+m+mi}{202}\PYG{p}{,}\PYG{+w}{ }\PYG{l+m+mi}{0}\PYG{p}{,}\PYG{+w}{ }\PYG{l+m+mi}{2}\PYG{p}{,}\PYG{+w}{ }\PYG{l+m+mi}{0}\PYG{p}{]}
\end{sphinxVerbatim}

\sphinxAtStartPar
\sphinxstylestrong{See Also:}

\sphinxAtStartPar
\sphinxcode{\sphinxupquote{ShowCursor()}} \sphinxcode{\sphinxupquote{WaitForMouseButton()}},   \sphinxcode{\sphinxupquote{SetMouseCursorPosition()}}, \sphinxcode{\sphinxupquote{GetMouseCursorPosition()}}

\index{GetNumJoystickAxes@\spxentry{GetNumJoystickAxes}}\ignorespaces 

\subsection{GetNumJoystickAxes()}
\label{\detokenize{reference/peblenvironment:getnumjoystickaxes}}\label{\detokenize{reference/peblenvironment:index-17}}
\sphinxAtStartPar
\sphinxstyleemphasis{Counts how many axes on a joystick}

\sphinxAtStartPar
\sphinxstylestrong{Description:}

\sphinxAtStartPar
This gets the number of axes on a joystick.  You need to specify a joystick object, which is created with OpenJoystick().

\sphinxAtStartPar
\sphinxstylestrong{See Also:}

\sphinxAtStartPar
GetNumJoysticks(), OpenJoystick(), GetNumJoystickAxes() GetNumJoystickBalls(), GetNumJoystickButtons(), GetNumJoystickHats() GetJoystickAxisState(), GetJoystickHatState(), GetJoystickButtonState()

\index{GetNumJoystickBalls@\spxentry{GetNumJoystickBalls}}\ignorespaces 

\subsection{GetNumJoystickBalls()}
\label{\detokenize{reference/peblenvironment:getnumjoystickballs}}\label{\detokenize{reference/peblenvironment:index-18}}
\sphinxAtStartPar
\sphinxstyleemphasis{Counts how many balls on a joystick}

\sphinxAtStartPar
\sphinxstylestrong{Description:}

\sphinxAtStartPar
This gets the number of joystick balls available on a particular joystick.  You need to specify a joystick object, which is created with OpenJoystick().

\sphinxAtStartPar
\sphinxstylestrong{See Also:}

\sphinxAtStartPar
GetNumJoysticks(), OpenJoystick(), GetNumJoystickAxes() GetNumJoystickBalls(), GetNumJoystickButtons(), GetNumJoystickHats() GetJoystickAxisState(), GetJoystickHatState(), GetJoystickButtonState()

\index{GetNumJoystickButtons@\spxentry{GetNumJoystickButtons}}\ignorespaces 

\subsection{GetNumJoystickButtons()}
\label{\detokenize{reference/peblenvironment:getnumjoystickbuttons}}\label{\detokenize{reference/peblenvironment:index-19}}
\sphinxAtStartPar
\sphinxstylestrong{Description:}

\sphinxAtStartPar
This gets the number of joystick buttons available on a particular joystick.  You need to specify a joystick object, which is created with OpenJoystick().

\sphinxAtStartPar
\sphinxstylestrong{See Also:}

\sphinxAtStartPar
GetNumJoysticks(), OpenJoystick(), GetNumJoystickAxes() GetNumJoystickBalls(), GetNumJoystickButtons(), GetNumJoystickHats() GetJoystickAxisState(), GetJoystickHatState(), GetJoystickButtonState()

\index{GetNumJoystickHats@\spxentry{GetNumJoystickHats}}\ignorespaces 

\subsection{GetNumJoystickHats()}
\label{\detokenize{reference/peblenvironment:getnumjoystickhats}}\label{\detokenize{reference/peblenvironment:index-20}}
\sphinxAtStartPar
\sphinxstyleemphasis{Counts how many hats on a joystick}

\sphinxAtStartPar
\sphinxstylestrong{Description:}

\sphinxAtStartPar
This gets the number of hats available on a particular joystick.  You need to specify a joystick object, which is created with OpenJoystick().

\sphinxAtStartPar
\sphinxstylestrong{See Also:}

\sphinxAtStartPar
GetNumJoysticks(), OpenJoystick(), GetNumJoystickAxes() GetNumJoystickBalls(), GetNumJoystickButtons(), GetNumJoystickHats() GetJoystickAxisState(), GetJoystickHatState(), GetJoystickButtonState()

\index{GetNumJoysticks@\spxentry{GetNumJoysticks}}\ignorespaces 

\subsection{GetNumJoysticks()}
\label{\detokenize{reference/peblenvironment:getnumjoysticks}}\label{\detokenize{reference/peblenvironment:index-21}}
\sphinxAtStartPar
\sphinxstyleemphasis{Determines how many joysticks are available}

\sphinxAtStartPar
\sphinxstylestrong{Description:}

\sphinxAtStartPar
This gets the number of joysticks available on a system. It returns an integer, which if greater than   you can open a joystick using the OpenJoystick() function..

\sphinxAtStartPar
\sphinxstylestrong{See Also:}

\sphinxAtStartPar
GetNumJoysticks(), OpenJoystick(), GetNumJoystickAxes() GetNumJoystickBalls(), GetNumJoystickButtons(), GetNumJoystickHats() GetJoystickAxisState(), GetJoystickHatState(), GetJoystickButtonState()

\index{GetPEBLVersion@\spxentry{GetPEBLVersion}}\ignorespaces 

\subsection{GetPEBLVersion()}
\label{\detokenize{reference/peblenvironment:getpeblversion}}\label{\detokenize{reference/peblenvironment:index-22}}
\sphinxAtStartPar
\sphinxstyleemphasis{Returns a string indicating which version of PEBL you are using}

\sphinxAtStartPar
\sphinxstylestrong{Description:}

\sphinxAtStartPar
Returns a string describing which version of PEBL you are running.

\sphinxAtStartPar
\sphinxstylestrong{Usage:}

\begin{sphinxVerbatim}[commandchars=\\\{\}]
\PYG{n+nf}{GetPEBLVersion}\PYG{p}{(}\PYG{p}{)}
\end{sphinxVerbatim}

\sphinxAtStartPar
\sphinxstylestrong{Example:}

\begin{sphinxVerbatim}[commandchars=\\\{\}]
\PYG{n+nf}{Print}\PYG{p}{(}\PYG{n+nf}{GetPEBLVersion}\PYG{p}{(}\PYG{p}{)}\PYG{p}{)}
\end{sphinxVerbatim}

\sphinxAtStartPar
\sphinxstylestrong{See Also:}

\sphinxAtStartPar
\sphinxcode{\sphinxupquote{TimeStamp()}}

\index{GetSystemType@\spxentry{GetSystemType}}\ignorespaces 

\subsection{GetSystemType()}
\label{\detokenize{reference/peblenvironment:getsystemtype}}\label{\detokenize{reference/peblenvironment:index-23}}
\sphinxAtStartPar
\sphinxstyleemphasis{Identifies the type of operating system being used.}

\sphinxAtStartPar
\sphinxstylestrong{Description:}

\sphinxAtStartPar
Returns a string identify what type of computer system you are using. It will return either: OSX, LINUX, or WINDOWS.

\sphinxAtStartPar
\sphinxstylestrong{Usage:}

\begin{sphinxVerbatim}[commandchars=\\\{\}]
\PYG{n+nf}{GetSystemType}\PYG{p}{(}\PYG{p}{)}
\end{sphinxVerbatim}

\sphinxAtStartPar
\sphinxstylestrong{Example:}

\begin{sphinxVerbatim}[commandchars=\\\{\}]
\PYG{c+c1}{\PYGZsh{}\PYGZsh{} Put this at the beginning of an experiment,}
\PYG{c+c1}{\PYGZsh{}\PYGZsh{} after a window gWin has been defined.}
\PYG{+w}{   }\PYG{k}{if}\PYG{p}{(}\PYG{n+nf}{GetSystemType}\PYG{p}{(}\PYG{p}{)}\PYG{+w}{ }\PYG{o}{==}\PYG{+w}{ }\PYG{l+s+s2}{\PYGZdq{}WINDOWS\PYGZdq{}}\PYG{p}{)}
\PYG{+w}{    }\PYG{p}{\PYGZob{}}
\PYG{+w}{      }\PYG{n+nf}{SignalFatalError}\PYG{p}{(}\PYG{l+s+s2}{\PYGZdq{}Experiment untested on windows\PYGZdq{}}\PYG{p}{)}
\PYG{+w}{    }\PYG{p}{\PYGZcb{}}
\end{sphinxVerbatim}

\sphinxAtStartPar
\sphinxstylestrong{See Also:}

\sphinxAtStartPar
\sphinxcode{\sphinxupquote{SystemCall()}}

\index{GetTextBoxCursorFromClick@\spxentry{GetTextBoxCursorFromClick}}\ignorespaces 

\subsection{GetTextBoxCursorFromClick()}
\label{\detokenize{reference/peblenvironment:gettextboxcursorfromclick}}\label{\detokenize{reference/peblenvironment:index-24}}
\sphinxAtStartPar
\sphinxstylestrong{Description:}

\sphinxAtStartPar
Returns the position (in characters) corresponding to a x,y click on a text box.  The X,Y position must be relative to the x,y position of the box, not absolute.  Once obtained, the cursor position can be set with SetCursorPosition().

\sphinxAtStartPar
\sphinxstylestrong{Usage:}

\begin{sphinxVerbatim}[commandchars=\\\{\}]
\PYG{n+nf}{GetTextBoxCursorFromClick}\PYG{p}{(}\PYG{o}{\PYGZlt{}}\PYG{n+nv}{widget}\PYG{o}{\PYGZgt{}}\PYG{p}{,}\PYG{o}{\PYGZlt{}}\PYG{n+nv}{x}\PYG{o}{\PYGZgt{}}\PYG{p}{,}\PYG{o}{\PYGZlt{}}\PYG{n+nv}{y}\PYG{o}{\PYGZgt{}}\PYG{p}{)}
\end{sphinxVerbatim}

\sphinxAtStartPar
\sphinxstylestrong{See Also:}

\sphinxAtStartPar
\sphinxcode{\sphinxupquote{SetCursorPosition()}}, \sphinxcode{\sphinxupquote{GetCursorPosition()}}, \sphinxcode{\sphinxupquote{SetEditable()}}, \sphinxcode{\sphinxupquote{MakeTextBox()}}

\index{GetTime@\spxentry{GetTime}}\ignorespaces 

\subsection{GetTime()}
\label{\detokenize{reference/peblenvironment:gettime}}\label{\detokenize{reference/peblenvironment:index-25}}
\sphinxAtStartPar
\sphinxstyleemphasis{Gets a number, in milliseconds, representing the time since the PEBL program began running.}

\sphinxAtStartPar
\sphinxstylestrong{Description:}

\sphinxAtStartPar
Gets time, in milliseconds, from when PEBL was   initialized.  Do not use as a seed for the RNG, because it will tend   to be about the same on each run. Instead, use \sphinxcode{\sphinxupquote{RandomizeTimer()}}.

\sphinxAtStartPar
\sphinxstylestrong{Usage:}

\begin{sphinxVerbatim}[commandchars=\\\{\}]
\PYG{n+nf}{GetTime}\PYG{p}{(}\PYG{p}{)}
\end{sphinxVerbatim}

\sphinxAtStartPar
\sphinxstylestrong{Example:}

\begin{sphinxVerbatim}[commandchars=\\\{\}]
\PYG{n+nv}{a}\PYG{+w}{ }\PYG{o}{\PYGZlt{}\PYGZhy{}}\PYG{+w}{ }\PYG{n+nf}{GetTime}\PYG{p}{(}\PYG{p}{)}
\PYG{n+nf}{WaitForKeyDown}\PYG{p}{(}\PYG{l+s+s2}{\PYGZdq{}A\PYGZdq{}}\PYG{p}{)}
\PYG{n+nv}{b}\PYG{+w}{ }\PYG{o}{\PYGZlt{}\PYGZhy{}}\PYG{+w}{ }\PYG{n+nf}{GetTime}\PYG{p}{(}\PYG{p}{)}
\PYG{n+nf}{Print}\PYG{p}{(}\PYG{l+s+s2}{\PYGZdq{}Response time is: \PYGZdq{}}\PYG{+w}{ }\PYG{o}{+}\PYG{+w}{ }\PYG{p}{(}\PYG{n+nv}{b}\PYG{+w}{ }\PYG{o}{\PYGZhy{}}\PYG{+w}{ }\PYG{n+nv}{a}\PYG{p}{)}\PYG{p}{)}
\end{sphinxVerbatim}

\sphinxAtStartPar
\sphinxstylestrong{See Also:}

\sphinxAtStartPar
\sphinxcode{\sphinxupquote{TimeStamp()}}

\index{GetTimeOfDay@\spxentry{GetTimeOfDay}}\ignorespaces 

\subsection{GetTimeOfDay()}
\label{\detokenize{reference/peblenvironment:gettimeofday}}\label{\detokenize{reference/peblenvironment:index-26}}
\sphinxAtStartPar
\sphinxstyleemphasis{Returns the current time in seconds since Unix epoch}

\sphinxAtStartPar
\sphinxstylestrong{Description:}

\sphinxAtStartPar
Returns the current time of day in seconds since the Unix epoch (January 1, 1970). This provides an absolute timestamp useful for logging when events occurred in real\sphinxhyphen{}world time.

\sphinxAtStartPar
\sphinxstylestrong{Usage:}

\begin{sphinxVerbatim}[commandchars=\\\{\}]
\PYG{n+nf}{GetTimeOfDay}\PYG{p}{(}\PYG{p}{)}
\end{sphinxVerbatim}

\sphinxAtStartPar
\sphinxstylestrong{Example:}

\begin{sphinxVerbatim}[commandchars=\\\{\}]
\PYG{n+nv}{timestamp}\PYG{+w}{ }\PYG{o}{\PYGZlt{}\PYGZhy{}}\PYG{+w}{ }\PYG{n+nf}{GetTimeOfDay}\PYG{p}{(}\PYG{p}{)}
\PYG{n+nf}{Print}\PYG{p}{(}\PYG{l+s+s2}{\PYGZdq{}Current Unix timestamp: \PYGZdq{}}\PYG{+w}{ }\PYG{o}{+}\PYG{+w}{ }\PYG{n+nv}{timestamp}\PYG{p}{)}
\end{sphinxVerbatim}

\sphinxAtStartPar
\sphinxstylestrong{See Also:}

\sphinxAtStartPar
\sphinxcode{\sphinxupquote{GetTime()}}, \sphinxcode{\sphinxupquote{TimeStamp()}}

\index{GetVideoModes@\spxentry{GetVideoModes}}\ignorespaces 

\subsection{GetVideoModes()}
\label{\detokenize{reference/peblenvironment:getvideomodes}}\label{\detokenize{reference/peblenvironment:index-27}}
\sphinxAtStartPar
\sphinxstyleemphasis{Gets list of available screen resolutions}

\sphinxAtStartPar
\sphinxstylestrong{Description:}

\sphinxAtStartPar
Gets a list of useable video modes (in width/height pixel pairs), as supplied by the video driver, for a specified screen. Screen is specified as an integer, with 0 being the default screen. If no screen is specified, screen 0 is used.

\sphinxAtStartPar
\sphinxstylestrong{Usage:}

\begin{sphinxVerbatim}[commandchars=\\\{\}]
\PYG{n+nv}{modes}\PYG{+w}{ }\PYG{o}{\PYGZlt{}\PYGZhy{}}\PYG{+w}{ }\PYG{n+nf}{GetVideoModes}\PYG{p}{(}\PYG{p}{)}
\end{sphinxVerbatim}

\sphinxAtStartPar
\sphinxstylestrong{Example:}

\begin{sphinxVerbatim}[commandchars=\\\{\}]
\PYG{n+nf}{Print}\PYG{p}{(}G\PYG{n+nv}{etVideoModes}\PYG{p}{)}
\PYG{c+c1}{\PYGZsh{}\PYGZsh{}Might return:}
\PYG{p}{[}\PYG{p}{[}\PYG{l+m+mi}{1440}\PYG{p}{,}\PYG{+w}{ }\PYG{l+m+mi}{900}\PYG{p}{]}
\PYG{p}{,}\PYG{+w}{ }\PYG{p}{[}\PYG{l+m+mi}{1360}\PYG{p}{,}\PYG{+w}{ }\PYG{l+m+mi}{768}\PYG{p}{]}
\PYG{p}{,}\PYG{+w}{ }\PYG{p}{[}\PYG{l+m+mi}{1152}\PYG{p}{,}\PYG{+w}{ }\PYG{l+m+mi}{864}\PYG{p}{]}
\PYG{p}{,}\PYG{+w}{ }\PYG{p}{[}\PYG{l+m+mi}{1024}\PYG{p}{,}\PYG{+w}{ }\PYG{l+m+mi}{768}\PYG{p}{]}
\PYG{p}{,}\PYG{+w}{ }\PYG{p}{[}\PYG{l+m+mi}{960}\PYG{p}{,}\PYG{+w}{ }\PYG{l+m+mi}{600}\PYG{p}{]}
\PYG{p}{,}\PYG{+w}{ }\PYG{p}{[}\PYG{l+m+mi}{960}\PYG{p}{,}\PYG{+w}{ }\PYG{l+m+mi}{540}\PYG{p}{]}
\PYG{p}{,}\PYG{+w}{ }\PYG{p}{[}\PYG{l+m+mi}{840}\PYG{p}{,}\PYG{+w}{ }\PYG{l+m+mi}{525}\PYG{p}{]}
\PYG{p}{,}\PYG{+w}{ }\PYG{p}{[}\PYG{l+m+mi}{832}\PYG{p}{,}\PYG{+w}{ }\PYG{l+m+mi}{624}\PYG{p}{]}
\PYG{p}{,}\PYG{+w}{ }\PYG{p}{[}\PYG{l+m+mi}{800}\PYG{p}{,}\PYG{+w}{ }\PYG{l+m+mi}{600}\PYG{p}{]}
\PYG{p}{,}\PYG{+w}{ }\PYG{p}{[}\PYG{l+m+mi}{800}\PYG{p}{,}\PYG{+w}{ }\PYG{l+m+mi}{512}\PYG{p}{]}
\PYG{p}{,}\PYG{+w}{ }\PYG{p}{[}\PYG{l+m+mi}{720}\PYG{p}{,}\PYG{+w}{ }\PYG{l+m+mi}{450}\PYG{p}{]}
\PYG{p}{,}\PYG{+w}{ }\PYG{p}{[}\PYG{l+m+mi}{720}\PYG{p}{,}\PYG{+w}{ }\PYG{l+m+mi}{400}\PYG{p}{]}
\PYG{p}{,}\PYG{+w}{ }\PYG{p}{[}\PYG{l+m+mi}{700}\PYG{p}{,}\PYG{+w}{ }\PYG{l+m+mi}{525}\PYG{p}{]}
\PYG{p}{]}
\end{sphinxVerbatim}

\sphinxAtStartPar
\sphinxstylestrong{See Also:}

\sphinxAtStartPar
\sphinxcode{\sphinxupquote{GetCurrentScreenResolution()}}, \sphinxcode{\sphinxupquote{gVideoWidth()}}, \sphinxcode{\sphinxupquote{gVideoHeight()}},     \sphinxcode{\sphinxupquote{GetDrivers()}}

\index{GetWorkingDirectory@\spxentry{GetWorkingDirectory}}\ignorespaces 

\subsection{GetWorkingDirectory()}
\label{\detokenize{reference/peblenvironment:getworkingdirectory}}\label{\detokenize{reference/peblenvironment:index-28}}
\sphinxAtStartPar
\sphinxstyleemphasis{Returns the current working directory}

\sphinxAtStartPar
\sphinxstylestrong{Description:}

\sphinxAtStartPar
Returns the current working directory path. This is the directory from which PEBL is currently executing and where relative file paths are resolved.

\sphinxAtStartPar
\sphinxstylestrong{Usage:}

\begin{sphinxVerbatim}[commandchars=\\\{\}]
\PYG{n+nf}{GetWorkingDirectory}\PYG{p}{(}\PYG{p}{)}
\end{sphinxVerbatim}

\sphinxAtStartPar
\sphinxstylestrong{Example:}

\begin{sphinxVerbatim}[commandchars=\\\{\}]
\PYG{n+nv}{cwd}\PYG{+w}{ }\PYG{o}{\PYGZlt{}\PYGZhy{}}\PYG{+w}{ }\PYG{n+nf}{GetWorkingDirectory}\PYG{p}{(}\PYG{p}{)}
\PYG{n+nf}{Print}\PYG{p}{(}\PYG{l+s+s2}{\PYGZdq{}Current directory: \PYGZdq{}}\PYG{+w}{ }\PYG{o}{+}\PYG{+w}{ }\PYG{n+nv}{cwd}\PYG{p}{)}
\end{sphinxVerbatim}

\sphinxAtStartPar
\sphinxstylestrong{See Also:}

\sphinxAtStartPar
\sphinxcode{\sphinxupquote{SetWorkingDirectory()}}, \sphinxcode{\sphinxupquote{GetHomeDirectory()}}, \sphinxcode{\sphinxupquote{GetDirectoryListing()}}

\index{IsAnyKeyDown@\spxentry{IsAnyKeyDown}}\ignorespaces 

\subsection{IsAnyKeyDown()}
\label{\detokenize{reference/peblenvironment:isanykeydown}}\label{\detokenize{reference/peblenvironment:index-29}}
\sphinxAtStartPar
\sphinxstyleemphasis{Determines whether any key is down.}

\sphinxAtStartPar
\sphinxstylestrong{Description:}

\begin{sphinxVerbatim}[commandchars=\\\{\}]
IsAnyKeyDown()
\end{sphinxVerbatim}

\sphinxAtStartPar
\sphinxstylestrong{Usage:}

\begin{sphinxVerbatim}[commandchars=\\\{\}]
\PYG{n+nf}{IsAnyKeyDown}\PYG{p}{(}\PYG{p}{)}
\end{sphinxVerbatim}

\index{IsAudioOut@\spxentry{IsAudioOut}}\ignorespaces 

\subsection{IsAudioOut()}
\label{\detokenize{reference/peblenvironment:isaudioout}}\label{\detokenize{reference/peblenvironment:index-30}}
\sphinxAtStartPar
\sphinxstylestrong{Description:}

\sphinxAtStartPar
Tests whether \sphinxcode{\sphinxupquote{\textless{}variant\textgreater{}}} is a AudioOut stream.

\sphinxAtStartPar
\sphinxstylestrong{Usage:}

\begin{sphinxVerbatim}[commandchars=\\\{\}]
\PYG{n+nf}{IsAudioOut}\PYG{p}{(}\PYG{o}{\PYGZlt{}}\PYG{n+nv}{variant}\PYG{o}{\PYGZgt{}}\PYG{p}{)}
\end{sphinxVerbatim}

\sphinxAtStartPar
\sphinxstylestrong{Example:}

\begin{sphinxVerbatim}[commandchars=\\\{\}]
\PYG{k}{if}\PYG{p}{(}\PYG{n+nf}{IsAudioOut}\PYG{p}{(}\PYG{n+nv}{x}\PYG{p}{)}\PYG{p}{)}
\PYG{p}{\PYGZob{}}
\PYG{+w}{ }\PYG{n+nf}{Play}\PYG{p}{(}\PYG{n+nv}{x}\PYG{p}{)}
\PYG{p}{\PYGZcb{}}
\end{sphinxVerbatim}

\sphinxAtStartPar
\sphinxstylestrong{See Also:}

\sphinxAtStartPar
\sphinxcode{\sphinxupquote{IsColor()}}, \sphinxcode{\sphinxupquote{IsImage()}},   \sphinxcode{\sphinxupquote{IsInteger()}}, \sphinxcode{\sphinxupquote{IsFileStream()}}, \sphinxcode{\sphinxupquote{IsFloat()}},   \sphinxcode{\sphinxupquote{IsFont()}}, \sphinxcode{\sphinxupquote{IsLabel()}}, \sphinxcode{\sphinxupquote{IsList()}},   \sphinxcode{\sphinxupquote{IsNumber()}}, \sphinxcode{\sphinxupquote{IsString()}}, \sphinxcode{\sphinxupquote{IsTextBox()}},   \sphinxcode{\sphinxupquote{IsWidget()}}

\index{IsCanvas@\spxentry{IsCanvas}}\ignorespaces 

\subsection{IsCanvas()}
\label{\detokenize{reference/peblenvironment:iscanvas}}\label{\detokenize{reference/peblenvironment:index-31}}
\sphinxAtStartPar
\sphinxstylestrong{Description:}

\sphinxAtStartPar
Tests whether \sphinxcode{\sphinxupquote{\textless{}variant\textgreater{}}} is a Canvas widget.

\sphinxAtStartPar
\sphinxstylestrong{Usage:}

\begin{sphinxVerbatim}[commandchars=\\\{\}]
\PYG{n+nf}{IsCanvas}\PYG{p}{(}\PYG{o}{\PYGZlt{}}\PYG{n+nv}{variant}\PYG{o}{\PYGZgt{}}\PYG{p}{)}
\end{sphinxVerbatim}

\sphinxAtStartPar
\sphinxstylestrong{Example:}

\begin{sphinxVerbatim}[commandchars=\\\{\}]
\PYG{k}{if}\PYG{p}{(}\PYG{n+nf}{IsCanvas}\PYG{p}{(}\PYG{n+nv}{x}\PYG{p}{)}
\PYG{p}{\PYGZob{}}
\PYG{+w}{   }\PYG{n+nf}{SetPixel}\PYG{p}{(}\PYG{n+nv}{x}\PYG{p}{,}\PYG{l+m+mi}{10}\PYG{p}{,}\PYG{l+m+mi}{10}\PYG{p}{,}\PYG{n+nf}{MakeColor}\PYG{p}{(}\PYG{l+s+s2}{\PYGZdq{}red\PYGZdq{}}\PYG{p}{)}\PYG{p}{)}
\PYG{p}{\PYGZcb{}}
\end{sphinxVerbatim}

\sphinxAtStartPar
\sphinxstylestrong{See Also:}

\sphinxAtStartPar
\sphinxcode{\sphinxupquote{IsAudioOut()}}, \sphinxcode{\sphinxupquote{IsImage()}},   \sphinxcode{\sphinxupquote{IsInteger()}}, \sphinxcode{\sphinxupquote{IsFileStream()}}, \sphinxcode{\sphinxupquote{IsFloat()}},   \sphinxcode{\sphinxupquote{IsFont()}}, \sphinxcode{\sphinxupquote{IsLabel()}}, \sphinxcode{\sphinxupquote{IsList()}},   \sphinxcode{\sphinxupquote{IsNumber()}}, \sphinxcode{\sphinxupquote{IsString()}}, \sphinxcode{\sphinxupquote{IsTextBox()}}, \sphinxcode{\sphinxupquote{IsText()}}   \sphinxcode{\sphinxupquote{IsWidget()}}, \sphinxcode{\sphinxupquote{IsWindow()}}

\index{IsColor@\spxentry{IsColor}}\ignorespaces 

\subsection{IsColor()}
\label{\detokenize{reference/peblenvironment:iscolor}}\label{\detokenize{reference/peblenvironment:index-32}}
\sphinxAtStartPar
\sphinxstylestrong{Description:}

\sphinxAtStartPar
Tests whether \sphinxcode{\sphinxupquote{\textless{}variant\textgreater{}}} is a Color.

\sphinxAtStartPar
\sphinxstylestrong{Usage:}

\begin{sphinxVerbatim}[commandchars=\\\{\}]
\PYG{n+nf}{IsColor}\PYG{p}{(}\PYG{o}{\PYGZlt{}}\PYG{n+nv}{variant}\PYG{o}{\PYGZgt{}}\PYG{p}{)}
\end{sphinxVerbatim}

\sphinxAtStartPar
\sphinxstylestrong{Example:}

\begin{sphinxVerbatim}[commandchars=\\\{\}]
\PYG{k}{if}\PYG{p}{(}\PYG{n+nf}{IsColor}\PYG{p}{(}\PYG{n+nv}{x}\PYG{p}{)}
\PYG{p}{\PYGZob{}}
\PYG{+w}{ }\PYG{n+nv+vg}{gWin}\PYG{+w}{ }\PYG{o}{\PYGZlt{}\PYGZhy{}}\PYG{+w}{ }\PYG{n+nf}{MakeWindow}\PYG{p}{(}\PYG{n+nv}{x}\PYG{p}{)}
\PYG{p}{\PYGZcb{}}
\end{sphinxVerbatim}

\sphinxAtStartPar
\sphinxstylestrong{See Also:}

\sphinxAtStartPar
\sphinxcode{\sphinxupquote{IsAudioOut()}}, \sphinxcode{\sphinxupquote{IsImage()}},   \sphinxcode{\sphinxupquote{IsInteger()}}, \sphinxcode{\sphinxupquote{IsFileStream()}}, \sphinxcode{\sphinxupquote{IsFloat()}},   \sphinxcode{\sphinxupquote{IsFont()}}, \sphinxcode{\sphinxupquote{IsLabel()}}, \sphinxcode{\sphinxupquote{IsList()}},   \sphinxcode{\sphinxupquote{IsNumber()}}, \sphinxcode{\sphinxupquote{IsString()}}, \sphinxcode{\sphinxupquote{IsTextBox()}},   \sphinxcode{\sphinxupquote{IsWidget()}}, \sphinxcode{\sphinxupquote{IsWindow()}}

\index{IsCustomObject@\spxentry{IsCustomObject}}\ignorespaces 

\subsection{IsCustomObject()}
\label{\detokenize{reference/peblenvironment:iscustomobject}}\label{\detokenize{reference/peblenvironment:index-33}}
\sphinxAtStartPar
\sphinxstyleemphasis{Tests whether object is a custom object.}

\sphinxAtStartPar
\sphinxstylestrong{Description:}

\sphinxAtStartPar
Tests whether \sphinxcode{\sphinxupquote{\textless{}variant\textgreater{}}} is a Custom object (created with \sphinxcode{\sphinxupquote{MakeCustomObject}}.) Return 1 if so, 0 if not.

\sphinxAtStartPar
\sphinxstylestrong{Usage:}

\begin{sphinxVerbatim}[commandchars=\\\{\}]
\PYG{n+nf}{IsCustomObject}\PYG{p}{(}\PYG{o}{\PYGZlt{}}\PYG{n+nv}{obj}\PYG{o}{\PYGZgt{}}\PYG{p}{)}
\end{sphinxVerbatim}

\sphinxAtStartPar
\sphinxstylestrong{Example:}

\begin{sphinxVerbatim}[commandchars=\\\{\}]
\PYG{k}{if}\PYG{p}{(}\PYG{n+nf}{IsCustomObject}\PYG{p}{(}\PYG{n+nv}{obj}\PYG{p}{)}
\PYG{p}{\PYGZob{}}
\PYG{+w}{   }\PYG{n+nf}{MoveObject}\PYG{p}{(}\PYG{n+nv}{obj}\PYG{p}{,}\PYG{n+nv}{x}\PYG{p}{,}\PYG{n+nv}{y}\PYG{p}{)}
\PYG{p}{\PYGZcb{}}\PYG{+w}{ }\PYG{k}{else}\PYG{+w}{ }\PYG{p}{\PYGZob{}}
\PYG{+w}{   }\PYG{n+nf}{Move}\PYG{p}{(}\PYG{n+nv}{obj}\PYG{p}{,}\PYG{n+nv}{x}\PYG{p}{,}\PYG{n+nv}{y}\PYG{p}{)}
\PYG{p}{\PYGZcb{}}
\end{sphinxVerbatim}

\sphinxAtStartPar
\sphinxstylestrong{See Also:}

\sphinxAtStartPar
\sphinxcode{\sphinxupquote{IsAudioOut()}}, \sphinxcode{\sphinxupquote{IsImage()}},   \sphinxcode{\sphinxupquote{IsInteger()}}, \sphinxcode{\sphinxupquote{IsFileStream()}}, \sphinxcode{\sphinxupquote{IsFloat()}},   \sphinxcode{\sphinxupquote{IsFont()}}, \sphinxcode{\sphinxupquote{IsLabel()}}, \sphinxcode{\sphinxupquote{IsList()}},   \sphinxcode{\sphinxupquote{IsNumber()}}, \sphinxcode{\sphinxupquote{IsString()}}, \sphinxcode{\sphinxupquote{IsTextBox()}}, \sphinxcode{\sphinxupquote{IsText()}}   \sphinxcode{\sphinxupquote{IsWidget()}}, \sphinxcode{\sphinxupquote{IsWindow()}}

\index{IsDirectory@\spxentry{IsDirectory}}\ignorespaces 

\subsection{IsDirectory()}
\label{\detokenize{reference/peblenvironment:isdirectory}}\label{\detokenize{reference/peblenvironment:index-34}}
\sphinxAtStartPar
\sphinxstyleemphasis{Checks whether a file is a directory}

\sphinxAtStartPar
\sphinxstylestrong{Description:}

\sphinxAtStartPar
Determines whether a named path is a directory.  Returns 1 if it exists and is a directory, and 0 otherwise.

\sphinxAtStartPar
\sphinxstylestrong{Usage:}

\begin{sphinxVerbatim}[commandchars=\\\{\}]
\PYG{n+nf}{IsDirectory}\PYG{p}{(}\PYG{o}{\PYGZlt{}}\PYG{n+nv}{path}\PYG{o}{\PYGZgt{}}\PYG{p}{)}
\end{sphinxVerbatim}

\sphinxAtStartPar
\sphinxstylestrong{Example:}

\begin{sphinxVerbatim}[commandchars=\\\{\}]
\PYG{n+nv}{filename}\PYG{+w}{ }\PYG{o}{\PYGZlt{}\PYGZhy{}}\PYG{+w}{ }\PYG{l+s+s2}{\PYGZdq{}data\PYGZhy{}\PYGZdq{}}\PYG{o}{+}\PYG{n+nv+vg}{gSubNum}\PYG{o}{+}\PYG{l+s+s2}{\PYGZdq{}.csv\PYGZdq{}}
\PYG{+w}{ }\PYG{n+nv}{exists}\PYG{+w}{ }\PYG{o}{\PYGZlt{}\PYGZhy{}}\PYG{+w}{  }\PYG{n+nf}{FileExists}\PYG{p}{(}\PYG{n+nv}{filename}\PYG{p}{)}
\PYG{+w}{  }\PYG{k}{if}\PYG{p}{(}\PYG{n+nv}{exists}\PYG{p}{)}
\PYG{+w}{   }\PYG{p}{\PYGZob{}}
\PYG{+w}{    }\PYG{n+nv}{out}\PYG{+w}{ }\PYG{o}{\PYGZlt{}\PYGZhy{}}\PYG{+w}{    }\PYG{n+nf}{IsDirectory}\PYG{p}{(}\PYG{n+nv}{filename}\PYG{p}{)}
\PYG{+w}{    }\PYG{n+nf}{Print}\PYG{p}{(}\PYG{n+nv}{out}\PYG{p}{)}
\PYG{+w}{   }\PYG{p}{\PYGZcb{}}
\end{sphinxVerbatim}

\sphinxAtStartPar
\sphinxstylestrong{See Also:}

\sphinxAtStartPar
\sphinxcode{\sphinxupquote{GetDirectoryListing()}}, \sphinxcode{\sphinxupquote{FileExists()}},       \sphinxcode{\sphinxupquote{IsDirectory()}},            \sphinxcode{\sphinxupquote{MakeDirectory()}}

\index{IsFileStream@\spxentry{IsFileStream}}\ignorespaces 

\subsection{IsFileStream()}
\label{\detokenize{reference/peblenvironment:isfilestream}}\label{\detokenize{reference/peblenvironment:index-35}}
\sphinxAtStartPar
\sphinxstylestrong{Description:}

\sphinxAtStartPar
Tests whether \sphinxcode{\sphinxupquote{\textless{}variant\textgreater{}}} is a FileStream object.

\sphinxAtStartPar
\sphinxstylestrong{Usage:}

\begin{sphinxVerbatim}[commandchars=\\\{\}]
\PYG{n+nf}{IsFileStream}\PYG{p}{(}\PYG{o}{\PYGZlt{}}\PYG{n+nv}{variant}\PYG{o}{\PYGZgt{}}\PYG{p}{)}
\end{sphinxVerbatim}

\sphinxAtStartPar
\sphinxstylestrong{Example:}

\begin{sphinxVerbatim}[commandchars=\\\{\}]
\PYG{k}{if}\PYG{p}{(}\PYG{n+nf}{IsFileStream}\PYG{p}{(}\PYG{n+nv}{x}\PYG{p}{)}\PYG{p}{)}
\PYG{p}{\PYGZob{}}
\PYG{+w}{ }\PYG{n+nf}{Print}\PYG{p}{(}\PYG{n+nf}{FileReadWord}\PYG{p}{(}\PYG{n+nv}{x}\PYG{p}{)}
\PYG{p}{\PYGZcb{}}
\end{sphinxVerbatim}

\sphinxAtStartPar
\sphinxstylestrong{See Also:}

\sphinxAtStartPar
\sphinxcode{\sphinxupquote{IsAudioOut()}}, \sphinxcode{\sphinxupquote{IsColor()}},   \sphinxcode{\sphinxupquote{IsImage()}}, \sphinxcode{\sphinxupquote{IsInteger()}}, \sphinxcode{\sphinxupquote{IsFloat()}},   \sphinxcode{\sphinxupquote{IsFont()}}, \sphinxcode{\sphinxupquote{IsLabel()}}, \sphinxcode{\sphinxupquote{IsList()}},   \sphinxcode{\sphinxupquote{IsNumber()}}, \sphinxcode{\sphinxupquote{IsString()}}, \sphinxcode{\sphinxupquote{IsTextBox()}},   \sphinxcode{\sphinxupquote{IsWidget()}}

\index{IsFloat@\spxentry{IsFloat}}\ignorespaces 

\subsection{IsFloat()}
\label{\detokenize{reference/peblenvironment:isfloat}}\label{\detokenize{reference/peblenvironment:index-36}}
\sphinxAtStartPar
\sphinxstylestrong{Description:}

\sphinxAtStartPar
Tests whether \sphinxcode{\sphinxupquote{\textless{}variant\textgreater{}}} is a floating\sphinxhyphen{}point   value. Note that floating\sphinxhyphen{}point can represent integers with great   precision, so that a number appearing as an integer can still be a   float.

\sphinxAtStartPar
\sphinxstylestrong{Usage:}

\begin{sphinxVerbatim}[commandchars=\\\{\}]
\PYG{n+nf}{IsFloat}\PYG{p}{(}\PYG{o}{\PYGZlt{}}\PYG{n+nv}{variant}\PYG{o}{\PYGZgt{}}\PYG{p}{)}
\end{sphinxVerbatim}

\sphinxAtStartPar
\sphinxstylestrong{Example:}

\begin{sphinxVerbatim}[commandchars=\\\{\}]
\PYG{n+nv}{x}\PYG{+w}{ }\PYG{o}{\PYGZlt{}\PYGZhy{}}\PYG{+w}{ }\PYG{l+m+mi}{44}
\PYG{n+nv}{y}\PYG{+w}{ }\PYG{o}{\PYGZlt{}\PYGZhy{}}\PYG{+w}{ }\PYG{l+m+mf}{23.5}
\PYG{n+nv}{z}\PYG{+w}{ }\PYG{o}{\PYGZlt{}\PYGZhy{}}\PYG{+w}{ }\PYG{l+m+mf}{6.5}
\PYG{n+nv}{test}\PYG{+w}{ }\PYG{o}{\PYGZlt{}\PYGZhy{}}\PYG{+w}{ }\PYG{n+nv}{x}\PYG{+w}{ }\PYG{o}{+}\PYG{+w}{ }\PYG{n+nv}{y}\PYG{+w}{ }\PYG{o}{+}\PYG{+w}{ }\PYG{n+nv}{z}

\PYG{n+nf}{IsFloat}\PYG{p}{(}\PYG{n+nv}{x}\PYG{p}{)}\PYG{+w}{           }\PYG{c+c1}{\PYGZsh{} false}
\PYG{n+nf}{IsFloat}\PYG{p}{(}\PYG{n+nv}{y}\PYG{p}{)}\PYG{+w}{           }\PYG{c+c1}{\PYGZsh{} true}
\PYG{n+nf}{IsFloat}\PYG{p}{(}\PYG{n+nv}{z}\PYG{p}{)}\PYG{+w}{           }\PYG{c+c1}{\PYGZsh{} true}
\PYG{n+nf}{IsFloat}\PYG{p}{(}\PYG{n+nv}{test}\PYG{p}{)}\PYG{+w}{        }\PYG{c+c1}{\PYGZsh{} true}
\end{sphinxVerbatim}

\sphinxAtStartPar
\sphinxstylestrong{See Also:}

\sphinxAtStartPar
\sphinxcode{\sphinxupquote{IsAudioOut()}}, \sphinxcode{\sphinxupquote{IsColor()}},   \sphinxcode{\sphinxupquote{IsImage()}}, \sphinxcode{\sphinxupquote{IsInteger()}}, \sphinxcode{\sphinxupquote{IsFileStream()}},   \sphinxcode{\sphinxupquote{IsFont()}}, \sphinxcode{\sphinxupquote{IsLabel()}}, \sphinxcode{\sphinxupquote{IsList()}},   \sphinxcode{\sphinxupquote{IsNumber()}}, \sphinxcode{\sphinxupquote{IsString()}}, \sphinxcode{\sphinxupquote{IsTextBox()}},   \sphinxcode{\sphinxupquote{IsWidget()}}

\index{IsFont@\spxentry{IsFont}}\ignorespaces 

\subsection{IsFont()}
\label{\detokenize{reference/peblenvironment:isfont}}\label{\detokenize{reference/peblenvironment:index-37}}
\sphinxAtStartPar
\sphinxstylestrong{Description:}

\sphinxAtStartPar
Tests whether \sphinxcode{\sphinxupquote{\textless{}variant\textgreater{}}} is a Font object.

\sphinxAtStartPar
\sphinxstylestrong{Usage:}

\begin{sphinxVerbatim}[commandchars=\\\{\}]
\PYG{n+nf}{IsFont}\PYG{p}{(}\PYG{o}{\PYGZlt{}}\PYG{n+nv}{variant}\PYG{o}{\PYGZgt{}}\PYG{p}{)}
\end{sphinxVerbatim}

\sphinxAtStartPar
\sphinxstylestrong{Example:}

\begin{sphinxVerbatim}[commandchars=\\\{\}]
\PYG{k}{if}\PYG{p}{(}\PYG{n+nf}{IsFont}\PYG{p}{(}\PYG{n+nv}{x}\PYG{p}{)}\PYG{p}{)}
\PYG{p}{\PYGZob{}}
\PYG{+w}{ }\PYG{n+nv}{y}\PYG{+w}{ }\PYG{o}{\PYGZlt{}\PYGZhy{}}\PYG{+w}{ }\PYG{n+nf}{MakeLabel}\PYG{p}{(}\PYG{l+s+s2}{\PYGZdq{}stimulus\PYGZdq{}}\PYG{p}{,}\PYG{+w}{ }\PYG{n+nv}{x}\PYG{p}{)}
\PYG{p}{\PYGZcb{}}
\end{sphinxVerbatim}

\sphinxAtStartPar
\sphinxstylestrong{See Also:}

\sphinxAtStartPar
\sphinxcode{\sphinxupquote{IsAudioOut()}}, \sphinxcode{\sphinxupquote{IsColor()}},   \sphinxcode{\sphinxupquote{IsImage()}}, \sphinxcode{\sphinxupquote{IsInteger()}}, \sphinxcode{\sphinxupquote{IsFileStream()}},   \sphinxcode{\sphinxupquote{IsFloat()}}, \sphinxcode{\sphinxupquote{IsLabel()}}, \sphinxcode{\sphinxupquote{IsList()}},   \sphinxcode{\sphinxupquote{IsNumber()}}, \sphinxcode{\sphinxupquote{IsString()}}, \sphinxcode{\sphinxupquote{IsTextBox()}},   \sphinxcode{\sphinxupquote{IsWidget()}}

\index{IsImage@\spxentry{IsImage}}\ignorespaces 

\subsection{IsImage()}
\label{\detokenize{reference/peblenvironment:isimage}}\label{\detokenize{reference/peblenvironment:index-38}}
\sphinxAtStartPar
\sphinxstylestrong{Description:}

\sphinxAtStartPar
Tests whether \sphinxcode{\sphinxupquote{\textless{}variant\textgreater{}}} is an Image.

\sphinxAtStartPar
\sphinxstylestrong{Usage:}

\begin{sphinxVerbatim}[commandchars=\\\{\}]
\PYG{n+nf}{IsImage}\PYG{p}{(}\PYG{o}{\PYGZlt{}}\PYG{n+nv}{variant}\PYG{o}{\PYGZgt{}}\PYG{p}{)}
\end{sphinxVerbatim}

\sphinxAtStartPar
\sphinxstylestrong{Example:}

\begin{sphinxVerbatim}[commandchars=\\\{\}]
\PYG{k}{if}\PYG{p}{(}\PYG{n+nf}{IsImage}\PYG{p}{(}\PYG{n+nv}{x}\PYG{p}{)}\PYG{p}{)}
\PYG{p}{\PYGZob{}}
\PYG{+w}{ }\PYG{n+nf}{AddObject}\PYG{p}{(}\PYG{n+nv+vg}{gWin}\PYG{p}{,}\PYG{+w}{ }\PYG{n+nv}{x}\PYG{p}{)}
\PYG{p}{\PYGZcb{}}
\end{sphinxVerbatim}

\sphinxAtStartPar
\sphinxstylestrong{See Also:}

\sphinxAtStartPar
\sphinxcode{\sphinxupquote{IsAudioOut()}}, \sphinxcode{\sphinxupquote{IsColor()}},   \sphinxcode{\sphinxupquote{IsInteger()}}, \sphinxcode{\sphinxupquote{IsFileStream()}}, \sphinxcode{\sphinxupquote{IsFloat()}},   \sphinxcode{\sphinxupquote{IsFont()}}, \sphinxcode{\sphinxupquote{IsLabel()}}, \sphinxcode{\sphinxupquote{IsList()}},   \sphinxcode{\sphinxupquote{IsNumber()}}, \sphinxcode{\sphinxupquote{IsString()}}, \sphinxcode{\sphinxupquote{IsTextBox()}},   \sphinxcode{\sphinxupquote{IsWidget()}}

\index{IsInteger@\spxentry{IsInteger}}\ignorespaces 

\subsection{IsInteger()}
\label{\detokenize{reference/peblenvironment:isinteger}}\label{\detokenize{reference/peblenvironment:index-39}}
\sphinxAtStartPar
\sphinxstylestrong{Description:}

\sphinxAtStartPar
Tests whether \sphinxcode{\sphinxupquote{\textless{}variant\textgreater{}}} is an integer type.   Note: a number represented internally as a floating\sphinxhyphen{}point type whose   is an integer will return false.  Floating\sphinxhyphen{}point numbers can be   converted to internally\sphinxhyphen{} represented integers with the   \sphinxcode{\sphinxupquote{ToInteger()}} or \sphinxcode{\sphinxupquote{Round()}} commands.

\sphinxAtStartPar
\sphinxstylestrong{Usage:}

\begin{sphinxVerbatim}[commandchars=\\\{\}]
\PYG{n+nf}{IsInteger}\PYG{p}{(}\PYG{o}{\PYGZlt{}}\PYG{n+nv}{variant}\PYG{o}{\PYGZgt{}}\PYG{p}{)}
\end{sphinxVerbatim}

\sphinxAtStartPar
\sphinxstylestrong{Example:}

\begin{sphinxVerbatim}[commandchars=\\\{\}]
\PYG{n+nv}{x}\PYG{+w}{ }\PYG{o}{\PYGZlt{}\PYGZhy{}}\PYG{+w}{ }\PYG{l+m+mi}{44}
\PYG{n+nv}{y}\PYG{+w}{ }\PYG{o}{\PYGZlt{}\PYGZhy{}}\PYG{+w}{ }\PYG{l+m+mf}{23.5}
\PYG{n+nv}{z}\PYG{+w}{ }\PYG{o}{\PYGZlt{}\PYGZhy{}}\PYG{+w}{ }\PYG{l+m+mf}{6.5}
\PYG{n+nv}{test}\PYG{+w}{ }\PYG{o}{\PYGZlt{}\PYGZhy{}}\PYG{+w}{ }\PYG{n+nv}{x}\PYG{+w}{ }\PYG{o}{+}\PYG{+w}{ }\PYG{n+nv}{y}\PYG{+w}{ }\PYG{o}{+}\PYG{+w}{ }\PYG{n+nv}{z}

\PYG{n+nf}{IsInteger}\PYG{p}{(}\PYG{n+nv}{x}\PYG{p}{)}\PYG{+w}{         }\PYG{c+c1}{\PYGZsh{} true}
\PYG{n+nf}{IsInteger}\PYG{p}{(}\PYG{n+nv}{y}\PYG{p}{)}\PYG{+w}{         }\PYG{c+c1}{\PYGZsh{} false}
\PYG{n+nf}{IsInteger}\PYG{p}{(}\PYG{n+nv}{z}\PYG{p}{)}\PYG{+w}{         }\PYG{c+c1}{\PYGZsh{} false}
\PYG{n+nf}{IsInteger}\PYG{p}{(}\PYG{n+nv}{test}\PYG{p}{)}\PYG{+w}{              }\PYG{c+c1}{\PYGZsh{} false}
\end{sphinxVerbatim}

\sphinxAtStartPar
\sphinxstylestrong{See Also:}

\sphinxAtStartPar
\sphinxcode{\sphinxupquote{IsAudioOut()}}, \sphinxcode{\sphinxupquote{IsColor()}},   \sphinxcode{\sphinxupquote{IsImage()}}, \sphinxcode{\sphinxupquote{IsFileStream()}}, \sphinxcode{\sphinxupquote{IsFloat()}},   \sphinxcode{\sphinxupquote{IsFont()}}, \sphinxcode{\sphinxupquote{IsLabel()}}, \sphinxcode{\sphinxupquote{IsList()}},   \sphinxcode{\sphinxupquote{IsNumber()}}, \sphinxcode{\sphinxupquote{IsString()}}, \sphinxcode{\sphinxupquote{IsTextBox()}},   \sphinxcode{\sphinxupquote{IsWidget()}}

\index{IsKeyDown@\spxentry{IsKeyDown}}\ignorespaces 

\subsection{IsKeyDown()}
\label{\detokenize{reference/peblenvironment:iskeydown}}\label{\detokenize{reference/peblenvironment:index-40}}
\sphinxAtStartPar
\sphinxstylestrong{Description:}

\sphinxAtStartPar
\sphinxstylestrong{See Also:}

\sphinxAtStartPar
\sphinxcode{\sphinxupquote{IsKeyUp()}}

\index{IsKeyUp@\spxentry{IsKeyUp}}\ignorespaces 

\subsection{IsKeyUp()}
\label{\detokenize{reference/peblenvironment:iskeyup}}\label{\detokenize{reference/peblenvironment:index-41}}
\sphinxAtStartPar
\sphinxstylestrong{Description:}

\sphinxAtStartPar
\sphinxstylestrong{See Also:}

\sphinxAtStartPar
\sphinxcode{\sphinxupquote{IsKeyDown()}}

\index{IsLabel@\spxentry{IsLabel}}\ignorespaces 

\subsection{IsLabel()}
\label{\detokenize{reference/peblenvironment:islabel}}\label{\detokenize{reference/peblenvironment:index-42}}
\sphinxAtStartPar
\sphinxstylestrong{Description:}

\sphinxAtStartPar
Tests whether \sphinxcode{\sphinxupquote{\textless{}variant\textgreater{}}} is a text Label object.

\sphinxAtStartPar
\sphinxstylestrong{Usage:}

\begin{sphinxVerbatim}[commandchars=\\\{\}]
\PYG{n+nf}{IsLabel}\PYG{p}{(}\PYG{o}{\PYGZlt{}}\PYG{n+nv}{variant}\PYG{o}{\PYGZgt{}}\PYG{p}{)}
\end{sphinxVerbatim}

\sphinxAtStartPar
\sphinxstylestrong{Example:}

\begin{sphinxVerbatim}[commandchars=\\\{\}]
\PYG{k}{if}\PYG{p}{(}\PYG{n+nf}{IsLabel}\PYG{p}{(}\PYG{n+nv}{x}\PYG{p}{)}
\PYG{p}{\PYGZob{}}
\PYG{+w}{ }\PYG{n+nv}{text}\PYG{+w}{ }\PYG{o}{\PYGZlt{}\PYGZhy{}}\PYG{+w}{ }\PYG{n+nf}{GetText}\PYG{p}{(}\PYG{n+nv}{x}\PYG{p}{)}
\PYG{p}{\PYGZcb{}}
\end{sphinxVerbatim}

\sphinxAtStartPar
\sphinxstylestrong{See Also:}

\sphinxAtStartPar
\sphinxcode{\sphinxupquote{IsAudioOut()}}, \sphinxcode{\sphinxupquote{IsColor()}},   \sphinxcode{\sphinxupquote{IsImage()}}, \sphinxcode{\sphinxupquote{IsInteger()}}, \sphinxcode{\sphinxupquote{IsFileStream()}},   \sphinxcode{\sphinxupquote{IsFloat()}}, \sphinxcode{\sphinxupquote{IsFont()}}, \sphinxcode{\sphinxupquote{IsList()}},   \sphinxcode{\sphinxupquote{IsNumber()}}, \sphinxcode{\sphinxupquote{IsString()}}, \sphinxcode{\sphinxupquote{IsTextBox()}},   \sphinxcode{\sphinxupquote{IsWidget()}}

\index{IsList@\spxentry{IsList}}\ignorespaces 

\subsection{IsList()}
\label{\detokenize{reference/peblenvironment:islist}}\label{\detokenize{reference/peblenvironment:index-43}}
\sphinxAtStartPar
\sphinxstylestrong{Description:}

\sphinxAtStartPar
Tests whether \sphinxcode{\sphinxupquote{\textless{}variant\textgreater{}}} is a PEBL list.

\sphinxAtStartPar
\sphinxstylestrong{Usage:}

\begin{sphinxVerbatim}[commandchars=\\\{\}]
\PYG{n+nf}{IsList}\PYG{p}{(}\PYG{o}{\PYGZlt{}}\PYG{n+nv}{variant}\PYG{o}{\PYGZgt{}}\PYG{p}{)}
\end{sphinxVerbatim}

\sphinxAtStartPar
\sphinxstylestrong{Example:}

\begin{sphinxVerbatim}[commandchars=\\\{\}]
\PYG{k}{if}\PYG{p}{(}\PYG{n+nf}{IsList}\PYG{p}{(}\PYG{n+nv}{x}\PYG{p}{)}\PYG{p}{)}
\PYG{p}{\PYGZob{}}
\PYG{+w}{ }\PYG{k}{loop}\PYG{p}{(}\PYG{n+nv}{item}\PYG{p}{,}\PYG{+w}{ }\PYG{n+nv}{x}\PYG{p}{)}
\PYG{+w}{ }\PYG{p}{\PYGZob{}}
\PYG{+w}{  }\PYG{n+nf}{Print}\PYG{p}{(}\PYG{n+nv}{item}\PYG{p}{)}
\PYG{+w}{ }\PYG{p}{\PYGZcb{}}
\PYG{p}{\PYGZcb{}}
\end{sphinxVerbatim}

\sphinxAtStartPar
\sphinxstylestrong{See Also:}

\sphinxAtStartPar
\sphinxcode{\sphinxupquote{IsAudioOut()}}, \sphinxcode{\sphinxupquote{IsColor()}},   \sphinxcode{\sphinxupquote{IsImage()}}, \sphinxcode{\sphinxupquote{IsInteger()}}, \sphinxcode{\sphinxupquote{IsFileStream()}},   \sphinxcode{\sphinxupquote{IsFloat()}}, \sphinxcode{\sphinxupquote{IsFont()}}, \sphinxcode{\sphinxupquote{IsLabel()}},   \sphinxcode{\sphinxupquote{IsNumber()}}, \sphinxcode{\sphinxupquote{IsString()}}, \sphinxcode{\sphinxupquote{IsTextBox()}},   \sphinxcode{\sphinxupquote{IsWidget()}}

\index{IsNumber@\spxentry{IsNumber}}\ignorespaces 

\subsection{IsNumber()}
\label{\detokenize{reference/peblenvironment:isnumber}}\label{\detokenize{reference/peblenvironment:index-44}}
\sphinxAtStartPar
\sphinxstylestrong{Description:}

\sphinxAtStartPar
Tests whether \sphinxcode{\sphinxupquote{\textless{}variant\textgreater{}}} is a number, either a               floating\sphinxhyphen{}point or an integer.

\sphinxAtStartPar
\sphinxstylestrong{Usage:}

\begin{sphinxVerbatim}[commandchars=\\\{\}]
\PYG{n+nf}{IsNumber}\PYG{p}{(}\PYG{o}{\PYGZlt{}}\PYG{n+nv}{variant}\PYG{o}{\PYGZgt{}}\PYG{p}{)}
\end{sphinxVerbatim}

\sphinxAtStartPar
\sphinxstylestrong{Example:}

\begin{sphinxVerbatim}[commandchars=\\\{\}]
\PYG{k}{if}\PYG{p}{(}\PYG{n+nf}{IsNumber}\PYG{p}{(}\PYG{n+nv}{x}\PYG{p}{)}\PYG{p}{)}
\PYG{p}{\PYGZob{}}
\PYG{+w}{ }\PYG{n+nf}{Print}\PYG{p}{(}\PYG{n+nf}{Sequence}\PYG{p}{(}\PYG{n+nv}{x}\PYG{p}{,}\PYG{+w}{ }\PYG{n+nv}{x}\PYG{o}{+}\PYG{l+m+mi}{10}\PYG{p}{,}\PYG{+w}{ }\PYG{l+m+mi}{1}\PYG{p}{)}\PYG{p}{)}
\PYG{p}{\PYGZcb{}}
\end{sphinxVerbatim}

\sphinxAtStartPar
\sphinxstylestrong{See Also:}

\sphinxAtStartPar
\sphinxcode{\sphinxupquote{IsAudioOut()}}, \sphinxcode{\sphinxupquote{IsColor()}},   \sphinxcode{\sphinxupquote{IsImage()}}, \sphinxcode{\sphinxupquote{IsInteger()}}, \sphinxcode{\sphinxupquote{IsFileStream()}},   \sphinxcode{\sphinxupquote{IsFloat()}}, \sphinxcode{\sphinxupquote{IsFont()}}, \sphinxcode{\sphinxupquote{IsLabel()}},   \sphinxcode{\sphinxupquote{IsList()}}, \sphinxcode{\sphinxupquote{IsString()}}, \sphinxcode{\sphinxupquote{IsTextBox()}},   \sphinxcode{\sphinxupquote{IsWidget()}}

\index{IsShape@\spxentry{IsShape}}\ignorespaces 

\subsection{IsShape()}
\label{\detokenize{reference/peblenvironment:isshape}}\label{\detokenize{reference/peblenvironment:index-45}}
\sphinxAtStartPar
\sphinxstylestrong{Description:}

\sphinxAtStartPar
Tests whether \sphinxcode{\sphinxupquote{\textless{}variant\textgreater{}}} is a drawable   shape, such as a circle, square rectangle, line, bezier curve, or   polygon.

\sphinxAtStartPar
\sphinxstylestrong{Usage:}

\begin{sphinxVerbatim}[commandchars=\\\{\}]
\PYG{n+nf}{IsShape}\PYG{p}{(}\PYG{o}{\PYGZlt{}}\PYG{n+nv}{variant}\PYG{o}{\PYGZgt{}}\PYG{p}{)}
\end{sphinxVerbatim}

\sphinxAtStartPar
\sphinxstylestrong{Example:}

\begin{sphinxVerbatim}[commandchars=\\\{\}]
\PYG{k}{if}\PYG{p}{(}\PYG{n+nf}{IsShape}\PYG{p}{(}\PYG{n+nv}{x}\PYG{p}{)}\PYG{p}{)}
\PYG{p}{\PYGZob{}}
\PYG{+w}{  }\PYG{n+nf}{Move}\PYG{p}{(}\PYG{n+nv}{x}\PYG{p}{,}\PYG{l+m+mi}{300}\PYG{p}{,}\PYG{l+m+mi}{300}\PYG{p}{)}
\PYG{p}{\PYGZcb{}}
\end{sphinxVerbatim}

\sphinxAtStartPar
\sphinxstylestrong{See Also:}

\sphinxAtStartPar
\sphinxcode{\sphinxupquote{Square()}}, \sphinxcode{\sphinxupquote{Circle()}},   \sphinxcode{\sphinxupquote{Rectangle()}}, \sphinxcode{\sphinxupquote{Line()}}, \sphinxcode{\sphinxupquote{Bezier()}}, \sphinxcode{\sphinxupquote{Polygon()}}   \sphinxcode{\sphinxupquote{IsAudioOut()}}, \sphinxcode{\sphinxupquote{IsColor()}},   \sphinxcode{\sphinxupquote{IsImage()}}, \sphinxcode{\sphinxupquote{IsInteger()}}, \sphinxcode{\sphinxupquote{IsFileStream()}},   \sphinxcode{\sphinxupquote{IsFloat()}}, \sphinxcode{\sphinxupquote{IsFont()}}, \sphinxcode{\sphinxupquote{IsLabel()}},   \sphinxcode{\sphinxupquote{IsList()}}, \sphinxcode{\sphinxupquote{IsNumber()}}, \sphinxcode{\sphinxupquote{IsString()}},   \sphinxcode{\sphinxupquote{IsTextBox()}}, \sphinxcode{\sphinxupquote{IsWindow()}}

\index{IsString@\spxentry{IsString}}\ignorespaces 

\subsection{IsString()}
\label{\detokenize{reference/peblenvironment:isstring}}\label{\detokenize{reference/peblenvironment:index-46}}
\sphinxAtStartPar
\sphinxstylestrong{Description:}

\sphinxAtStartPar
Tests whether \sphinxcode{\sphinxupquote{\textless{}variant\textgreater{}}} is a text string.

\sphinxAtStartPar
\sphinxstylestrong{Usage:}

\begin{sphinxVerbatim}[commandchars=\\\{\}]
\PYG{n+nf}{IsString}\PYG{p}{(}\PYG{o}{\PYGZlt{}}\PYG{n+nv}{variant}\PYG{o}{\PYGZgt{}}\PYG{p}{)}
\end{sphinxVerbatim}

\sphinxAtStartPar
\sphinxstylestrong{Example:}

\begin{sphinxVerbatim}[commandchars=\\\{\}]
\PYG{k}{if}\PYG{p}{(}\PYG{n+nf}{IsString}\PYG{p}{(}\PYG{n+nv}{x}\PYG{p}{)}\PYG{p}{)}
\PYG{p}{\PYGZob{}}
\PYG{+w}{ }\PYG{n+nv}{tb}\PYG{+w}{ }\PYG{o}{\PYGZlt{}\PYGZhy{}}\PYG{+w}{ }\PYG{n+nf}{MakeTextBox}\PYG{p}{(}\PYG{n+nv}{x}\PYG{p}{,}\PYG{+w}{ }\PYG{l+m+mi}{100}\PYG{p}{,}\PYG{+w}{ }\PYG{l+m+mi}{100}\PYG{p}{)}
\PYG{p}{\PYGZcb{}}
\end{sphinxVerbatim}

\sphinxAtStartPar
\sphinxstylestrong{See Also:}

\sphinxAtStartPar
\sphinxcode{\sphinxupquote{IsText()}}        \sphinxcode{\sphinxupquote{IsAudioOut()}}, \sphinxcode{\sphinxupquote{IsColor()}}, \sphinxcode{\sphinxupquote{IsImage()}}, \sphinxcode{\sphinxupquote{IsInteger()}},                \sphinxcode{\sphinxupquote{IsFileStream()}}, \sphinxcode{\sphinxupquote{IsFloat()}}, \sphinxcode{\sphinxupquote{IsFont()}}, \sphinxcode{\sphinxupquote{IsLabel()}},                 \sphinxcode{\sphinxupquote{IsList()}}, \sphinxcode{\sphinxupquote{IsNumber()}}, \sphinxcode{\sphinxupquote{IsTextBox()}}, \sphinxcode{\sphinxupquote{IsWidget()}}

\index{IsText@\spxentry{IsText}}\ignorespaces 

\subsection{IsText()}
\label{\detokenize{reference/peblenvironment:istext}}\label{\detokenize{reference/peblenvironment:index-47}}
\sphinxAtStartPar
\sphinxstylestrong{Description:}

\sphinxAtStartPar
Tests whether \sphinxcode{\sphinxupquote{\textless{}variant\textgreater{}}} is a text string.   Same as IsString().

\sphinxAtStartPar
\sphinxstylestrong{Usage:}

\begin{sphinxVerbatim}[commandchars=\\\{\}]
\PYG{n+nf}{IsString}\PYG{p}{(}\PYG{o}{\PYGZlt{}}\PYG{n+nv}{variant}\PYG{o}{\PYGZgt{}}\PYG{p}{)}
\end{sphinxVerbatim}

\sphinxAtStartPar
\sphinxstylestrong{Example:}

\begin{sphinxVerbatim}[commandchars=\\\{\}]
\PYG{k}{if}\PYG{p}{(}\PYG{n+nf}{IsText}\PYG{p}{(}\PYG{n+nv}{x}\PYG{p}{)}\PYG{p}{)}
\PYG{p}{\PYGZob{}}
\PYG{+w}{ }\PYG{n+nv}{tb}\PYG{+w}{ }\PYG{o}{\PYGZlt{}\PYGZhy{}}\PYG{+w}{ }\PYG{n+nf}{MakeTextBox}\PYG{p}{(}\PYG{n+nv}{x}\PYG{p}{,}\PYG{+w}{ }\PYG{l+m+mi}{100}\PYG{p}{,}\PYG{+w}{ }\PYG{l+m+mi}{100}\PYG{p}{)}
\PYG{p}{\PYGZcb{}}
\end{sphinxVerbatim}

\sphinxAtStartPar
\sphinxstylestrong{See Also:}

\sphinxAtStartPar
\sphinxcode{\sphinxupquote{IsString()}}      \sphinxcode{\sphinxupquote{IsAudioOut()}}, \sphinxcode{\sphinxupquote{IsColor()}}, \sphinxcode{\sphinxupquote{IsImage()}}, \sphinxcode{\sphinxupquote{IsInteger()}},                \sphinxcode{\sphinxupquote{IsFileStream()}}, \sphinxcode{\sphinxupquote{IsFloat()}}, \sphinxcode{\sphinxupquote{IsFont()}}, \sphinxcode{\sphinxupquote{IsLabel()}},                 \sphinxcode{\sphinxupquote{IsList()}}, \sphinxcode{\sphinxupquote{IsNumber()}}, \sphinxcode{\sphinxupquote{IsTextBox()}}, \sphinxcode{\sphinxupquote{IsWidget()}}

\index{IsTextBox@\spxentry{IsTextBox}}\ignorespaces 

\subsection{IsTextBox()}
\label{\detokenize{reference/peblenvironment:istextbox}}\label{\detokenize{reference/peblenvironment:index-48}}
\sphinxAtStartPar
\sphinxstylestrong{Description:}

\sphinxAtStartPar
Tests whether \sphinxcode{\sphinxupquote{\textless{}variant\textgreater{}}} is a TextBox Object

\sphinxAtStartPar
\sphinxstylestrong{Usage:}

\begin{sphinxVerbatim}[commandchars=\\\{\}]
\PYG{n+nf}{IsTextBox}\PYG{p}{(}\PYG{o}{\PYGZlt{}}\PYG{n+nv}{variant}\PYG{o}{\PYGZgt{}}\PYG{p}{)}
\end{sphinxVerbatim}

\sphinxAtStartPar
\sphinxstylestrong{Example:}

\begin{sphinxVerbatim}[commandchars=\\\{\}]
\PYG{k}{if}\PYG{p}{(}\PYG{n+nf}{IsTextBox}\PYG{p}{(}\PYG{n+nv}{x}\PYG{p}{)}\PYG{p}{)}
\PYG{p}{\PYGZob{}}
\PYG{+w}{ }\PYG{n+nf}{Print}\PYG{p}{(}\PYG{n+nf}{GetText}\PYG{p}{(}\PYG{n+nv}{x}\PYG{p}{)}\PYG{p}{)}
\PYG{p}{\PYGZcb{}}
\end{sphinxVerbatim}

\sphinxAtStartPar
\sphinxstylestrong{See Also:}

\sphinxAtStartPar
\sphinxcode{\sphinxupquote{IsAudioOut()}}, \sphinxcode{\sphinxupquote{IsColor()}},   \sphinxcode{\sphinxupquote{IsImage()}}, \sphinxcode{\sphinxupquote{IsInteger()}}, \sphinxcode{\sphinxupquote{IsFileStream()}},   \sphinxcode{\sphinxupquote{IsFloat()}}, \sphinxcode{\sphinxupquote{IsFont()}}, \sphinxcode{\sphinxupquote{IsLabel()}},   \sphinxcode{\sphinxupquote{IsList()}}, \sphinxcode{\sphinxupquote{IsNumber()}}, \sphinxcode{\sphinxupquote{IsString()}},   \sphinxcode{\sphinxupquote{IsWidget()}}

\index{IsWidget@\spxentry{IsWidget}}\ignorespaces 

\subsection{IsWidget()}
\label{\detokenize{reference/peblenvironment:iswidget}}\label{\detokenize{reference/peblenvironment:index-49}}
\sphinxAtStartPar
\sphinxstylestrong{Description:}

\sphinxAtStartPar
Tests whether \sphinxcode{\sphinxupquote{\textless{}variant\textgreater{}}} is any kind of a widget object              (image, label, or textbox).

\sphinxAtStartPar
\sphinxstylestrong{Usage:}

\begin{sphinxVerbatim}[commandchars=\\\{\}]
\PYG{n+nf}{IsWidget}\PYG{p}{(}\PYG{o}{\PYGZlt{}}\PYG{n+nv}{variant}\PYG{o}{\PYGZgt{}}\PYG{p}{)}
\end{sphinxVerbatim}

\sphinxAtStartPar
\sphinxstylestrong{Example:}

\begin{sphinxVerbatim}[commandchars=\\\{\}]
\PYG{k}{if}\PYG{p}{(}\PYG{n+nf}{IsWidget}\PYG{p}{(}\PYG{n+nv}{x}\PYG{p}{)}\PYG{p}{)}
\PYG{p}{\PYGZob{}}
\PYG{+w}{ }\PYG{n+nf}{Move}\PYG{p}{(}\PYG{n+nv}{x}\PYG{p}{,}\PYG{+w}{ }\PYG{l+m+mi}{200}\PYG{p}{,}\PYG{l+m+mi}{300}\PYG{p}{)}
\PYG{p}{\PYGZcb{}}
\end{sphinxVerbatim}

\sphinxAtStartPar
\sphinxstylestrong{See Also:}

\sphinxAtStartPar
\sphinxcode{\sphinxupquote{IsAudioOut()}}, \sphinxcode{\sphinxupquote{IsColor()}},   \sphinxcode{\sphinxupquote{IsImage()}}, \sphinxcode{\sphinxupquote{IsInteger()}}, \sphinxcode{\sphinxupquote{IsFileStream()}},   \sphinxcode{\sphinxupquote{IsFloat()}}, \sphinxcode{\sphinxupquote{IsFont()}}, \sphinxcode{\sphinxupquote{IsLabel()}},   \sphinxcode{\sphinxupquote{IsList()}}, \sphinxcode{\sphinxupquote{IsNumber()}}, \sphinxcode{\sphinxupquote{IsString()}},   \sphinxcode{\sphinxupquote{IsTextBox()}}

\index{IsWindow@\spxentry{IsWindow}}\ignorespaces 

\subsection{IsWindow()}
\label{\detokenize{reference/peblenvironment:iswindow}}\label{\detokenize{reference/peblenvironment:index-50}}
\sphinxAtStartPar
\sphinxstylestrong{Description:}

\sphinxAtStartPar
Tests whether \sphinxcode{\sphinxupquote{\textless{}variant\textgreater{}}} is a window.

\sphinxAtStartPar
\sphinxstylestrong{Usage:}

\begin{sphinxVerbatim}[commandchars=\\\{\}]
\PYG{n+nf}{IsWindow}\PYG{p}{(}\PYG{o}{\PYGZlt{}}\PYG{n+nv}{variant}\PYG{o}{\PYGZgt{}}\PYG{p}{)}
\end{sphinxVerbatim}

\sphinxAtStartPar
\sphinxstylestrong{Example:}

\begin{sphinxVerbatim}[commandchars=\\\{\}]
\PYG{k}{if}\PYG{p}{(}\PYG{n+nf}{IsWindow}\PYG{p}{(}\PYG{n+nv}{x}\PYG{p}{)}\PYG{p}{)}
\PYG{p}{\PYGZob{}}
\PYG{+w}{  }\PYG{n+nf}{AddObject}\PYG{p}{(}\PYG{n+nv}{y}\PYG{p}{,}\PYG{n+nv}{x}\PYG{p}{)}
\PYG{p}{\PYGZcb{}}
\end{sphinxVerbatim}

\sphinxAtStartPar
\sphinxstylestrong{See Also:}

\sphinxAtStartPar
\sphinxcode{\sphinxupquote{IsAudioOut()}}, \sphinxcode{\sphinxupquote{IsColor()}},   \sphinxcode{\sphinxupquote{IsImage()}}, \sphinxcode{\sphinxupquote{IsInteger()}}, \sphinxcode{\sphinxupquote{IsFileStream()}},   \sphinxcode{\sphinxupquote{IsFloat()}}, \sphinxcode{\sphinxupquote{IsFont()}}, \sphinxcode{\sphinxupquote{IsLabel()}},   \sphinxcode{\sphinxupquote{IsList()}}, \sphinxcode{\sphinxupquote{IsNumber()}}, \sphinxcode{\sphinxupquote{IsString()}},   \sphinxcode{\sphinxupquote{IsTextBox()}}

\index{LaunchFile@\spxentry{LaunchFile}}\ignorespaces 

\subsection{LaunchFile()}
\label{\detokenize{reference/peblenvironment:launchfile}}\label{\detokenize{reference/peblenvironment:index-51}}
\sphinxAtStartPar
\sphinxstyleemphasis{Launches a file using platform\sphinxhyphen{}specific handlers}

\sphinxAtStartPar
\sphinxstylestrong{Description:}

\sphinxAtStartPar
Launch a specified file or URI with a platform\sphinxhyphen{}specific handler.

\sphinxAtStartPar
\sphinxstylestrong{Usage:}

\begin{sphinxVerbatim}[commandchars=\\\{\}]
\PYG{n+nf}{LaunchFile}\PYG{p}{(}\PYG{l+s+s2}{\PYGZdq{}filename\PYGZdq{}}\PYG{p}{)}
\end{sphinxVerbatim}

\sphinxAtStartPar
\sphinxstylestrong{See Also:}

\sphinxAtStartPar
\sphinxcode{\sphinxupquote{SystemCall()}}

\index{MakeDirectory@\spxentry{MakeDirectory}}\ignorespaces 

\subsection{MakeDirectory()}
\label{\detokenize{reference/peblenvironment:makedirectory}}\label{\detokenize{reference/peblenvironment:index-52}}
\sphinxAtStartPar
\sphinxstyleemphasis{Creates a directory in path}

\sphinxAtStartPar
\sphinxstylestrong{Description:}

\sphinxAtStartPar
Creates a directory with a particular name. It will have no effect of the directory already exists.

\sphinxAtStartPar
\sphinxstylestrong{Usage:}

\begin{sphinxVerbatim}[commandchars=\\\{\}]
\PYG{n+nf}{FileExists}\PYG{p}{(}\PYG{o}{\PYGZlt{}}\PYG{n+nv}{path}\PYG{o}{\PYGZgt{}}\PYG{p}{)}
\end{sphinxVerbatim}

\sphinxAtStartPar
\sphinxstylestrong{Example:}

\begin{sphinxVerbatim}[commandchars=\\\{\}]
\PYG{c+c1}{\PYGZsh{}create data subdirectory + subject\PYGZhy{}specific directory}
\PYG{+w}{ }\PYG{n+nf}{MakeDirectory}\PYG{p}{(}\PYG{l+s+s2}{\PYGZdq{}data\PYGZdq{}}\PYG{p}{)}
\PYG{+w}{ }\PYG{n+nf}{MakeDirectory}\PYG{p}{(}\PYG{l+s+s2}{\PYGZdq{}data/\PYGZdq{}}\PYG{o}{+}\PYG{n+nv+vg}{gsubnum}\PYG{p}{)}
\PYG{+w}{ }\PYG{n+nv}{filename}\PYG{+w}{ }\PYG{o}{\PYGZlt{}\PYGZhy{}}\PYG{+w}{ }\PYG{l+s+s2}{\PYGZdq{}data/\PYGZdq{}}\PYG{o}{+}\PYG{n+nv+vg}{gsubnum}\PYG{o}{+}\PYG{l+s+s2}{\PYGZdq{}/output.csv\PYGZdq{}}
\end{sphinxVerbatim}

\sphinxAtStartPar
\sphinxstylestrong{See Also:}

\sphinxAtStartPar
\sphinxcode{\sphinxupquote{GetDirectoryListing()}}, \sphinxcode{\sphinxupquote{FileExists()}},       \sphinxcode{\sphinxupquote{IsDirectory()}},            \sphinxcode{\sphinxupquote{MakeDirectory()}}

\index{OpenJoystick@\spxentry{OpenJoystick}}\ignorespaces 

\subsection{OpenJoystick()}
\label{\detokenize{reference/peblenvironment:openjoystick}}\label{\detokenize{reference/peblenvironment:index-53}}
\sphinxAtStartPar
\sphinxstyleemphasis{Gets a joystick object}

\sphinxAtStartPar
\sphinxstylestrong{Description:}

\sphinxAtStartPar
This opens an available joystick, as specified by its index.  The returned object can then be used in to access the state of the joystick.  It takes an integer argument, and for the most part, if you have a single joystick attached to your system, you will use OpenJoystick(1).  If you want to use a second joystick, use OpenJoystick(2), and so on.

\sphinxAtStartPar
\sphinxstylestrong{See Also:}

\sphinxAtStartPar
GetNumJoysticks(), OpenJoystick(), GetNumJoystickAxes() GetNumJoystickBalls(), GetNumJoystickButtons(), GetNumJoystickHats() GetJoystickAxisState(), GetJoystickHatState(), GetJoystickButtonState()

\index{PlayMovie@\spxentry{PlayMovie}}\ignorespaces 

\subsection{PlayMovie()}
\label{\detokenize{reference/peblenvironment:playmovie}}\label{\detokenize{reference/peblenvironment:index-54}}
\sphinxAtStartPar
\sphinxstylestrong{*(CURRENTLY NOT WORKING)*}
\sphinxstyleemphasis{Plays a movie until its end}

\sphinxAtStartPar
\sphinxstylestrong{Description:}

\sphinxAtStartPar
Plays the movie (or other multimedia file) loaded via either the LoadMovie or LoadAudioFile function.  Note that this functionality uses a  different underlying system than the sound playing functions PlayBackground and PlayForeground, and they are not interchangeable.

\sphinxAtStartPar
\sphinxstylestrong{Usage:}

\begin{sphinxVerbatim}[commandchars=\\\{\}]
\PYG{n+nf}{PlayMovie}\PYG{p}{(}\PYG{n+nv}{movie}\PYG{p}{)}
\end{sphinxVerbatim}

\sphinxAtStartPar
\sphinxstylestrong{Example:}

\begin{sphinxVerbatim}[commandchars=\\\{\}]
\PYG{n+nv}{movie}\PYG{+w}{ }\PYG{o}{\PYGZlt{}\PYGZhy{}}\PYG{+w}{ }\PYG{n+nf}{LoadMovie}\PYG{p}{(}\PYG{l+s+s2}{\PYGZdq{}movie.avi\PYGZdq{}}\PYG{p}{,}\PYG{n+nv+vg}{gWin}\PYG{p}{,}\PYG{l+m+mi}{640}\PYG{p}{,}\PYG{l+m+mi}{480}\PYG{p}{)}
\PYG{+w}{   }\PYG{n+nf}{PrintProperties}\PYG{p}{(}\PYG{n+nv}{movie}\PYG{p}{)}
\PYG{+w}{   }\PYG{n+nf}{Move}\PYG{p}{(}\PYG{n+nv}{movie}\PYG{p}{,}\PYG{l+m+mi}{20}\PYG{p}{,}\PYG{l+m+mi}{20}\PYG{p}{)}
\PYG{+w}{   }\PYG{n+nv}{movie.volume}\PYG{+w}{ }\PYG{o}{\PYGZlt{}\PYGZhy{}}\PYG{+w}{ }\PYG{l+m+mf}{.1}
\PYG{+w}{   }\PYG{n+nv}{status}\PYG{+w}{ }\PYG{o}{\PYGZlt{}\PYGZhy{}}\PYG{+w}{ }\PYG{n+nf}{EasyLabel}\PYG{p}{(}\PYG{l+s+s2}{\PYGZdq{}Demo Movie Player\PYGZdq{}}\PYG{p}{,}\PYG{l+m+mi}{300}\PYG{p}{,}\PYG{l+m+mi}{25}\PYG{p}{,}\PYG{n+nv+vg}{gWin}\PYG{p}{,}\PYG{l+m+mi}{22}\PYG{p}{)}
\PYG{+w}{   }\PYG{n+nf}{Draw}\PYG{p}{(}\PYG{p}{)}
\PYG{+w}{   }\PYG{n+nf}{PlayMovie}\PYG{p}{(}\PYG{n+nv}{movie}\PYG{p}{)}
\end{sphinxVerbatim}

\sphinxAtStartPar
\sphinxstylestrong{See Also:}

\sphinxAtStartPar
\sphinxcode{\sphinxupquote{LoadAudioFile()}}, \sphinxcode{\sphinxupquote{LoadMovie()}}, \sphinxcode{\sphinxupquote{StartPlayback()}}, \sphinxcode{\sphinxupquote{PausePlayback()}}

\index{RegisterEvent@\spxentry{RegisterEvent}}\ignorespaces 

\subsection{RegisterEvent()}
\label{\detokenize{reference/peblenvironment:registerevent}}\label{\detokenize{reference/peblenvironment:index-55}}
\sphinxAtStartPar
\sphinxstyleemphasis{Registers events to trigger based on particular conditions}

\sphinxAtStartPar
\sphinxstylestrong{Description:}

\sphinxAtStartPar
Adds an event to the event loop.  This function is currently experimental, and its usage may change in future versions of PEBL.

\sphinxAtStartPar
\sphinxstylestrong{Usage:}

\begin{sphinxVerbatim}[commandchars=\\\{\}]
\PYG{c+c1}{\PYGZsh{}\PYGZsh{} shows a way to generate custom WaitForMouseButton}
\PYG{n+nf}{RegisterEvent}\PYG{p}{(}\PYG{l+s+s2}{\PYGZdq{}\PYGZlt{}MOUSE\PYGZus{}BUTTON\PYGZus{}PRESS\PYGZgt{}\PYGZdq{}}\PYG{p}{,}\PYG{l+m+mi}{1}\PYG{p}{,}\PYG{l+m+mi}{1}\PYG{p}{,}\PYG{l+s+s2}{\PYGZdq{}\PYGZlt{}EQUAL\PYGZgt{}\PYGZdq{}}\PYG{p}{,}\PYG{l+s+s2}{\PYGZdq{}\PYGZdq{}}\PYG{p}{,}\PYG{+w}{ }\PYG{p}{[}\PYG{p}{]}\PYG{p}{)}
\PYG{n+nv}{out}\PYG{+w}{ }\PYG{o}{\PYGZlt{}\PYGZhy{}}\PYG{+w}{   }\PYG{n+nf}{StartEventLoop}\PYG{p}{(}\PYG{p}{)}
\PYG{n+nf}{ClearEventLoop}\PYG{p}{(}\PYG{p}{)}
\end{sphinxVerbatim}

\sphinxAtStartPar
\sphinxstylestrong{See Also:}

\sphinxAtStartPar
\sphinxcode{\sphinxupquote{ClearEventLoop()}}, \sphinxcode{\sphinxupquote{StartEventLoop()}}

\index{SetMouseCursorPosition@\spxentry{SetMouseCursorPosition}}\ignorespaces 

\subsection{SetMouseCursorPosition()}
\label{\detokenize{reference/peblenvironment:setmousecursorposition}}\label{\detokenize{reference/peblenvironment:index-56}}
\sphinxAtStartPar
\sphinxstylestrong{Description:}

\sphinxAtStartPar
Sets the current x,y coordinates of the mouse   pointer, ‘warping’ the mouse to that location immediately

\sphinxAtStartPar
\sphinxstylestrong{Usage:}

\begin{sphinxVerbatim}[commandchars=\\\{\}]
\PYG{n+nf}{SetMouseCursorPosition}\PYG{p}{(}\PYG{o}{\PYGZlt{}}\PYG{n+nv}{x}\PYG{o}{\PYGZgt{}}\PYG{p}{,}\PYG{o}{\PYGZlt{}}\PYG{n+nv}{y}\PYG{o}{\PYGZgt{}}\PYG{p}{)}
\end{sphinxVerbatim}

\sphinxAtStartPar
\sphinxstylestrong{Example:}

\begin{sphinxVerbatim}[commandchars=\\\{\}]
\PYG{c+c1}{\PYGZsh{}\PYGZsh{}Set mouse to center of screen:}
\PYG{+w}{  }\PYG{n+nf}{SetMouseCursorPosition}\PYG{p}{(}\PYG{n+nv+vg}{gVideoWidth}\PYG{o}{/}\PYG{l+m+mi}{2}\PYG{p}{,}
\PYG{+w}{                         }\PYG{n+nv+vg}{gVideoHeight}\PYG{o}{/}\PYG{l+m+mi}{2}\PYG{p}{)}
\end{sphinxVerbatim}

\sphinxAtStartPar
\sphinxstylestrong{See Also:}

\sphinxAtStartPar
\sphinxcode{\sphinxupquote{ShowCursor()}}, \sphinxcode{\sphinxupquote{WaitForMouseButton()}},   \sphinxcode{\sphinxupquote{SetMouseCursorPosition()}}, \sphinxcode{\sphinxupquote{GetMouseCursorPosition()}}

\index{SetWorkingDirectory@\spxentry{SetWorkingDirectory}}\ignorespaces 

\subsection{SetWorkingDirectory()}
\label{\detokenize{reference/peblenvironment:setworkingdirectory}}\label{\detokenize{reference/peblenvironment:index-57}}
\sphinxAtStartPar
\sphinxstyleemphasis{Changes the current working directory}

\sphinxAtStartPar
\sphinxstylestrong{Description:}

\sphinxAtStartPar
Changes the current working directory to the specified path. This affects how relative file paths are resolved in subsequent file operations.

\sphinxAtStartPar
\sphinxstylestrong{Usage:}

\begin{sphinxVerbatim}[commandchars=\\\{\}]
\PYG{n+nf}{SetWorkingDirectory}\PYG{p}{(}\PYG{o}{\PYGZlt{}}\PYG{n+nv}{path}\PYG{o}{\PYGZgt{}}\PYG{p}{)}
\end{sphinxVerbatim}

\sphinxAtStartPar
\sphinxstylestrong{Example:}

\begin{sphinxVerbatim}[commandchars=\\\{\}]
\PYG{n+nf}{SetWorkingDirectory}\PYG{p}{(}\PYG{l+s+s2}{\PYGZdq{}./data\PYGZdq{}}\PYG{p}{)}
\PYG{n+nf}{Print}\PYG{p}{(}\PYG{n+nf}{GetWorkingDirectory}\PYG{p}{(}\PYG{p}{)}\PYG{p}{)}\PYG{+w}{  }\PYG{c+c1}{\PYGZsh{}\PYGZsh{}Shows new directory}

\PYG{c+c1}{\PYGZsh{}\PYGZsh{}Now relative paths work from ./data}
\PYG{n+nv}{file}\PYG{+w}{ }\PYG{o}{\PYGZlt{}\PYGZhy{}}\PYG{+w}{ }\PYG{n+nf}{FileOpenRead}\PYG{p}{(}\PYG{l+s+s2}{\PYGZdq{}output.csv\PYGZdq{}}\PYG{p}{)}
\end{sphinxVerbatim}

\sphinxAtStartPar
\sphinxstylestrong{See Also:}

\sphinxAtStartPar
\sphinxcode{\sphinxupquote{GetWorkingDirectory()}}, \sphinxcode{\sphinxupquote{GetHomeDirectory()}}, \sphinxcode{\sphinxupquote{FileExists()}}

\index{ShowCursor@\spxentry{ShowCursor}}\ignorespaces 

\subsection{ShowCursor()}
\label{\detokenize{reference/peblenvironment:showcursor}}\label{\detokenize{reference/peblenvironment:index-58}}
\sphinxAtStartPar
\sphinxstyleemphasis{Hides or show mouse cursor.}

\sphinxAtStartPar
\sphinxstylestrong{Description:}

\sphinxAtStartPar
Hides or shows the mouse cursor.  Currently, the   mouse is not used, but on some systems in some configurations, the   mouse cursor shows up.  Calling \sphinxcode{\sphinxupquote{ShowCursor(0)}} will turn off the   cursor, and \sphinxcode{\sphinxupquote{ShowCursor(1)}} will turn it back on.  Be sure to turn it   on at the end of the experiment, or you may actually lose the cursor   for good.

\sphinxAtStartPar
\sphinxstylestrong{Usage:}

\begin{sphinxVerbatim}[commandchars=\\\{\}]
\PYG{n+nf}{ShowCursor}\PYG{p}{(}\PYG{o}{\PYGZlt{}}\PYG{n+nv}{value}\PYG{o}{\PYGZgt{}}\PYG{p}{)}
\end{sphinxVerbatim}

\sphinxAtStartPar
\sphinxstylestrong{Example:}

\begin{sphinxVerbatim}[commandchars=\\\{\}]
\PYG{n+nv}{window}\PYG{+w}{ }\PYG{o}{\PYGZlt{}\PYGZhy{}}\PYG{+w}{ }\PYG{n+nf}{MakeWindow}\PYG{p}{(}\PYG{p}{)}
\PYG{n+nf}{ShowCursor}\PYG{p}{(}\PYG{l+m+mi}{0}\PYG{p}{)}
\PYG{c+c1}{\PYGZsh{}\PYGZsh{} Do experiment here}
\PYG{c+c1}{\PYGZsh{}\PYGZsh{}}

\PYG{c+c1}{\PYGZsh{}\PYGZsh{} Turn mouse back on.}
\PYG{n+nf}{ShowCursor}\PYG{p}{(}\PYG{l+m+mi}{1}\PYG{p}{)}
\end{sphinxVerbatim}

\index{SignalFatalError@\spxentry{SignalFatalError}}\ignorespaces 

\subsection{SignalFatalError()}
\label{\detokenize{reference/peblenvironment:signalfatalerror}}\label{\detokenize{reference/peblenvironment:index-59}}
\sphinxAtStartPar
\sphinxstyleemphasis{Halts execution, printing out message}

\sphinxAtStartPar
\sphinxstylestrong{Description:}

\sphinxAtStartPar
Stops PEBL and prints \sphinxcode{\sphinxupquote{\textless{}message\textgreater{}}} to stderr. In addition, when possible, it will pop\sphinxhyphen{}up a window with the error message. Useful for type\sphinxhyphen{}checking in user\sphinxhyphen{}defined functions.  If you want to end an experiment directly, use \sphinxcode{\sphinxupquote{ExitQuietly}} instead.

\sphinxAtStartPar
\sphinxstylestrong{Usage:}

\begin{sphinxVerbatim}[commandchars=\\\{\}]
\PYG{n+nf}{SignalFatalError}\PYG{p}{(}\PYG{o}{\PYGZlt{}}\PYG{n+nv}{message}\PYG{o}{\PYGZgt{}}\PYG{p}{)}
\end{sphinxVerbatim}

\sphinxAtStartPar
\sphinxstylestrong{Example:}

\begin{sphinxVerbatim}[commandchars=\\\{\}]
\PYG{k}{If}\PYG{p}{(}\PYG{k}{not}\PYG{+w}{ }\PYG{n+nf}{IsList}\PYG{p}{(}\PYG{n+nv}{x}\PYG{p}{)}\PYG{p}{)}
\PYG{p}{\PYGZob{}}
\PYG{+w}{ }\PYG{n+nf}{SignalFatalError}\PYG{p}{(}\PYG{l+s+s2}{\PYGZdq{}Tried to frobnicate a List.\PYGZdq{}}\PYG{p}{)}
\PYG{p}{\PYGZcb{}}
\PYG{c+c1}{\PYGZsh{}\PYGZsh{}Prints out error message and}
\PYG{c+c1}{\PYGZsh{}\PYGZsh{}line/filename of function}
\end{sphinxVerbatim}

\sphinxAtStartPar
\sphinxstylestrong{See Also:}

\sphinxAtStartPar
\sphinxcode{\sphinxupquote{Print()}}, \sphinxcode{\sphinxupquote{ExitQuietly()}}

\index{StartEventLoop@\spxentry{StartEventLoop}}\ignorespaces 

\subsection{StartEventLoop()}
\label{\detokenize{reference/peblenvironment:starteventloop}}\label{\detokenize{reference/peblenvironment:index-60}}
\sphinxAtStartPar
\sphinxstyleemphasis{Starts the event loop}

\sphinxAtStartPar
\sphinxstylestrong{Description:}

\sphinxAtStartPar
Starts the event loop with currently\sphinxhyphen{}registered events.  This function is currently experimental, and its usage may change in future versions of PEBL.

\sphinxAtStartPar
\sphinxstylestrong{Usage:}

\begin{sphinxVerbatim}[commandchars=\\\{\}]
\PYG{c+c1}{\PYGZsh{}\PYGZsh{} shows a way to generate custom WaitForMouseButton}
\PYG{n+nf}{RegisterEvent}\PYG{p}{(}\PYG{l+s+s2}{\PYGZdq{}\PYGZlt{}MOUSE\PYGZus{}BUTTON\PYGZus{}PRESS\PYGZgt{}\PYGZdq{}}\PYG{p}{,}\PYG{l+m+mi}{1}\PYG{p}{,}\PYG{l+m+mi}{1}\PYG{p}{,}\PYG{l+s+s2}{\PYGZdq{}\PYGZlt{}EQUAL\PYGZgt{}\PYGZdq{}}\PYG{p}{,}\PYG{l+s+s2}{\PYGZdq{}\PYGZdq{}}\PYG{p}{,}\PYG{+w}{ }\PYG{p}{[}\PYG{p}{]}\PYG{p}{)}
\PYG{n+nv}{out}\PYG{+w}{ }\PYG{o}{\PYGZlt{}\PYGZhy{}}\PYG{+w}{   }\PYG{n+nf}{StartEventLoop}\PYG{p}{(}\PYG{p}{)}
\PYG{n+nf}{ClearEventLoop}\PYG{p}{(}\PYG{p}{)}
\end{sphinxVerbatim}

\sphinxAtStartPar
\sphinxstylestrong{See Also:}

\sphinxAtStartPar
\sphinxcode{\sphinxupquote{RegisterEvent()}}, \sphinxcode{\sphinxupquote{ClearEventLoop()}}

\index{SystemCall@\spxentry{SystemCall}}\ignorespaces 

\subsection{SystemCall()}
\label{\detokenize{reference/peblenvironment:systemcall}}\label{\detokenize{reference/peblenvironment:index-61}}
\sphinxAtStartPar
\sphinxstyleemphasis{Executes command in operating system}

\sphinxAtStartPar
\sphinxstylestrong{Description:}

\sphinxAtStartPar
Calls/runs another operating system command.  Can also be used to  launch another PEBL program.  Useful to check GetSystemType() before running.   Note that the output of a    command\sphinxhyphen{}line argument is generally not passed back into PEBL; just    the function’s return code, which is usually 0 on success or some    other number on failure (depending upon the type of failure).  Some    uses might include:

\sphinxAtStartPar
\sphinxstylestrong{Usage:}

\begin{sphinxVerbatim}[commandchars=\\\{\}]
\PYG{n+nf}{SystemCall}\PYG{p}{(}\PYG{l+s+s2}{\PYGZdq{}text\PYGZhy{}of\PYGZhy{}command\PYGZdq{}}\PYG{p}{)}
\PYG{n+nf}{SystemCall}\PYG{p}{(}\PYG{l+s+s2}{\PYGZdq{}text\PYGZhy{}of\PYGZhy{}command\PYGZdq{}}\PYG{p}{,}\PYG{l+s+s2}{\PYGZdq{}command\PYGZhy{}line\PYGZhy{}options\PYGZdq{}}\PYG{p}{)}
\end{sphinxVerbatim}

\sphinxAtStartPar
\sphinxstylestrong{Example:}

\begin{sphinxVerbatim}[commandchars=\\\{\}]
\PYG{k}{if}\PYG{p}{(}\PYG{n+nf}{GetSystemType}\PYG{p}{(}\PYG{p}{)}\PYG{+w}{ }\PYG{o}{==}\PYG{+w}{ }\PYG{l+s+s2}{\PYGZdq{}WINDOWS\PYGZdq{}}\PYG{p}{)}
\PYG{+w}{     }\PYG{p}{\PYGZob{}}
\PYG{+w}{       }\PYG{n+nv}{x}\PYG{+w}{ }\PYG{o}{\PYGZlt{}\PYGZhy{}}\PYG{+w}{ }\PYG{n+nf}{SystemCall}\PYG{p}{(}\PYG{l+s+s2}{\PYGZdq{}dir input.txt\PYGZdq{}}\PYG{p}{)}
\PYG{+w}{     }\PYG{p}{\PYGZcb{}}\PYG{+w}{ }\PYG{k}{else}\PYG{+w}{ }\PYG{p}{\PYGZob{}}
\PYG{+w}{       }\PYG{n+nv}{x}\PYG{+w}{ }\PYG{o}{\PYGZlt{}\PYGZhy{}}\PYG{+w}{ }\PYG{n+nf}{SystemCall}\PYG{p}{(}\PYG{l+s+s2}{\PYGZdq{}ls input.txt\PYGZdq{}}\PYG{p}{)}
\PYG{+w}{     }\PYG{p}{\PYGZcb{}}
\PYG{+w}{      }\PYG{k}{if}\PYG{p}{(}\PYG{n+nv}{x}\PYG{+w}{ }\PYG{o}{\PYGZlt{}\PYGZgt{}}\PYG{+w}{ }\PYG{l+m+mi}{0}\PYG{p}{)}
\PYG{+w}{      }\PYG{p}{\PYGZob{}}
\PYG{+w}{         }\PYG{n+nf}{SignalFatalError}\PYG{p}{(}\PYG{l+s+s2}{\PYGZdq{}Expected file [\PYGZdq{}}\PYG{o}{+}
\PYG{+w}{               }\PYG{l+s+s2}{\PYGZdq{}input.txt] does not exist\PYGZdq{}}\PYG{p}{)}
\PYG{+w}{      }\PYG{p}{\PYGZcb{}}
\end{sphinxVerbatim}

\sphinxAtStartPar
\sphinxstylestrong{See Also:}

\sphinxAtStartPar
\sphinxcode{\sphinxupquote{GetSystemType()}}

\index{SystemCallUpdate@\spxentry{SystemCallUpdate}}\ignorespaces 

\subsection{SystemCallUpdate()}
\label{\detokenize{reference/peblenvironment:systemcallupdate}}\label{\detokenize{reference/peblenvironment:index-62}}
\sphinxAtStartPar
\sphinxstyleemphasis{Executes an OS command with real\sphinxhyphen{}time output updates}

\sphinxAtStartPar
\sphinxstylestrong{Description:}

\sphinxAtStartPar
Calls an operating system command similar to SystemCall(), but with support for receiving output updates during execution. This is useful for long\sphinxhyphen{}running commands where you want to see progress.

\sphinxAtStartPar
\sphinxstylestrong{Usage:}

\begin{sphinxVerbatim}[commandchars=\\\{\}]
\PYG{n+nf}{SystemCallUpdate}\PYG{p}{(}\PYG{o}{\PYGZlt{}}\PYG{n+nv}{command}\PYG{o}{\PYGZgt{}}\PYG{p}{)}
\PYG{n+nf}{SystemCallUpdate}\PYG{p}{(}\PYG{o}{\PYGZlt{}}\PYG{n+nv}{command}\PYG{o}{\PYGZgt{}}\PYG{p}{,}\PYG{+w}{ }\PYG{o}{\PYGZlt{}}\PYG{n+nv}{arguments}\PYG{o}{\PYGZgt{}}\PYG{p}{)}
\end{sphinxVerbatim}

\sphinxAtStartPar
\sphinxstylestrong{Example:}

\begin{sphinxVerbatim}[commandchars=\\\{\}]
\PYG{c+c1}{\PYGZsh{}\PYGZsh{}Run a command with arguments}
\PYG{n+nv}{result}\PYG{+w}{ }\PYG{o}{\PYGZlt{}\PYGZhy{}}\PYG{+w}{ }\PYG{n+nf}{SystemCallUpdate}\PYG{p}{(}\PYG{l+s+s2}{\PYGZdq{}ls\PYGZdq{}}\PYG{p}{,}\PYG{+w}{ }\PYG{l+s+s2}{\PYGZdq{}\PYGZhy{}la\PYGZdq{}}\PYG{p}{)}
\end{sphinxVerbatim}

\sphinxAtStartPar
\sphinxstylestrong{See Also:}

\sphinxAtStartPar
\sphinxcode{\sphinxupquote{SystemCall()}}, \sphinxcode{\sphinxupquote{GetSystemType()}}

\index{TimeStamp@\spxentry{TimeStamp}}\ignorespaces 

\subsection{TimeStamp()}
\label{\detokenize{reference/peblenvironment:timestamp}}\label{\detokenize{reference/peblenvironment:index-63}}
\sphinxAtStartPar
\sphinxstyleemphasis{Returns a string containing the current date and time}

\sphinxAtStartPar
\sphinxstylestrong{Description:}

\sphinxAtStartPar
Returns a string containing the date\sphinxhyphen{}and\sphinxhyphen{}time,   formatted according to local conventions. Should be used for   documenting the time\sphinxhyphen{}of\sphinxhyphen{}day and date an experiment was run, but not   for keeping track of timing accuracy.  For that, use   \sphinxcode{\sphinxupquote{GetTime()}}.

\sphinxAtStartPar
\sphinxstylestrong{Usage:}

\begin{sphinxVerbatim}[commandchars=\\\{\}]
\PYG{n+nf}{TimeStamp}\PYG{p}{(}\PYG{p}{)}
\end{sphinxVerbatim}

\sphinxAtStartPar
\sphinxstylestrong{Example:}

\begin{sphinxVerbatim}[commandchars=\\\{\}]
\PYG{n+nv}{a}\PYG{+w}{ }\PYG{o}{\PYGZlt{}\PYGZhy{}}\PYG{+w}{ }\PYG{n+nf}{TimeStamp}\PYG{p}{(}\PYG{p}{)}
\PYG{n+nf}{Print}\PYG{p}{(}\PYG{n+nv}{a}\PYG{p}{)}
\end{sphinxVerbatim}

\sphinxAtStartPar
\sphinxstylestrong{See Also:}

\sphinxAtStartPar
\sphinxcode{\sphinxupquote{GetTime()}}

\index{TranslateKeyCode@\spxentry{TranslateKeyCode}}\ignorespaces 

\subsection{TranslateKeyCode()}
\label{\detokenize{reference/peblenvironment:translatekeycode}}\label{\detokenize{reference/peblenvironment:index-64}}
\sphinxAtStartPar
\sphinxstyleemphasis{Converts a keycode to a key name}

\sphinxAtStartPar
\sphinxstylestrong{Description:}

\sphinxAtStartPar
Translates a code corresponding to a keyboard key   into a keyboard value.  This code is returned by some event/device   polling functions.

\index{TranslateString@\spxentry{TranslateString}}\ignorespaces 

\subsection{TranslateString()}
\label{\detokenize{reference/peblenvironment:translatestring}}\label{\detokenize{reference/peblenvironment:index-65}}
\sphinxAtStartPar
\sphinxstyleemphasis{Converts a key name string to its keycode}

\sphinxAtStartPar
\sphinxstylestrong{Description:}

\sphinxAtStartPar
Translates a string representation of a key (like “a”, “space”, “return”) into its corresponding internal keycode value. This is useful for programmatically working with keyboard input.

\sphinxAtStartPar
\sphinxstylestrong{Usage:}

\begin{sphinxVerbatim}[commandchars=\\\{\}]
\PYG{n+nf}{TranslateString}\PYG{p}{(}\PYG{o}{\PYGZlt{}}\PYG{n+nv}{key\PYGZus{}string}\PYG{o}{\PYGZgt{}}\PYG{p}{)}
\end{sphinxVerbatim}

\sphinxAtStartPar
\sphinxstylestrong{Example:}

\begin{sphinxVerbatim}[commandchars=\\\{\}]
\PYG{n+nv}{keycode}\PYG{+w}{ }\PYG{o}{\PYGZlt{}\PYGZhy{}}\PYG{+w}{ }\PYG{n+nf}{TranslateString}\PYG{p}{(}\PYG{l+s+s2}{\PYGZdq{}a\PYGZdq{}}\PYG{p}{)}
\PYG{n+nv}{spaceCode}\PYG{+w}{ }\PYG{o}{\PYGZlt{}\PYGZhy{}}\PYG{+w}{ }\PYG{n+nf}{TranslateString}\PYG{p}{(}\PYG{l+s+s2}{\PYGZdq{}space\PYGZdq{}}\PYG{p}{)}
\PYG{n+nv}{enterCode}\PYG{+w}{ }\PYG{o}{\PYGZlt{}\PYGZhy{}}\PYG{+w}{ }\PYG{n+nf}{TranslateString}\PYG{p}{(}\PYG{l+s+s2}{\PYGZdq{}return\PYGZdq{}}\PYG{p}{)}
\end{sphinxVerbatim}

\sphinxAtStartPar
\sphinxstylestrong{See Also:}

\sphinxAtStartPar
\sphinxcode{\sphinxupquote{TranslateKeyCode()}}, \sphinxcode{\sphinxupquote{WaitForKeyPress()}}

\index{VariableExists@\spxentry{VariableExists}}\ignorespaces 

\subsection{VariableExists()}
\label{\detokenize{reference/peblenvironment:variableexists}}\label{\detokenize{reference/peblenvironment:index-66}}
\sphinxAtStartPar
\sphinxstylestrong{Description:}

\sphinxAtStartPar
Tests whether a variable exists.

\sphinxAtStartPar
\sphinxstylestrong{Usage:}

\begin{sphinxVerbatim}[commandchars=\\\{\}]
\PYG{n+nf}{Uppercase}\PYG{p}{(}\PYG{l+s+s2}{\PYGZdq{}variablename\PYGZdq{}}\PYG{p}{)}
\end{sphinxVerbatim}

\sphinxAtStartPar
\sphinxstylestrong{Example:}

\begin{sphinxVerbatim}[commandchars=\\\{\}]
\PYG{k}{if}\PYG{p}{(}\PYG{k}{not}\PYG{+w}{ }\PYG{n+nf}{VariableExists}\PYG{p}{(}\PYG{l+s+s2}{\PYGZdq{}underwear\PYGZdq{}}\PYG{p}{)}\PYG{p}{)}
\PYG{+w}{     }\PYG{p}{\PYGZob{}}
\PYG{+w}{       }\PYG{n+nv}{underwear}\PYG{+w}{ }\PYG{o}{\PYGZlt{}\PYGZhy{}}\PYG{+w}{ }\PYG{l+s+s2}{\PYGZdq{}Under there\PYGZdq{}}
\PYG{+w}{     }\PYG{p}{\PYGZcb{}}
\end{sphinxVerbatim}

\sphinxAtStartPar
\sphinxstylestrong{See Also:}

\sphinxAtStartPar
\sphinxcode{\sphinxupquote{PropertyExists()}}

\index{Wait@\spxentry{Wait}}\ignorespaces 

\subsection{Wait()}
\label{\detokenize{reference/peblenvironment:wait}}\label{\detokenize{reference/peblenvironment:index-67}}
\sphinxAtStartPar
\sphinxstylestrong{Description:}

\sphinxAtStartPar
Waits the specified number of milliseconds, then returns.

\sphinxAtStartPar
\sphinxstylestrong{Usage:}

\begin{sphinxVerbatim}[commandchars=\\\{\}]
\PYG{n+nf}{Wait}\PYG{p}{(}\PYG{o}{\PYGZlt{}}\PYG{n+nv}{time}\PYG{o}{\PYGZgt{}}\PYG{p}{)}
\end{sphinxVerbatim}

\sphinxAtStartPar
\sphinxstylestrong{Example:}

\begin{sphinxVerbatim}[commandchars=\\\{\}]
\PYG{n+nf}{Wait}\PYG{p}{(}\PYG{l+m+mi}{100}\PYG{p}{)}
\PYG{n+nf}{Wait}\PYG{p}{(}\PYG{l+m+mi}{15}\PYG{p}{)}
\end{sphinxVerbatim}

\index{WaitForAllKeysUp@\spxentry{WaitForAllKeysUp}}\ignorespaces 

\subsection{WaitForAllKeysUp()}
\label{\detokenize{reference/peblenvironment:waitforallkeysup}}\label{\detokenize{reference/peblenvironment:index-68}}
\sphinxAtStartPar
\sphinxstyleemphasis{Waits until all keys are in up state}

\sphinxAtStartPar
\sphinxstylestrong{Description:}

\sphinxAtStartPar
Wait until all keyboard keys are in the up                position. This includes numlock, capslock, etc.

\index{WaitForKeyDown@\spxentry{WaitForKeyDown}}\ignorespaces 

\subsection{WaitForKeyDown()}
\label{\detokenize{reference/peblenvironment:waitforkeydown}}\label{\detokenize{reference/peblenvironment:index-69}}
\sphinxAtStartPar
\sphinxstyleemphasis{Waits until a specific key is in the down state}

\sphinxAtStartPar
\sphinxstylestrong{Description:}

\sphinxAtStartPar
Waits for a specific key to be detected in the down position. Unlike WaitForKeyPress(), this tests the state of the key rather than waiting for a keypress event. Will return immediately if the key is already down when called.

\sphinxAtStartPar
\sphinxstylestrong{Usage:}

\begin{sphinxVerbatim}[commandchars=\\\{\}]
\PYG{n+nf}{WaitForKeyDown}\PYG{p}{(}\PYG{o}{\PYGZlt{}}\PYG{n+nv}{key}\PYG{o}{\PYGZgt{}}\PYG{p}{)}
\end{sphinxVerbatim}

\sphinxAtStartPar
\sphinxstylestrong{Example:}

\begin{sphinxVerbatim}[commandchars=\\\{\}]
\PYG{n+nf}{WaitForKeyDown}\PYG{p}{(}\PYG{l+s+s2}{\PYGZdq{}a\PYGZdq{}}\PYG{p}{)}
\PYG{n+nf}{Print}\PYG{p}{(}\PYG{l+s+s2}{\PYGZdq{}The \PYGZsq{}a\PYGZsq{} key is now down\PYGZdq{}}\PYG{p}{)}
\end{sphinxVerbatim}

\sphinxAtStartPar
\sphinxstylestrong{See Also:}

\sphinxAtStartPar
\sphinxcode{\sphinxupquote{WaitForKeyPress()}}, \sphinxcode{\sphinxupquote{WaitForKeyRelease()}}, \sphinxcode{\sphinxupquote{WaitForAnyKeyDown()}}

\index{WaitForAnyKeyDown@\spxentry{WaitForAnyKeyDown}}\ignorespaces 

\subsection{WaitForAnyKeyDown()}
\label{\detokenize{reference/peblenvironment:waitforanykeydown}}\label{\detokenize{reference/peblenvironment:index-70}}
\sphinxAtStartPar
\sphinxstyleemphasis{Waits until any key is detected in down state}

\sphinxAtStartPar
\sphinxstylestrong{Description:}

\sphinxAtStartPar
Waits for any key to be detected in the down position.              This includes numlock, capslock, etc, which can be locked              in the down position even if they are not being held              down.  Will return immediately if a key is being held              down before the function is called.

\sphinxAtStartPar
\sphinxstylestrong{See Also:}

\sphinxAtStartPar
\sphinxcode{\sphinxupquote{WaitForAnyKeyPress()}}

\index{WaitForAnyKeyDownWithTimeout@\spxentry{WaitForAnyKeyDownWithTimeout}}\ignorespaces 

\subsection{WaitForAnyKeyDownWithTimeout()}
\label{\detokenize{reference/peblenvironment:waitforanykeydownwithtimeout}}\label{\detokenize{reference/peblenvironment:index-71}}
\sphinxAtStartPar
\sphinxstylestrong{Description:}

\sphinxAtStartPar
Waits until any key is detected in the down position, but will return   after a specified number of milliseconds.  This tests for the key position on each cycle; users should prefer using WaitForAnyKeyPressWithTimout() which waits for the keypress event.

\sphinxAtStartPar
\sphinxstylestrong{Usage:}

\begin{sphinxVerbatim}[commandchars=\\\{\}]
\PYG{n+nf}{WaitForAnyKeyDownWithTimeout}\PYG{p}{(}\PYG{o}{\PYGZlt{}}\PYG{n+nv}{time}\PYG{o}{\PYGZgt{}}\PYG{p}{)}
\end{sphinxVerbatim}

\sphinxAtStartPar
\sphinxstylestrong{See Also:}

\sphinxAtStartPar
\sphinxcode{\sphinxupquote{WaitForAnyKeyPressWithTimeout()}}, \sphinxcode{\sphinxupquote{WaitListKeyPressWithTimeout()}},  \sphinxcode{\sphinxupquote{WaitForAnyKeyPress()}}, \sphinxcode{\sphinxupquote{WaitListKeyPress()}}

\index{WaitForAnyKeyPress@\spxentry{WaitForAnyKeyPress}}\ignorespaces 

\subsection{WaitForAnyKeyPress()}
\label{\detokenize{reference/peblenvironment:waitforanykeypress}}\label{\detokenize{reference/peblenvironment:index-72}}
\sphinxAtStartPar
\sphinxstyleemphasis{Waits until any key is pressed}

\sphinxAtStartPar
\sphinxstylestrong{Description:}

\sphinxAtStartPar
Waits until any key is pressed, and returns the key pressed. This waits for the keyboard event, which is typically more reliable and less computationally taxing than waiting for the keyboard state (which updates based on those events anyway).

\sphinxAtStartPar
\sphinxstylestrong{Usage:}

\begin{sphinxVerbatim}[commandchars=\\\{\}]
\PYG{n+nf}{WaitForKeyPress}\PYG{p}{(}\PYG{o}{\PYGZlt{}}\PYG{n+nv}{time}\PYG{o}{\PYGZgt{}}\PYG{p}{)}
\end{sphinxVerbatim}

\sphinxAtStartPar
\sphinxstylestrong{Example:}

\begin{sphinxVerbatim}[commandchars=\\\{\}]
\PYG{n+nv}{cont}\PYG{+w}{ }\PYG{o}{\PYGZlt{}\PYGZhy{}}\PYG{+w}{ }\PYG{l+m+mi}{1}
\PYG{+w}{        }\PYG{k}{while}\PYG{p}{(}\PYG{n+nv}{cont}\PYG{p}{)}
\PYG{+w}{         }\PYG{p}{\PYGZob{}}
\PYG{+w}{         }\PYG{n+nv}{key}\PYG{+w}{ }\PYG{o}{\PYGZlt{}\PYGZhy{}}\PYG{+w}{ }\PYG{n+nf}{WaitForAnyKEyPress}\PYG{p}{(}\PYG{p}{)}
\PYG{+w}{         }\PYG{k}{if}\PYG{p}{(}\PYG{n+nv}{key}\PYG{+w}{ }\PYG{o}{==}\PYG{+w}{ }\PYG{l+s+s2}{\PYGZdq{}x\PYGZdq{}}\PYG{p}{)}
\PYG{+w}{          }\PYG{p}{\PYGZob{}}
\PYG{+w}{              }\PYG{n+nv}{cont}\PYG{+w}{ }\PYG{o}{\PYGZlt{}\PYGZhy{}}\PYG{+w}{ }\PYG{l+m+mi}{0}
\PYG{+w}{          }\PYG{p}{\PYGZcb{}}
\PYG{+w}{         }\PYG{p}{\PYGZcb{}}
\end{sphinxVerbatim}

\sphinxAtStartPar
\sphinxstylestrong{See Also:}

\sphinxAtStartPar
\sphinxcode{\sphinxupquote{WaitForAnyKeyPressWithTimeout()}}, \sphinxcode{\sphinxupquote{WaitListKeyPressWithTimeout()}},   \sphinxcode{\sphinxupquote{WaitListKeyPress()}}

\index{WaitForAnyKeyPressWithTimeout@\spxentry{WaitForAnyKeyPressWithTimeout}}\ignorespaces 

\subsection{WaitForAnyKeyPressWithTimeout()}
\label{\detokenize{reference/peblenvironment:waitforanykeypresswithtimeout}}\label{\detokenize{reference/peblenvironment:index-73}}
\sphinxAtStartPar
\sphinxstylestrong{Description:}

\sphinxAtStartPar
Waits until any key is detected in the down position, but will return   after a specified number of milliseconds.  This tests for the key position on each cycle; users should prefer using WaitForAnyKeyPressWithTimout() which waits for the keypress event.

\sphinxAtStartPar
\sphinxstylestrong{Usage:}

\begin{sphinxVerbatim}[commandchars=\\\{\}]
\PYG{n+nf}{WaitForAnyKeyDownWithTimeout}\PYG{p}{(}\PYG{o}{\PYGZlt{}}\PYG{n+nv}{time}\PYG{o}{\PYGZgt{}}\PYG{p}{)}
\end{sphinxVerbatim}

\sphinxAtStartPar
\sphinxstylestrong{See Also:}

\sphinxAtStartPar
\sphinxcode{\sphinxupquote{WaitForAnyKeyPressWithTimeout()}}, \sphinxcode{\sphinxupquote{WaitListKeyPressWithTimeout()}}

\index{WaitForKeyListDown@\spxentry{WaitForKeyListDown}}\ignorespaces 

\subsection{WaitForKeyListDown()}
\label{\detokenize{reference/peblenvironment:waitforkeylistdown}}\label{\detokenize{reference/peblenvironment:index-74}}
\sphinxAtStartPar
\sphinxstyleemphasis{Waits until one of the keys is in down state}

\sphinxAtStartPar
\sphinxstylestrong{Description:}

\sphinxAtStartPar
Returns when any one of the keys specified in the   argument is down. If a key is down when called, it will return immediately.

\sphinxAtStartPar
\sphinxstylestrong{Usage:}

\begin{sphinxVerbatim}[commandchars=\\\{\}]
\PYG{n+nf}{WaitForKeyListDown}\PYG{p}{(}\PYG{o}{\PYGZlt{}}\PYG{n+nv}{list}\PYG{o}{\PYGZhy{}}\PYG{n+nv}{of}\PYG{o}{\PYGZhy{}}\PYG{n+nv}{keys}\PYG{o}{\PYGZgt{}}\PYG{p}{)}
\end{sphinxVerbatim}

\sphinxAtStartPar
\sphinxstylestrong{Example:}

\begin{sphinxVerbatim}[commandchars=\\\{\}]
\PYG{n+nf}{WaitForKeyListDown}\PYG{p}{(}\PYG{p}{[}\PYG{l+s+s2}{\PYGZdq{}a\PYGZdq{}}\PYG{p}{,}\PYG{l+s+s2}{\PYGZdq{}z\PYGZdq{}}\PYG{p}{]}\PYG{p}{)}
\end{sphinxVerbatim}

\index{WaitForKeyPress@\spxentry{WaitForKeyPress}}\ignorespaces 

\subsection{WaitForKeyPress()}
\label{\detokenize{reference/peblenvironment:waitforkeypress}}\label{\detokenize{reference/peblenvironment:index-75}}
\sphinxAtStartPar
\sphinxstylestrong{Description:}

\sphinxAtStartPar
Waits for a keypress event that matches the   specified key.  Usage of this function is preferred over   \sphinxcode{\sphinxupquote{WaitForKeyDown()}}, which tests the state of the key. Returns the   value of the key pressed.

\sphinxAtStartPar
\sphinxstylestrong{Usage:}

\begin{sphinxVerbatim}[commandchars=\\\{\}]
\PYG{n+nf}{WaitForKeyPress}\PYG{p}{(}\PYG{o}{\PYGZlt{}}\PYG{n+nv}{key}\PYG{o}{\PYGZgt{}}\PYG{p}{)}
\end{sphinxVerbatim}

\sphinxAtStartPar
\sphinxstylestrong{See Also:}

\sphinxAtStartPar
\sphinxcode{\sphinxupquote{WaitForAnyKeyPress()}}, \sphinxcode{\sphinxupquote{WaitForKeyRelease()}}, \sphinxcode{\sphinxupquote{WaitForListKeyPress()}}

\index{WaitForKeyUp@\spxentry{WaitForKeyUp}}\ignorespaces 

\subsection{WaitForKeyUp()}
\label{\detokenize{reference/peblenvironment:waitforkeyup}}\label{\detokenize{reference/peblenvironment:index-76}}
\sphinxAtStartPar
\sphinxstylestrong{Description:}

\index{WaitForKeyRelease@\spxentry{WaitForKeyRelease}}\ignorespaces 

\subsection{WaitForKeyRelease()}
\label{\detokenize{reference/peblenvironment:waitforkeyrelease}}\label{\detokenize{reference/peblenvironment:index-77}}
\sphinxAtStartPar
\sphinxstyleemphasis{Waits until a specific key is released}

\sphinxAtStartPar
\sphinxstylestrong{Description:}

\sphinxAtStartPar
Waits for a specific key to be released (transition from down to up state). This is useful for ensuring a key has been released before continuing, preventing accidental repeated inputs.

\sphinxAtStartPar
\sphinxstylestrong{Usage:}

\begin{sphinxVerbatim}[commandchars=\\\{\}]
\PYG{n+nf}{WaitForKeyRelease}\PYG{p}{(}\PYG{o}{\PYGZlt{}}\PYG{n+nv}{key}\PYG{o}{\PYGZgt{}}\PYG{p}{)}
\end{sphinxVerbatim}

\sphinxAtStartPar
\sphinxstylestrong{Example:}

\begin{sphinxVerbatim}[commandchars=\\\{\}]
\PYG{n+nf}{WaitForKeyPress}\PYG{p}{(}\PYG{l+s+s2}{\PYGZdq{}space\PYGZdq{}}\PYG{p}{)}
\PYG{n+nf}{Print}\PYG{p}{(}\PYG{l+s+s2}{\PYGZdq{}Space pressed\PYGZdq{}}\PYG{p}{)}
\PYG{n+nf}{WaitForKeyRelease}\PYG{p}{(}\PYG{l+s+s2}{\PYGZdq{}space\PYGZdq{}}\PYG{p}{)}
\PYG{n+nf}{Print}\PYG{p}{(}\PYG{l+s+s2}{\PYGZdq{}Space released\PYGZdq{}}\PYG{p}{)}
\end{sphinxVerbatim}

\sphinxAtStartPar
\sphinxstylestrong{See Also:}

\sphinxAtStartPar
\sphinxcode{\sphinxupquote{WaitForKeyDown()}}, \sphinxcode{\sphinxupquote{WaitForKeyPress()}}, \sphinxcode{\sphinxupquote{WaitForAnyKeyPress()}}

\index{WaitForListKeyPress@\spxentry{WaitForListKeyPress}}\ignorespaces 

\subsection{WaitForListKeyPress()}
\label{\detokenize{reference/peblenvironment:waitforlistkeypress}}\label{\detokenize{reference/peblenvironment:index-78}}
\sphinxAtStartPar
\sphinxstylestrong{Description:}

\sphinxAtStartPar
Returns when any one of the keys specified in the   argument is pressed. Will only return on a new keyboard event, and   so a previously pressed key will not trip this function, unlike   \sphinxcode{\sphinxupquote{WaitForKeyListDown()}}  Returns a string indicating the value   of the keypress.

\sphinxAtStartPar
\sphinxstylestrong{Usage:}

\begin{sphinxVerbatim}[commandchars=\\\{\}]
\PYG{n+nf}{WaitForListKeyPress}\PYG{p}{(}\PYG{o}{\PYGZlt{}}\PYG{n+nv}{list}\PYG{o}{\PYGZhy{}}\PYG{n+nv}{of}\PYG{o}{\PYGZhy{}}\PYG{n+nv}{keys}\PYG{o}{\PYGZgt{}}\PYG{p}{)}
\end{sphinxVerbatim}

\sphinxAtStartPar
\sphinxstylestrong{Example:}

\begin{sphinxVerbatim}[commandchars=\\\{\}]
\PYG{n+nf}{WaitForListKeyPress}\PYG{p}{(}\PYG{p}{[}\PYG{l+s+s2}{\PYGZdq{}a\PYGZdq{}}\PYG{p}{,}\PYG{l+s+s2}{\PYGZdq{}z\PYGZdq{}}\PYG{p}{]}\PYG{p}{)}
\end{sphinxVerbatim}

\sphinxAtStartPar
\sphinxstylestrong{See Also:}

\sphinxAtStartPar
\sphinxcode{\sphinxupquote{WaitForKeyListDown()}}, \sphinxcode{\sphinxupquote{WaitForListKeyPressWithTimeout()}}

\index{WaitForListKeyPressWithTimeout@\spxentry{WaitForListKeyPressWithTimeout}}\ignorespaces 

\subsection{WaitForListKeyPressWithTimeout()}
\label{\detokenize{reference/peblenvironment:waitforlistkeypresswithtimeout}}\label{\detokenize{reference/peblenvironment:index-79}}
\sphinxAtStartPar
\sphinxstylestrong{Description:}

\sphinxAtStartPar
Returns when any one of the keys specified in the   argument is pressed, or when the timeout has elapsed; whichever   comes first. Will only return on a new keyboard/timeout events, and   so a previously pressed key will not trip this function, unlike   \sphinxcode{\sphinxupquote{WaitForKeyListDown()}}.  The optional \sphinxcode{\sphinxupquote{\textless{}style\textgreater{}}} parameter is currently   unused, but may be deployed in the future for differences in how   or when things should be returned.  Returns the value of the pressed   key.  If the function terminates by exceeding the \sphinxcode{\sphinxupquote{\textless{}timeout\textgreater{}}},   it will return the string \sphinxcode{\sphinxupquote{"\textless{}timeout\textgreater{}"}}.  Note: previous to 2.0, returned a list {[}“\textless{}timeout\textgreater{}”{]}, which may mean updating logic for tests designed in the 0.x series.

\sphinxAtStartPar
\sphinxstylestrong{Usage:}

\begin{sphinxVerbatim}[commandchars=\\\{\}]
\PYG{n+nf}{WaitForListKeyPressWithTimeout}\PYG{p}{(}\PYG{o}{\PYGZlt{}}\PYG{n+nv}{list}\PYG{o}{\PYGZhy{}}\PYG{n+nv}{of}\PYG{o}{\PYGZhy{}}\PYG{n+nv}{keys}\PYG{o}{\PYGZgt{}}\PYG{p}{,}
\PYG{+w}{                                }\PYG{o}{\PYGZlt{}}\PYG{n+nv}{timeout}\PYG{o}{\PYGZgt{}}\PYG{p}{,}\PYG{n+nv}{opt}\PYG{o}{:}\PYG{o}{\PYGZlt{}}\PYG{n+nv}{style}\PYG{o}{\PYGZgt{}}\PYG{p}{)}
\end{sphinxVerbatim}

\sphinxAtStartPar
\sphinxstylestrong{Example:}

\begin{sphinxVerbatim}[commandchars=\\\{\}]
\PYG{n+nv}{x}\PYG{+w}{ }\PYG{o}{\PYGZlt{}\PYGZhy{}}\PYG{+w}{ }\PYG{n+nf}{WaitForListKeyPressWithTimeout}\PYG{p}{(}\PYG{p}{[}\PYG{l+s+s2}{\PYGZdq{}a\PYGZdq{}}\PYG{p}{,}\PYG{l+s+s2}{\PYGZdq{}z\PYGZdq{}}\PYG{p}{]}\PYG{p}{,}
\PYG{+w}{                                       }\PYG{l+m+mi}{2000}\PYG{p}{)}
\PYG{+w}{  }\PYG{k}{if}\PYG{p}{(}\PYG{n+nf}{IsList}\PYG{p}{(}\PYG{n+nv}{x}\PYG{p}{)}\PYG{p}{)}
\PYG{+w}{  }\PYG{p}{\PYGZob{}}
\PYG{+w}{     }\PYG{n+nf}{Print}\PYG{p}{(}\PYG{l+s+s2}{\PYGZdq{}Did Not Respond.\PYGZdq{}}\PYG{p}{)}
\PYG{+w}{  }\PYG{p}{\PYGZcb{}}
\end{sphinxVerbatim}

\sphinxAtStartPar
\sphinxstylestrong{See Also:}

\sphinxAtStartPar
\sphinxcode{\sphinxupquote{WaitForKeyListDown()}}, \sphinxcode{\sphinxupquote{WaitForListKeyPress()}}, \sphinxcode{\sphinxupquote{WaitForKeyPressWithTimeout()}}

\index{WaitForMouseButton@\spxentry{WaitForMouseButton}}\ignorespaces 

\subsection{WaitForMouseButton()}
\label{\detokenize{reference/peblenvironment:waitformousebutton}}\label{\detokenize{reference/peblenvironment:index-80}}
\sphinxAtStartPar
\sphinxstyleemphasis{Waits until any of the mouse buttons is pressed or released, and returns message indicating what happened}

\sphinxAtStartPar
\sphinxstylestrong{Description:}

\sphinxAtStartPar
Waits for a mouse click event to occur.   This takes no arguments, and returns a 4\sphinxhyphen{}tuple list, indicating:

\begin{sphinxVerbatim}[commandchars=\\\{\}]
[xpos,   ypos,   button id [1\PYGZhy{}3],   \PYGZdq{}\PYGZlt{}pressed\PYGZgt{}\PYGZdq{} or \PYGZdq{}\PYGZlt{}released\PYGZgt{}\PYGZdq{}]
\end{sphinxVerbatim}

\sphinxAtStartPar
\sphinxstylestrong{Usage:}

\begin{sphinxVerbatim}[commandchars=\\\{\}]
\PYG{n+nf}{WaitForMouseButton}\PYG{p}{(}\PYG{p}{)}
\end{sphinxVerbatim}

\sphinxAtStartPar
\sphinxstylestrong{Example:}

\begin{sphinxVerbatim}[commandchars=\\\{\}]
\PYG{c+c1}{\PYGZsh{}\PYGZsh{} Here is how to wait for a mouse down\PYGZhy{}click}

\PYG{+w}{ }\PYG{n+nv}{continue}\PYG{+w}{ }\PYG{o}{\PYGZlt{}\PYGZhy{}}\PYG{+w}{ }\PYG{l+m+mi}{1}
\PYG{+w}{ }\PYG{k}{while}\PYG{p}{(}\PYG{n+nv}{continue}\PYG{p}{)}
\PYG{+w}{ }\PYG{p}{\PYGZob{}}
\PYG{+w}{     }\PYG{n+nv}{x}\PYG{+w}{ }\PYG{o}{\PYGZlt{}\PYGZhy{}}\PYG{+w}{ }\PYG{n+nf}{WaitForMouseButton}\PYG{p}{(}\PYG{p}{)}
\PYG{+w}{     }\PYG{k}{if}\PYG{p}{(}\PYG{n+nf}{Nth}\PYG{p}{(}\PYG{n+nv}{x}\PYG{p}{,}\PYG{l+m+mi}{4}\PYG{p}{)}\PYG{o}{==}\PYG{l+s+s2}{\PYGZdq{}\PYGZlt{}pressed\PYGZgt{}\PYGZdq{}}\PYG{p}{)}
\PYG{+w}{      }\PYG{p}{\PYGZob{}}
\PYG{+w}{          }\PYG{n+nv}{continue}\PYG{+w}{ }\PYG{o}{\PYGZlt{}\PYGZhy{}}\PYG{+w}{ }\PYG{l+m+mi}{0}
\PYG{+w}{      }\PYG{p}{\PYGZcb{}}
\PYG{+w}{ }\PYG{p}{\PYGZcb{}}
\PYG{+w}{ }\PYG{n+nf}{Print}\PYG{p}{(}\PYG{l+s+s2}{\PYGZdq{}Clicked\PYGZdq{}}\PYG{p}{)}
\end{sphinxVerbatim}

\sphinxAtStartPar
\sphinxstylestrong{See Also:}

\sphinxAtStartPar
\sphinxcode{\sphinxupquote{ShowCursor()}}, \sphinxcode{\sphinxupquote{WaitForMouseButtonWithTimeout()}}   \sphinxcode{\sphinxupquote{SetMouseCursorPosition()}}, \sphinxcode{\sphinxupquote{GetMouseCursorPosition()}}

\index{WaitForMouseButtonWithTimeout@\spxentry{WaitForMouseButtonWithTimeout}}\ignorespaces 

\subsection{WaitForMouseButtonWithTimeout()}
\label{\detokenize{reference/peblenvironment:waitformousebuttonwithtimeout}}\label{\detokenize{reference/peblenvironment:index-81}}
\sphinxAtStartPar
\sphinxstylestrong{Description:}

\sphinxAtStartPar
Waits for a mouse click event to occur, or a   timeout to be reached.   This takes a single argument: timeout delay in ms. When clicked, it returns a   4\sphinxhyphen{}tuple list, indicating:

\begin{sphinxVerbatim}[commandchars=\\\{\}]
[xpos,  ypos,   button id [1\PYGZhy{}3],   \PYGZdq{}\PYGZlt{}pressed\PYGZgt{}\PYGZdq{} or \PYGZdq{}\PYGZlt{}released\PYGZgt{}\PYGZdq{}]

when not click and timeout is reached, it returns a list:    ``[timeout]``
\end{sphinxVerbatim}

\sphinxAtStartPar
\sphinxstylestrong{Usage:}

\begin{sphinxVerbatim}[commandchars=\\\{\}]
\PYG{n+nf}{WaitForMouseButtonWithTimeOut}\PYG{p}{(}\PYG{l+m+mi}{10}\PYG{p}{)}
\end{sphinxVerbatim}

\sphinxAtStartPar
\sphinxstylestrong{Example:}

\begin{sphinxVerbatim}[commandchars=\\\{\}]
\PYG{c+c1}{\PYGZsh{}\PYGZsh{} Here is how to wait for a mouse down\PYGZhy{}click}

\PYG{+w}{ }\PYG{n+nv}{continue}\PYG{+w}{ }\PYG{o}{\PYGZlt{}\PYGZhy{}}\PYG{+w}{ }\PYG{l+m+mi}{1}
\PYG{+w}{ }\PYG{k}{while}\PYG{p}{(}\PYG{n+nv}{continue}\PYG{p}{)}
\PYG{+w}{ }\PYG{p}{\PYGZob{}}
\PYG{+w}{     }\PYG{n+nv}{x}\PYG{+w}{ }\PYG{o}{\PYGZlt{}\PYGZhy{}}\PYG{+w}{ }\PYG{n+nf}{WaitForMouseButtonWithTimeout}\PYG{p}{(}\PYG{l+m+mi}{500}\PYG{p}{)}
\PYG{+w}{     }\PYG{k}{if}\PYG{p}{(}\PYG{n+nf}{First}\PYG{p}{(}\PYG{n+nv}{x}\PYG{p}{)}\PYG{o}{==}\PYG{l+s+s2}{\PYGZdq{}\PYGZlt{}timeout\PYGZgt{}\PYGZdq{}}\PYG{p}{)}
\PYG{+w}{      }\PYG{p}{\PYGZob{}}
\PYG{+w}{         }\PYG{n+nf}{Print}\PYG{p}{(}\PYG{l+s+s2}{\PYGZdq{}time is \PYGZdq{}}\PYG{o}{+}\PYG{n+nf}{GetTime}\PYG{p}{(}\PYG{p}{)}\PYG{p}{)}
\PYG{+w}{          }\PYG{n+nv}{continue}\PYG{+w}{ }\PYG{o}{\PYGZlt{}\PYGZhy{}}\PYG{+w}{ }\PYG{l+m+mi}{1}
\PYG{+w}{      }\PYG{p}{\PYGZcb{}}\PYG{+w}{ }\PYG{k}{else}\PYG{+w}{ }\PYG{p}{\PYGZob{}}
\PYG{+w}{          }\PYG{n+nv}{continue}\PYG{+w}{ }\PYG{o}{\PYGZlt{}\PYGZhy{}}\PYG{+w}{ }\PYG{l+m+mi}{0}
\PYG{+w}{      }\PYG{p}{\PYGZcb{}}
\PYG{+w}{ }\PYG{p}{\PYGZcb{}}
\PYG{+w}{ }\PYG{n+nf}{Print}\PYG{p}{(}\PYG{l+s+s2}{\PYGZdq{}Clicked\PYGZdq{}}\PYG{p}{)}
\end{sphinxVerbatim}

\sphinxAtStartPar
\sphinxstylestrong{See Also:}

\sphinxAtStartPar
\sphinxcode{\sphinxupquote{ShowCursor()}},   \sphinxcode{\sphinxupquote{SetMouseCursorPosition()}}, \sphinxcode{\sphinxupquote{GetMouseCursorPosition()}}

\sphinxstepscope


\section{PEBLList \sphinxhyphen{} List Manipulation}
\label{\detokenize{reference/pebllist:pebllist-list-manipulation}}\label{\detokenize{reference/pebllist::doc}}
\sphinxAtStartPar
This module contains functions for creating, manipulating, and querying lists.

\begin{sphinxShadowBox}
\sphinxstyletopictitle{Function Index}
\begin{itemize}
\item {} 
\sphinxAtStartPar
\phantomsection\label{\detokenize{reference/pebllist:id3}}{\hyperref[\detokenize{reference/pebllist:append}]{\sphinxcrossref{Append()}}}

\item {} 
\sphinxAtStartPar
\phantomsection\label{\detokenize{reference/pebllist:id4}}{\hyperref[\detokenize{reference/pebllist:crossfactorwithoutduplicates}]{\sphinxcrossref{CrossFactorWithoutDuplicates()}}}

\item {} 
\sphinxAtStartPar
\phantomsection\label{\detokenize{reference/pebllist:id5}}{\hyperref[\detokenize{reference/pebllist:designfullcounterbalance}]{\sphinxcrossref{DesignFullCounterbalance()}}}

\item {} 
\sphinxAtStartPar
\phantomsection\label{\detokenize{reference/pebllist:id6}}{\hyperref[\detokenize{reference/pebllist:first}]{\sphinxcrossref{First()}}}

\item {} 
\sphinxAtStartPar
\phantomsection\label{\detokenize{reference/pebllist:id7}}{\hyperref[\detokenize{reference/pebllist:ismember}]{\sphinxcrossref{IsMember()}}}

\item {} 
\sphinxAtStartPar
\phantomsection\label{\detokenize{reference/pebllist:id8}}{\hyperref[\detokenize{reference/pebllist:last}]{\sphinxcrossref{Last()}}}

\item {} 
\sphinxAtStartPar
\phantomsection\label{\detokenize{reference/pebllist:id9}}{\hyperref[\detokenize{reference/pebllist:length}]{\sphinxcrossref{Length()}}}

\item {} 
\sphinxAtStartPar
\phantomsection\label{\detokenize{reference/pebllist:id10}}{\hyperref[\detokenize{reference/pebllist:list}]{\sphinxcrossref{List()}}}

\item {} 
\sphinxAtStartPar
\phantomsection\label{\detokenize{reference/pebllist:id11}}{\hyperref[\detokenize{reference/pebllist:listtostring}]{\sphinxcrossref{ListToString()}}}

\item {} 
\sphinxAtStartPar
\phantomsection\label{\detokenize{reference/pebllist:id12}}{\hyperref[\detokenize{reference/pebllist:merge}]{\sphinxcrossref{Merge()}}}

\item {} 
\sphinxAtStartPar
\phantomsection\label{\detokenize{reference/pebllist:id13}}{\hyperref[\detokenize{reference/pebllist:modlist}]{\sphinxcrossref{ModList()}}}

\item {} 
\sphinxAtStartPar
\phantomsection\label{\detokenize{reference/pebllist:id14}}{\hyperref[\detokenize{reference/pebllist:nth}]{\sphinxcrossref{Nth()}}}

\item {} 
\sphinxAtStartPar
\phantomsection\label{\detokenize{reference/pebllist:id15}}{\hyperref[\detokenize{reference/pebllist:second}]{\sphinxcrossref{Second()}}}

\item {} 
\sphinxAtStartPar
\phantomsection\label{\detokenize{reference/pebllist:id16}}{\hyperref[\detokenize{reference/pebllist:third}]{\sphinxcrossref{Third()}}}

\item {} 
\sphinxAtStartPar
\phantomsection\label{\detokenize{reference/pebllist:id17}}{\hyperref[\detokenize{reference/pebllist:fourth}]{\sphinxcrossref{Fourth()}}}

\item {} 
\sphinxAtStartPar
\phantomsection\label{\detokenize{reference/pebllist:id18}}{\hyperref[\detokenize{reference/pebllist:fifth}]{\sphinxcrossref{Fifth()}}}

\item {} 
\sphinxAtStartPar
\phantomsection\label{\detokenize{reference/pebllist:id19}}{\hyperref[\detokenize{reference/pebllist:pushonend}]{\sphinxcrossref{PushOnEnd()}}}

\item {} 
\sphinxAtStartPar
\phantomsection\label{\detokenize{reference/pebllist:id20}}{\hyperref[\detokenize{reference/pebllist:repeat}]{\sphinxcrossref{Repeat()}}}

\item {} 
\sphinxAtStartPar
\phantomsection\label{\detokenize{reference/pebllist:id21}}{\hyperref[\detokenize{reference/pebllist:repeatlist}]{\sphinxcrossref{RepeatList()}}}

\item {} 
\sphinxAtStartPar
\phantomsection\label{\detokenize{reference/pebllist:id22}}{\hyperref[\detokenize{reference/pebllist:rotate}]{\sphinxcrossref{Rotate()}}}

\item {} 
\sphinxAtStartPar
\phantomsection\label{\detokenize{reference/pebllist:id23}}{\hyperref[\detokenize{reference/pebllist:sequence}]{\sphinxcrossref{Sequence()}}}

\item {} 
\sphinxAtStartPar
\phantomsection\label{\detokenize{reference/pebllist:id24}}{\hyperref[\detokenize{reference/pebllist:setelement}]{\sphinxcrossref{SetElement()}}}

\item {} 
\sphinxAtStartPar
\phantomsection\label{\detokenize{reference/pebllist:id25}}{\hyperref[\detokenize{reference/pebllist:shuffle}]{\sphinxcrossref{Shuffle()}}}

\item {} 
\sphinxAtStartPar
\phantomsection\label{\detokenize{reference/pebllist:id26}}{\hyperref[\detokenize{reference/pebllist:sort}]{\sphinxcrossref{Sort()}}}

\item {} 
\sphinxAtStartPar
\phantomsection\label{\detokenize{reference/pebllist:id27}}{\hyperref[\detokenize{reference/pebllist:sortby}]{\sphinxcrossref{SortBy()}}}

\item {} 
\sphinxAtStartPar
\phantomsection\label{\detokenize{reference/pebllist:id28}}{\hyperref[\detokenize{reference/pebllist:sublist}]{\sphinxcrossref{SubList()}}}

\item {} 
\sphinxAtStartPar
\phantomsection\label{\detokenize{reference/pebllist:id29}}{\hyperref[\detokenize{reference/pebllist:transpose}]{\sphinxcrossref{Transpose()}}}

\end{itemize}
\end{sphinxShadowBox}

\index{Append@\spxentry{Append}}\ignorespaces 

\subsection{Append()}
\label{\detokenize{reference/pebllist:append}}\label{\detokenize{reference/pebllist:index-0}}
\sphinxAtStartPar
\sphinxstylestrong{Description:}

\sphinxAtStartPar
Appends an item to a list.  Useful for constructing lists in conjunction with the loop statement.  Note: \sphinxcode{\sphinxupquote{Append()}} is useful, but inefficent for large data structures, because it requires making a copy of the entire data list and then overwriting it, if you use \sphinxcode{\sphinxupquote{list \textless{}\sphinxhyphen{} Append(list, item)}}.  The overhead will be hardly noticeable unless you are building lists hundreds of elements long.  In that case you shuold either create the list upfront and use \sphinxcode{\sphinxupquote{SetElement}}, or you \sphinxcode{\sphinxupquote{PushOnEnd}} to modify the list directly.

\sphinxAtStartPar
\sphinxstylestrong{Usage:}

\begin{sphinxVerbatim}[commandchars=\\\{\}]
\PYG{n+nf}{Append}\PYG{p}{(}\PYG{o}{\PYGZlt{}}\PYG{n+nv}{list}\PYG{o}{\PYGZgt{}}\PYG{p}{,}\PYG{+w}{ }\PYG{o}{\PYGZlt{}}\PYG{n+nv}{item}\PYG{o}{\PYGZgt{}}\PYG{p}{)}
\end{sphinxVerbatim}

\sphinxAtStartPar
\sphinxstylestrong{Example:}

\begin{sphinxVerbatim}[commandchars=\\\{\}]
\PYG{n+nv}{list}\PYG{+w}{ }\PYG{o}{\PYGZlt{}\PYGZhy{}}\PYG{+w}{ }\PYG{n+nf}{Sequence}\PYG{p}{(}\PYG{l+m+mi}{1}\PYG{p}{,}\PYG{l+m+mi}{5}\PYG{p}{,}\PYG{l+m+mi}{1}\PYG{p}{)}
\PYG{n+nv}{double}\PYG{+w}{  }\PYG{o}{\PYGZlt{}\PYGZhy{}}\PYG{+w}{ }\PYG{p}{[}\PYG{p}{]}
\PYG{k}{loop}\PYG{p}{(}\PYG{n+nv}{i}\PYG{p}{,}\PYG{+w}{ }\PYG{n+nv}{list}\PYG{p}{)}
\PYG{p}{\PYGZob{}}
\PYG{+w}{ }\PYG{n+nv}{double}\PYG{+w}{ }\PYG{o}{\PYGZlt{}\PYGZhy{}}\PYG{+w}{ }\PYG{n+nf}{Append}\PYG{p}{(}\PYG{n+nv}{double}\PYG{p}{,}\PYG{+w}{ }\PYG{p}{[}\PYG{n+nv}{i}\PYG{p}{,}\PYG{n+nv}{i}\PYG{p}{]}\PYG{p}{)}
\PYG{p}{\PYGZcb{}}
\PYG{n+nf}{Print}\PYG{p}{(}\PYG{n+nv}{double}\PYG{p}{)}
\PYG{c+c1}{\PYGZsh{} Produces [[1,1],[2,2],[3,3],[4,4],[5,5]]}
\end{sphinxVerbatim}

\sphinxAtStartPar
\sphinxstylestrong{See Also:}

\sphinxAtStartPar
\sphinxcode{\sphinxupquote{SetElement()}} \sphinxcode{\sphinxupquote{List()}}, \sphinxcode{\sphinxupquote{{[} {]}()}}, \sphinxcode{\sphinxupquote{Merge()}}, \sphinxcode{\sphinxupquote{PushOnEnd()}}

\index{CrossFactorWithoutDuplicates@\spxentry{CrossFactorWithoutDuplicates}}\ignorespaces 

\subsection{CrossFactorWithoutDuplicates()}
\label{\detokenize{reference/pebllist:crossfactorwithoutduplicates}}\label{\detokenize{reference/pebllist:index-1}}
\sphinxAtStartPar
\sphinxstylestrong{Description:}

\sphinxAtStartPar
This function takes a single list, and returns a list of all                    pairs, excluding the pairs that have two of the same item.                      To achieve the same effect but include the duplicates, use: \sphinxcode{\sphinxupquote{DesignFullCounterBalance(x,x)}}.

\sphinxAtStartPar
\sphinxstylestrong{Usage:}

\begin{sphinxVerbatim}[commandchars=\\\{\}]
\PYG{n+nf}{CrossFactorWithoutDuplicates}\PYG{p}{(}\PYG{o}{\PYGZlt{}}\PYG{n+nv}{list}\PYG{o}{\PYGZgt{}}\PYG{p}{)}
\end{sphinxVerbatim}

\sphinxAtStartPar
\sphinxstylestrong{Example:}

\begin{sphinxVerbatim}[commandchars=\\\{\}]
\PYG{n+nf}{CrossFactorWithoutDuplicates}\PYG{p}{(}\PYG{p}{[}\PYG{n+nv}{a}\PYG{p}{,}\PYG{n+nv}{b}\PYG{p}{,}\PYG{n+nv}{c}\PYG{p}{]}\PYG{p}{)}
\PYG{c+c1}{\PYGZsh{} == [[a,b],[a,c],[b,a],[b,c],[c,a],[c,b]]}
\end{sphinxVerbatim}

\sphinxAtStartPar
\sphinxstylestrong{See Also:}

\sphinxAtStartPar
\sphinxcode{\sphinxupquote{DesignFullCounterBalance()}}, \sphinxcode{\sphinxupquote{Repeat()}}, \sphinxcode{\sphinxupquote{DesignBalancedSampling()}},  \sphinxcode{\sphinxupquote{DesignGrecoLatinSquare()}}, \sphinxcode{\sphinxupquote{DesignLatinSquare()}},  \sphinxcode{\sphinxupquote{RepeatList()}},    \sphinxcode{\sphinxupquote{LatinSquare()}}, \sphinxcode{\sphinxupquote{Shuffle()}}

\index{DesignFullCounterbalance@\spxentry{DesignFullCounterbalance}}\ignorespaces 

\subsection{DesignFullCounterbalance()}
\label{\detokenize{reference/pebllist:designfullcounterbalance}}\label{\detokenize{reference/pebllist:index-2}}
\sphinxAtStartPar
\sphinxstylestrong{Description:}

\sphinxAtStartPar
This takes two lists as parameters, and returns a nested list           of lists that includes the full counterbalancing of both                parameter lists.  Use cautiously; this gets very large.

\sphinxAtStartPar
\sphinxstylestrong{Usage:}

\begin{sphinxVerbatim}[commandchars=\\\{\}]
\PYG{n+nf}{DesignFullCounterbalance}\PYG{p}{(}\PYG{o}{\PYGZlt{}}\PYG{n+nv}{lista}\PYG{o}{\PYGZgt{}}\PYG{p}{,}\PYG{+w}{ }\PYG{o}{\PYGZlt{}}\PYG{n+nv}{listb}\PYG{o}{\PYGZgt{}}\PYG{p}{)}
\end{sphinxVerbatim}

\sphinxAtStartPar
\sphinxstylestrong{Example:}

\begin{sphinxVerbatim}[commandchars=\\\{\}]
\PYG{n+nv}{a}\PYG{+w}{ }\PYG{o}{\PYGZlt{}\PYGZhy{}}\PYG{+w}{ }\PYG{p}{[}\PYG{l+m+mi}{1}\PYG{p}{,}\PYG{l+m+mi}{2}\PYG{p}{,}\PYG{l+m+mi}{3}\PYG{p}{]}
\PYG{n+nv}{b}\PYG{+w}{ }\PYG{o}{\PYGZlt{}\PYGZhy{}}\PYG{+w}{ }\PYG{p}{[}\PYG{l+m+mi}{9}\PYG{p}{,}\PYG{l+m+mi}{8}\PYG{p}{,}\PYG{l+m+mi}{7}\PYG{p}{]}
\PYG{n+nf}{DesignFullCounterbalance}\PYG{p}{(}\PYG{n+nv}{a}\PYG{p}{,}\PYG{n+nv}{b}\PYG{p}{)}\PYG{+w}{        }\PYG{c+c1}{\PYGZsh{} == [[1,9],[1,8],[1,7],}
\PYG{+w}{                             }\PYG{c+c1}{\PYGZsh{}     [2,9],[2,8],[2,7],}
\PYG{+w}{                             }\PYG{c+c1}{\PYGZsh{}     [3,9],[3,8],[3,7]]}
\end{sphinxVerbatim}

\sphinxAtStartPar
\sphinxstylestrong{See Also:}

\sphinxAtStartPar
\sphinxcode{\sphinxupquote{CrossFactorWithoutDuplicates()}},   \sphinxcode{\sphinxupquote{LatinSquare()}}, \sphinxcode{\sphinxupquote{Shuffle()}},   \sphinxcode{\sphinxupquote{DesignBalancedSampling()}}, \sphinxcode{\sphinxupquote{DesignGrecoLatinSquare()}},    \sphinxcode{\sphinxupquote{DesignLatinSquare()}}, \sphinxcode{\sphinxupquote{Repeat()}}, \sphinxcode{\sphinxupquote{RepeatList()}},

\index{First@\spxentry{First}}\ignorespaces 

\subsection{First()}
\label{\detokenize{reference/pebllist:first}}\label{\detokenize{reference/pebllist:index-3}}
\sphinxAtStartPar
\sphinxstyleemphasis{Returns the first item in a list.}

\sphinxAtStartPar
\sphinxstylestrong{Description:}

\sphinxAtStartPar
Returns the first item of a list.

\sphinxAtStartPar
\sphinxstylestrong{Usage:}

\begin{sphinxVerbatim}[commandchars=\\\{\}]
\PYG{n+nf}{First}\PYG{p}{(}\PYG{o}{\PYGZlt{}}\PYG{n+nv}{list}\PYG{o}{\PYGZgt{}}\PYG{p}{)}
\end{sphinxVerbatim}

\sphinxAtStartPar
\sphinxstylestrong{Example:}

\begin{sphinxVerbatim}[commandchars=\\\{\}]
\PYG{n+nf}{First}\PYG{p}{(}\PYG{p}{[}\PYG{l+m+mi}{3}\PYG{p}{,}\PYG{l+m+mi}{33}\PYG{p}{,}\PYG{l+m+mi}{132}\PYG{p}{]}\PYG{p}{)}\PYG{+w}{            }\PYG{c+c1}{\PYGZsh{} == 3}
\end{sphinxVerbatim}

\sphinxAtStartPar
\sphinxstylestrong{See Also:}

\sphinxAtStartPar
\sphinxcode{\sphinxupquote{Nth()}}, \sphinxcode{\sphinxupquote{Last()}}

\index{IsMember@\spxentry{IsMember}}\ignorespaces 

\subsection{IsMember()}
\label{\detokenize{reference/pebllist:ismember}}\label{\detokenize{reference/pebllist:index-4}}
\sphinxAtStartPar
\sphinxstylestrong{Description:}

\sphinxAtStartPar
Returns true if \sphinxcode{\sphinxupquote{\textless{}element\textgreater{}}} is a member of \sphinxcode{\sphinxupquote{\textless{}list\textgreater{}}}.

\sphinxAtStartPar
\sphinxstylestrong{Usage:}

\begin{sphinxVerbatim}[commandchars=\\\{\}]
\PYG{n+nf}{IsMember}\PYG{p}{(}\PYG{o}{\PYGZlt{}}\PYG{n+nv}{element}\PYG{o}{\PYGZgt{}}\PYG{p}{,}\PYG{o}{\PYGZlt{}}\PYG{n+nv}{list}\PYG{o}{\PYGZgt{}}\PYG{p}{)}
\end{sphinxVerbatim}

\sphinxAtStartPar
\sphinxstylestrong{Example:}

\begin{sphinxVerbatim}[commandchars=\\\{\}]
\PYG{n+nf}{IsMember}\PYG{p}{(}\PYG{l+m+mi}{2}\PYG{p}{,}\PYG{p}{[}\PYG{l+m+mi}{1}\PYG{p}{,}\PYG{l+m+mi}{4}\PYG{p}{,}\PYG{l+m+mi}{6}\PYG{p}{,}\PYG{l+m+mi}{7}\PYG{p}{,}\PYG{l+m+mi}{7}\PYG{p}{,}\PYG{l+m+mi}{7}\PYG{p}{,}\PYG{l+m+mi}{7}\PYG{p}{]}\PYG{p}{)}\PYG{+w}{          }\PYG{c+c1}{\PYGZsh{} false}
\PYG{n+nf}{IsMember}\PYG{p}{(}\PYG{l+m+mi}{2}\PYG{p}{,}\PYG{p}{[}\PYG{l+m+mi}{1}\PYG{p}{,}\PYG{l+m+mi}{4}\PYG{p}{,}\PYG{l+m+mi}{6}\PYG{p}{,}\PYG{l+m+mi}{7}\PYG{p}{,}\PYG{l+m+mi}{2}\PYG{p}{,}\PYG{l+m+mi}{7}\PYG{p}{,}\PYG{l+m+mi}{7}\PYG{p}{,}\PYG{l+m+mi}{7}\PYG{p}{]}\PYG{p}{)}\PYG{+w}{                }\PYG{c+c1}{\PYGZsh{} true}
\end{sphinxVerbatim}

\index{Last@\spxentry{Last}}\ignorespaces 

\subsection{Last()}
\label{\detokenize{reference/pebllist:last}}\label{\detokenize{reference/pebllist:index-5}}
\sphinxAtStartPar
\sphinxstyleemphasis{Returns the last item in a list.}

\sphinxAtStartPar
\sphinxstylestrong{Description:}

\sphinxAtStartPar
Returns the last item in a list. Provides faster                access to the last item of a list than does Nth().

\sphinxAtStartPar
\sphinxstylestrong{Usage:}

\begin{sphinxVerbatim}[commandchars=\\\{\}]
\PYG{n+nf}{Last}\PYG{p}{(}\PYG{o}{\PYGZlt{}}\PYG{n+nv}{list}\PYG{o}{\PYGZgt{}}\PYG{p}{)}
\end{sphinxVerbatim}

\sphinxAtStartPar
\sphinxstylestrong{Example:}

\begin{sphinxVerbatim}[commandchars=\\\{\}]
\PYG{n+nf}{Last}\PYG{p}{(}\PYG{p}{[}\PYG{l+m+mi}{1}\PYG{p}{,}\PYG{l+m+mi}{2}\PYG{p}{,}\PYG{l+m+mi}{3}\PYG{p}{,}\PYG{l+m+mi}{444}\PYG{p}{]}\PYG{p}{)}\PYG{+w}{    }\PYG{c+c1}{\PYGZsh{} == 444}
\end{sphinxVerbatim}

\sphinxAtStartPar
\sphinxstylestrong{See Also:}

\sphinxAtStartPar
\sphinxcode{\sphinxupquote{Nth()}}, \sphinxcode{\sphinxupquote{First()}}

\index{Length@\spxentry{Length}}\ignorespaces 

\subsection{Length()}
\label{\detokenize{reference/pebllist:length}}\label{\detokenize{reference/pebllist:index-6}}
\sphinxAtStartPar
\sphinxstyleemphasis{Returns the number of elements in a list.}

\sphinxAtStartPar
\sphinxstylestrong{Description:}

\sphinxAtStartPar
Returns the number of items in a list.

\sphinxAtStartPar
\sphinxstylestrong{Usage:}

\begin{sphinxVerbatim}[commandchars=\\\{\}]
\PYG{n+nf}{Length}\PYG{p}{(}\PYG{o}{\PYGZlt{}}\PYG{n+nv}{list}\PYG{o}{\PYGZgt{}}\PYG{p}{)}
\end{sphinxVerbatim}

\sphinxAtStartPar
\sphinxstylestrong{Example:}

\begin{sphinxVerbatim}[commandchars=\\\{\}]
\PYG{n+nf}{Length}\PYG{p}{(}\PYG{p}{[}\PYG{l+m+mi}{1}\PYG{p}{,}\PYG{l+m+mi}{3}\PYG{p}{,}\PYG{l+m+mi}{55}\PYG{p}{,}\PYG{l+m+mi}{1515}\PYG{p}{]}\PYG{p}{)}\PYG{+w}{        }\PYG{c+c1}{\PYGZsh{} == 4}
\end{sphinxVerbatim}

\sphinxAtStartPar
\sphinxstylestrong{See Also:}

\sphinxAtStartPar
\sphinxcode{\sphinxupquote{StringLength()}}

\index{List@\spxentry{List}}\ignorespaces 

\subsection{List()}
\label{\detokenize{reference/pebllist:list}}\label{\detokenize{reference/pebllist:index-7}}
\sphinxAtStartPar
\sphinxstyleemphasis{Makes a list out of items}

\sphinxAtStartPar
\sphinxstylestrong{Description:}

\sphinxAtStartPar
Creates a list of items. Functional version of \sphinxcode{\sphinxupquote{{[}{]}}}.

\sphinxAtStartPar
\sphinxstylestrong{Usage:}

\begin{sphinxVerbatim}[commandchars=\\\{\}]
\PYG{n+nf}{List}\PYG{p}{(}\PYG{o}{\PYGZlt{}}\PYG{n+nv}{item1}\PYG{o}{\PYGZgt{}}\PYG{p}{,}\PYG{+w}{ }\PYG{o}{\PYGZlt{}}\PYG{n+nv}{item2}\PYG{o}{\PYGZgt{}}\PYG{p}{,}\PYG{+w}{ }\PYG{p}{.}\PYG{p}{.}\PYG{p}{.}\PYG{p}{.}\PYG{p}{)}
\end{sphinxVerbatim}

\sphinxAtStartPar
\sphinxstylestrong{Example:}

\begin{sphinxVerbatim}[commandchars=\\\{\}]
\PYG{n+nf}{List}\PYG{p}{(}\PYG{l+m+mi}{1}\PYG{p}{,}\PYG{l+m+mi}{2}\PYG{p}{,}\PYG{l+m+mi}{3}\PYG{p}{,}\PYG{l+m+mi}{444}\PYG{p}{)}\PYG{+w}{              }\PYG{c+c1}{\PYGZsh{} == [1,2,3,444]}
\end{sphinxVerbatim}

\sphinxAtStartPar
\sphinxstylestrong{See Also:}

\sphinxAtStartPar
\sphinxcode{\sphinxupquote{{[} {]}()}}, \sphinxcode{\sphinxupquote{Merge()}}, \sphinxcode{\sphinxupquote{Append()}}

\index{ListToString@\spxentry{ListToString}}\ignorespaces 

\subsection{ListToString()}
\label{\detokenize{reference/pebllist:listtostring}}\label{\detokenize{reference/pebllist:index-8}}
\sphinxAtStartPar
\sphinxstylestrong{Description:}

\sphinxAtStartPar
Converts a list of things to a single string

\sphinxAtStartPar
\sphinxstylestrong{Usage:}

\begin{sphinxVerbatim}[commandchars=\\\{\}]
\PYG{n+nf}{ListToString}\PYG{p}{(}\PYG{o}{\PYGZlt{}}\PYG{n+nv}{list}\PYG{o}{\PYGZgt{}}\PYG{p}{)}
\end{sphinxVerbatim}

\sphinxAtStartPar
\sphinxstylestrong{Example:}

\begin{sphinxVerbatim}[commandchars=\\\{\}]
\PYG{n+nf}{ListToString}\PYG{p}{(}\PYG{p}{[}\PYG{l+m+mi}{1}\PYG{p}{,}\PYG{l+m+mi}{2}\PYG{p}{,}\PYG{l+m+mi}{3}\PYG{p}{,}\PYG{l+m+mi}{444}\PYG{p}{]}\PYG{p}{)}\PYG{+w}{            }\PYG{c+c1}{\PYGZsh{} == \PYGZdq{}123444\PYGZdq{}}
\PYG{n+nf}{ListToString}\PYG{p}{(}\PYG{p}{[}\PYG{l+s+s2}{\PYGZdq{}a\PYGZdq{}}\PYG{p}{,}\PYG{l+s+s2}{\PYGZdq{}b\PYGZdq{}}\PYG{p}{,}\PYG{l+s+s2}{\PYGZdq{}c\PYGZdq{}}\PYG{p}{,}\PYG{l+s+s2}{\PYGZdq{}d\PYGZdq{}}\PYG{p}{,}\PYG{l+s+s2}{\PYGZdq{}e\PYGZdq{}}\PYG{p}{]}\PYG{p}{)}\PYG{+w}{          }\PYG{c+c1}{\PYGZsh{} == \PYGZdq{}abcde\PYGZdq{}}
\end{sphinxVerbatim}

\sphinxAtStartPar
\sphinxstylestrong{See Also:}

\sphinxAtStartPar
\sphinxcode{\sphinxupquote{SubString()}}, \sphinxcode{\sphinxupquote{StringLength()}}, \sphinxcode{\sphinxupquote{ConcatenateList}}

\index{Merge@\spxentry{Merge}}\ignorespaces 

\subsection{Merge()}
\label{\detokenize{reference/pebllist:merge}}\label{\detokenize{reference/pebllist:index-9}}
\sphinxAtStartPar
\sphinxstyleemphasis{Combines two lists.}

\sphinxAtStartPar
\sphinxstylestrong{Description:}

\sphinxAtStartPar
Combines two lists, \sphinxcode{\sphinxupquote{\textless{}lista\textgreater{}}} and \sphinxcode{\sphinxupquote{\textless{}listb\textgreater{}}}, into a single list.

\sphinxAtStartPar
\sphinxstylestrong{Usage:}

\begin{sphinxVerbatim}[commandchars=\\\{\}]
\PYG{n+nf}{Merge}\PYG{p}{(}\PYG{o}{\PYGZlt{}}\PYG{n+nv}{lista}\PYG{o}{\PYGZgt{}}\PYG{p}{,}\PYG{o}{\PYGZlt{}}\PYG{n+nv}{listb}\PYG{o}{\PYGZgt{}}\PYG{p}{)}
\end{sphinxVerbatim}

\sphinxAtStartPar
\sphinxstylestrong{Example:}

\begin{sphinxVerbatim}[commandchars=\\\{\}]
\PYG{n+nf}{Merge}\PYG{p}{(}\PYG{p}{[}\PYG{l+m+mi}{1}\PYG{p}{,}\PYG{l+m+mi}{2}\PYG{p}{,}\PYG{l+m+mi}{3}\PYG{p}{]}\PYG{p}{,}\PYG{p}{[}\PYG{l+m+mi}{8}\PYG{p}{,}\PYG{l+m+mi}{9}\PYG{p}{]}\PYG{p}{)}\PYG{+w}{         }\PYG{c+c1}{\PYGZsh{} == [1,2,3,8,9]}
\end{sphinxVerbatim}

\sphinxAtStartPar
\sphinxstylestrong{See Also:}

\sphinxAtStartPar
\sphinxcode{\sphinxupquote{{[} {]}()}}, \sphinxcode{\sphinxupquote{Append()}}, \sphinxcode{\sphinxupquote{List()}}

\index{ModList@\spxentry{ModList}}\ignorespaces 

\subsection{ModList()}
\label{\detokenize{reference/pebllist:modlist}}\label{\detokenize{reference/pebllist:index-10}}
\sphinxAtStartPar
\sphinxstyleemphasis{Adds pre\sphinxhyphen{} and post\sphinxhyphen{} elements to each list member}

\sphinxAtStartPar
\sphinxstylestrong{Description:}

\sphinxAtStartPar
Modifies each element of a list with a pre\sphinxhyphen{} and post\sphinxhyphen{} string. If the list item is not a string, it will use whatever string it turns into.  This creates a new list, so it could be used to make a copy of a string\sphinxhyphen{}based list.

\sphinxAtStartPar
\sphinxstylestrong{Usage:}

\begin{sphinxVerbatim}[commandchars=\\\{\}]
\PYG{n+nf}{ModList}\PYG{p}{(}\PYG{o}{\PYGZlt{}}\PYG{n+nv}{list}\PYG{o}{\PYGZgt{}}\PYG{p}{,}\PYG{o}{\PYGZlt{}}\PYG{n+nv}{pre}\PYG{o}{\PYGZgt{}}\PYG{p}{,}\PYG{o}{\PYGZlt{}}\PYG{n+nv}{post}\PYG{o}{\PYGZgt{}}\PYG{p}{)}
\PYG{+w}{     }\PYG{n+nf}{ModList}\PYG{p}{(}\PYG{n+nv}{list}\PYG{p}{,}\PYG{l+s+s2}{\PYGZdq{}\PYGZlt{}\PYGZdq{}}\PYG{p}{,}\PYG{l+s+s2}{\PYGZdq{}\PYGZgt{}\PYGZdq{}}\PYG{p}{)}\PYG{+w}{  }\PYG{c+c1}{\PYGZsh{}\PYGZsh{}encloses each list item in brackets}
\end{sphinxVerbatim}

\sphinxAtStartPar
\sphinxstylestrong{Example:}

\begin{sphinxVerbatim}[commandchars=\\\{\}]
\PYG{n+nf}{ModList}\PYG{p}{(}\PYG{p}{[}\PYG{l+m+mi}{1}\PYG{p}{,}\PYG{l+m+mi}{2}\PYG{p}{,}\PYG{l+m+mi}{3}\PYG{p}{,}\PYG{l+m+mi}{444}\PYG{p}{]}\PYG{p}{,}\PYG{l+s+s2}{\PYGZdq{} \PYGZdq{}}\PYG{p}{,}\PYG{l+s+s2}{\PYGZdq{}\PYGZdq{}}\PYG{p}{)}
\PYG{+w}{     }\PYG{n+nf}{ModList}\PYG{p}{(}\PYG{p}{[}\PYG{l+s+s2}{\PYGZdq{}a\PYGZdq{}}\PYG{p}{,}\PYG{l+s+s2}{\PYGZdq{}b\PYGZdq{}}\PYG{p}{,}\PYG{l+s+s2}{\PYGZdq{}c\PYGZdq{}}\PYG{p}{,}\PYG{l+s+s2}{\PYGZdq{}d\PYGZdq{}}\PYG{p}{,}\PYG{l+s+s2}{\PYGZdq{}e\PYGZdq{}}\PYG{p}{]}\PYG{p}{,}\PYG{l+s+s2}{\PYGZdq{},\PYGZdq{}}\PYG{p}{,}\PYG{l+s+s2}{\PYGZdq{}\PYGZhy{}\PYGZdq{}}\PYG{p}{)}
\end{sphinxVerbatim}

\sphinxAtStartPar
\sphinxstylestrong{See Also:}

\sphinxAtStartPar
\sphinxcode{\sphinxupquote{SubString()}}, \sphinxcode{\sphinxupquote{StringLength()}}, \sphinxcode{\sphinxupquote{FoldList}},  \sphinxcode{\sphinxupquote{ConcatenateList}},

\index{Nth@\spxentry{Nth}}\ignorespaces 

\subsection{Nth()}
\label{\detokenize{reference/pebllist:nth}}\label{\detokenize{reference/pebllist:index-11}}
\sphinxAtStartPar
\sphinxstyleemphasis{Returns the nth item in a list.}

\sphinxAtStartPar
\sphinxstylestrong{Description:}

\sphinxAtStartPar
Extracts the Nth item from a list.  Indexes from 1 upwards.             \sphinxcode{\sphinxupquote{Last()}} provides faster access than \sphinxcode{\sphinxupquote{Nth()}} to the end of a list,                  which must walk along the list to the desired position.

\sphinxAtStartPar
\sphinxstylestrong{Usage:}

\begin{sphinxVerbatim}[commandchars=\\\{\}]
\PYG{n+nf}{Nth}\PYG{p}{(}\PYG{o}{\PYGZlt{}}\PYG{n+nv}{list}\PYG{o}{\PYGZgt{}}\PYG{p}{,}\PYG{+w}{ }\PYG{o}{\PYGZlt{}}\PYG{n+nv}{index}\PYG{o}{\PYGZgt{}}\PYG{p}{)}
\end{sphinxVerbatim}

\sphinxAtStartPar
\sphinxstylestrong{Example:}

\begin{sphinxVerbatim}[commandchars=\\\{\}]
\PYG{n+nv}{a}\PYG{+w}{ }\PYG{o}{\PYGZlt{}\PYGZhy{}}\PYG{+w}{ }\PYG{p}{[}\PYG{l+s+s2}{\PYGZdq{}a\PYGZdq{}}\PYG{p}{,}\PYG{l+s+s2}{\PYGZdq{}b\PYGZdq{}}\PYG{p}{,}\PYG{l+s+s2}{\PYGZdq{}c\PYGZdq{}}\PYG{p}{,}\PYG{l+s+s2}{\PYGZdq{}d\PYGZdq{}}\PYG{p}{]}
\PYG{n+nf}{Print}\PYG{p}{(}\PYG{n+nf}{Nth}\PYG{p}{(}\PYG{n+nv}{a}\PYG{p}{,}\PYG{l+m+mi}{3}\PYG{p}{)}\PYG{p}{)}\PYG{+w}{              }\PYG{c+c1}{\PYGZsh{} == \PYGZsq{}c\PYGZsq{}}
\end{sphinxVerbatim}

\sphinxAtStartPar
\sphinxstylestrong{See Also:}

\sphinxAtStartPar
\sphinxcode{\sphinxupquote{First()}}, \sphinxcode{\sphinxupquote{Last()}}, \sphinxcode{\sphinxupquote{Second()}}, \sphinxcode{\sphinxupquote{Third()}}, \sphinxcode{\sphinxupquote{Fourth()}}, \sphinxcode{\sphinxupquote{Fifth()}}

\index{Second@\spxentry{Second}}\ignorespaces 

\subsection{Second()}
\label{\detokenize{reference/pebllist:second}}\label{\detokenize{reference/pebllist:index-12}}
\sphinxAtStartPar
\sphinxstyleemphasis{Returns the second item in a list.}

\sphinxAtStartPar
\sphinxstylestrong{Description:}

\sphinxAtStartPar
Returns the second item of a list. Provides convenient access to the second element without using Nth().

\sphinxAtStartPar
\sphinxstylestrong{Usage:}

\begin{sphinxVerbatim}[commandchars=\\\{\}]
\PYG{n+nf}{Second}\PYG{p}{(}\PYG{o}{\PYGZlt{}}\PYG{n+nv}{list}\PYG{o}{\PYGZgt{}}\PYG{p}{)}
\end{sphinxVerbatim}

\sphinxAtStartPar
\sphinxstylestrong{Example:}

\begin{sphinxVerbatim}[commandchars=\\\{\}]
\PYG{n+nf}{Second}\PYG{p}{(}\PYG{p}{[}\PYG{l+m+mi}{3}\PYG{p}{,}\PYG{l+m+mi}{33}\PYG{p}{,}\PYG{l+m+mi}{132}\PYG{p}{,}\PYG{l+m+mi}{200}\PYG{p}{]}\PYG{p}{)}\PYG{+w}{               }\PYG{c+c1}{\PYGZsh{} == 33}
\end{sphinxVerbatim}

\sphinxAtStartPar
\sphinxstylestrong{See Also:}

\sphinxAtStartPar
\sphinxcode{\sphinxupquote{First()}}, \sphinxcode{\sphinxupquote{Third()}}, \sphinxcode{\sphinxupquote{Fourth()}}, \sphinxcode{\sphinxupquote{Fifth()}}, \sphinxcode{\sphinxupquote{Nth()}}, \sphinxcode{\sphinxupquote{Last()}}

\index{Third@\spxentry{Third}}\ignorespaces 

\subsection{Third()}
\label{\detokenize{reference/pebllist:third}}\label{\detokenize{reference/pebllist:index-13}}
\sphinxAtStartPar
\sphinxstyleemphasis{Returns the third item in a list.}

\sphinxAtStartPar
\sphinxstylestrong{Description:}

\sphinxAtStartPar
Returns the third item of a list. Provides convenient access to the third element without using Nth().

\sphinxAtStartPar
\sphinxstylestrong{Usage:}

\begin{sphinxVerbatim}[commandchars=\\\{\}]
\PYG{n+nf}{Third}\PYG{p}{(}\PYG{o}{\PYGZlt{}}\PYG{n+nv}{list}\PYG{o}{\PYGZgt{}}\PYG{p}{)}
\end{sphinxVerbatim}

\sphinxAtStartPar
\sphinxstylestrong{Example:}

\begin{sphinxVerbatim}[commandchars=\\\{\}]
\PYG{n+nf}{Third}\PYG{p}{(}\PYG{p}{[}\PYG{l+m+mi}{3}\PYG{p}{,}\PYG{l+m+mi}{33}\PYG{p}{,}\PYG{l+m+mi}{132}\PYG{p}{,}\PYG{l+m+mi}{200}\PYG{p}{]}\PYG{p}{)}\PYG{+w}{                }\PYG{c+c1}{\PYGZsh{} == 132}
\end{sphinxVerbatim}

\sphinxAtStartPar
\sphinxstylestrong{See Also:}

\sphinxAtStartPar
\sphinxcode{\sphinxupquote{First()}}, \sphinxcode{\sphinxupquote{Second()}}, \sphinxcode{\sphinxupquote{Fourth()}}, \sphinxcode{\sphinxupquote{Fifth()}}, \sphinxcode{\sphinxupquote{Nth()}}, \sphinxcode{\sphinxupquote{Last()}}

\index{Fourth@\spxentry{Fourth}}\ignorespaces 

\subsection{Fourth()}
\label{\detokenize{reference/pebllist:fourth}}\label{\detokenize{reference/pebllist:index-14}}
\sphinxAtStartPar
\sphinxstyleemphasis{Returns the fourth item in a list.}

\sphinxAtStartPar
\sphinxstylestrong{Description:}

\sphinxAtStartPar
Returns the fourth item of a list. Provides convenient access to the fourth element without using Nth().

\sphinxAtStartPar
\sphinxstylestrong{Usage:}

\begin{sphinxVerbatim}[commandchars=\\\{\}]
\PYG{n+nf}{Fourth}\PYG{p}{(}\PYG{o}{\PYGZlt{}}\PYG{n+nv}{list}\PYG{o}{\PYGZgt{}}\PYG{p}{)}
\end{sphinxVerbatim}

\sphinxAtStartPar
\sphinxstylestrong{Example:}

\begin{sphinxVerbatim}[commandchars=\\\{\}]
\PYG{n+nf}{Fourth}\PYG{p}{(}\PYG{p}{[}\PYG{l+m+mi}{3}\PYG{p}{,}\PYG{l+m+mi}{33}\PYG{p}{,}\PYG{l+m+mi}{132}\PYG{p}{,}\PYG{l+m+mi}{200}\PYG{p}{]}\PYG{p}{)}\PYG{+w}{               }\PYG{c+c1}{\PYGZsh{} == 200}
\end{sphinxVerbatim}

\sphinxAtStartPar
\sphinxstylestrong{See Also:}

\sphinxAtStartPar
\sphinxcode{\sphinxupquote{First()}}, \sphinxcode{\sphinxupquote{Second()}}, \sphinxcode{\sphinxupquote{Third()}}, \sphinxcode{\sphinxupquote{Fifth()}}, \sphinxcode{\sphinxupquote{Nth()}}, \sphinxcode{\sphinxupquote{Last()}}

\index{Fifth@\spxentry{Fifth}}\ignorespaces 

\subsection{Fifth()}
\label{\detokenize{reference/pebllist:fifth}}\label{\detokenize{reference/pebllist:index-15}}
\sphinxAtStartPar
\sphinxstyleemphasis{Returns the fifth item in a list.}

\sphinxAtStartPar
\sphinxstylestrong{Description:}

\sphinxAtStartPar
Returns the fifth item of a list. Provides convenient access to the fifth element without using Nth().

\sphinxAtStartPar
\sphinxstylestrong{Usage:}

\begin{sphinxVerbatim}[commandchars=\\\{\}]
\PYG{n+nf}{Fifth}\PYG{p}{(}\PYG{o}{\PYGZlt{}}\PYG{n+nv}{list}\PYG{o}{\PYGZgt{}}\PYG{p}{)}
\end{sphinxVerbatim}

\sphinxAtStartPar
\sphinxstylestrong{Example:}

\begin{sphinxVerbatim}[commandchars=\\\{\}]
\PYG{n+nf}{Fifth}\PYG{p}{(}\PYG{p}{[}\PYG{l+m+mi}{3}\PYG{p}{,}\PYG{l+m+mi}{33}\PYG{p}{,}\PYG{l+m+mi}{132}\PYG{p}{,}\PYG{l+m+mi}{200}\PYG{p}{,}\PYG{l+m+mi}{999}\PYG{p}{]}\PYG{p}{)}\PYG{+w}{            }\PYG{c+c1}{\PYGZsh{} == 999}
\end{sphinxVerbatim}

\sphinxAtStartPar
\sphinxstylestrong{See Also:}

\sphinxAtStartPar
\sphinxcode{\sphinxupquote{First()}}, \sphinxcode{\sphinxupquote{Second()}}, \sphinxcode{\sphinxupquote{Third()}}, \sphinxcode{\sphinxupquote{Fourth()}}, \sphinxcode{\sphinxupquote{Nth()}}, \sphinxcode{\sphinxupquote{Last()}}

\index{PushOnEnd@\spxentry{PushOnEnd}}\ignorespaces 

\subsection{PushOnEnd()}
\label{\detokenize{reference/pebllist:pushonend}}\label{\detokenize{reference/pebllist:index-16}}
\sphinxAtStartPar
\sphinxstylestrong{Description:}

\sphinxAtStartPar
Pushes an item onto the end of a list, modifying the list itself.  Note: \sphinxcode{\sphinxupquote{PushOnEnd}} is a more efficient replacement for \sphinxcode{\sphinxupquote{Append()}}. Unlike \sphinxcode{\sphinxupquote{Append}}, it will modify the original list as a side effect, so the following works:

\begin{sphinxVerbatim}[commandchars=\\\{\}]
  PushOnEnd(list, item)

There is no need to set the original list to the result of PushOnEnd, like you must do with Append.  However, it does in fact work, and incurs only a slight overhead, so that Append can often be replaced with PushOnEnd without worry.
\end{sphinxVerbatim}

\begin{sphinxVerbatim}[commandchars=\\\{\}]
list \PYGZlt{}\PYGZhy{}  PushOnEnd(list, item)
\end{sphinxVerbatim}

\sphinxAtStartPar
\sphinxstylestrong{Usage:}

\begin{sphinxVerbatim}[commandchars=\\\{\}]
\PYG{n+nf}{PushOnEnd}\PYG{p}{(}\PYG{o}{\PYGZlt{}}\PYG{n+nv}{list}\PYG{o}{\PYGZgt{}}\PYG{p}{,}\PYG{+w}{ }\PYG{o}{\PYGZlt{}}\PYG{n+nv}{item}\PYG{o}{\PYGZgt{}}\PYG{p}{)}
\end{sphinxVerbatim}

\sphinxAtStartPar
\sphinxstylestrong{Example:}

\begin{sphinxVerbatim}[commandchars=\\\{\}]
\PYG{n+nv}{list}\PYG{+w}{ }\PYG{o}{\PYGZlt{}\PYGZhy{}}\PYG{+w}{ }\PYG{n+nf}{Sequence}\PYG{p}{(}\PYG{l+m+mi}{1}\PYG{p}{,}\PYG{l+m+mi}{5}\PYG{p}{,}\PYG{l+m+mi}{1}\PYG{p}{)}
\PYG{n+nv}{double}\PYG{+w}{  }\PYG{o}{\PYGZlt{}\PYGZhy{}}\PYG{+w}{ }\PYG{p}{[}\PYG{p}{]}
\PYG{k}{loop}\PYG{p}{(}\PYG{n+nv}{i}\PYG{p}{,}\PYG{+w}{ }\PYG{n+nv}{list}\PYG{p}{)}
\PYG{p}{\PYGZob{}}
\PYG{+w}{  }\PYG{n+nf}{PushOnEnd}\PYG{p}{(}\PYG{n+nv}{double}\PYG{p}{,}\PYG{+w}{ }\PYG{p}{[}\PYG{n+nv}{i}\PYG{p}{,}\PYG{n+nv}{i}\PYG{p}{]}\PYG{p}{)}
\PYG{p}{\PYGZcb{}}
\PYG{n+nf}{Print}\PYG{p}{(}\PYG{n+nv}{double}\PYG{p}{)}
\PYG{c+c1}{\PYGZsh{} Produces [[1,1],[2,2],[3,3],[4,4],[5,5]]}
\end{sphinxVerbatim}

\sphinxAtStartPar
\sphinxstylestrong{See Also:}

\sphinxAtStartPar
\sphinxcode{\sphinxupquote{SetElement()}} \sphinxcode{\sphinxupquote{List()}}, \sphinxcode{\sphinxupquote{{[} {]}()}}, \sphinxcode{\sphinxupquote{Merge()}}, \sphinxcode{\sphinxupquote{PushOnEnd()}}

\index{Repeat@\spxentry{Repeat}}\ignorespaces 

\subsection{Repeat()}
\label{\detokenize{reference/pebllist:repeat}}\label{\detokenize{reference/pebllist:index-17}}
\sphinxAtStartPar
\sphinxstylestrong{Description:}

\sphinxAtStartPar
Makes and returns a list by repeating \sphinxcode{\sphinxupquote{\textless{}object\textgreater{}}} \sphinxcode{\sphinxupquote{\textless{}n\textgreater{}}} times.               Has no effect on the object. Repeat will not make new copies            of the object. If you later change the object,                  you will change every object in the list.

\sphinxAtStartPar
\sphinxstylestrong{Usage:}

\begin{sphinxVerbatim}[commandchars=\\\{\}]
\PYG{n+nf}{Repeat}\PYG{p}{(}\PYG{o}{\PYGZlt{}}\PYG{n+nv}{object}\PYG{o}{\PYGZgt{}}\PYG{p}{,}\PYG{+w}{ }\PYG{o}{\PYGZlt{}}\PYG{n+nv}{n}\PYG{o}{\PYGZgt{}}\PYG{p}{)}
\end{sphinxVerbatim}

\sphinxAtStartPar
\sphinxstylestrong{Example:}

\begin{sphinxVerbatim}[commandchars=\\\{\}]
\PYG{n+nv}{x}\PYG{+w}{ }\PYG{o}{\PYGZlt{}\PYGZhy{}}\PYG{+w}{ }\PYG{l+s+s2}{\PYGZdq{}potato\PYGZdq{}}
\PYG{n+nv}{y}\PYG{+w}{ }\PYG{o}{\PYGZlt{}\PYGZhy{}}\PYG{+w}{ }\PYG{n+nv}{repeat}\PYG{p}{(}\PYG{n+nv}{x}\PYG{p}{,}\PYG{+w}{ }\PYG{l+m+mi}{10}\PYG{p}{)}
\PYG{n+nf}{Print}\PYG{p}{(}\PYG{n+nv}{y}\PYG{p}{)}
\PYG{c+c1}{\PYGZsh{} produces [\PYGZdq{}potato\PYGZdq{},\PYGZdq{}potato\PYGZdq{},\PYGZdq{}potato\PYGZdq{},}
\PYG{+w}{            }\PYG{l+s+s2}{\PYGZdq{}potato\PYGZdq{}}\PYG{p}{,}\PYG{l+s+s2}{\PYGZdq{}potato\PYGZdq{}}\PYG{p}{,}\PYG{+w}{ }\PYG{l+s+s2}{\PYGZdq{}potato\PYGZdq{}}\PYG{p}{,}
\PYG{+w}{            }\PYG{l+s+s2}{\PYGZdq{}potato\PYGZdq{}}\PYG{p}{,}\PYG{l+s+s2}{\PYGZdq{}potato\PYGZdq{}}\PYG{p}{,}\PYG{l+s+s2}{\PYGZdq{}potato\PYGZdq{}}\PYG{p}{,}\PYG{l+s+s2}{\PYGZdq{}potato\PYGZdq{}}\PYG{p}{]}
\end{sphinxVerbatim}

\sphinxAtStartPar
\sphinxstylestrong{See Also:}

\sphinxAtStartPar
\sphinxcode{\sphinxupquote{RepeatList()}}

\index{RepeatList@\spxentry{RepeatList}}\ignorespaces 

\subsection{RepeatList()}
\label{\detokenize{reference/pebllist:repeatlist}}\label{\detokenize{reference/pebllist:index-18}}
\sphinxAtStartPar
\sphinxstylestrong{Description:}

\sphinxAtStartPar
Makes a longer list by repeating a shorter list \sphinxcode{\sphinxupquote{\textless{}n\textgreater{}}} times.          Has no effect on the list itself, but changes made to objects   in the new list will also affect the old list.

\sphinxAtStartPar
\sphinxstylestrong{Usage:}

\begin{sphinxVerbatim}[commandchars=\\\{\}]
\PYG{n+nf}{RepeatList}\PYG{p}{(}\PYG{o}{\PYGZlt{}}\PYG{n+nv}{list}\PYG{o}{\PYGZgt{}}\PYG{p}{,}\PYG{+w}{ }\PYG{o}{\PYGZlt{}}\PYG{n+nv}{n}\PYG{o}{\PYGZgt{}}\PYG{p}{)}
\end{sphinxVerbatim}

\sphinxAtStartPar
\sphinxstylestrong{Example:}

\begin{sphinxVerbatim}[commandchars=\\\{\}]
\PYG{n+nf}{RepeatList}\PYG{p}{(}\PYG{p}{[}\PYG{l+m+mi}{1}\PYG{p}{,}\PYG{l+m+mi}{2}\PYG{p}{]}\PYG{p}{,}\PYG{l+m+mi}{3}\PYG{p}{)}\PYG{+w}{ }\PYG{c+c1}{\PYGZsh{} == [1,2,1,2,1,2]}
\end{sphinxVerbatim}

\sphinxAtStartPar
\sphinxstylestrong{See Also:}

\sphinxAtStartPar
\sphinxcode{\sphinxupquote{Repeat()}}, \sphinxcode{\sphinxupquote{Merge()}}, \sphinxcode{\sphinxupquote{{[} {]}()}}

\index{Rotate@\spxentry{Rotate}}\ignorespaces 

\subsection{Rotate()}
\label{\detokenize{reference/pebllist:rotate}}\label{\detokenize{reference/pebllist:index-19}}
\sphinxAtStartPar
\sphinxstylestrong{Description:}

\sphinxAtStartPar
Returns a list created by rotating a list by \sphinxcode{\sphinxupquote{\textless{}n\textgreater{}}} items.             The new list will begin with the \sphinxcode{\sphinxupquote{\textless{}n+1\textgreater{}\textasciigrave{}\textasciigrave{}th item of the old            list (modulo its length), and contain all of its items in               order, jumping back to the beginning and ending with the \textasciigrave{}\textasciigrave{}\textless{}n\textgreater{}\textasciigrave{}\textasciigrave{}th              item. Rotate(\textasciigrave{}\textasciigrave{}\textless{}list\textgreater{}}},0) has no effect.  Rotate does not modify               the original list.

\sphinxAtStartPar
\sphinxstylestrong{Usage:}

\begin{sphinxVerbatim}[commandchars=\\\{\}]
\PYG{n+nf}{Rotate}\PYG{p}{(}\PYG{o}{\PYGZlt{}}\PYG{n+nv}{list}\PYG{o}{\PYGZhy{}}\PYG{n+nv}{of}\PYG{o}{\PYGZhy{}}\PYG{n+nv}{items}\PYG{o}{\PYGZgt{}}\PYG{p}{,}\PYG{+w}{ }\PYG{o}{\PYGZlt{}}\PYG{n+nv}{n}\PYG{o}{\PYGZgt{}}\PYG{p}{)}
\end{sphinxVerbatim}

\sphinxAtStartPar
\sphinxstylestrong{Example:}

\begin{sphinxVerbatim}[commandchars=\\\{\}]
\PYG{n+nf}{Rotate}\PYG{p}{(}\PYG{p}{[}\PYG{l+m+mi}{1}\PYG{p}{,}\PYG{l+m+mi}{11}\PYG{p}{,}\PYG{l+m+mi}{111}\PYG{p}{]}\PYG{p}{,}\PYG{l+m+mi}{1}\PYG{p}{)}\PYG{+w}{  }\PYG{c+c1}{\PYGZsh{} == [11,111,1]}
\end{sphinxVerbatim}

\sphinxAtStartPar
\sphinxstylestrong{See Also:}

\sphinxAtStartPar
\sphinxcode{\sphinxupquote{Transpose()}}

\index{Sequence@\spxentry{Sequence}}\ignorespaces 

\subsection{Sequence()}
\label{\detokenize{reference/pebllist:sequence}}\label{\detokenize{reference/pebllist:index-20}}
\sphinxAtStartPar
\sphinxstylestrong{Description:}

\sphinxAtStartPar
Makes a sequence of numbers from \sphinxcode{\sphinxupquote{\textless{}start\textgreater{}}} to   \sphinxcode{\sphinxupquote{\textless{}end\textgreater{}}} at \sphinxcode{\sphinxupquote{\textless{}step\textgreater{}}}\sphinxhyphen{}sized increments. If \sphinxcode{\sphinxupquote{\textless{}step\textgreater{}}} is   positive, \sphinxcode{\sphinxupquote{\textless{}end\textgreater{}}} must be larger than \sphinxcode{\sphinxupquote{\textless{}start\textgreater{}}}, and if   \sphinxcode{\sphinxupquote{\textless{}step\textgreater{}}} is negative, \sphinxcode{\sphinxupquote{\textless{}end\textgreater{}}} must be smaller than   \sphinxcode{\sphinxupquote{\textless{}start\textgreater{}}}. If \sphinxcode{\sphinxupquote{\textless{}start\textgreater{} + n*\textless{}step\textgreater{}}} does not exactly equal   \sphinxcode{\sphinxupquote{\textless{}end\textgreater{}}}, the last item in the sequence will be the number   closest number to \sphinxcode{\sphinxupquote{\textless{}end\textgreater{}}} in the direction of \sphinxcode{\sphinxupquote{\textless{}start\textgreater{}}}   (and thus \sphinxcode{\sphinxupquote{\textless{}step\textgreater{}}}).

\sphinxAtStartPar
\sphinxstylestrong{Usage:}

\begin{sphinxVerbatim}[commandchars=\\\{\}]
\PYG{n+nf}{Sequence}\PYG{p}{(}\PYG{o}{\PYGZlt{}}\PYG{n+nv}{start}\PYG{o}{\PYGZgt{}}\PYG{p}{,}\PYG{+w}{ }\PYG{o}{\PYGZlt{}}\PYG{n+nv}{end}\PYG{o}{\PYGZgt{}}\PYG{p}{,}\PYG{+w}{ }\PYG{o}{\PYGZlt{}}\PYG{n+nv}{step}\PYG{o}{\PYGZgt{}}\PYG{p}{)}
\end{sphinxVerbatim}

\sphinxAtStartPar
\sphinxstylestrong{Example:}

\begin{sphinxVerbatim}[commandchars=\\\{\}]
\PYG{n+nf}{Sequence}\PYG{p}{(}\PYG{l+m+mi}{0}\PYG{p}{,}\PYG{l+m+mi}{10}\PYG{p}{,}\PYG{l+m+mi}{3}\PYG{p}{)}\PYG{+w}{    }\PYG{c+c1}{\PYGZsh{} == [0,3,6,9]}
\PYG{n+nf}{Sequence}\PYG{p}{(}\PYG{l+m+mi}{0}\PYG{p}{,}\PYG{l+m+mi}{10}\PYG{p}{,}\PYG{l+m+mf}{1.5}\PYG{p}{)}\PYG{+w}{  }\PYG{c+c1}{\PYGZsh{} == [0,1.5,3,4.5, 6, 7.5, 9]}
\PYG{n+nf}{Sequence}\PYG{p}{(}\PYG{l+m+mi}{10}\PYG{p}{,}\PYG{l+m+mi}{1}\PYG{p}{,}\PYG{l+m+mi}{3}\PYG{p}{)}\PYG{+w}{    }\PYG{c+c1}{\PYGZsh{} error}
\PYG{n+nf}{Sequence}\PYG{p}{(}\PYG{l+m+mi}{10}\PYG{p}{,}\PYG{l+m+mi}{0}\PYG{p}{,}\PYG{o}{\PYGZhy{}}\PYG{l+m+mi}{1}\PYG{p}{)}\PYG{+w}{   }\PYG{c+c1}{\PYGZsh{} == [10,9,8,7,6,5,4,3,2,1]}
\end{sphinxVerbatim}

\sphinxAtStartPar
\sphinxstylestrong{See Also:}

\sphinxAtStartPar
\sphinxcode{\sphinxupquote{Repeat()}}, \sphinxcode{\sphinxupquote{RepeatList()}}

\index{SetElement@\spxentry{SetElement}}\ignorespaces 

\subsection{SetElement()}
\label{\detokenize{reference/pebllist:setelement}}\label{\detokenize{reference/pebllist:index-21}}
\sphinxAtStartPar
\sphinxstyleemphasis{Sets an element of list to value}

\sphinxAtStartPar
\sphinxstylestrong{Description:}

\sphinxAtStartPar
Efficiently alter a specific item from a list.  \sphinxcode{\sphinxupquote{SetElement}} has  length\sphinxhyphen{}constant access time, and so it can be efficient to pre\sphinxhyphen{}create a list structure and then populate it one\sphinxhyphen{}by\sphinxhyphen{}one.

\sphinxAtStartPar
\sphinxstylestrong{Usage:}

\begin{sphinxVerbatim}[commandchars=\\\{\}]
\PYG{n+nf}{SetElement}\PYG{p}{(}\PYG{o}{\PYGZlt{}}\PYG{n+nv}{list}\PYG{o}{\PYGZgt{}}\PYG{p}{,}\PYG{+w}{ }\PYG{o}{\PYGZlt{}}\PYG{n+nv}{index}\PYG{o}{\PYGZgt{}}\PYG{p}{,}\PYG{+w}{ }\PYG{o}{\PYGZlt{}}\PYG{n+nv}{value}\PYG{o}{\PYGZgt{}}\PYG{p}{)}
\end{sphinxVerbatim}

\sphinxAtStartPar
\sphinxstylestrong{Example:}

\begin{sphinxVerbatim}[commandchars=\\\{\}]
\PYG{c+c1}{\PYGZsh{}\PYGZsh{}Set a random subset of elements to their index:}
\PYG{+w}{ }\PYG{n+nv}{list}\PYG{+w}{ }\PYG{o}{\PYGZlt{}\PYGZhy{}}\PYG{+w}{ }\PYG{n+nf}{Repeat}\PYG{p}{(}\PYG{l+m+mi}{0}\PYG{p}{,}\PYG{l+m+mi}{10}\PYG{p}{)}
\PYG{+w}{  }\PYG{n+nv}{index}\PYG{+w}{ }\PYG{o}{\PYGZlt{}\PYGZhy{}}\PYG{+w}{ }\PYG{l+m+mi}{1}
\PYG{+w}{  }\PYG{k}{while}\PYG{p}{(}\PYG{n+nv}{index}\PYG{+w}{ }\PYG{o}{\PYGZlt{}=}\PYG{+w}{ }\PYG{l+m+mi}{10}\PYG{p}{)}
\PYG{+w}{  }\PYG{p}{\PYGZob{}}
\PYG{+w}{    }\PYG{k}{if}\PYG{p}{(}\PYG{n+nf}{Random}\PYG{p}{(}\PYG{p}{)}\PYG{o}{\PYGZlt{}}\PYG{l+m+mf}{.2}\PYG{p}{)}
\PYG{+w}{     }\PYG{p}{\PYGZob{}}
\PYG{+w}{        }\PYG{n+nf}{SetElement}\PYG{p}{(}\PYG{n+nv}{list}\PYG{p}{,}\PYG{n+nv}{index}\PYG{p}{,}\PYG{n+nv}{index}\PYG{p}{)}
\PYG{+w}{      }\PYG{p}{\PYGZcb{}}
\PYG{+w}{    }\PYG{n+nv}{index}\PYG{+w}{ }\PYG{o}{\PYGZlt{}\PYGZhy{}}\PYG{+w}{ }\PYG{n+nv}{index}\PYG{+w}{ }\PYG{o}{+}\PYG{+w}{ }\PYG{l+m+mi}{1}
\PYG{+w}{   }\PYG{p}{\PYGZcb{}}
\end{sphinxVerbatim}

\sphinxAtStartPar
\sphinxstylestrong{See Also:}

\sphinxAtStartPar
\sphinxcode{\sphinxupquote{Nth()}}, \sphinxcode{\sphinxupquote{Append()}}, \sphinxcode{\sphinxupquote{PushOnEnd()}}

\index{Shuffle@\spxentry{Shuffle}}\ignorespaces 

\subsection{Shuffle()}
\label{\detokenize{reference/pebllist:shuffle}}\label{\detokenize{reference/pebllist:index-22}}
\sphinxAtStartPar
\sphinxstyleemphasis{Returns a new list with the items in list shuffled randomly.}

\sphinxAtStartPar
\sphinxstylestrong{Description:}

\sphinxAtStartPar
Randomly shuffles a list.

\sphinxAtStartPar
\sphinxstylestrong{Usage:}

\begin{sphinxVerbatim}[commandchars=\\\{\}]
\PYG{n+nf}{Shuffle}\PYG{p}{(}\PYG{n+nv}{list}\PYG{p}{)}
\end{sphinxVerbatim}

\sphinxAtStartPar
\sphinxstylestrong{Example:}

\begin{sphinxVerbatim}[commandchars=\\\{\}]
\PYG{n+nf}{Print}\PYG{p}{(}\PYG{n+nf}{Shuffle}\PYG{p}{(}\PYG{p}{[}\PYG{l+m+mi}{1}\PYG{p}{,}\PYG{l+m+mi}{2}\PYG{p}{,}\PYG{l+m+mi}{3}\PYG{p}{,}\PYG{l+m+mi}{4}\PYG{p}{,}\PYG{l+m+mi}{5}\PYG{p}{]}\PYG{p}{)}\PYG{p}{)}
\PYG{c+c1}{\PYGZsh{} Results might be anything, like [5,3,2,1,4]}
\end{sphinxVerbatim}

\sphinxAtStartPar
\sphinxstylestrong{See Also:}

\sphinxAtStartPar
\sphinxcode{\sphinxupquote{Sort()}}, \sphinxcode{\sphinxupquote{SortBy()}}, \sphinxcode{\sphinxupquote{ShuffleRepeat()}},                     \sphinxcode{\sphinxupquote{ShuffleWithoutAdjacents()}}

\index{Sort@\spxentry{Sort}}\ignorespaces 

\subsection{Sort()}
\label{\detokenize{reference/pebllist:sort}}\label{\detokenize{reference/pebllist:index-23}}
\sphinxAtStartPar
\sphinxstyleemphasis{Sorts a list by its values.}

\sphinxAtStartPar
\sphinxstylestrong{Description:}

\sphinxAtStartPar
Sorts a list by its values from smallest to largest.

\sphinxAtStartPar
\sphinxstylestrong{Usage:}

\begin{sphinxVerbatim}[commandchars=\\\{\}]
\PYG{n+nf}{Sort}\PYG{p}{(}\PYG{o}{\PYGZlt{}}\PYG{n+nv}{list}\PYG{o}{\PYGZgt{}}\PYG{p}{)}
\end{sphinxVerbatim}

\sphinxAtStartPar
\sphinxstylestrong{Example:}

\begin{sphinxVerbatim}[commandchars=\\\{\}]
\PYG{n+nf}{Sort}\PYG{p}{(}\PYG{p}{[}\PYG{l+m+mi}{3}\PYG{p}{,}\PYG{l+m+mi}{4}\PYG{p}{,}\PYG{l+m+mi}{2}\PYG{p}{,}\PYG{l+m+mi}{1}\PYG{p}{,}\PYG{l+m+mi}{5}\PYG{p}{]}\PYG{p}{)}\PYG{+w}{ }\PYG{c+c1}{\PYGZsh{} == [1,2,3,4,5]}
\end{sphinxVerbatim}

\sphinxAtStartPar
\sphinxstylestrong{See Also:}

\sphinxAtStartPar
\sphinxcode{\sphinxupquote{SortBy()}}, \sphinxcode{\sphinxupquote{Shuffle()}}

\index{SortBy@\spxentry{SortBy}}\ignorespaces 

\subsection{SortBy()}
\label{\detokenize{reference/pebllist:sortby}}\label{\detokenize{reference/pebllist:index-24}}
\sphinxAtStartPar
\sphinxstylestrong{Description:}

\sphinxAtStartPar
Sorts a list by the values in another list, in ascending                order.

\sphinxAtStartPar
\sphinxstylestrong{Usage:}

\begin{sphinxVerbatim}[commandchars=\\\{\}]
\PYG{n+nf}{SortBy}\PYG{p}{(}\PYG{o}{\PYGZlt{}}\PYG{n+nv}{value}\PYG{o}{\PYGZhy{}}\PYG{n+nv}{list}\PYG{o}{\PYGZgt{}}\PYG{p}{,}\PYG{+w}{ }\PYG{o}{\PYGZlt{}}\PYG{n+nv}{key}\PYG{o}{\PYGZhy{}}\PYG{n+nv}{list}\PYG{o}{\PYGZgt{}}\PYG{p}{)}
\end{sphinxVerbatim}

\sphinxAtStartPar
\sphinxstylestrong{Example:}

\begin{sphinxVerbatim}[commandchars=\\\{\}]
\PYG{n+nf}{SortBy}\PYG{p}{(}\PYG{p}{[}\PYG{l+s+s2}{\PYGZdq{}Bobby\PYGZdq{}}\PYG{p}{,}\PYG{l+s+s2}{\PYGZdq{}Greg\PYGZdq{}}\PYG{p}{,}\PYG{l+s+s2}{\PYGZdq{}Peter\PYGZdq{}}\PYG{p}{]}\PYG{p}{,}\PYG{+w}{ }\PYG{p}{[}\PYG{l+m+mi}{3}\PYG{p}{,}\PYG{l+m+mi}{1}\PYG{p}{,}\PYG{l+m+mi}{2}\PYG{p}{]}\PYG{p}{)}
\PYG{c+c1}{\PYGZsh{} == [\PYGZdq{}Greg\PYGZdq{},\PYGZdq{}Peter\PYGZdq{},\PYGZdq{}Bobby\PYGZdq{}]}
\end{sphinxVerbatim}

\sphinxAtStartPar
\sphinxstylestrong{See Also:}

\sphinxAtStartPar
\sphinxcode{\sphinxupquote{Shuffle()}}, \sphinxcode{\sphinxupquote{Sort()}}

\index{SubList@\spxentry{SubList}}\ignorespaces 

\subsection{SubList()}
\label{\detokenize{reference/pebllist:sublist}}\label{\detokenize{reference/pebllist:index-25}}
\sphinxAtStartPar
\sphinxstyleemphasis{Returns a sublist of a list.}

\sphinxAtStartPar
\sphinxstylestrong{Description:}

\sphinxAtStartPar
Extracts a list from another list, by specifying                beginning and end points of new sublist.

\sphinxAtStartPar
\sphinxstylestrong{Usage:}

\begin{sphinxVerbatim}[commandchars=\\\{\}]
\PYG{n+nf}{SubList}\PYG{p}{(}\PYG{o}{\PYGZlt{}}\PYG{n+nv}{list}\PYG{o}{\PYGZgt{}}\PYG{p}{,}\PYG{+w}{ }\PYG{o}{\PYGZlt{}}\PYG{n+nv}{begin}\PYG{o}{\PYGZgt{}}\PYG{p}{,}\PYG{+w}{ }\PYG{o}{\PYGZlt{}}\PYG{n+nv}{end}\PYG{o}{\PYGZgt{}}\PYG{p}{)}
\end{sphinxVerbatim}

\sphinxAtStartPar
\sphinxstylestrong{Example:}

\begin{sphinxVerbatim}[commandchars=\\\{\}]
\PYG{n+nf}{SubList}\PYG{p}{(}\PYG{p}{[}\PYG{l+m+mi}{1}\PYG{p}{,}\PYG{l+m+mi}{2}\PYG{p}{,}\PYG{l+m+mi}{3}\PYG{p}{,}\PYG{l+m+mi}{4}\PYG{p}{,}\PYG{l+m+mi}{5}\PYG{p}{,}\PYG{l+m+mi}{6}\PYG{p}{]}\PYG{p}{,}\PYG{l+m+mi}{3}\PYG{p}{,}\PYG{l+m+mi}{5}\PYG{p}{)}\PYG{+w}{   }\PYG{c+c1}{\PYGZsh{} == [3,4,5]}
\end{sphinxVerbatim}

\sphinxAtStartPar
\sphinxstylestrong{See Also:}

\sphinxAtStartPar
\sphinxcode{\sphinxupquote{SubSet()}}, \sphinxcode{\sphinxupquote{ExtractListItems()}}

\index{Transpose@\spxentry{Transpose}}\ignorespaces 

\subsection{Transpose()}
\label{\detokenize{reference/pebllist:transpose}}\label{\detokenize{reference/pebllist:index-26}}
\sphinxAtStartPar
\sphinxstyleemphasis{Transposes a list of equal\sphinxhyphen{}length lists.}

\sphinxAtStartPar
\sphinxstylestrong{Description:}

\sphinxAtStartPar
Transposes or {\color{red}\bfseries{}\textasciigrave{}\textasciigrave{}}rotates’’ a list of lists.  Each   sublist must be of the same length.

\sphinxAtStartPar
\sphinxstylestrong{Usage:}

\begin{sphinxVerbatim}[commandchars=\\\{\}]
\PYG{n+nf}{Transpose}\PYG{p}{(}\PYG{o}{\PYGZlt{}}\PYG{n+nv}{list}\PYG{o}{\PYGZhy{}}\PYG{n+nv}{of}\PYG{o}{\PYGZhy{}}\PYG{n+nv}{lists}\PYG{o}{\PYGZgt{}}\PYG{p}{)}
\end{sphinxVerbatim}

\sphinxAtStartPar
\sphinxstylestrong{Example:}

\begin{sphinxVerbatim}[commandchars=\\\{\}]
\PYG{n+nf}{Transpose}\PYG{p}{(}\PYG{p}{[}\PYG{p}{[}\PYG{l+m+mi}{1}\PYG{p}{,}\PYG{l+m+mi}{11}\PYG{p}{,}\PYG{l+m+mi}{111}\PYG{p}{]}\PYG{p}{,}\PYG{p}{[}\PYG{l+m+mi}{2}\PYG{p}{,}\PYG{l+m+mi}{22}\PYG{p}{,}\PYG{l+m+mi}{222}\PYG{p}{]}\PYG{p}{,}
\PYG{+w}{           }\PYG{p}{[}\PYG{l+m+mi}{3}\PYG{p}{,}\PYG{l+m+mi}{33}\PYG{p}{,}\PYG{l+m+mi}{333}\PYG{p}{]}\PYG{p}{,}\PYG{+w}{ }\PYG{p}{[}\PYG{l+m+mi}{4}\PYG{p}{,}\PYG{l+m+mi}{44}\PYG{p}{,}\PYG{l+m+mi}{444}\PYG{p}{]}\PYG{p}{]}\PYG{p}{)}
\PYG{c+c1}{\PYGZsh{} == [[1,2,3,4],[11,22,33,44],}
\PYG{c+c1}{\PYGZsh{}      [111,222,333,444]]}
\end{sphinxVerbatim}

\sphinxAtStartPar
\sphinxstylestrong{See Also:}

\sphinxAtStartPar
\sphinxcode{\sphinxupquote{Rotate()}}


\bigskip\hrule\bigskip


\sphinxstepscope


\section{PEBLObjects \sphinxhyphen{} Graphics and Objects}
\label{\detokenize{reference/peblobjects:peblobjects-graphics-and-objects}}\label{\detokenize{reference/peblobjects::doc}}
\sphinxAtStartPar
This module contains functions for creating and manipulating graphical objects, windows, and visual elements.

\begin{sphinxShadowBox}
\sphinxstyletopictitle{Function Index}
\begin{itemize}
\item {} 
\sphinxAtStartPar
\phantomsection\label{\detokenize{reference/peblobjects:id1}}{\hyperref[\detokenize{reference/peblobjects:addobject}]{\sphinxcrossref{AddObject()}}}

\item {} 
\sphinxAtStartPar
\phantomsection\label{\detokenize{reference/peblobjects:id2}}{\hyperref[\detokenize{reference/peblobjects:bezier}]{\sphinxcrossref{Bezier()}}}

\item {} 
\sphinxAtStartPar
\phantomsection\label{\detokenize{reference/peblobjects:id3}}{\hyperref[\detokenize{reference/peblobjects:circle}]{\sphinxcrossref{Circle()}}}

\item {} 
\sphinxAtStartPar
\phantomsection\label{\detokenize{reference/peblobjects:id4}}{\hyperref[\detokenize{reference/peblobjects:draw}]{\sphinxcrossref{Draw()}}}

\item {} 
\sphinxAtStartPar
\phantomsection\label{\detokenize{reference/peblobjects:id5}}{\hyperref[\detokenize{reference/peblobjects:drawfor}]{\sphinxcrossref{DrawFor()}}}

\item {} 
\sphinxAtStartPar
\phantomsection\label{\detokenize{reference/peblobjects:id6}}{\hyperref[\detokenize{reference/peblobjects:ellipse}]{\sphinxcrossref{Ellipse()}}}

\item {} 
\sphinxAtStartPar
\phantomsection\label{\detokenize{reference/peblobjects:id7}}{\hyperref[\detokenize{reference/peblobjects:getcursorposition}]{\sphinxcrossref{GetCursorPosition()}}}

\item {} 
\sphinxAtStartPar
\phantomsection\label{\detokenize{reference/peblobjects:id8}}{\hyperref[\detokenize{reference/peblobjects:getlinebreaks}]{\sphinxcrossref{GetLineBreaks()}}}

\item {} 
\sphinxAtStartPar
\phantomsection\label{\detokenize{reference/peblobjects:id9}}{\hyperref[\detokenize{reference/peblobjects:getparent}]{\sphinxcrossref{GetParent()}}}

\item {} 
\sphinxAtStartPar
\phantomsection\label{\detokenize{reference/peblobjects:id10}}{\hyperref[\detokenize{reference/peblobjects:getpixelcolor}]{\sphinxcrossref{GetPixelColor()}}}

\item {} 
\sphinxAtStartPar
\phantomsection\label{\detokenize{reference/peblobjects:id11}}{\hyperref[\detokenize{reference/peblobjects:getproperty}]{\sphinxcrossref{GetProperty()}}}

\item {} 
\sphinxAtStartPar
\phantomsection\label{\detokenize{reference/peblobjects:id12}}{\hyperref[\detokenize{reference/peblobjects:getpropertylist}]{\sphinxcrossref{GetPropertyList()}}}

\item {} 
\sphinxAtStartPar
\phantomsection\label{\detokenize{reference/peblobjects:id13}}{\hyperref[\detokenize{reference/peblobjects:getsize}]{\sphinxcrossref{GetSize()}}}

\item {} 
\sphinxAtStartPar
\phantomsection\label{\detokenize{reference/peblobjects:id14}}{\hyperref[\detokenize{reference/peblobjects:gettext}]{\sphinxcrossref{GetText()}}}

\item {} 
\sphinxAtStartPar
\phantomsection\label{\detokenize{reference/peblobjects:id15}}{\hyperref[\detokenize{reference/peblobjects:getvocalresponsetime}]{\sphinxcrossref{GetVocalResponseTime()}}}

\item {} 
\sphinxAtStartPar
\phantomsection\label{\detokenize{reference/peblobjects:id16}}{\hyperref[\detokenize{reference/peblobjects:hide}]{\sphinxcrossref{Hide()}}}

\item {} 
\sphinxAtStartPar
\phantomsection\label{\detokenize{reference/peblobjects:id17}}{\hyperref[\detokenize{reference/peblobjects:line}]{\sphinxcrossref{Line()}}}

\item {} 
\sphinxAtStartPar
\phantomsection\label{\detokenize{reference/peblobjects:id18}}{\hyperref[\detokenize{reference/peblobjects:loadaudiofile}]{\sphinxcrossref{LoadAudioFile()}}}

\item {} 
\sphinxAtStartPar
\phantomsection\label{\detokenize{reference/peblobjects:id19}}{\hyperref[\detokenize{reference/peblobjects:loadmovie}]{\sphinxcrossref{LoadMovie()}}}

\item {} 
\sphinxAtStartPar
\phantomsection\label{\detokenize{reference/peblobjects:id20}}{\hyperref[\detokenize{reference/peblobjects:loadsound}]{\sphinxcrossref{LoadSound()}}}

\item {} 
\sphinxAtStartPar
\phantomsection\label{\detokenize{reference/peblobjects:id21}}{\hyperref[\detokenize{reference/peblobjects:makeaudioinputbuffer}]{\sphinxcrossref{MakeAudioInputBuffer()}}}

\item {} 
\sphinxAtStartPar
\phantomsection\label{\detokenize{reference/peblobjects:id22}}{\hyperref[\detokenize{reference/peblobjects:makecanvas}]{\sphinxcrossref{MakeCanvas()}}}

\item {} 
\sphinxAtStartPar
\phantomsection\label{\detokenize{reference/peblobjects:id23}}{\hyperref[\detokenize{reference/peblobjects:makecolor}]{\sphinxcrossref{MakeColor()}}}

\item {} 
\sphinxAtStartPar
\phantomsection\label{\detokenize{reference/peblobjects:id24}}{\hyperref[\detokenize{reference/peblobjects:makecolorrgb}]{\sphinxcrossref{MakeColorRGB()}}}

\item {} 
\sphinxAtStartPar
\phantomsection\label{\detokenize{reference/peblobjects:id25}}{\hyperref[\detokenize{reference/peblobjects:makecustomobject}]{\sphinxcrossref{MakeCustomObject()}}}

\item {} 
\sphinxAtStartPar
\phantomsection\label{\detokenize{reference/peblobjects:id26}}{\hyperref[\detokenize{reference/peblobjects:makefont}]{\sphinxcrossref{MakeFont()}}}

\item {} 
\sphinxAtStartPar
\phantomsection\label{\detokenize{reference/peblobjects:id27}}{\hyperref[\detokenize{reference/peblobjects:makeimage}]{\sphinxcrossref{MakeImage()}}}

\item {} 
\sphinxAtStartPar
\phantomsection\label{\detokenize{reference/peblobjects:id28}}{\hyperref[\detokenize{reference/peblobjects:makelabel}]{\sphinxcrossref{MakeLabel()}}}

\item {} 
\sphinxAtStartPar
\phantomsection\label{\detokenize{reference/peblobjects:id29}}{\hyperref[\detokenize{reference/peblobjects:makesinewave}]{\sphinxcrossref{MakeSineWave()}}}

\item {} 
\sphinxAtStartPar
\phantomsection\label{\detokenize{reference/peblobjects:id30}}{\hyperref[\detokenize{reference/peblobjects:maketextbox}]{\sphinxcrossref{MakeTextBox()}}}

\item {} 
\sphinxAtStartPar
\phantomsection\label{\detokenize{reference/peblobjects:id31}}{\hyperref[\detokenize{reference/peblobjects:makewindow}]{\sphinxcrossref{MakeWindow()}}}

\item {} 
\sphinxAtStartPar
\phantomsection\label{\detokenize{reference/peblobjects:id32}}{\hyperref[\detokenize{reference/peblobjects:move}]{\sphinxcrossref{Move()}}}

\item {} 
\sphinxAtStartPar
\phantomsection\label{\detokenize{reference/peblobjects:id33}}{\hyperref[\detokenize{reference/peblobjects:pauseplayback}]{\sphinxcrossref{PausePlayback()}}}

\item {} 
\sphinxAtStartPar
\phantomsection\label{\detokenize{reference/peblobjects:id34}}{\hyperref[\detokenize{reference/peblobjects:playbackground}]{\sphinxcrossref{PlayBackground()}}}

\item {} 
\sphinxAtStartPar
\phantomsection\label{\detokenize{reference/peblobjects:id35}}{\hyperref[\detokenize{reference/peblobjects:playforeground}]{\sphinxcrossref{PlayForeground()}}}

\item {} 
\sphinxAtStartPar
\phantomsection\label{\detokenize{reference/peblobjects:id36}}{\hyperref[\detokenize{reference/peblobjects:polygon}]{\sphinxcrossref{Polygon()}}}

\item {} 
\sphinxAtStartPar
\phantomsection\label{\detokenize{reference/peblobjects:id37}}{\hyperref[\detokenize{reference/peblobjects:printproperties}]{\sphinxcrossref{PrintProperties()}}}

\item {} 
\sphinxAtStartPar
\phantomsection\label{\detokenize{reference/peblobjects:id38}}{\hyperref[\detokenize{reference/peblobjects:propertyexists}]{\sphinxcrossref{PropertyExists()}}}

\item {} 
\sphinxAtStartPar
\phantomsection\label{\detokenize{reference/peblobjects:id39}}{\hyperref[\detokenize{reference/peblobjects:rectangle}]{\sphinxcrossref{Rectangle()}}}

\item {} 
\sphinxAtStartPar
\phantomsection\label{\detokenize{reference/peblobjects:id40}}{\hyperref[\detokenize{reference/peblobjects:removeobject}]{\sphinxcrossref{RemoveObject()}}}

\item {} 
\sphinxAtStartPar
\phantomsection\label{\detokenize{reference/peblobjects:id41}}{\hyperref[\detokenize{reference/peblobjects:resizewindow}]{\sphinxcrossref{ResizeWindow()}}}

\item {} 
\sphinxAtStartPar
\phantomsection\label{\detokenize{reference/peblobjects:id42}}{\hyperref[\detokenize{reference/peblobjects:rotozoom}]{\sphinxcrossref{RotoZoom()}}}

\item {} 
\sphinxAtStartPar
\phantomsection\label{\detokenize{reference/peblobjects:id43}}{\hyperref[\detokenize{reference/peblobjects:saveaudiotowavefile}]{\sphinxcrossref{SaveAudioToWaveFile()}}}

\item {} 
\sphinxAtStartPar
\phantomsection\label{\detokenize{reference/peblobjects:id44}}{\hyperref[\detokenize{reference/peblobjects:recordtobuffer}]{\sphinxcrossref{RecordToBuffer()}}}

\item {} 
\sphinxAtStartPar
\phantomsection\label{\detokenize{reference/peblobjects:id45}}{\hyperref[\detokenize{reference/peblobjects:startaudiomonitor}]{\sphinxcrossref{StartAudioMonitor()}}}

\item {} 
\sphinxAtStartPar
\phantomsection\label{\detokenize{reference/peblobjects:id46}}{\hyperref[\detokenize{reference/peblobjects:stopaudiomonitor}]{\sphinxcrossref{StopAudioMonitor()}}}

\item {} 
\sphinxAtStartPar
\phantomsection\label{\detokenize{reference/peblobjects:id47}}{\hyperref[\detokenize{reference/peblobjects:getaudiostats}]{\sphinxcrossref{GetAudioStats()}}}

\item {} 
\sphinxAtStartPar
\phantomsection\label{\detokenize{reference/peblobjects:id48}}{\hyperref[\detokenize{reference/peblobjects:setcursorposition}]{\sphinxcrossref{SetCursorPosition()}}}

\item {} 
\sphinxAtStartPar
\phantomsection\label{\detokenize{reference/peblobjects:id49}}{\hyperref[\detokenize{reference/peblobjects:seteditable}]{\sphinxcrossref{SetEditable()}}}

\item {} 
\sphinxAtStartPar
\phantomsection\label{\detokenize{reference/peblobjects:id50}}{\hyperref[\detokenize{reference/peblobjects:setfont}]{\sphinxcrossref{SetFont()}}}

\item {} 
\sphinxAtStartPar
\phantomsection\label{\detokenize{reference/peblobjects:id51}}{\hyperref[\detokenize{reference/peblobjects:setpanning}]{\sphinxcrossref{SetPanning()}}}

\item {} 
\sphinxAtStartPar
\phantomsection\label{\detokenize{reference/peblobjects:id52}}{\hyperref[\detokenize{reference/peblobjects:setplayrepeats}]{\sphinxcrossref{SetPlayRepeats()}}}

\item {} 
\sphinxAtStartPar
\phantomsection\label{\detokenize{reference/peblobjects:id53}}{\hyperref[\detokenize{reference/peblobjects:setproperty}]{\sphinxcrossref{SetProperty()}}}

\item {} 
\sphinxAtStartPar
\phantomsection\label{\detokenize{reference/peblobjects:id54}}{\hyperref[\detokenize{reference/peblobjects:settext}]{\sphinxcrossref{SetText()}}}

\item {} 
\sphinxAtStartPar
\phantomsection\label{\detokenize{reference/peblobjects:id55}}{\hyperref[\detokenize{reference/peblobjects:show}]{\sphinxcrossref{Show()}}}

\item {} 
\sphinxAtStartPar
\phantomsection\label{\detokenize{reference/peblobjects:id56}}{\hyperref[\detokenize{reference/peblobjects:square}]{\sphinxcrossref{Square()}}}

\item {} 
\sphinxAtStartPar
\phantomsection\label{\detokenize{reference/peblobjects:id57}}{\hyperref[\detokenize{reference/peblobjects:startplayback}]{\sphinxcrossref{StartPlayback()}}}

\item {} 
\sphinxAtStartPar
\phantomsection\label{\detokenize{reference/peblobjects:id58}}{\hyperref[\detokenize{reference/peblobjects:stop}]{\sphinxcrossref{Stop()}}}

\item {} 
\sphinxAtStartPar
\phantomsection\label{\detokenize{reference/peblobjects:id59}}{\hyperref[\detokenize{reference/peblobjects:thickline}]{\sphinxcrossref{ThickLine()}}}

\end{itemize}
\end{sphinxShadowBox}

\index{AddObject@\spxentry{AddObject}}\ignorespaces 

\subsection{AddObject()}
\label{\detokenize{reference/peblobjects:addobject}}\label{\detokenize{reference/peblobjects:index-0}}
\sphinxAtStartPar
\sphinxstyleemphasis{Adds an object to a parent object (window)}

\sphinxAtStartPar
\sphinxstylestrong{Description:}

\sphinxAtStartPar
Adds a widget to a parent window, at the top of the object stack.  Once added, the object will be drawn onto the parent last, meaning it will be on top of anything previously added.   In general, objects can be added to other objects as well as windows.  For example, you can add drawing objects (circles, etc.) to an image to annotate the image and maintain its proper x,y coordinates.  Also, if you ‘re\sphinxhyphen{}add’ an object that is already on a widget, it will get automatically removed from the window first.  This is an easy way to reorder elements on a screen.

\begin{sphinxVerbatim}[commandchars=\\\{\}]
AddObject(\PYGZlt{}obj\PYGZgt{}, \PYGZlt{}window\PYGZgt{}) AddObject(\PYGZlt{}obj\PYGZgt{}, \PYGZlt{}canvas\PYGZgt{}) AddObject(\PYGZlt{}obj\PYGZgt{}, \PYGZlt{}widget\PYGZgt{})
\end{sphinxVerbatim}

\sphinxAtStartPar
\sphinxstylestrong{Example:}

\begin{sphinxVerbatim}[commandchars=\\\{\}]
\PYG{k}{define}\PYG{+w}{ }\PYG{n+nf}{Start}\PYG{p}{(}\PYG{n+nv}{p}\PYG{p}{)}
\PYG{p}{\PYGZob{}}
\PYG{+w}{ }\PYG{n+nv}{win}\PYG{+w}{ }\PYG{o}{\PYGZlt{}\PYGZhy{}}\PYG{+w}{ }\PYG{n+nf}{MakeWindow}\PYG{p}{(}\PYG{p}{)}
\PYG{+w}{ }\PYG{n+nv}{img}\PYG{+w}{ }\PYG{o}{\PYGZlt{}\PYGZhy{}}\PYG{+w}{ }\PYG{n+nf}{MakeImage}\PYG{p}{(}\PYG{l+s+s2}{\PYGZdq{}pebl.png\PYGZdq{}}\PYG{p}{)}
\PYG{+w}{ }\PYG{n+nv}{circ}\PYG{+w}{ }\PYG{o}{\PYGZlt{}\PYGZhy{}}\PYG{+w}{ }\PYG{n+nf}{Circle}\PYG{p}{(}\PYG{l+m+mi}{20}\PYG{p}{,}\PYG{l+m+mi}{20}\PYG{p}{,}\PYG{l+m+mi}{10}\PYG{p}{,}\PYG{n+nf}{MakeColor}\PYG{p}{(}\PYG{l+s+s2}{\PYGZdq{}red\PYGZdq{}}\PYG{p}{)}\PYG{p}{,}\PYG{l+m+mi}{1}\PYG{p}{)}
\PYG{+w}{ }\PYG{n+nf}{AddObject}\PYG{p}{(}\PYG{n+nv}{circ}\PYG{p}{,}\PYG{n+nv}{img}\PYG{p}{)}
\PYG{+w}{ }\PYG{n+nf}{AddObject}\PYG{p}{(}\PYG{n+nv}{img}\PYG{p}{,}\PYG{n+nv}{win}\PYG{p}{)}
\PYG{+w}{ }\PYG{n+nf}{Move}\PYG{p}{(}\PYG{n+nv}{img}\PYG{p}{,}\PYG{l+m+mi}{100}\PYG{p}{,}\PYG{l+m+mi}{100}\PYG{p}{)}
\PYG{+w}{ }\PYG{n+nf}{Draw}\PYG{p}{(}\PYG{p}{)}
\PYG{+w}{ }\PYG{n+nf}{WaitForAnyKeyPress}\PYG{p}{(}\PYG{p}{)}
\PYG{p}{\PYGZcb{}}
\end{sphinxVerbatim}

\sphinxAtStartPar
\sphinxstylestrong{See Also:}

\sphinxAtStartPar
\sphinxcode{\sphinxupquote{RemoveObject()}}

\index{Bezier@\spxentry{Bezier}}\ignorespaces 

\subsection{Bezier()}
\label{\detokenize{reference/peblobjects:bezier}}\label{\detokenize{reference/peblobjects:index-1}}
\sphinxAtStartPar
\sphinxstyleemphasis{Creates bezier curve centered at x,y with relative points}

\sphinxAtStartPar
\sphinxstylestrong{Description:}

\sphinxAtStartPar
Creates a smoothed line through the  points specified by \sphinxcode{\sphinxupquote{\textless{}xpoints\textgreater{}}}, \sphinxcode{\sphinxupquote{\textless{}ypoints\textgreater{}}}. The lists \sphinxcode{\sphinxupquote{\textless{}xpoints\textgreater{}}} and \sphinxcode{\sphinxupquote{\textless{}ypoints\textgreater{}}} are adjusted by  \sphinxcode{\sphinxupquote{\textless{}x\textgreater{}}} and \sphinxcode{\sphinxupquote{\textless{}y\textgreater{}}}, so they should be relative to 0, not the location you want the points to be at.  Like other drawn objects, the bezier must then be added to the window to appear. \textless{}steps\textgreater{} denotes how smooth the approximation will be.

\sphinxAtStartPar
\sphinxstylestrong{Usage:}

\begin{sphinxVerbatim}[commandchars=\\\{\}]
\PYG{n+nf}{Bezier}\PYG{p}{(}\PYG{o}{\PYGZlt{}}\PYG{n+nv}{x}\PYG{o}{\PYGZgt{}}\PYG{p}{,}\PYG{o}{\PYGZlt{}}\PYG{n+nv}{y}\PYG{o}{\PYGZgt{}}\PYG{p}{,}\PYG{o}{\PYGZlt{}}\PYG{n+nv}{xpoints}\PYG{o}{\PYGZgt{}}\PYG{p}{,}\PYG{o}{\PYGZlt{}}\PYG{n+nv}{ypoints}\PYG{o}{\PYGZgt{}}\PYG{p}{,}
\PYG{+w}{         }\PYG{o}{\PYGZlt{}}\PYG{n+nv}{steps}\PYG{o}{\PYGZgt{}}\PYG{p}{,}\PYG{o}{\PYGZlt{}}\PYG{n+nv}{color}\PYG{o}{\PYGZgt{}}\PYG{p}{)}
\end{sphinxVerbatim}

\sphinxAtStartPar
\sphinxstylestrong{Example:}

\begin{sphinxVerbatim}[commandchars=\\\{\}]
\PYG{n+nv}{win}\PYG{+w}{ }\PYG{o}{\PYGZlt{}\PYGZhy{}}\PYG{+w}{ }\PYG{n+nf}{MakeWindow}\PYG{p}{(}\PYG{p}{)}
\PYG{+w}{   }\PYG{c+c1}{\PYGZsh{}This makes a T}
\PYG{+w}{   }\PYG{n+nv}{xpoints}\PYG{+w}{ }\PYG{o}{\PYGZlt{}\PYGZhy{}}\PYG{+w}{ }\PYG{p}{[}\PYG{o}{\PYGZhy{}}\PYG{l+m+mi}{10}\PYG{p}{,}\PYG{l+m+mi}{10}\PYG{p}{,}\PYG{l+m+mi}{10}\PYG{p}{,}\PYG{l+m+mi}{20}\PYG{p}{,}\PYG{l+m+mi}{20}\PYG{p}{,}\PYG{o}{\PYGZhy{}}\PYG{l+m+mi}{20}\PYG{p}{,}\PYG{o}{\PYGZhy{}}\PYG{l+m+mi}{20}\PYG{p}{,}\PYG{o}{\PYGZhy{}}\PYG{l+m+mi}{10}\PYG{p}{]}
\PYG{+w}{   }\PYG{n+nv}{ypoints}\PYG{+w}{ }\PYG{o}{\PYGZlt{}\PYGZhy{}}\PYG{+w}{ }\PYG{p}{[}\PYG{o}{\PYGZhy{}}\PYG{l+m+mi}{20}\PYG{p}{,}\PYG{o}{\PYGZhy{}}\PYG{l+m+mi}{20}\PYG{p}{,}\PYG{l+m+mi}{40}\PYG{p}{,}\PYG{l+m+mi}{40}\PYG{p}{,}\PYG{l+m+mi}{50}\PYG{p}{,}\PYG{l+m+mi}{50}\PYG{p}{,}\PYG{l+m+mi}{40}\PYG{p}{,}\PYG{l+m+mi}{40}\PYG{p}{]}
\PYG{+w}{  }\PYG{n+nv}{p1}\PYG{+w}{ }\PYG{o}{\PYGZlt{}\PYGZhy{}}\PYG{+w}{    }\PYG{n+nf}{Bezier}\PYG{p}{(}\PYG{l+m+mi}{100}\PYG{p}{,}\PYG{l+m+mi}{100}\PYG{p}{,}\PYG{n+nv}{xpoints}\PYG{p}{,}\PYG{+w}{ }\PYG{n+nv}{ypoints}\PYG{p}{,}
\PYG{+w}{           }\PYG{l+m+mi}{5}\PYG{p}{,}\PYG{+w}{ }\PYG{n+nf}{MakeColor}\PYG{p}{(}\PYG{l+s+s2}{\PYGZdq{}black\PYGZdq{}}\PYG{p}{)}\PYG{p}{)}
\PYG{+w}{  }\PYG{n+nf}{AddObject}\PYG{p}{(}\PYG{n+nv}{p1}\PYG{p}{,}\PYG{n+nv}{win}\PYG{p}{)}
\PYG{+w}{  }\PYG{n+nf}{Draw}\PYG{p}{(}\PYG{p}{)}
\end{sphinxVerbatim}

\sphinxAtStartPar
\sphinxstylestrong{See Also:}

\sphinxAtStartPar
\sphinxcode{\sphinxupquote{BlockE()}}, \sphinxcode{\sphinxupquote{Polygon()}}, \sphinxcode{\sphinxupquote{MakeStarPoints()}}, \sphinxcode{\sphinxupquote{MakeNGonPoints()}}

\index{Circle@\spxentry{Circle}}\ignorespaces 

\subsection{Circle()}
\label{\detokenize{reference/peblobjects:circle}}\label{\detokenize{reference/peblobjects:index-2}}
\sphinxAtStartPar
\sphinxstyleemphasis{Creates circle with radius r centered at position x,y}

\sphinxAtStartPar
\sphinxstylestrong{Description:}

\sphinxAtStartPar
Creates a circle for graphing at x,y with radius r.   Circles must be added to a parent widget before it can be drawn; it   may be added to widgets other than a base window. The properties of   circles may be changed by accessing their properties directly,   including the FILLED property which makes the object an outline   versus a filled shape.

\sphinxAtStartPar
\sphinxstylestrong{Usage:}

\begin{sphinxVerbatim}[commandchars=\\\{\}]
\PYG{n+nf}{Circle}\PYG{p}{(}\PYG{o}{\PYGZlt{}}\PYG{n+nv}{x}\PYG{o}{\PYGZgt{}}\PYG{p}{,}\PYG{+w}{ }\PYG{o}{\PYGZlt{}}\PYG{n+nv}{y}\PYG{o}{\PYGZgt{}}\PYG{p}{,}\PYG{+w}{ }\PYG{o}{\PYGZlt{}}\PYG{n+nv}{r}\PYG{o}{\PYGZgt{}}\PYG{p}{,}\PYG{o}{\PYGZlt{}}\PYG{n+nv}{color}\PYG{o}{\PYGZgt{}}\PYG{p}{)}
\end{sphinxVerbatim}

\sphinxAtStartPar
\sphinxstylestrong{Example:}

\begin{sphinxVerbatim}[commandchars=\\\{\}]
\PYG{n+nv}{c}\PYG{+w}{ }\PYG{o}{\PYGZlt{}\PYGZhy{}}\PYG{+w}{ }\PYG{n+nf}{Circle}\PYG{p}{(}\PYG{l+m+mi}{30}\PYG{p}{,}\PYG{l+m+mi}{30}\PYG{p}{,}\PYG{l+m+mi}{20}\PYG{p}{,}\PYG{+w}{ }\PYG{n+nf}{MakeColor}\PYG{p}{(}\PYG{n+nv+vg}{green}\PYG{p}{)}\PYG{p}{)}
\PYG{+w}{  }\PYG{n+nf}{AddObject}\PYG{p}{(}\PYG{n+nv}{c}\PYG{p}{,}\PYG{+w}{ }\PYG{n+nv}{win}\PYG{p}{)}
\PYG{+w}{  }\PYG{n+nf}{Draw}\PYG{p}{(}\PYG{p}{)}
\end{sphinxVerbatim}

\sphinxAtStartPar
\sphinxstylestrong{See Also:}

\sphinxAtStartPar
\sphinxcode{\sphinxupquote{Square()}}, \sphinxcode{\sphinxupquote{Ellipse()}}, \sphinxcode{\sphinxupquote{Rectangle()}}, \sphinxcode{\sphinxupquote{Line()}}

\index{Draw@\spxentry{Draw}}\ignorespaces 

\subsection{Draw()}
\label{\detokenize{reference/peblobjects:draw}}\label{\detokenize{reference/peblobjects:index-3}}
\sphinxAtStartPar
\sphinxstyleemphasis{Redraws a widget and its children}

\sphinxAtStartPar
\sphinxstylestrong{Description:}

\sphinxAtStartPar
Redraws the screen or a specific widget.

\sphinxAtStartPar
\sphinxstylestrong{Usage:}

\begin{sphinxVerbatim}[commandchars=\\\{\}]
\PYG{n+nf}{Draw}\PYG{p}{(}\PYG{p}{)}
\PYG{n+nf}{Draw}\PYG{p}{(}\PYG{o}{\PYGZlt{}}\PYG{n+nv}{object}\PYG{o}{\PYGZgt{}}\PYG{p}{)}
\end{sphinxVerbatim}

\sphinxAtStartPar
\sphinxstylestrong{See Also:}

\sphinxAtStartPar
\sphinxcode{\sphinxupquote{DrawFor()}}, \sphinxcode{\sphinxupquote{Show()}}, \sphinxcode{\sphinxupquote{Hide()}}

\index{DrawFor@\spxentry{DrawFor}}\ignorespaces 

\subsection{DrawFor()}
\label{\detokenize{reference/peblobjects:drawfor}}\label{\detokenize{reference/peblobjects:index-4}}
\sphinxAtStartPar
\sphinxstylestrong{Description:}

\sphinxAtStartPar
Draws a screen or widget, returning after   \sphinxcode{\sphinxupquote{\textless{}cycles\textgreater{}}} refreshes. This function currently does not work as   intended in the SDL implementation, because of a lack of control   over the refresh blank.  It may work in the future.

\sphinxAtStartPar
\sphinxstylestrong{Usage:}

\begin{sphinxVerbatim}[commandchars=\\\{\}]
\PYG{n+nf}{DrawFor}\PYG{p}{(}\PYG{+w}{ }\PYG{o}{\PYGZlt{}}\PYG{n+nv}{object}\PYG{o}{\PYGZgt{}}\PYG{p}{,}\PYG{+w}{ }\PYG{o}{\PYGZlt{}}\PYG{n+nv}{cycles}\PYG{o}{\PYGZgt{}}\PYG{p}{)}
\end{sphinxVerbatim}

\sphinxAtStartPar
\sphinxstylestrong{See Also:}

\sphinxAtStartPar
\sphinxcode{\sphinxupquote{Draw()}}, \sphinxcode{\sphinxupquote{Show()}}, \sphinxcode{\sphinxupquote{Hide()}}

\index{Ellipse@\spxentry{Ellipse}}\ignorespaces 

\subsection{Ellipse()}
\label{\detokenize{reference/peblobjects:ellipse}}\label{\detokenize{reference/peblobjects:index-5}}
\sphinxAtStartPar
\sphinxstyleemphasis{Creates ellipse with radii rx and ry centered at position x,y}

\sphinxAtStartPar
\sphinxstylestrong{Description:}

\sphinxAtStartPar
Creates a ellipse for graphing at x,y with radii   rx and ry. Ellipses are only currently definable oriented in   horizontal/vertical directions.  Ellipses  must be added   to a parent widget before it can be drawn; it may be added to   widgets other than a base window.  The properties of ellipses may be   changed by accessing their properties directly, including the FILLED   property which makes the object an outline versus a filled shape.

\sphinxAtStartPar
\sphinxstylestrong{Usage:}

\begin{sphinxVerbatim}[commandchars=\\\{\}]
\PYG{n+nf}{Ellipse}\PYG{p}{(}\PYG{o}{\PYGZlt{}}\PYG{n+nv}{x}\PYG{o}{\PYGZgt{}}\PYG{p}{,}\PYG{+w}{ }\PYG{o}{\PYGZlt{}}\PYG{n+nv}{y}\PYG{o}{\PYGZgt{}}\PYG{p}{,}\PYG{+w}{ }\PYG{o}{\PYGZlt{}}\PYG{n+nv}{rx}\PYG{o}{\PYGZgt{}}\PYG{p}{,}\PYG{+w}{ }\PYG{o}{\PYGZlt{}}\PYG{n+nv}{ry}\PYG{o}{\PYGZgt{}}\PYG{p}{,}\PYG{o}{\PYGZlt{}}\PYG{n+nv}{color}\PYG{o}{\PYGZgt{}}\PYG{p}{)}
\end{sphinxVerbatim}

\sphinxAtStartPar
\sphinxstylestrong{Example:}

\begin{sphinxVerbatim}[commandchars=\\\{\}]
\PYG{n+nv}{e}\PYG{+w}{ }\PYG{o}{\PYGZlt{}\PYGZhy{}}\PYG{+w}{ }\PYG{n+nf}{Ellipse}\PYG{p}{(}\PYG{l+m+mi}{30}\PYG{p}{,}\PYG{l+m+mi}{30}\PYG{p}{,}\PYG{l+m+mi}{20}\PYG{p}{,}\PYG{l+m+mi}{10}\PYG{p}{,}\PYG{+w}{ }\PYG{n+nf}{MakeColor}\PYG{p}{(}\PYG{n+nv+vg}{green}\PYG{p}{)}\PYG{p}{)}
\PYG{+w}{  }\PYG{n+nf}{AddObject}\PYG{p}{(}\PYG{n+nv}{e}\PYG{p}{,}\PYG{+w}{ }\PYG{n+nv}{win}\PYG{p}{)}
\PYG{+w}{  }\PYG{n+nf}{Draw}\PYG{p}{(}\PYG{p}{)}
\end{sphinxVerbatim}

\sphinxAtStartPar
\sphinxstylestrong{See Also:}

\sphinxAtStartPar
\sphinxcode{\sphinxupquote{Square()}}, \sphinxcode{\sphinxupquote{Circle()}}, \sphinxcode{\sphinxupquote{Rectangle()}}, \sphinxcode{\sphinxupquote{Line()}}

\index{GetCursorPosition@\spxentry{GetCursorPosition}}\ignorespaces 

\subsection{GetCursorPosition()}
\label{\detokenize{reference/peblobjects:getcursorposition}}\label{\detokenize{reference/peblobjects:index-6}}
\sphinxAtStartPar
\sphinxstylestrong{Description:}

\sphinxAtStartPar
Returns an integer specifying where in a textbox the edit cursor is.  The value indicates which character it is on.

\sphinxAtStartPar
\sphinxstylestrong{Usage:}

\begin{sphinxVerbatim}[commandchars=\\\{\}]
\PYG{n+nf}{GetCursorPosition}\PYG{p}{(}\PYG{o}{\PYGZlt{}}\PYG{n+nv}{textbox}\PYG{o}{\PYGZgt{}}\PYG{p}{)}
\end{sphinxVerbatim}

\sphinxAtStartPar
\sphinxstylestrong{See Also:}

\sphinxAtStartPar
\sphinxcode{\sphinxupquote{SetCursorPosition()}}, \sphinxcode{\sphinxupquote{MakeTextBox()}}, \sphinxcode{\sphinxupquote{SetText()}}

\index{GetLineBreaks@\spxentry{GetLineBreaks}}\ignorespaces 

\subsection{GetLineBreaks()}
\label{\detokenize{reference/peblobjects:getlinebreaks}}\label{\detokenize{reference/peblobjects:index-7}}
\sphinxAtStartPar
\sphinxstylestrong{Description:}

\sphinxAtStartPar
This gets linebreaks for a textbox.  It is mainly used internally for   text rendering/layout, but could be useful in other contexts.

\sphinxAtStartPar
\sphinxstylestrong{Example:}

\begin{sphinxVerbatim}[commandchars=\\\{\}]
\PYG{n+nv+vg}{gWin}\PYG{+w}{ }\PYG{o}{\PYGZlt{}\PYGZhy{}}\PYG{+w}{ }\PYG{n+nf}{MakeWindow}\PYG{p}{(}\PYG{p}{)}
\PYG{+w}{     }\PYG{n+nv}{obj}\PYG{+w}{ }\PYG{o}{\PYGZlt{}\PYGZhy{}}\PYG{+w}{ }\PYG{n+nf}{EasyTextbox}\PYG{p}{(}\PYG{l+s+s2}{\PYGZdq{}test a b c}
\PYG{l+s+s2}{      d e f}
\PYG{l+s+s2}{      g h i j k}
\PYG{l+s+s2}{      l m n o p q r}
\PYG{l+s+s2}{      s t u v\PYGZdq{}}\PYG{p}{,}\PYG{l+m+mi}{30}\PYG{p}{,}\PYG{l+m+mi}{30}\PYG{p}{,}\PYG{n+nv+vg}{gWin}\PYG{p}{,}\PYG{l+m+mi}{22}\PYG{p}{,}\PYG{+w}{ }\PYG{l+m+mi}{40}\PYG{p}{,}\PYG{l+m+mi}{200}\PYG{p}{)}

\PYG{+w}{     }\PYG{n+nv}{breaks}\PYG{+w}{ }\PYG{o}{\PYGZlt{}\PYGZhy{}}\PYG{+w}{ }\PYG{n+nf}{GetLineBreaks}\PYG{p}{(}\PYG{n+nv}{obj}\PYG{p}{)}
\PYG{+w}{     }\PYG{n+nf}{Print}\PYG{p}{(}\PYG{l+s+s2}{\PYGZdq{}Number of lines:\PYGZdq{}}\PYG{+w}{ }\PYG{o}{+}\PYG{+w}{ }\PYG{n+nf}{Length}\PYG{p}{(}\PYG{n+nv}{breaks}\PYG{p}{)}\PYG{p}{)}
\end{sphinxVerbatim}

\index{GetParent@\spxentry{GetParent}}\ignorespaces 

\subsection{GetParent()}
\label{\detokenize{reference/peblobjects:getparent}}\label{\detokenize{reference/peblobjects:index-8}}
\sphinxAtStartPar
\sphinxstylestrong{Description:}

\sphinxAtStartPar
This gets parent of a widget.

\sphinxAtStartPar
\sphinxstylestrong{Example:}

\begin{sphinxVerbatim}[commandchars=\\\{\}]
\PYG{n+nv+vg}{gWin}\PYG{+w}{ }\PYG{o}{\PYGZlt{}\PYGZhy{}}\PYG{+w}{ }\PYG{n+nf}{MakeWindow}\PYG{p}{(}\PYG{p}{)}
\PYG{+w}{   }\PYG{n+nv}{obj}\PYG{+w}{ }\PYG{o}{\PYGZlt{}\PYGZhy{}}\PYG{+w}{ }\PYG{n+nf}{EasyLabel}\PYG{p}{(}\PYG{l+s+s2}{\PYGZdq{}test\PYGZdq{}}\PYG{p}{,}\PYG{l+m+mi}{30}\PYG{p}{,}\PYG{l+m+mi}{30}\PYG{p}{,}\PYG{n+nv+vg}{gWin}\PYG{p}{,}\PYG{l+m+mi}{22}\PYG{p}{)}

\PYG{+w}{   }\PYG{c+c1}{\PYGZsh{}\PYGZsh{} later}

\PYG{+w}{   }\PYG{n+nv}{win}\PYG{+w}{ }\PYG{o}{\PYGZlt{}\PYGZhy{}}\PYG{+w}{ }\PYG{n+nf}{GetParent}\PYG{p}{(}\PYG{n+nv}{obj}\PYG{p}{)}\PYG{+w}{ }\PYG{c+c1}{\PYGZsh{}\PYGZsh{}should be gWin}
\end{sphinxVerbatim}

\index{GetPixelColor@\spxentry{GetPixelColor}}\ignorespaces 

\subsection{GetPixelColor()}
\label{\detokenize{reference/peblobjects:getpixelcolor}}\label{\detokenize{reference/peblobjects:index-9}}
\sphinxAtStartPar
\sphinxstyleemphasis{Gets the color of a specified pixel on a widget}

\sphinxAtStartPar
\sphinxstylestrong{Description:}

\sphinxAtStartPar
Gets a color object specifying the color of a particular pixel on a widget.

\sphinxAtStartPar
\sphinxstylestrong{Usage:}

\begin{sphinxVerbatim}[commandchars=\\\{\}]
\PYG{n+nv}{color}\PYG{+w}{ }\PYG{o}{\PYGZlt{}\PYGZhy{}}\PYG{+w}{ }\PYG{n+nf}{GetPixelColor}\PYG{p}{(}\PYG{n+nv}{widget}\PYG{p}{,}\PYG{n+nv}{x}\PYG{p}{,}\PYG{n+nv}{y}\PYG{p}{)}
\end{sphinxVerbatim}

\sphinxAtStartPar
\sphinxstylestrong{Example:}

\begin{sphinxVerbatim}[commandchars=\\\{\}]
\PYG{c+c1}{\PYGZsh{}\PYGZsh{}Judge brightness of a pixel}
\PYG{+w}{  }\PYG{n+nv}{img}\PYG{+w}{ }\PYG{o}{\PYGZlt{}\PYGZhy{}}\PYG{+w}{ }\PYG{n+nf}{MakeImage}\PYG{p}{(}\PYG{l+s+s2}{\PYGZdq{}test.png\PYGZdq{}}\PYG{p}{)}
\PYG{+w}{  }\PYG{n+nv}{col}\PYG{+w}{ }\PYG{o}{\PYGZlt{}\PYGZhy{}}\PYG{+w}{ }\PYG{n+nf}{GetPixelColor}\PYG{p}{(}\PYG{n+nv}{img}\PYG{p}{,}\PYG{l+m+mi}{20}\PYG{p}{,}\PYG{l+m+mi}{20}\PYG{p}{)}
\PYG{+w}{  }\PYG{n+nv}{hsv}\PYG{+w}{ }\PYG{o}{\PYGZlt{}\PYGZhy{}}\PYG{+w}{ }\PYG{n+nf}{RGBtoHSV}\PYG{p}{(}\PYG{n+nv}{col}\PYG{p}{)}
\PYG{+w}{  }\PYG{n+nf}{Print}\PYG{p}{(}\PYG{n+nf}{Third}\PYG{p}{(}\PYG{n+nv}{hsv}\PYG{p}{)}\PYG{p}{)}
\end{sphinxVerbatim}

\sphinxAtStartPar
\sphinxstylestrong{See Also:}

\sphinxAtStartPar
\sphinxcode{\sphinxupquote{SetPixel()}}

\index{GetProperty@\spxentry{GetProperty}}\ignorespaces 

\subsection{GetProperty()}
\label{\detokenize{reference/peblobjects:getproperty}}\label{\detokenize{reference/peblobjects:index-10}}
\sphinxAtStartPar
\sphinxstyleemphasis{Returns value of property}

\sphinxAtStartPar
\sphinxstylestrong{Description:}

\sphinxAtStartPar
Gets a particular named property of an object. This works for custom or built\sphinxhyphen{}in objects.  If the property does not exist, a fatal error will be signaled, and so you should check using PropertyExists() if there is any chance the property does not exist.

\sphinxAtStartPar
\sphinxstylestrong{Example:}

\begin{sphinxVerbatim}[commandchars=\\\{\}]
\PYG{n+nv}{obj}\PYG{+w}{ }\PYG{o}{\PYGZlt{}\PYGZhy{}}\PYG{+w}{ }\PYG{n+nf}{MakeCustomObject}\PYG{p}{(}\PYG{l+s+s2}{\PYGZdq{}myobject\PYGZdq{}}\PYG{p}{)}
\PYG{n+nv}{obj.taste}\PYG{+w}{ }\PYG{o}{\PYGZlt{}\PYGZhy{}}\PYG{+w}{ }\PYG{l+s+s2}{\PYGZdq{}buttery\PYGZdq{}}
\PYG{n+nv}{obj.texture}\PYG{+w}{ }\PYG{o}{\PYGZlt{}\PYGZhy{}}\PYG{+w}{ }\PYG{l+s+s2}{\PYGZdq{}creamy\PYGZdq{}}
\PYG{n+nf}{SetProperty}\PYG{p}{(}\PYG{n+nv}{obj}\PYG{p}{,}\PYG{l+s+s2}{\PYGZdq{}flavor\PYGZdq{}}\PYG{p}{,}\PYG{l+s+s2}{\PYGZdq{}tasty\PYGZdq{}}\PYG{p}{)}

\PYG{n+nv}{list}\PYG{+w}{ }\PYG{o}{\PYGZlt{}\PYGZhy{}}\PYG{+w}{ }\PYG{n+nf}{GetPropertyList}\PYG{p}{(}\PYG{n+nv}{obj}\PYG{p}{)}
\PYG{k}{loop}\PYG{p}{(}\PYG{n+nv}{i}\PYG{p}{,}\PYG{n+nv}{list}\PYG{p}{)}
\PYG{+w}{   }\PYG{p}{\PYGZob{}}
\PYG{+w}{     }\PYG{k}{if}\PYG{p}{(}\PYG{n+nf}{PropertyExists}\PYG{p}{(}\PYG{n+nv}{obj}\PYG{p}{,}\PYG{n+nv}{i}\PYG{p}{)}
\PYG{+w}{      }\PYG{p}{\PYGZob{}}
\PYG{+w}{        }\PYG{n+nf}{Print}\PYG{p}{(}\PYG{n+nv}{i}\PYG{+w}{  }\PYG{o}{+}\PYG{+w}{ }\PYG{l+s+s2}{\PYGZdq{}:  \PYGZdq{}}\PYG{+w}{ }\PYG{o}{+}\PYG{+w}{ }\PYG{n+nf}{GetProperty}\PYG{p}{(}\PYG{n+nv}{obj}\PYG{p}{,}\PYG{n+nv}{i}\PYG{p}{)}\PYG{p}{)}
\PYG{+w}{      }\PYG{p}{\PYGZcb{}}
\PYG{+w}{   }\PYG{p}{\PYGZcb{}}
\end{sphinxVerbatim}

\sphinxAtStartPar
\sphinxstylestrong{See Also:}

\sphinxAtStartPar
\sphinxcode{\sphinxupquote{GetPropertyList()}}, \sphinxcode{\sphinxupquote{PropertyExists()}}, \sphinxcode{\sphinxupquote{SetProperty()}}, \sphinxcode{\sphinxupquote{MakeCustomObject()}}, \sphinxcode{\sphinxupquote{PrintProperties()}}

\index{GetPropertyList@\spxentry{GetPropertyList}}\ignorespaces 

\subsection{GetPropertyList()}
\label{\detokenize{reference/peblobjects:getpropertylist}}\label{\detokenize{reference/peblobjects:index-11}}
\sphinxAtStartPar
\sphinxstyleemphasis{Gets a list of all the property names of an object}

\sphinxAtStartPar
\sphinxstylestrong{Description:}

\sphinxAtStartPar
Gets a list of all of the properties an object has.  This works for custom or built\sphinxhyphen{}in objects.

\sphinxAtStartPar
\sphinxstylestrong{Example:}

\begin{sphinxVerbatim}[commandchars=\\\{\}]
\PYG{n+nv}{obj}\PYG{+w}{ }\PYG{o}{\PYGZlt{}\PYGZhy{}}\PYG{+w}{ }\PYG{n+nf}{MakeCustomObject}\PYG{p}{(}\PYG{l+s+s2}{\PYGZdq{}myobject\PYGZdq{}}\PYG{p}{)}
\PYG{+w}{  }\PYG{n+nv}{obj.taste}\PYG{+w}{ }\PYG{o}{\PYGZlt{}\PYGZhy{}}\PYG{+w}{ }\PYG{l+s+s2}{\PYGZdq{}buttery\PYGZdq{}}
\PYG{+w}{  }\PYG{n+nv}{obj.texture}\PYG{+w}{ }\PYG{o}{\PYGZlt{}\PYGZhy{}}\PYG{+w}{ }\PYG{l+s+s2}{\PYGZdq{}creamy\PYGZdq{}}
\PYG{+w}{  }\PYG{n+nf}{SetProperty}\PYG{p}{(}\PYG{n+nv}{obj}\PYG{p}{,}\PYG{l+s+s2}{\PYGZdq{}flavor\PYGZdq{}}\PYG{p}{,}\PYG{l+s+s2}{\PYGZdq{}tasty\PYGZdq{}}\PYG{p}{)}

\PYG{+w}{  }\PYG{n+nv}{list}\PYG{+w}{ }\PYG{o}{\PYGZlt{}\PYGZhy{}}\PYG{+w}{ }\PYG{n+nf}{GetPropertyList}\PYG{p}{(}\PYG{n+nv}{obj}\PYG{p}{)}
\PYG{+w}{  }\PYG{k}{loop}\PYG{p}{(}\PYG{n+nv}{i}\PYG{p}{,}\PYG{n+nv}{list}\PYG{p}{)}
\PYG{+w}{   }\PYG{p}{\PYGZob{}}
\PYG{+w}{     }\PYG{k}{if}\PYG{p}{(}\PYG{n+nf}{PropertyExists}\PYG{p}{(}\PYG{n+nv}{obj}\PYG{p}{,}\PYG{n+nv}{i}\PYG{p}{)}
\PYG{+w}{      }\PYG{p}{\PYGZob{}}
\PYG{+w}{        }\PYG{n+nf}{Print}\PYG{p}{(}\PYG{n+nv}{i}\PYG{+w}{  }\PYG{o}{+}\PYG{+w}{ }\PYG{l+s+s2}{\PYGZdq{}:  \PYGZdq{}}\PYG{+w}{ }\PYG{o}{+}\PYG{+w}{ }\PYG{n+nf}{GetProperty}\PYG{p}{(}\PYG{n+nv}{obj}\PYG{p}{,}\PYG{n+nv}{i}\PYG{p}{)}\PYG{p}{)}
\PYG{+w}{      }\PYG{p}{\PYGZcb{}}
\PYG{+w}{   }\PYG{p}{\PYGZcb{}}
\end{sphinxVerbatim}

\sphinxAtStartPar
\sphinxstylestrong{See Also:}

\sphinxAtStartPar
\sphinxcode{\sphinxupquote{GetProperty()}}, \sphinxcode{\sphinxupquote{PropertyExists()}}, \sphinxcode{\sphinxupquote{SetProperty()}} \sphinxcode{\sphinxupquote{MakeCustomObject()}}, \sphinxcode{\sphinxupquote{PrintProperties()}}

\index{GetSize@\spxentry{GetSize}}\ignorespaces 

\subsection{GetSize()}
\label{\detokenize{reference/peblobjects:getsize}}\label{\detokenize{reference/peblobjects:index-12}}
\sphinxAtStartPar
\sphinxstylestrong{Description:}

\sphinxAtStartPar
Returns a list of \sphinxcode{\sphinxupquote{{[}height, width{]}}},   specifying the size of the widget.   The .width and .height properties can also be used instead of this function

\sphinxAtStartPar
\sphinxstylestrong{Usage:}

\begin{sphinxVerbatim}[commandchars=\\\{\}]
\PYG{n+nf}{GetSize}\PYG{p}{(}\PYG{o}{\PYGZlt{}}\PYG{n+nv}{widget}\PYG{o}{\PYGZgt{}}\PYG{p}{)}
\end{sphinxVerbatim}

\sphinxAtStartPar
\sphinxstylestrong{Example:}

\begin{sphinxVerbatim}[commandchars=\\\{\}]
\PYG{n+nv}{image}\PYG{+w}{ }\PYG{o}{\PYGZlt{}\PYGZhy{}}\PYG{+w}{ }\PYG{n+nf}{MakeImage}\PYG{p}{(}\PYG{l+s+s2}{\PYGZdq{}stim1.bmp\PYGZdq{}}\PYG{p}{)}
\PYG{n+nv}{xy}\PYG{+w}{ }\PYG{o}{\PYGZlt{}\PYGZhy{}}\PYG{+w}{ }\PYG{n+nf}{GetSize}\PYG{p}{(}\PYG{n+nv}{image}\PYG{p}{)}
\PYG{n+nv}{x}\PYG{+w}{ }\PYG{o}{\PYGZlt{}\PYGZhy{}}\PYG{+w}{ }\PYG{n+nf}{Nth}\PYG{p}{(}\PYG{n+nv}{xy}\PYG{p}{,}\PYG{+w}{ }\PYG{l+m+mi}{1}\PYG{p}{)}
\PYG{n+nv}{y}\PYG{+w}{ }\PYG{o}{\PYGZlt{}\PYGZhy{}}\PYG{+w}{ }\PYG{n+nf}{Nth}\PYG{p}{(}\PYG{n+nv}{xy}\PYG{p}{,}\PYG{+w}{ }\PYG{l+m+mi}{2}\PYG{p}{)}
\end{sphinxVerbatim}

\index{GetText@\spxentry{GetText}}\ignorespaces 

\subsection{GetText()}
\label{\detokenize{reference/peblobjects:gettext}}\label{\detokenize{reference/peblobjects:index-13}}
\sphinxAtStartPar
\sphinxstyleemphasis{Returns the text in a textbox or label}

\sphinxAtStartPar
\sphinxstylestrong{Description:}

\sphinxAtStartPar
Returns the text stored in a text object                (either a textbox or a label).  The .text properties can also   be used instead of this function.

\sphinxAtStartPar
\sphinxstylestrong{Usage:}

\begin{sphinxVerbatim}[commandchars=\\\{\}]
\PYG{n+nf}{GetText}\PYG{p}{(}\PYG{o}{\PYGZlt{}}\PYG{n+nv}{widget}\PYG{o}{\PYGZgt{}}\PYG{p}{)}
\end{sphinxVerbatim}

\sphinxAtStartPar
\sphinxstylestrong{See Also:}

\sphinxAtStartPar
\sphinxcode{\sphinxupquote{SetCursorPosition()}}, \sphinxcode{\sphinxupquote{GetCursorPosition()}}, \sphinxcode{\sphinxupquote{SetEditable()}}, \sphinxcode{\sphinxupquote{MakeTextBox()}}

\index{GetVocalResponseTime@\spxentry{GetVocalResponseTime}}\ignorespaces 

\subsection{GetVocalResponseTime()}
\label{\detokenize{reference/peblobjects:getvocalresponsetime}}\label{\detokenize{reference/peblobjects:index-14}}
\sphinxAtStartPar
\sphinxstyleemphasis{A simple voice key}

\sphinxAtStartPar
\sphinxstylestrong{Description:}

\sphinxAtStartPar
This is a simple audio amplitude voice key controlled by two parameters  \sphinxstyleemphasis{ONLY AVAILABLE ON WINDOWS AND LINUX}.

\sphinxAtStartPar
\sphinxstylestrong{Usage:}

\begin{sphinxVerbatim}[commandchars=\\\{\}]
\PYG{n+nf}{GetVocalResponseTime}\PYG{p}{(}\PYG{n+nv}{buffer}\PYG{p}{,}
\PYG{+w}{                     }\PYG{n+nv}{timethreshold}\PYG{p}{,}
\PYG{+w}{                     }\PYG{n+nv}{energythreshold}\PYG{p}{)}
\end{sphinxVerbatim}

\sphinxAtStartPar
\sphinxstylestrong{Example:}

\begin{sphinxVerbatim}[commandchars=\\\{\}]
\PYG{n+nv}{buffer}\PYG{+w}{ }\PYG{o}{\PYGZlt{}\PYGZhy{}}\PYG{+w}{ }\PYG{n+nf}{MakeAudioInputBuffer}\PYG{p}{(}\PYG{l+m+mi}{5000}\PYG{p}{)}
\PYG{+w}{  }\PYG{n+nv}{resp0}\PYG{+w}{ }\PYG{o}{\PYGZlt{}\PYGZhy{}}\PYG{+w}{  }\PYG{n+nf}{GetVocalResponseTime}\PYG{p}{(}\PYG{n+nv}{buffer}\PYG{p}{,}\PYG{l+m+mf}{.35}\PYG{p}{,}\PYG{+w}{ }\PYG{l+m+mi}{200}\PYG{p}{)}
\PYG{+w}{  }\PYG{n+nf}{SaveAudioToWaveFile}\PYG{p}{(}\PYG{l+s+s2}{\PYGZdq{}output.wav\PYGZdq{}}\PYG{p}{,}\PYG{n+nv}{buffer}\PYG{p}{)}
\end{sphinxVerbatim}

\sphinxAtStartPar
\sphinxstylestrong{See Also:}

\sphinxAtStartPar
\sphinxcode{\sphinxupquote{MakeAudioInputBuffer()}}, \sphinxcode{\sphinxupquote{SaveAudioToWaveFile()}},

\index{Hide@\spxentry{Hide}}\ignorespaces 

\subsection{Hide()}
\label{\detokenize{reference/peblobjects:hide}}\label{\detokenize{reference/peblobjects:index-15}}
\sphinxAtStartPar
\sphinxstyleemphasis{Hides an object}

\sphinxAtStartPar
\sphinxstylestrong{Description:}

\sphinxAtStartPar
Makes an object invisible, so it will not be drawn.

\sphinxAtStartPar
\sphinxstylestrong{Usage:}

\begin{sphinxVerbatim}[commandchars=\\\{\}]
\PYG{n+nf}{Hide}\PYG{p}{(}\PYG{o}{\PYGZlt{}}\PYG{n+nv}{object}\PYG{o}{\PYGZgt{}}\PYG{p}{)}
\end{sphinxVerbatim}

\sphinxAtStartPar
\sphinxstylestrong{Example:}

\begin{sphinxVerbatim}[commandchars=\\\{\}]
\PYG{n+nv}{window}\PYG{+w}{ }\PYG{o}{\PYGZlt{}\PYGZhy{}}\PYG{+w}{ }\PYG{n+nf}{MakeWindow}\PYG{p}{(}\PYG{p}{)}
\PYG{n+nv}{image1}\PYG{+w}{  }\PYG{o}{\PYGZlt{}\PYGZhy{}}\PYG{+w}{ }\PYG{n+nf}{MakeImage}\PYG{p}{(}\PYG{l+s+s2}{\PYGZdq{}pebl.bmp\PYGZdq{}}\PYG{p}{)}
\PYG{n+nv}{image2}\PYG{+w}{  }\PYG{o}{\PYGZlt{}\PYGZhy{}}\PYG{+w}{ }\PYG{n+nf}{MakeImage}\PYG{p}{(}\PYG{l+s+s2}{\PYGZdq{}pebl.bmp\PYGZdq{}}\PYG{p}{)}
\PYG{n+nf}{AddObject}\PYG{p}{(}\PYG{n+nv}{image1}\PYG{p}{,}\PYG{+w}{ }\PYG{n+nv}{window}\PYG{p}{)}
\PYG{n+nf}{AddObject}\PYG{p}{(}\PYG{n+nv}{image2}\PYG{p}{,}\PYG{+w}{ }\PYG{n+nv}{window}\PYG{p}{)}
\PYG{n+nf}{Hide}\PYG{p}{(}\PYG{n+nv}{image1}\PYG{p}{)}
\PYG{n+nf}{Hide}\PYG{p}{(}\PYG{n+nv}{image2}\PYG{p}{)}
\PYG{n+nf}{Draw}\PYG{p}{(}\PYG{p}{)}\PYG{+w}{               }\PYG{c+c1}{\PYGZsh{} empty screen will be drawn.}

\PYG{n+nf}{Wait}\PYG{p}{(}\PYG{l+m+mi}{3000}\PYG{p}{)}
\PYG{n+nf}{Show}\PYG{p}{(}\PYG{n+nv}{image2}\PYG{p}{)}
\PYG{n+nf}{Draw}\PYG{p}{(}\PYG{p}{)}\PYG{+w}{               }\PYG{c+c1}{\PYGZsh{} image2 will appear.}

\PYG{n+nf}{Hide}\PYG{p}{(}\PYG{n+nv}{image2}\PYG{p}{)}
\PYG{n+nf}{Draw}\PYG{p}{(}\PYG{p}{)}\PYG{+w}{               }\PYG{c+c1}{\PYGZsh{} image2 will disappear.}

\PYG{n+nf}{Wait}\PYG{p}{(}\PYG{l+m+mi}{1000}\PYG{p}{)}
\PYG{n+nf}{Show}\PYG{p}{(}\PYG{n+nv}{image1}\PYG{p}{)}
\PYG{n+nf}{Draw}\PYG{p}{(}\PYG{p}{)}\PYG{+w}{               }\PYG{c+c1}{\PYGZsh{} image1 will appear.}
\end{sphinxVerbatim}

\sphinxAtStartPar
\sphinxstylestrong{See Also:}

\sphinxAtStartPar
\sphinxcode{\sphinxupquote{Show()}}

\index{Line@\spxentry{Line}}\ignorespaces 

\subsection{Line()}
\label{\detokenize{reference/peblobjects:line}}\label{\detokenize{reference/peblobjects:index-16}}
\sphinxAtStartPar
\sphinxstyleemphasis{Creates line starting at x,y and ending at x+dx, y+dy}

\sphinxAtStartPar
\sphinxstylestrong{Description:}

\sphinxAtStartPar
Creates a line for graphing at x,y ending at x+dx,   y+dy.  dx and dy describe the size of the line.  Lines must be added   to a parent widget before it can be drawn; it may be added to   widgets other than a base window. Properties of lines may be   accessed and set later.

\sphinxAtStartPar
\sphinxstylestrong{Usage:}

\begin{sphinxVerbatim}[commandchars=\\\{\}]
\PYG{n+nf}{Line}\PYG{p}{(}\PYG{o}{\PYGZlt{}}\PYG{n+nv}{x}\PYG{o}{\PYGZgt{}}\PYG{p}{,}\PYG{+w}{ }\PYG{o}{\PYGZlt{}}\PYG{n+nv}{y}\PYG{o}{\PYGZgt{}}\PYG{p}{,}\PYG{+w}{ }\PYG{o}{\PYGZlt{}}\PYG{n+nv}{dx}\PYG{o}{\PYGZgt{}}\PYG{p}{,}\PYG{+w}{ }\PYG{o}{\PYGZlt{}}\PYG{n+nv}{dy}\PYG{o}{\PYGZgt{}}\PYG{p}{,}\PYG{+w}{ }\PYG{o}{\PYGZlt{}}\PYG{n+nv}{color}\PYG{o}{\PYGZgt{}}\PYG{p}{)}
\end{sphinxVerbatim}

\sphinxAtStartPar
\sphinxstylestrong{Example:}

\begin{sphinxVerbatim}[commandchars=\\\{\}]
\PYG{n+nv}{l}\PYG{+w}{ }\PYG{o}{\PYGZlt{}\PYGZhy{}}\PYG{+w}{ }\PYG{n+nf}{Line}\PYG{p}{(}\PYG{l+m+mi}{30}\PYG{p}{,}\PYG{l+m+mi}{30}\PYG{p}{,}\PYG{l+m+mi}{20}\PYG{p}{,}\PYG{l+m+mi}{20}\PYG{p}{,}\PYG{+w}{ }\PYG{n+nf}{MakeColor}\PYG{p}{(}\PYG{l+s+s2}{\PYGZdq{}green\PYGZdq{}}\PYG{p}{)}
\PYG{+w}{  }\PYG{n+nf}{AddObject}\PYG{p}{(}\PYG{n+nv}{l}\PYG{p}{,}\PYG{+w}{ }\PYG{n+nv}{win}\PYG{p}{)}
\PYG{+w}{  }\PYG{n+nf}{Draw}\PYG{p}{(}\PYG{p}{)}
\end{sphinxVerbatim}

\sphinxAtStartPar
\sphinxstylestrong{See Also:}

\sphinxAtStartPar
\sphinxcode{\sphinxupquote{Square()}}, \sphinxcode{\sphinxupquote{Ellipse()}}, \sphinxcode{\sphinxupquote{Rectangle()}}, \sphinxcode{\sphinxupquote{Circle()}}

\index{LoadAudioFile@\spxentry{LoadAudioFile}}\ignorespaces 

\subsection{LoadAudioFile()}
\label{\detokenize{reference/peblobjects:loadaudiofile}}\label{\detokenize{reference/peblobjects:index-17}}
\sphinxAtStartPar
\sphinxstyleemphasis{Load an audio file}

\sphinxAtStartPar
\sphinxstylestrong{Description:}

\sphinxAtStartPar
Loads an audio file supported by  the ffmpeg library.  It is nearly identical to LoadMovie(), but only works for audio files (.ogg, .mp3, .wav, .aiff, .wma, et.).  It creates a movie object, which can then be played using PlayMovie() or StartPlayback() functions.  Currently, only supported on Windows and Linux.  The ffmpeg (\sphinxcode{\sphinxupquote{http://ffmpeg.org}}) library supports a wide range of audio formats, including most .wav, .mp3, .ogg, .flac, .aiff, .wma, and others.   Currently, there appears to sometimes be playback problems if the audio stream is not stereo, so be sure to convert your audio to stereo. Also, there appears to be some problems with .flac data formats.  If you have problems with playback,  you should verify that your media file loads with another ffmpeg media player.

\sphinxAtStartPar
\sphinxstylestrong{Usage:}

\begin{sphinxVerbatim}[commandchars=\\\{\}]
\PYG{n+nf}{LoadAudioFile}\PYG{p}{(}\PYG{n+nv}{audiofile}\PYG{p}{)}
\end{sphinxVerbatim}

\sphinxAtStartPar
\sphinxstylestrong{Example:}

\begin{sphinxVerbatim}[commandchars=\\\{\}]
\PYG{n+nv}{movie}\PYG{+w}{ }\PYG{o}{\PYGZlt{}\PYGZhy{}}\PYG{+w}{ }\PYG{n+nf}{LoadAudioFile}\PYG{p}{(}\PYG{l+s+s2}{\PYGZdq{}instuctions.mp3\PYGZdq{}}\PYG{p}{)}
\PYG{+w}{   }\PYG{n+nf}{PrintProperties}\PYG{p}{(}\PYG{n+nv}{inst}\PYG{p}{)}
\PYG{+w}{   }\PYG{n+nf}{PlayMovie}\PYG{p}{(}\PYG{n+nv}{inst}\PYG{p}{)}
\PYG{+w}{   }\PYG{n+nf}{PausePlayback}\PYG{p}{(}\PYG{n+nv}{insnt}\PYG{p}{)}
\end{sphinxVerbatim}

\sphinxAtStartPar
\sphinxstylestrong{See Also:}

\sphinxAtStartPar
\sphinxcode{\sphinxupquote{LoadMovie()}}, \sphinxcode{\sphinxupquote{PlayMovie()}}, \sphinxcode{\sphinxupquote{StartPlayback()}}, \sphinxcode{\sphinxupquote{PausePlayback()}}

\index{LoadMovie@\spxentry{LoadMovie}}\ignorespaces 

\subsection{LoadMovie()}
\label{\detokenize{reference/peblobjects:loadmovie}}\label{\detokenize{reference/peblobjects:index-18}}
\sphinxAtStartPar
\sphinxstyleemphasis{Load a movie file}

\sphinxAtStartPar
\sphinxstylestrong{Description:}

\sphinxAtStartPar
DOES NOT WORK IN PEBL 2.0+    Loads a movie file using the ffmpeg library.  It creates a movie object, which can then be played using PlayMovie() or StartPlayback() functions.  Currently, only supported on Windows and Linux.  The ffmpeg (\sphinxcode{\sphinxupquote{http://ffmpeg.org}}) library supports a wide range of video and audio formats, including most .mpg, .avi, .ogg and .mp3 type formats.  Audio\sphinxhyphen{}only formats should load and play with LoadMovie, but another function, LoadAudioFile(), has been created for these, as they do not need to be added to a window to work.  If you have problems with playback,  you should verify that your media file loads with another ffmpeg media player.  For technical reasons, a movie MUST be loaded directly onto a window, and not another widget.

\sphinxAtStartPar
\sphinxstylestrong{Usage:}

\begin{sphinxVerbatim}[commandchars=\\\{\}]
\PYG{n+nf}{LoadMovie}\PYG{p}{(}\PYG{n+nv}{movie}\PYG{p}{,}\PYG{n+nv}{window}\PYG{p}{,}\PYG{+w}{ }\PYG{n+nv}{width}\PYG{p}{,}\PYG{+w}{ }\PYG{n+nv}{height}\PYG{p}{)}
\end{sphinxVerbatim}

\sphinxAtStartPar
\sphinxstylestrong{Example:}

\begin{sphinxVerbatim}[commandchars=\\\{\}]
\PYG{n+nv}{movie}\PYG{+w}{ }\PYG{o}{\PYGZlt{}\PYGZhy{}}\PYG{+w}{ }\PYG{n+nf}{LoadMovie}\PYG{p}{(}\PYG{l+s+s2}{\PYGZdq{}movie.avi\PYGZdq{}}\PYG{p}{,}\PYG{n+nv+vg}{gWin}\PYG{p}{,}\PYG{l+m+mi}{640}\PYG{p}{,}\PYG{l+m+mi}{480}\PYG{p}{)}
\PYG{+w}{   }\PYG{n+nf}{PrintProperties}\PYG{p}{(}\PYG{n+nv}{movie}\PYG{p}{)}
\PYG{+w}{   }\PYG{n+nf}{Move}\PYG{p}{(}\PYG{n+nv}{movie}\PYG{p}{,}\PYG{l+m+mi}{20}\PYG{p}{,}\PYG{l+m+mi}{20}\PYG{p}{)}
\PYG{+w}{   }\PYG{n+nf}{Draw}\PYG{p}{(}\PYG{p}{)}
\PYG{+w}{   }\PYG{n+nf}{StartPlayback}\PYG{p}{(}\PYG{n+nv}{movie}\PYG{p}{)}
\PYG{+w}{   }\PYG{n+nf}{Wait}\PYG{p}{(}\PYG{l+m+mi}{500}\PYG{p}{)}\PYG{+w}{ }\PYG{c+c1}{\PYGZsh{}Play 500 ms of the movie.}
\PYG{+w}{   }\PYG{n+nf}{PausePlayback}\PYG{p}{(}\PYG{n+nv}{movie}\PYG{p}{)}
\end{sphinxVerbatim}

\sphinxAtStartPar
\sphinxstylestrong{See Also:}

\sphinxAtStartPar
\sphinxcode{\sphinxupquote{LoadAudioFile()}}, \sphinxcode{\sphinxupquote{LoadMovie()}}, \sphinxcode{\sphinxupquote{PlayMovie()}}, \sphinxcode{\sphinxupquote{StartPlayback()}}, \sphinxcode{\sphinxupquote{PausePlayback()}}

\index{LoadSound@\spxentry{LoadSound}}\ignorespaces 

\subsection{LoadSound()}
\label{\detokenize{reference/peblobjects:loadsound}}\label{\detokenize{reference/peblobjects:index-19}}
\sphinxAtStartPar
\sphinxstyleemphasis{Loads a soundfile from the filename, returning a variable that can be played}

\sphinxAtStartPar
\sphinxstylestrong{Description:}

\sphinxAtStartPar
Loads a soundfile from \sphinxcode{\sphinxupquote{\textless{}filename\textgreater{}}},  returning a variable that can be played using the PlayForeground or PlayBackground functions.  \sphinxcode{\sphinxupquote{LoadSound}} As of PEBL version 2.1, LoadSound will load raw and compressed audio files of various sorts.  This includes uncompressed .wav files, .mp3, .ogg, .flac, and .midi files. This is based on the sdl2\_mixer library, and so more details about the file formats accepted can be found by examining that library.  Examples of using LoadSound are found in \sphinxcode{\sphinxupquote{demo tests testaudio.pbl}}  When the file gets loaded, it gets automatically transcoded into a stereo 44100\sphinxhyphen{}sampling rate audio stream, regardless of its original playback rate.  We have reports that in some cases, this can cause some problems, especially if a mono file gets loaded multiple times in an experiment. If you experience playback problems, try converting your audio to  stereo 44100 hz and see if it helps.

\sphinxAtStartPar
\sphinxstylestrong{Usage:}

\begin{sphinxVerbatim}[commandchars=\\\{\}]
\PYG{n+nf}{LoadSound}\PYG{p}{(}\PYG{o}{\PYGZlt{}}\PYG{n+nv}{filename}\PYG{o}{\PYGZgt{}}\PYG{p}{)}
\end{sphinxVerbatim}

\sphinxAtStartPar
\sphinxstylestrong{Example:}

\begin{sphinxVerbatim}[commandchars=\\\{\}]
\PYG{n+nv}{woof}\PYG{+w}{   }\PYG{o}{\PYGZlt{}\PYGZhy{}}\PYG{+w}{ }\PYG{n+nf}{LoadSound}\PYG{p}{(}\PYG{l+s+s2}{\PYGZdq{}dog.wav\PYGZdq{}}\PYG{p}{)}
\PYG{+w}{  }\PYG{n+nf}{PlayBackground}\PYG{p}{(}\PYG{n+nv}{woof}\PYG{p}{)}
\PYG{+w}{  }\PYG{n+nf}{Wait}\PYG{p}{(}\PYG{l+m+mi}{200}\PYG{p}{)}
\PYG{+w}{  }\PYG{n+nf}{Stop}\PYG{p}{(}\PYG{n+nv}{woof}\PYG{p}{)}
\PYG{+w}{  }\PYG{n+nf}{PlayForeground}\PYG{p}{(}\PYG{n+nv}{woof}\PYG{p}{)}
\end{sphinxVerbatim}

\sphinxAtStartPar
\sphinxstylestrong{See Also:}

\sphinxAtStartPar
\sphinxcode{\sphinxupquote{PlayForeground()}}, \sphinxcode{\sphinxupquote{PlayBackground()}}, \sphinxcode{\sphinxupquote{LoadAudioFile()}}, \sphinxcode{\sphinxupquote{LoadMovie()}}

\index{MakeAudioInputBuffer@\spxentry{MakeAudioInputBuffer}}\ignorespaces 

\subsection{MakeAudioInputBuffer()}
\label{\detokenize{reference/peblobjects:makeaudioinputbuffer}}\label{\detokenize{reference/peblobjects:index-20}}
\sphinxAtStartPar
\sphinxstyleemphasis{Creates a buffer to record audio input}

\sphinxAtStartPar
\sphinxstylestrong{Description:}

\sphinxAtStartPar
Creates a sound buffer to use for audio recording or voicekey sound input.  It is currently very simple, allowing only to set the duration.  By default, it record mono at 44100 hz.

\sphinxAtStartPar
\sphinxstylestrong{Usage:}

\begin{sphinxVerbatim}[commandchars=\\\{\}]
\PYG{n+nf}{MakeAudioInputBuffer}\PYG{p}{(}\PYG{o}{\PYGZlt{}}\PYG{n+nv}{time}\PYG{o}{\PYGZhy{}}\PYG{n+nv}{in}\PYG{o}{\PYGZhy{}}\PYG{n+nv}{ms}\PYG{o}{\PYGZgt{}}\PYG{p}{)}
\end{sphinxVerbatim}

\sphinxAtStartPar
\sphinxstylestrong{Example:}

\begin{sphinxVerbatim}[commandchars=\\\{\}]
\PYG{n+nv}{buffer}\PYG{+w}{ }\PYG{o}{\PYGZlt{}\PYGZhy{}}\PYG{+w}{ }\PYG{n+nf}{MakeAudioInputBuffer}\PYG{p}{(}\PYG{l+m+mi}{5000}\PYG{p}{)}
\PYG{+w}{  }\PYG{n+nv}{resp0}\PYG{+w}{ }\PYG{o}{\PYGZlt{}\PYGZhy{}}\PYG{+w}{  }\PYG{n+nf}{GetVocalResponseTime}\PYG{p}{(}\PYG{n+nv}{buffer}\PYG{p}{,}\PYG{l+m+mf}{.35}\PYG{p}{,}\PYG{+w}{ }\PYG{l+m+mi}{200}\PYG{p}{)}
\PYG{+w}{  }\PYG{n+nf}{SaveAudioToWaveFile}\PYG{p}{(}\PYG{l+s+s2}{\PYGZdq{}output.wav\PYGZdq{}}\PYG{p}{,}\PYG{n+nv}{buffer}\PYG{p}{)}
\end{sphinxVerbatim}

\sphinxAtStartPar
\sphinxstylestrong{See Also:}

\sphinxAtStartPar
\sphinxcode{\sphinxupquote{GetVocalResponseTime()}}, \sphinxcode{\sphinxupquote{SaveAudioToWaveFile()}},

\index{MakeCanvas@\spxentry{MakeCanvas}}\ignorespaces 

\subsection{MakeCanvas()}
\label{\detokenize{reference/peblobjects:makecanvas}}\label{\detokenize{reference/peblobjects:index-21}}
\sphinxAtStartPar
\sphinxstyleemphasis{Creates a blank canvas}

\sphinxAtStartPar
\sphinxstylestrong{Description:}

\sphinxAtStartPar
Makes a canvas object  \sphinxcode{\sphinxupquote{\textless{}x\textgreater{}}} pixels by   \sphinxcode{\sphinxupquote{y}} pixels, in color \sphinxcode{\sphinxupquote{\textless{}color\textgreater{}}}.  A canvas is an object that other objects can be attached to, and imprinted upon. When the canvas gets moved, the attached objects move as well. The background of a canvas can be made invisible by using a color with alpha channel == 0. The Setpixel and SetPoint functions let you change individual pixels on a canvas, to enable adding noise, drawing functional images, etc. A canvas gets ‘cleared’ by calling ResetCanvas(canvas). Any object added to a canvas creates an ‘imprint’ on the canvas that remains if the object is moved.  This allows you to use another image as a paintbrush on the canvas, and lets you to add noise to text. Because a text label gets re\sphinxhyphen{}rendered when its drawn, if you want to add pixel noise to a stimulus, you can create a label, add it to a canvas, then add pixel noise to the canvas.

\sphinxAtStartPar
\sphinxstylestrong{Usage:}

\begin{sphinxVerbatim}[commandchars=\\\{\}]
\PYG{n+nf}{MakeCanvas}\PYG{p}{(}\PYG{o}{\PYGZlt{}}\PYG{n+nv}{x}\PYG{o}{\PYGZgt{}}\PYG{p}{,}\PYG{+w}{ }\PYG{o}{\PYGZlt{}}\PYG{n+nv}{y}\PYG{o}{\PYGZgt{}}\PYG{p}{,}\PYG{+w}{ }\PYG{o}{\PYGZlt{}}\PYG{n+nv}{color}\PYG{o}{\PYGZgt{}}\PYG{p}{)}
\end{sphinxVerbatim}

\sphinxAtStartPar
\sphinxstylestrong{Example:}

\begin{sphinxVerbatim}[commandchars=\\\{\}]
\PYG{n+nv+vg}{gWin}\PYG{+w}{ }\PYG{o}{\PYGZlt{}\PYGZhy{}}\PYG{+w}{ }\PYG{n+nf}{MakeWindow}\PYG{p}{(}\PYG{p}{)}
\PYG{+w}{  }\PYG{n+nv}{clear}\PYG{+w}{ }\PYG{o}{\PYGZlt{}\PYGZhy{}}\PYG{+w}{ }\PYG{n+nf}{MakeColor}\PYG{p}{(}\PYG{l+s+s2}{\PYGZdq{}white\PYGZdq{}}\PYG{p}{)}
\PYG{+w}{  }\PYG{n+nv}{clear.alpha}\PYG{+w}{ }\PYG{o}{\PYGZlt{}\PYGZhy{}}\PYG{+w}{ }\PYG{l+m+mi}{0}
\PYG{+w}{  }\PYG{c+c1}{\PYGZsh{}make a transparent canvas:}
\PYG{+w}{  }\PYG{n+nv}{x}\PYG{+w}{ }\PYG{o}{\PYGZlt{}\PYGZhy{}}\PYG{+w}{ }\PYG{n+nf}{MakeCanvas}\PYG{p}{(}\PYG{l+m+mi}{300}\PYG{p}{,}\PYG{l+m+mi}{300}\PYG{p}{,}\PYG{n+nv}{clear}\PYG{p}{)}
\PYG{+w}{  }\PYG{n+nf}{AddObject}\PYG{p}{(}\PYG{n+nv}{x}\PYG{p}{,}\PYG{n+nv+vg}{gWin}\PYG{p}{)}
\PYG{+w}{  }\PYG{n+nf}{Move}\PYG{p}{(}\PYG{n+nv}{x}\PYG{p}{,}\PYG{l+m+mi}{300}\PYG{p}{,}\PYG{l+m+mi}{300}\PYG{p}{)}
\PYG{+w}{  }\PYG{n+nv}{img}\PYG{+w}{ }\PYG{o}{\PYGZlt{}\PYGZhy{}}\PYG{+w}{ }\PYG{n+nf}{MakeImage}\PYG{p}{(}\PYG{l+s+s2}{\PYGZdq{}pebl.png\PYGZdq{}}\PYG{p}{)}
\PYG{+w}{  }\PYG{n+nf}{AddObject}\PYG{p}{(}\PYG{n+nv}{img}\PYG{p}{,}\PYG{n+nv}{x}\PYG{p}{)}
\PYG{+w}{  }\PYG{n+nf}{Move}\PYG{p}{(}\PYG{n+nv}{img}\PYG{p}{,}\PYG{l+m+mi}{100}\PYG{p}{,}\PYG{l+m+mi}{100}\PYG{p}{)}
\PYG{+w}{  }\PYG{n+nf}{Draw}\PYG{p}{(}\PYG{n+nv}{x}\PYG{p}{)}\PYG{+w}{          }\PYG{c+c1}{\PYGZsh{}imprint the image on the canvas}
\PYG{+w}{  }\PYG{n+nf}{Move}\PYG{p}{(}\PYG{n+nv}{img}\PYG{p}{,}\PYG{l+m+mi}{100}\PYG{p}{,}\PYG{l+m+mi}{200}\PYG{p}{)}
\PYG{+w}{  }\PYG{n+nf}{Draw}\PYG{p}{(}\PYG{n+nv}{x}\PYG{p}{)}\PYG{+w}{          }\PYG{c+c1}{\PYGZsh{}imprint the image on the canvas}
\PYG{+w}{  }\PYG{n+nf}{Hide}\PYG{p}{(}\PYG{n+nv}{img}\PYG{p}{)}

\PYG{+w}{  }\PYG{c+c1}{\PYGZsh{}draw a line on the canvas}
\PYG{+w}{   }\PYG{n+nv}{i}\PYG{+w}{ }\PYG{o}{\PYGZlt{}\PYGZhy{}}\PYG{+w}{ }\PYG{l+m+mi}{10}
\PYG{+w}{   }\PYG{n+nv}{red}\PYG{+w}{ }\PYG{o}{\PYGZlt{}\PYGZhy{}}\PYG{+w}{ }\PYG{n+nf}{MakeColor}\PYG{p}{(}\PYG{l+s+s2}{\PYGZdq{}red\PYGZdq{}}\PYG{p}{)}
\PYG{+w}{  }\PYG{k}{while}\PYG{p}{(}\PYG{n+nv}{i}\PYG{+w}{ }\PYG{o}{\PYGZlt{}}\PYG{+w}{ }\PYG{l+m+mi}{200}\PYG{p}{)}
\PYG{+w}{   }\PYG{p}{\PYGZob{}}
\PYG{+w}{     }\PYG{n+nf}{SetPixel}\PYG{p}{(}\PYG{n+nv}{x}\PYG{p}{,}\PYG{l+m+mi}{20}\PYG{p}{,}\PYG{n+nv}{i}\PYG{p}{,}\PYG{n+nv}{red}\PYG{p}{)}
\PYG{+w}{     }\PYG{n+nv}{i}\PYG{+w}{ }\PYG{o}{\PYGZlt{}\PYGZhy{}}\PYG{+w}{ }\PYG{n+nv}{i}\PYG{+w}{ }\PYG{o}{+}\PYG{+w}{ }\PYG{l+m+mi}{1}
\PYG{+w}{   }\PYG{p}{\PYGZcb{}}
\PYG{+w}{  }\PYG{n+nf}{Draw}\PYG{p}{(}\PYG{p}{)}
\PYG{+w}{  }\PYG{n+nf}{WaitForAnyKeyPress}\PYG{p}{(}\PYG{p}{)}
\end{sphinxVerbatim}

\sphinxAtStartPar
\sphinxstylestrong{See Also:}

\sphinxAtStartPar
\sphinxcode{\sphinxupquote{MakeImage()}}, \sphinxcode{\sphinxupquote{SetPixel()}},   \sphinxcode{\sphinxupquote{MakeGabor()}}, \sphinxcode{\sphinxupquote{ResetCanvas()}}

\index{MakeColor@\spxentry{MakeColor}}\ignorespaces 

\subsection{MakeColor()}
\label{\detokenize{reference/peblobjects:makecolor}}\label{\detokenize{reference/peblobjects:index-22}}
\sphinxAtStartPar
\sphinxstyleemphasis{Creates a color based on a color name}

\sphinxAtStartPar
\sphinxstylestrong{Description:}

\sphinxAtStartPar
Makes a color from \sphinxcode{\sphinxupquote{\textless{}colorname\textgreater{}}} such as   \sphinxcode{\sphinxupquote{red\textquotesingle{}\textquotesingle{}, \textasciigrave{}\textasciigrave{}green\textquotesingle{}\textquotesingle{}, and nearly 800 others.  Color names and   corresponding RGB values can be found in \textasciigrave{}\textasciigrave{}doc/colors.txt}}.

\sphinxAtStartPar
\sphinxstylestrong{Usage:}

\begin{sphinxVerbatim}[commandchars=\\\{\}]
\PYG{n+nf}{MakeColor}\PYG{p}{(}\PYG{o}{\PYGZlt{}}\PYG{n+nv}{colorname}\PYG{o}{\PYGZgt{}}\PYG{p}{)}
\end{sphinxVerbatim}

\sphinxAtStartPar
\sphinxstylestrong{Example:}

\begin{sphinxVerbatim}[commandchars=\\\{\}]
\PYG{n+nv+vg}{green}\PYG{+w}{ }\PYG{o}{\PYGZlt{}\PYGZhy{}}\PYG{+w}{ }\PYG{n+nf}{MakeColor}\PYG{p}{(}\PYG{l+s+s2}{\PYGZdq{}green\PYGZdq{}}\PYG{p}{)}
\PYG{+w}{ }\PYG{n+nv}{black}\PYG{+w}{ }\PYG{o}{\PYGZlt{}\PYGZhy{}}\PYG{+w}{ }\PYG{n+nf}{MakeColor}\PYG{p}{(}\PYG{l+s+s2}{\PYGZdq{}black\PYGZdq{}}\PYG{p}{)}
\end{sphinxVerbatim}

\sphinxAtStartPar
\sphinxstylestrong{See Also:}

\sphinxAtStartPar
\sphinxcode{\sphinxupquote{MakeColorRGB()}}, \sphinxcode{\sphinxupquote{RGBtoHSV()}}

\index{MakeColorRGB@\spxentry{MakeColorRGB}}\ignorespaces 

\subsection{MakeColorRGB()}
\label{\detokenize{reference/peblobjects:makecolorrgb}}\label{\detokenize{reference/peblobjects:index-23}}
\sphinxAtStartPar
\sphinxstyleemphasis{Creates a color based on red, green, and blue values}

\sphinxAtStartPar
\sphinxstylestrong{Description:}

\sphinxAtStartPar
Makes an RGB color by specifying \sphinxcode{\sphinxupquote{\textless{}red\textgreater{}}},   \sphinxcode{\sphinxupquote{\textless{}green\textgreater{}}}, and \sphinxcode{\sphinxupquote{\textless{}blue\textgreater{}}} values (between 0 and 255).

\sphinxAtStartPar
\sphinxstylestrong{Usage:}

\begin{sphinxVerbatim}[commandchars=\\\{\}]
\PYG{n+nf}{MakeColorRGB}\PYG{p}{(}\PYG{o}{\PYGZlt{}}\PYG{n+nv}{red}\PYG{o}{\PYGZgt{}}\PYG{p}{,}\PYG{+w}{ }\PYG{o}{\PYGZlt{}}\PYG{n+nv+vg}{green}\PYG{o}{\PYGZgt{}}\PYG{p}{,}\PYG{+w}{ }\PYG{o}{\PYGZlt{}}\PYG{n+nv}{blue}\PYG{o}{\PYGZgt{}}\PYG{p}{)}
\end{sphinxVerbatim}

\sphinxAtStartPar
\sphinxstylestrong{See Also:}

\sphinxAtStartPar
\sphinxcode{\sphinxupquote{MakeColor()}}, \sphinxcode{\sphinxupquote{RGBtoHSV()}}

\index{MakeCustomObject@\spxentry{MakeCustomObject}}\ignorespaces 

\subsection{MakeCustomObject()}
\label{\detokenize{reference/peblobjects:makecustomobject}}\label{\detokenize{reference/peblobjects:index-24}}
\sphinxAtStartPar
\sphinxstyleemphasis{Creates custom object.}

\sphinxAtStartPar
\sphinxstylestrong{Description:}

\sphinxAtStartPar
Creates a ‘custom’ object that can encapsulate multiple properties. It takes a name as an argument, but this is currently not accessible.

\sphinxAtStartPar
\sphinxstylestrong{Example:}

\begin{sphinxVerbatim}[commandchars=\\\{\}]
\PYG{n+nv}{obj}\PYG{+w}{ }\PYG{o}{\PYGZlt{}\PYGZhy{}}\PYG{+w}{ }\PYG{n+nf}{MakeCustomObject}\PYG{p}{(}\PYG{l+s+s2}{\PYGZdq{}myobject\PYGZdq{}}\PYG{p}{)}
\PYG{+w}{  }\PYG{n+nv}{obj.taste}\PYG{+w}{ }\PYG{o}{\PYGZlt{}\PYGZhy{}}\PYG{+w}{ }\PYG{l+s+s2}{\PYGZdq{}buttery\PYGZdq{}}
\PYG{+w}{  }\PYG{n+nv}{obj.texture}\PYG{+w}{ }\PYG{o}{\PYGZlt{}\PYGZhy{}}\PYG{+w}{ }\PYG{l+s+s2}{\PYGZdq{}creamy\PYGZdq{}}
\PYG{+w}{  }\PYG{n+nf}{SetProperty}\PYG{p}{(}\PYG{n+nv}{obj}\PYG{p}{,}\PYG{l+s+s2}{\PYGZdq{}flavor\PYGZdq{}}\PYG{p}{,}\PYG{l+s+s2}{\PYGZdq{}tasty\PYGZdq{}}\PYG{p}{)}

\PYG{+w}{  }\PYG{n+nv}{list}\PYG{+w}{ }\PYG{o}{\PYGZlt{}\PYGZhy{}}\PYG{+w}{ }\PYG{n+nf}{GetPropertyList}\PYG{p}{(}\PYG{n+nv}{obj}\PYG{p}{)}
\PYG{+w}{  }\PYG{k}{loop}\PYG{p}{(}\PYG{n+nv}{i}\PYG{p}{,}\PYG{n+nv}{list}\PYG{p}{)}
\PYG{+w}{   }\PYG{p}{\PYGZob{}}
\PYG{+w}{     }\PYG{k}{if}\PYG{p}{(}\PYG{n+nf}{PropertyExists}\PYG{p}{(}\PYG{n+nv}{obj}\PYG{p}{,}\PYG{n+nv}{i}\PYG{p}{)}
\PYG{+w}{      }\PYG{p}{\PYGZob{}}
\PYG{+w}{        }\PYG{n+nf}{Print}\PYG{p}{(}\PYG{n+nv}{i}\PYG{+w}{  }\PYG{o}{+}\PYG{+w}{ }\PYG{l+s+s2}{\PYGZdq{}:  \PYGZdq{}}\PYG{+w}{ }\PYG{o}{+}\PYG{+w}{ }\PYG{n+nf}{GetProperty}\PYG{p}{(}\PYG{n+nv}{obj}\PYG{p}{,}\PYG{n+nv}{i}\PYG{p}{)}\PYG{p}{)}
\PYG{+w}{      }\PYG{p}{\PYGZcb{}}
\PYG{+w}{   }\PYG{p}{\PYGZcb{}}
\end{sphinxVerbatim}

\sphinxAtStartPar
\sphinxstylestrong{See Also:}

\sphinxAtStartPar
\sphinxcode{\sphinxupquote{GetPropertyList()}}, \sphinxcode{\sphinxupquote{PropertyExists()}}, \sphinxcode{\sphinxupquote{SetProperty()}}, \sphinxcode{\sphinxupquote{IsCustomObject()}}, \sphinxcode{\sphinxupquote{PrintProperties()}}, \sphinxcode{\sphinxupquote{GetProperty()}}

\index{MakeFont@\spxentry{MakeFont}}\ignorespaces 

\subsection{MakeFont()}
\label{\detokenize{reference/peblobjects:makefont}}\label{\detokenize{reference/peblobjects:index-25}}
\sphinxAtStartPar
\sphinxstyleemphasis{Creates a font which can be used to make labels}

\sphinxAtStartPar
\sphinxstylestrong{Description:}

\sphinxAtStartPar
Makes a font.  The first argument must be a text   name of a font.  The font can reside anywhere in PEBL’s search path,   which would primarily include the media/fonts directory, and the   working directory (where the script is saved).
\begin{itemize}
\item {} 
\sphinxAtStartPar
style changes from normal to bold/underline, italic. 0=normal, 1=underline, 2=italic,3=bolditalic

\item {} 
\sphinxAtStartPar
fgcolor and bgcolor need to be colors, not just names of colors

\item {} 
\sphinxAtStartPar
if show\sphinxhyphen{}backing is 0, the font gets rendered with an invisible

\item {} 
\sphinxAtStartPar
background; otherwise with a bgcolor background. (Note: previous to PEBL 0.11, the final argument = 0 rendered the font  with non anti\sphinxhyphen{}aliased background, which I can see almost no use for.)

\end{itemize}

\sphinxAtStartPar
\sphinxstylestrong{Usage:}

\begin{sphinxVerbatim}[commandchars=\\\{\}]
\PYG{n+nf}{MakeFont}\PYG{p}{(}\PYG{o}{\PYGZlt{}}\PYG{n+nv}{ttf\PYGZus{}filename}\PYG{o}{\PYGZgt{}}\PYG{p}{,}\PYG{+w}{ }\PYG{o}{\PYGZlt{}}\PYG{n+nv}{style}\PYG{o}{\PYGZgt{}}\PYG{p}{,}\PYG{+w}{ }\PYG{o}{\PYGZlt{}}\PYG{n+nv}{size}\PYG{o}{\PYGZgt{}}\PYG{p}{,}
\PYG{+w}{         }\PYG{o}{\PYGZlt{}}\PYG{n+nv}{fgcolor}\PYG{o}{\PYGZgt{}}\PYG{p}{,}\PYG{+w}{ }\PYG{o}{\PYGZlt{}}\PYG{n+nv}{bgcolor}\PYG{o}{\PYGZgt{}}\PYG{p}{,}\PYG{+w}{ }\PYG{o}{\PYGZlt{}}\PYG{n+nv}{show}\PYG{o}{\PYGZhy{}}\PYG{n+nv}{backing}\PYG{o}{\PYGZgt{}}\PYG{p}{)}
\end{sphinxVerbatim}

\sphinxAtStartPar
\sphinxstylestrong{Example:}

\begin{sphinxVerbatim}[commandchars=\\\{\}]
\PYG{n+nv}{font}\PYG{+w}{ }\PYG{o}{\PYGZlt{}\PYGZhy{}}\PYG{+w}{ }\PYG{n+nf}{MakeFont}\PYG{p}{(}\PYG{l+s+s2}{\PYGZdq{}Vera.ttf\PYGZdq{}}\PYG{p}{,}\PYG{l+m+mi}{0}\PYG{p}{,}\PYG{l+m+mi}{22}\PYG{p}{,}\PYG{n+nf}{MakeColor}\PYG{p}{(}\PYG{l+s+s2}{\PYGZdq{}black\PYGZdq{}}\PYG{p}{)}\PYG{p}{,}
\PYG{+w}{                    }\PYG{n+nf}{MakeColor}\PYG{p}{(}\PYG{l+s+s2}{\PYGZdq{}white\PYGZdq{}}\PYG{p}{)}\PYG{p}{,}\PYG{l+m+mi}{1}\PYG{p}{)}
\end{sphinxVerbatim}

\index{MakeImage@\spxentry{MakeImage}}\ignorespaces 

\subsection{MakeImage()}
\label{\detokenize{reference/peblobjects:makeimage}}\label{\detokenize{reference/peblobjects:index-26}}
\sphinxAtStartPar
\sphinxstyleemphasis{Creates an image by reading in an image file (jpg, gif, png, bmp, etc.)}

\sphinxAtStartPar
\sphinxstylestrong{Description:}

\sphinxAtStartPar
Makes an image widget from an image file.               \sphinxcode{\sphinxupquote{.bmp}} formats should be supported; others may be as well.

\sphinxAtStartPar
\sphinxstylestrong{Usage:}

\begin{sphinxVerbatim}[commandchars=\\\{\}]
\PYG{n+nf}{MakeImage}\PYG{p}{(}\PYG{o}{\PYGZlt{}}\PYG{n+nv}{filename}\PYG{o}{\PYGZgt{}}\PYG{p}{)}
\end{sphinxVerbatim}

\index{MakeLabel@\spxentry{MakeLabel}}\ignorespaces 

\subsection{MakeLabel()}
\label{\detokenize{reference/peblobjects:makelabel}}\label{\detokenize{reference/peblobjects:index-27}}
\sphinxAtStartPar
\sphinxstylestrong{Description:}

\sphinxAtStartPar
Makes a text label for display on\sphinxhyphen{}screen. Text will   be on a single line, and the \sphinxcode{\sphinxupquote{Move()}} command centers   \sphinxcode{\sphinxupquote{\textless{}text\textgreater{}}} on the specified point.

\sphinxAtStartPar
\sphinxstylestrong{Usage:}

\begin{sphinxVerbatim}[commandchars=\\\{\}]
\PYG{n+nf}{MakeLabel}\PYG{p}{(}\PYG{o}{\PYGZlt{}}\PYG{n+nv}{text}\PYG{o}{\PYGZgt{}}\PYG{p}{,}\PYG{+w}{ }\PYG{o}{\PYGZlt{}}\PYG{n+nv}{font}\PYG{o}{\PYGZgt{}}\PYG{p}{)}
\end{sphinxVerbatim}

\index{MakeSineWave@\spxentry{MakeSineWave}}\ignorespaces 

\subsection{MakeSineWave()}
\label{\detokenize{reference/peblobjects:makesinewave}}\label{\detokenize{reference/peblobjects:index-28}}
\sphinxAtStartPar
\sphinxstyleemphasis{Creates a pure sine wave.}

\sphinxAtStartPar
\sphinxstylestrong{Description:}

\sphinxAtStartPar
Creates a sine wave that can be played using the Play() or PlayBackground() functions.  It will create a single\sphinxhyphen{}channel sound at 44100 bitrate, 16 bit precision.

\sphinxAtStartPar
\sphinxstylestrong{Usage:}

\begin{sphinxVerbatim}[commandchars=\\\{\}]
\PYG{n+nf}{MakeSineWave}\PYG{p}{(}\PYG{o}{\PYGZlt{}}\PYG{n+nv}{duration\PYGZus{}in\PYGZus{}ms}\PYG{o}{\PYGZgt{}}\PYG{p}{,}\PYG{+w}{ }\PYG{o}{\PYGZlt{}}\PYG{n+nv}{hz}\PYG{o}{\PYGZgt{}}\PYG{p}{,}\PYG{+w}{ }\PYG{o}{\PYGZlt{}}\PYG{n+nv}{amplitude}\PYG{o}{\PYGZgt{}}\PYG{p}{)}
\end{sphinxVerbatim}

\sphinxAtStartPar
\sphinxstylestrong{Example:}

\begin{sphinxVerbatim}[commandchars=\\\{\}]
\PYG{c+c1}{\PYGZsh{}\PYGZsh{}Make a sound that is 1000 ms, but just play 300 ms}
\PYG{+w}{   }\PYG{n+nv}{sound}\PYG{+w}{  }\PYG{o}{\PYGZlt{}\PYGZhy{}}\PYG{+w}{ }\PYG{n+nf}{MakeSineWave}\PYG{p}{(}\PYG{l+m+mi}{200}\PYG{p}{,}\PYG{+w}{ }\PYG{l+m+mi}{220}\PYG{p}{,}\PYG{+w}{ }\PYG{l+m+mi}{1000}\PYG{p}{)}
\PYG{+w}{   }\PYG{n+nf}{PlayBackground}\PYG{p}{(}\PYG{n+nv}{sound}\PYG{p}{)}
\PYG{+w}{   }\PYG{n+nf}{Wait}\PYG{p}{(}\PYG{l+m+mi}{300}\PYG{p}{)}
\PYG{+w}{   }\PYG{n+nf}{Stop}\PYG{p}{(}\PYG{n+nv}{sound}\PYG{p}{)}
\end{sphinxVerbatim}

\sphinxAtStartPar
\sphinxstylestrong{See Also:}

\sphinxAtStartPar
\sphinxcode{\sphinxupquote{PlayForeground()}}, \sphinxcode{\sphinxupquote{PlayBackGround()}}, \sphinxcode{\sphinxupquote{Stop()}}

\index{MakeTextBox@\spxentry{MakeTextBox}}\ignorespaces 

\subsection{MakeTextBox()}
\label{\detokenize{reference/peblobjects:maketextbox}}\label{\detokenize{reference/peblobjects:index-29}}
\sphinxAtStartPar
\sphinxstyleemphasis{Creates a sized box filled}

\sphinxAtStartPar
\sphinxstylestrong{Description:}

\sphinxAtStartPar
Creates a textbox in which to display text.             Textboxes allow multiple lines of text to be rendered;          automatically breaking the text into lines.

\sphinxAtStartPar
\sphinxstylestrong{Usage:}

\begin{sphinxVerbatim}[commandchars=\\\{\}]
\PYG{n+nf}{MakeTextbox}\PYG{p}{(}\PYG{o}{\PYGZlt{}}\PYG{n+nv}{text}\PYG{o}{\PYGZgt{}}\PYG{p}{,}\PYG{o}{\PYGZlt{}}\PYG{n+nv}{font}\PYG{o}{\PYGZgt{}}\PYG{p}{,}\PYG{o}{\PYGZlt{}}\PYG{n+nv}{width}\PYG{o}{\PYGZgt{}}\PYG{p}{,}\PYG{o}{\PYGZlt{}}\PYG{n+nv}{height}\PYG{o}{\PYGZgt{}}\PYG{p}{)}
\end{sphinxVerbatim}

\sphinxAtStartPar
\sphinxstylestrong{Example:}

\begin{sphinxVerbatim}[commandchars=\\\{\}]
\PYG{n+nv}{font}\PYG{+w}{ }\PYG{o}{\PYGZlt{}\PYGZhy{}}\PYG{n+nf}{MakeFont}\PYG{p}{(}\PYG{l+s+s2}{\PYGZdq{}Vera.ttf\PYGZdq{}}\PYG{p}{,}\PYG{+w}{ }\PYG{l+m+mi}{1}\PYG{p}{,}\PYG{+w}{ }\PYG{l+m+mi}{12}\PYG{p}{,}\PYG{+w}{ }\PYG{n+nf}{MakeColor}\PYG{p}{(}\PYG{l+s+s2}{\PYGZdq{}red\PYGZdq{}}\PYG{p}{)}\PYG{p}{,}
\PYG{n+nf}{MakeColor}\PYG{p}{(}\PYG{l+s+s2}{\PYGZdq{}green\PYGZdq{}}\PYG{p}{)}\PYG{p}{,}\PYG{+w}{ }\PYG{l+m+mi}{1}\PYG{p}{)}
\PYG{n+nv}{tb}\PYG{+w}{ }\PYG{o}{\PYGZlt{}\PYGZhy{}}\PYG{+w}{ }\PYG{n+nf}{MakeTextBox}\PYG{p}{(}\PYG{l+s+s2}{\PYGZdq{}This is the text in the textbox\PYGZdq{}}\PYG{p}{,}
\PYG{n+nv}{font}\PYG{p}{,}\PYG{+w}{ }\PYG{l+m+mi}{100}\PYG{p}{,}\PYG{+w}{ }\PYG{l+m+mi}{250}\PYG{p}{)}
\end{sphinxVerbatim}

\sphinxAtStartPar
\sphinxstylestrong{See Also:}

\sphinxAtStartPar
\sphinxcode{\sphinxupquote{MakeLabel()}}, \sphinxcode{\sphinxupquote{GetText()}}, \sphinxcode{\sphinxupquote{SetText()}}, \sphinxcode{\sphinxupquote{SetCursorPosition()}},                 \sphinxcode{\sphinxupquote{GetCursorPosition()}}, \sphinxcode{\sphinxupquote{SetEditable()}}

\index{MakeWindow@\spxentry{MakeWindow}}\ignorespaces 

\subsection{MakeWindow()}
\label{\detokenize{reference/peblobjects:makewindow}}\label{\detokenize{reference/peblobjects:index-30}}
\sphinxAtStartPar
\sphinxstyleemphasis{Creates main window, in color named by argument, or grey if no argument is named}

\sphinxAtStartPar
\sphinxstylestrong{Description:}

\sphinxAtStartPar
Creates a window to display things in.          Background is specified by \sphinxcode{\sphinxupquote{\textless{}color\textgreater{}}}.

\sphinxAtStartPar
\sphinxstylestrong{Usage:}

\begin{sphinxVerbatim}[commandchars=\\\{\}]
\PYG{n+nf}{MakeWindow}\PYG{p}{(}\PYG{n+nv}{opt}\PYG{o}{:}\PYG{o}{\PYGZlt{}}\PYG{n+nv}{color}\PYG{o}{\PYGZgt{}}\PYG{p}{,}\PYG{+w}{ }\PYG{n+nv}{opt}\PYG{o}{:}\PYG{o}{\PYGZlt{}}\PYG{n+nv}{width}\PYG{o}{\PYGZgt{}}\PYG{p}{,}\PYG{n+nv}{opt}\PYG{o}{:}\PYG{o}{\PYGZlt{}}\PYG{n+nv}{height}\PYG{o}{\PYGZgt{}}\PYG{p}{)}
\end{sphinxVerbatim}

\sphinxAtStartPar
\sphinxstylestrong{Example:}

\begin{sphinxVerbatim}[commandchars=\\\{\}]
\PYG{n+nv}{win}\PYG{+w}{ }\PYG{o}{\PYGZlt{}\PYGZhy{}}\PYG{+w}{ }\PYG{n+nf}{MakeWindow}\PYG{p}{(}\PYG{p}{)}
\PYG{+w}{  }\PYG{n+nv+vg}{gWin}\PYG{+w}{ }\PYG{o}{\PYGZlt{}\PYGZhy{}}\PYG{+w}{ }\PYG{n+nf}{MakeWindow}\PYG{p}{(}\PYG{l+s+s2}{\PYGZdq{}white\PYGZdq{}}\PYG{p}{)}

\PYG{+w}{  }\PYG{c+c1}{\PYGZsh{}\PYGZsh{}make a second window for debugging or experimenter data entry.}
\PYG{+w}{  }\PYG{n+nv+vg}{gWin2}\PYG{+w}{ }\PYG{o}{\PYGZlt{}\PYGZhy{}}\PYG{+w}{ }\PYG{n+nf}{MakeWindow}\PYG{p}{(}\PYG{l+s+s2}{\PYGZdq{}black\PYGZdq{}}\PYG{p}{,}\PYG{l+m+mi}{400}\PYG{p}{,}\PYG{l+m+mi}{200}\PYG{p}{)}
\end{sphinxVerbatim}

\index{Move@\spxentry{Move}}\ignorespaces 

\subsection{Move()}
\label{\detokenize{reference/peblobjects:move}}\label{\detokenize{reference/peblobjects:index-31}}
\sphinxAtStartPar
\sphinxstylestrong{Description:}

\sphinxAtStartPar
Moves an object to a specified location.                Images and Labels are moved according to their center;                  TextBoxes are moved according to their upper left corner.

\sphinxAtStartPar
\sphinxstylestrong{Usage:}

\begin{sphinxVerbatim}[commandchars=\\\{\}]
\PYG{n+nf}{Move}\PYG{p}{(}\PYG{o}{\PYGZlt{}}\PYG{n+nv}{object}\PYG{o}{\PYGZgt{}}\PYG{p}{,}\PYG{+w}{ }\PYG{o}{\PYGZlt{}}\PYG{n+nv}{x}\PYG{o}{\PYGZgt{}}\PYG{p}{,}\PYG{+w}{ }\PYG{o}{\PYGZlt{}}\PYG{n+nv}{y}\PYG{o}{\PYGZgt{}}\PYG{p}{)}
\end{sphinxVerbatim}

\sphinxAtStartPar
\sphinxstylestrong{Example:}

\begin{sphinxVerbatim}[commandchars=\\\{\}]
\PYG{n+nf}{Move}\PYG{p}{(}\PYG{n+nv}{label}\PYG{p}{,}\PYG{+w}{ }\PYG{l+m+mi}{33}\PYG{p}{,}\PYG{+w}{ }\PYG{l+m+mi}{100}\PYG{p}{)}
\end{sphinxVerbatim}

\sphinxAtStartPar
\sphinxstylestrong{See Also:}

\sphinxAtStartPar
\sphinxcode{\sphinxupquote{MoveCorner()}}, \sphinxcode{\sphinxupquote{MoveCenter()}}, \sphinxcode{\sphinxupquote{X()}} and \sphinxcode{\sphinxupquote{Y()}} properties.

\index{PausePlayback@\spxentry{PausePlayback}}\ignorespaces 

\subsection{PausePlayback()}
\label{\detokenize{reference/peblobjects:pauseplayback}}\label{\detokenize{reference/peblobjects:index-32}}
\sphinxAtStartPar
\sphinxstyleemphasis{Pauses playback of movie}

\sphinxAtStartPar
\sphinxstylestrong{Description:}

\sphinxAtStartPar
Pauses a playing movie or audio stream.  This is used for  movies whose playback was initiated using \sphinxcode{\sphinxupquote{StartPlayback}}, which then ran as background threads during a Wait() function.

\sphinxAtStartPar
\sphinxstylestrong{Usage:}

\begin{sphinxVerbatim}[commandchars=\\\{\}]
\PYG{n+nf}{PausePlayBack}\PYG{p}{(}\PYG{n+nv}{movie}\PYG{p}{)}
\end{sphinxVerbatim}

\sphinxAtStartPar
\sphinxstylestrong{Example:}

\begin{sphinxVerbatim}[commandchars=\\\{\}]
\PYG{n+nv}{movie}\PYG{+w}{ }\PYG{o}{\PYGZlt{}\PYGZhy{}}\PYG{+w}{ }\PYG{n+nf}{LoadMovie}\PYG{p}{(}\PYG{l+s+s2}{\PYGZdq{}movie.avi\PYGZdq{}}\PYG{p}{,}\PYG{n+nv+vg}{gWin}\PYG{p}{,}\PYG{l+m+mi}{640}\PYG{p}{,}\PYG{l+m+mi}{480}\PYG{p}{)}
\PYG{+w}{   }\PYG{n+nf}{PrintProperties}\PYG{p}{(}\PYG{n+nv}{movie}\PYG{p}{)}
\PYG{+w}{   }\PYG{n+nf}{Move}\PYG{p}{(}\PYG{n+nv}{movie}\PYG{p}{,}\PYG{l+m+mi}{20}\PYG{p}{,}\PYG{l+m+mi}{20}\PYG{p}{)}
\PYG{+w}{   }\PYG{n+nf}{Draw}\PYG{p}{(}\PYG{p}{)}
\PYG{+w}{   }\PYG{n+nf}{StartPlayback}\PYG{p}{(}\PYG{n+nv}{movie}\PYG{p}{)}
\PYG{+w}{   }\PYG{n+nf}{Wait}\PYG{p}{(}\PYG{l+m+mi}{500}\PYG{p}{)}\PYG{+w}{ }\PYG{c+c1}{\PYGZsh{}Play 500 ms of the movie.}
\PYG{+w}{   }\PYG{n+nf}{PausePlayback}\PYG{p}{(}\PYG{n+nv}{movie}\PYG{p}{)}
\PYG{+w}{   }\PYG{n+nf}{Wait}\PYG{p}{(}\PYG{l+m+mi}{500}\PYG{p}{)}
\end{sphinxVerbatim}

\sphinxAtStartPar
\sphinxstylestrong{See Also:}

\sphinxAtStartPar
\sphinxcode{\sphinxupquote{LoadAudioFile()}}, \sphinxcode{\sphinxupquote{LoadMovie()}}, \sphinxcode{\sphinxupquote{PlayMovie()}}, \sphinxcode{\sphinxupquote{StartPlayback()}}

\index{PlayBackground@\spxentry{PlayBackground}}\ignorespaces 

\subsection{PlayBackground()}
\label{\detokenize{reference/peblobjects:playbackground}}\label{\detokenize{reference/peblobjects:index-33}}
\sphinxAtStartPar
\sphinxstyleemphasis{Plays the sound ‘in the background’, returning immediately}

\sphinxAtStartPar
\sphinxstylestrong{Description:}

\sphinxAtStartPar
Plays the sound ‘in the background’, returning immediately.

\sphinxAtStartPar
\sphinxstylestrong{Usage:}

\begin{sphinxVerbatim}[commandchars=\\\{\}]
\PYG{n+nf}{PlayBackground}\PYG{p}{(}\PYG{o}{\PYGZlt{}}\PYG{n+nv}{sound}\PYG{o}{\PYGZgt{}}\PYG{p}{)}
\end{sphinxVerbatim}

\sphinxAtStartPar
\sphinxstylestrong{Example:}

\begin{sphinxVerbatim}[commandchars=\\\{\}]
\PYG{n+nv}{sound}\PYG{+w}{  }\PYG{o}{\PYGZlt{}\PYGZhy{}}\PYG{+w}{ }\PYG{n+nf}{MakeSineWave}\PYG{p}{(}\PYG{l+m+mi}{200}\PYG{p}{,}\PYG{+w}{ }\PYG{l+m+mi}{220}\PYG{p}{,}\PYG{+w}{ }\PYG{l+m+mi}{1000}\PYG{p}{)}
\PYG{+w}{   }\PYG{n+nf}{PlayBackground}\PYG{p}{(}\PYG{n+nv}{sound}\PYG{p}{)}
\PYG{+w}{   }\PYG{n+nf}{Wait}\PYG{p}{(}\PYG{l+m+mi}{300}\PYG{p}{)}
\PYG{+w}{   }\PYG{n+nf}{Stop}\PYG{p}{(}\PYG{n+nv}{sound}\PYG{p}{)}
\end{sphinxVerbatim}

\sphinxAtStartPar
\sphinxstylestrong{See Also:}

\sphinxAtStartPar
\sphinxcode{\sphinxupquote{PlayForeground()}}, \sphinxcode{\sphinxupquote{Stop()}}

\index{PlayForeground@\spxentry{PlayForeground}}\ignorespaces 

\subsection{PlayForeground()}
\label{\detokenize{reference/peblobjects:playforeground}}\label{\detokenize{reference/peblobjects:index-34}}
\sphinxAtStartPar
\sphinxstyleemphasis{Plays the sound ‘in the foreground’, not returning until the sound is complete}

\sphinxAtStartPar
\sphinxstylestrong{Description:}

\sphinxAtStartPar
Plays the sound ‘in the foreground’;            does not return until the sound is complete.

\sphinxAtStartPar
\sphinxstylestrong{Usage:}

\begin{sphinxVerbatim}[commandchars=\\\{\}]
\PYG{n+nf}{PlayForeground}\PYG{p}{(}\PYG{o}{\PYGZlt{}}\PYG{n+nv}{sound}\PYG{o}{\PYGZgt{}}\PYG{p}{)}
\end{sphinxVerbatim}

\sphinxAtStartPar
\sphinxstylestrong{Example:}

\begin{sphinxVerbatim}[commandchars=\\\{\}]
\PYG{n+nv}{sound}\PYG{+w}{  }\PYG{o}{\PYGZlt{}\PYGZhy{}}\PYG{+w}{ }\PYG{n+nf}{MakeSineWave}\PYG{p}{(}\PYG{l+m+mi}{200}\PYG{p}{,}\PYG{+w}{ }\PYG{l+m+mi}{220}\PYG{p}{,}\PYG{+w}{ }\PYG{l+m+mi}{1000}\PYG{p}{)}
\PYG{+w}{   }\PYG{n+nf}{PlayForeground}\PYG{p}{(}\PYG{n+nv}{sound}\PYG{p}{)}
\end{sphinxVerbatim}

\sphinxAtStartPar
\sphinxstylestrong{See Also:}

\sphinxAtStartPar
\sphinxcode{\sphinxupquote{PlayBackground()}}, \sphinxcode{\sphinxupquote{Stop()}}

\index{Polygon@\spxentry{Polygon}}\ignorespaces 

\subsection{Polygon()}
\label{\detokenize{reference/peblobjects:polygon}}\label{\detokenize{reference/peblobjects:index-35}}
\sphinxAtStartPar
\sphinxstylestrong{Description:}

\sphinxAtStartPar
Creates a polygon in the shape of the points specified by \sphinxcode{\sphinxupquote{\textless{}xpoints\textgreater{}}}, \sphinxcode{\sphinxupquote{\textless{}ypoints\textgreater{}}}. The lists \sphinxcode{\sphinxupquote{\textless{}xpoints\textgreater{}}} and \sphinxcode{\sphinxupquote{\textless{}ypoints\textgreater{}}} are adjusted by  \sphinxcode{\sphinxupquote{\textless{}x\textgreater{}}} and \sphinxcode{\sphinxupquote{\textless{}y\textgreater{}}}, so they should be relative to 0, not the location you want the points to be at.  Like other drawn objects, the polygon must then be added to the window to appear.

\sphinxAtStartPar
\sphinxstylestrong{Usage:}

\begin{sphinxVerbatim}[commandchars=\\\{\}]
\PYG{n+nf}{Polygon}\PYG{p}{(}\PYG{o}{\PYGZlt{}}\PYG{n+nv}{x}\PYG{o}{\PYGZgt{}}\PYG{p}{,}\PYG{o}{\PYGZlt{}}\PYG{n+nv}{y}\PYG{o}{\PYGZgt{}}\PYG{p}{,}\PYG{o}{\PYGZlt{}}\PYG{n+nv}{xpoints}\PYG{o}{\PYGZgt{}}\PYG{p}{,}\PYG{o}{\PYGZlt{}}\PYG{n+nv}{ypoints}\PYG{o}{\PYGZgt{}}\PYG{p}{,}
\PYG{+w}{          }\PYG{o}{\PYGZlt{}}\PYG{n+nv}{color}\PYG{o}{\PYGZgt{}}\PYG{p}{,}\PYG{o}{\PYGZlt{}}\PYG{n+nv}{filled}\PYG{o}{\PYGZgt{}}\PYG{p}{)}
\end{sphinxVerbatim}

\sphinxAtStartPar
\sphinxstylestrong{Example:}

\begin{sphinxVerbatim}[commandchars=\\\{\}]
\PYG{n+nv}{win}\PYG{+w}{ }\PYG{o}{\PYGZlt{}\PYGZhy{}}\PYG{+w}{ }\PYG{n+nf}{MakeWindow}\PYG{p}{(}\PYG{p}{)}
\PYG{+w}{   }\PYG{c+c1}{\PYGZsh{}This makes a T}
\PYG{+w}{   }\PYG{n+nv}{xpoints}\PYG{+w}{ }\PYG{o}{\PYGZlt{}\PYGZhy{}}\PYG{+w}{ }\PYG{p}{[}\PYG{o}{\PYGZhy{}}\PYG{l+m+mi}{10}\PYG{p}{,}\PYG{l+m+mi}{10}\PYG{p}{,}\PYG{l+m+mi}{10}\PYG{p}{,}\PYG{l+m+mi}{20}\PYG{p}{,}\PYG{l+m+mi}{20}\PYG{p}{,}\PYG{o}{\PYGZhy{}}\PYG{l+m+mi}{20}\PYG{p}{,}\PYG{o}{\PYGZhy{}}\PYG{l+m+mi}{20}\PYG{p}{,}\PYG{o}{\PYGZhy{}}\PYG{l+m+mi}{10}\PYG{p}{]}
\PYG{+w}{   }\PYG{n+nv}{ypoints}\PYG{+w}{ }\PYG{o}{\PYGZlt{}\PYGZhy{}}\PYG{+w}{ }\PYG{p}{[}\PYG{o}{\PYGZhy{}}\PYG{l+m+mi}{20}\PYG{p}{,}\PYG{o}{\PYGZhy{}}\PYG{l+m+mi}{20}\PYG{p}{,}\PYG{l+m+mi}{40}\PYG{p}{,}\PYG{l+m+mi}{40}\PYG{p}{,}\PYG{l+m+mi}{50}\PYG{p}{,}\PYG{l+m+mi}{50}\PYG{p}{,}\PYG{l+m+mi}{40}\PYG{p}{,}\PYG{l+m+mi}{40}\PYG{p}{]}
\PYG{+w}{  }\PYG{n+nv}{p1}\PYG{+w}{ }\PYG{o}{\PYGZlt{}\PYGZhy{}}\PYG{+w}{    }\PYG{n+nf}{Polygon}\PYG{p}{(}\PYG{l+m+mi}{100}\PYG{p}{,}\PYG{l+m+mi}{100}\PYG{p}{,}\PYG{n+nv}{xpoints}\PYG{p}{,}\PYG{+w}{ }\PYG{n+nv}{ypoints}\PYG{p}{,}
\PYG{+w}{                   }\PYG{n+nf}{MakeColor}\PYG{p}{(}\PYG{l+s+s2}{\PYGZdq{}black\PYGZdq{}}\PYG{p}{)}\PYG{p}{,}\PYG{l+m+mi}{1}\PYG{p}{)}
\PYG{+w}{  }\PYG{n+nf}{AddObject}\PYG{p}{(}\PYG{n+nv}{p1}\PYG{p}{,}\PYG{n+nv}{win}\PYG{p}{)}
\PYG{+w}{  }\PYG{n+nf}{Draw}\PYG{p}{(}\PYG{p}{)}
\end{sphinxVerbatim}

\sphinxAtStartPar
\sphinxstylestrong{See Also:}

\sphinxAtStartPar
\sphinxcode{\sphinxupquote{BlockE()}}, \sphinxcode{\sphinxupquote{Bezier()}}, \sphinxcode{\sphinxupquote{MakeStarPoints()}}, \sphinxcode{\sphinxupquote{MakeNGonPoints()}}

\index{PrintProperties@\spxentry{PrintProperties}}\ignorespaces 

\subsection{PrintProperties()}
\label{\detokenize{reference/peblobjects:printproperties}}\label{\detokenize{reference/peblobjects:index-36}}
\sphinxAtStartPar
\sphinxstyleemphasis{Prints a list of all available properties of an object (for debugging)}

\sphinxAtStartPar
\sphinxstylestrong{Description:}

\sphinxAtStartPar
Prints .properties/values for any complex object.   These include textboxes, fonts, colors, images, shapes, etc. Mostly   useful as a debugging tool.

\sphinxAtStartPar
\sphinxstylestrong{Usage:}

\begin{sphinxVerbatim}[commandchars=\\\{\}]
\PYG{n+nf}{PrintProperties}\PYG{p}{(}\PYG{o}{\PYGZlt{}}\PYG{n+nv}{object}\PYG{o}{\PYGZgt{}}\PYG{p}{)}
\end{sphinxVerbatim}

\sphinxAtStartPar
\sphinxstylestrong{Example:}

\begin{sphinxVerbatim}[commandchars=\\\{\}]
\PYG{n+nv}{win}\PYG{+w}{ }\PYG{o}{\PYGZlt{}\PYGZhy{}}\PYG{+w}{ }\PYG{n+nf}{MakeWindow}\PYG{p}{(}\PYG{p}{)}
\PYG{+w}{   }\PYG{n+nv}{tb}\PYG{+w}{ }\PYG{o}{\PYGZlt{}\PYGZhy{}}\PYG{+w}{ }\PYG{n+nf}{EasyTextbox}\PYG{p}{(}\PYG{l+s+s2}{\PYGZdq{}one\PYGZdq{}}\PYG{p}{,}\PYG{l+m+mi}{20}\PYG{p}{,}\PYG{l+m+mi}{20}\PYG{p}{,}\PYG{n+nv}{win}\PYG{p}{,}\PYG{l+m+mi}{22}\PYG{p}{,}\PYG{l+m+mi}{400}\PYG{p}{,}\PYG{l+m+mi}{80}\PYG{p}{)}
\PYG{+w}{   }\PYG{n+nf}{PrintProperties}\PYG{p}{(}\PYG{n+nv}{tb}\PYG{p}{)}

\PYG{c+c1}{\PYGZsh{}\PYGZsh{}Output:}
\PYG{o}{\PYGZhy{}}\PYG{o}{\PYGZhy{}}\PYG{o}{\PYGZhy{}}\PYG{o}{\PYGZhy{}}\PYG{o}{\PYGZhy{}}\PYG{o}{\PYGZhy{}}\PYG{o}{\PYGZhy{}}\PYG{o}{\PYGZhy{}}\PYG{o}{\PYGZhy{}}\PYG{o}{\PYGZhy{}}
\PYG{p}{[}CURSORPOS\PYG{p}{]}\PYG{o}{:}\PYG{+w}{ }\PYG{l+m+mi}{0}
\PYG{p}{[}EDITABLE\PYG{p}{]}\PYG{o}{:}\PYG{+w}{ }\PYG{l+m+mi}{0}
\PYG{p}{[}HEIGHT\PYG{p}{]}\PYG{o}{:}\PYG{+w}{ }\PYG{l+m+mi}{80}
\PYG{p}{[}ROTATION\PYG{p}{]}\PYG{o}{:}\PYG{+w}{ }\PYG{l+m+mi}{0}
\PYG{p}{[}TEXT\PYG{p}{]}\PYG{o}{:}\PYG{+w}{ }\PYG{n+nv}{one}
\PYG{p}{[}VISIBLE\PYG{p}{]}\PYG{o}{:}\PYG{+w}{ }\PYG{l+m+mi}{1}
\PYG{p}{[}WIDTH\PYG{p}{]}\PYG{o}{:}\PYG{+w}{ }\PYG{l+m+mi}{400}
\PYG{p}{[}X\PYG{p}{]}\PYG{o}{:}\PYG{+w}{ }\PYG{l+m+mi}{20}
\PYG{p}{[}Y\PYG{p}{]}\PYG{o}{:}\PYG{+w}{ }\PYG{l+m+mi}{20}
\PYG{p}{[}ZOOMX\PYG{p}{]}\PYG{o}{:}\PYG{+w}{ }\PYG{l+m+mi}{1}
\PYG{p}{[}ZOOMY\PYG{p}{]}\PYG{o}{:}\PYG{+w}{ }\PYG{l+m+mi}{1}
\PYG{o}{\PYGZhy{}}\PYG{o}{\PYGZhy{}}\PYG{o}{\PYGZhy{}}\PYG{o}{\PYGZhy{}}\PYG{o}{\PYGZhy{}}\PYG{o}{\PYGZhy{}}\PYG{o}{\PYGZhy{}}\PYG{o}{\PYGZhy{}}\PYG{o}{\PYGZhy{}}\PYG{o}{\PYGZhy{}}
\end{sphinxVerbatim}

\sphinxAtStartPar
\sphinxstylestrong{See Also:}

\sphinxAtStartPar
\sphinxcode{\sphinxupquote{Print()}}

\index{PropertyExists@\spxentry{PropertyExists}}\ignorespaces 

\subsection{PropertyExists()}
\label{\detokenize{reference/peblobjects:propertyexists}}\label{\detokenize{reference/peblobjects:index-37}}
\sphinxAtStartPar
\sphinxstyleemphasis{Determines whether a particular property exists}

\sphinxAtStartPar
\sphinxstylestrong{Description:}

\sphinxAtStartPar
Tests whether a particular named property exists. This works for custom or built\sphinxhyphen{}in objects. This is important to check properties that might not exist, because trying to \sphinxcode{\sphinxupquote{GetProperty}} of a non\sphinxhyphen{}existent property will cause a fatal error.

\sphinxAtStartPar
\sphinxstylestrong{Example:}

\begin{sphinxVerbatim}[commandchars=\\\{\}]
\PYG{n+nv}{obj}\PYG{+w}{ }\PYG{o}{\PYGZlt{}\PYGZhy{}}\PYG{+w}{ }\PYG{n+nf}{MakeCustomObject}\PYG{p}{(}\PYG{l+s+s2}{\PYGZdq{}myobject\PYGZdq{}}\PYG{p}{)}
\PYG{+w}{  }\PYG{n+nv}{obj.taste}\PYG{+w}{ }\PYG{o}{\PYGZlt{}\PYGZhy{}}\PYG{+w}{ }\PYG{l+s+s2}{\PYGZdq{}buttery\PYGZdq{}}
\PYG{+w}{  }\PYG{n+nv}{obj.texture}\PYG{+w}{ }\PYG{o}{\PYGZlt{}\PYGZhy{}}\PYG{+w}{ }\PYG{l+s+s2}{\PYGZdq{}creamy\PYGZdq{}}
\PYG{+w}{  }\PYG{n+nf}{SetProperty}\PYG{p}{(}\PYG{n+nv}{obj}\PYG{p}{,}\PYG{l+s+s2}{\PYGZdq{}flavor\PYGZdq{}}\PYG{p}{,}\PYG{l+s+s2}{\PYGZdq{}tasty\PYGZdq{}}\PYG{p}{)}

\PYG{+w}{  }\PYG{n+nv}{list}\PYG{+w}{ }\PYG{o}{\PYGZlt{}\PYGZhy{}}\PYG{+w}{ }\PYG{n+nf}{GetPropertyList}\PYG{p}{(}\PYG{n+nv}{obj}\PYG{p}{)}
\PYG{+w}{  }\PYG{k}{loop}\PYG{p}{(}\PYG{n+nv}{i}\PYG{p}{,}\PYG{n+nv}{list}\PYG{p}{)}
\PYG{+w}{   }\PYG{p}{\PYGZob{}}
\PYG{+w}{     }\PYG{k}{if}\PYG{p}{(}\PYG{n+nf}{PropertyExists}\PYG{p}{(}\PYG{n+nv}{obj}\PYG{p}{,}\PYG{n+nv}{i}\PYG{p}{)}
\PYG{+w}{      }\PYG{p}{\PYGZob{}}
\PYG{+w}{        }\PYG{n+nf}{Print}\PYG{p}{(}\PYG{n+nv}{i}\PYG{+w}{  }\PYG{o}{+}\PYG{+w}{ }\PYG{l+s+s2}{\PYGZdq{}:  \PYGZdq{}}\PYG{+w}{ }\PYG{o}{+}\PYG{+w}{ }\PYG{n+nf}{GetProperty}\PYG{p}{(}\PYG{n+nv}{obj}\PYG{p}{,}\PYG{n+nv}{i}\PYG{p}{)}\PYG{p}{)}
\PYG{+w}{      }\PYG{p}{\PYGZcb{}}
\PYG{+w}{   }\PYG{p}{\PYGZcb{}}
\end{sphinxVerbatim}

\sphinxAtStartPar
\sphinxstylestrong{See Also:}

\sphinxAtStartPar
\sphinxcode{\sphinxupquote{GetPropertyList()}}, \sphinxcode{\sphinxupquote{GetProperty()}}, \sphinxcode{\sphinxupquote{SetProperty()}} \sphinxcode{\sphinxupquote{MakeCustomObject()}}, \sphinxcode{\sphinxupquote{PrintProperties()}}

\index{Rectangle@\spxentry{Rectangle}}\ignorespaces 

\subsection{Rectangle()}
\label{\detokenize{reference/peblobjects:rectangle}}\label{\detokenize{reference/peblobjects:index-38}}
\sphinxAtStartPar
\sphinxstyleemphasis{Creates rectangle with size (dx, dy) centered at position x,y}

\sphinxAtStartPar
\sphinxstylestrong{Description:}

\sphinxAtStartPar
Creates a rectangle for graphing at x,y with size   dx and dy. Rectangles are only currently definable oriented in   horizontal/vertical directions.  A rectangle  must be added   to a parent widget before it can be drawn; it may be added to   widgets other than a base window.  The properties of rectangles may be   changed by accessing their properties directly, including the FILLED   property which makes the object an outline versus a filled shape.

\sphinxAtStartPar
\sphinxstylestrong{Usage:}

\begin{sphinxVerbatim}[commandchars=\\\{\}]
\PYG{n+nf}{Rectangle}\PYG{p}{(}\PYG{o}{\PYGZlt{}}\PYG{n+nv}{x}\PYG{o}{\PYGZgt{}}\PYG{p}{,}\PYG{+w}{ }\PYG{o}{\PYGZlt{}}\PYG{n+nv}{y}\PYG{o}{\PYGZgt{}}\PYG{p}{,}\PYG{+w}{ }\PYG{o}{\PYGZlt{}}\PYG{n+nv}{dx}\PYG{o}{\PYGZgt{}}\PYG{p}{,}\PYG{+w}{ }\PYG{o}{\PYGZlt{}}\PYG{n+nv}{dy}\PYG{o}{\PYGZgt{}}\PYG{p}{,}\PYG{+w}{ }\PYG{o}{\PYGZlt{}}\PYG{n+nv}{color}\PYG{o}{\PYGZgt{}}\PYG{p}{)}
\end{sphinxVerbatim}

\sphinxAtStartPar
\sphinxstylestrong{Example:}

\begin{sphinxVerbatim}[commandchars=\\\{\}]
\PYG{n+nv}{r}\PYG{+w}{ }\PYG{o}{\PYGZlt{}\PYGZhy{}}\PYG{+w}{ }\PYG{n+nf}{Rectangle}\PYG{p}{(}\PYG{l+m+mi}{30}\PYG{p}{,}\PYG{l+m+mi}{30}\PYG{p}{,}\PYG{l+m+mi}{20}\PYG{p}{,}\PYG{l+m+mi}{10}\PYG{p}{,}\PYG{+w}{ }\PYG{n+nf}{MakeColor}\PYG{p}{(}\PYG{n+nv+vg}{green}\PYG{p}{)}\PYG{p}{)}
\PYG{+w}{  }\PYG{n+nf}{AddObject}\PYG{p}{(}\PYG{n+nv}{r}\PYG{p}{,}\PYG{+w}{ }\PYG{n+nv}{win}\PYG{p}{)}
\PYG{+w}{  }\PYG{n+nf}{Draw}\PYG{p}{(}\PYG{p}{)}
\end{sphinxVerbatim}

\sphinxAtStartPar
\sphinxstylestrong{See Also:}

\sphinxAtStartPar
\sphinxcode{\sphinxupquote{Circle()}}, \sphinxcode{\sphinxupquote{Ellipse()}}, \sphinxcode{\sphinxupquote{Square()}}, \sphinxcode{\sphinxupquote{Line()}}

\index{RemoveObject@\spxentry{RemoveObject}}\ignorespaces 

\subsection{RemoveObject()}
\label{\detokenize{reference/peblobjects:removeobject}}\label{\detokenize{reference/peblobjects:index-39}}
\sphinxAtStartPar
\sphinxstyleemphasis{Removes an object from a parent window}

\sphinxAtStartPar
\sphinxstylestrong{Description:}

\sphinxAtStartPar
Removes a child widget from a parent.  Useful if   you are adding a local widget to a global window inside a loop.  If   you do not remove the object and only \sphinxcode{\sphinxupquote{Hide()}} it, drawing will   be sluggish.  Objects that are local to a function are removed   automatically when the function terminates, so you do not need to   call \sphinxcode{\sphinxupquote{RemoveObject()}} on them at the end of a function.

\sphinxAtStartPar
\sphinxstylestrong{Usage:}

\begin{sphinxVerbatim}[commandchars=\\\{\}]
\PYG{n+nf}{RemoveObject}\PYG{p}{(}\PYG{+w}{ }\PYG{o}{\PYGZlt{}}\PYG{n+nv}{object}\PYG{o}{\PYGZgt{}}\PYG{p}{,}\PYG{+w}{ }\PYG{o}{\PYGZlt{}}\PYG{n+nv}{parent}\PYG{o}{\PYGZgt{}}\PYG{p}{)}
\end{sphinxVerbatim}

\index{ResizeWindow@\spxentry{ResizeWindow}}\ignorespaces 

\subsection{ResizeWindow()}
\label{\detokenize{reference/peblobjects:resizewindow}}\label{\detokenize{reference/peblobjects:index-40}}
\sphinxAtStartPar
\sphinxstyleemphasis{Resizes a window to a specified width and height}

\sphinxAtStartPar
\sphinxstylestrong{Description:}

\sphinxAtStartPar
Resizes an existing window to the specified dimensions. This allows you to dynamically change the size of a window during program execution.

\sphinxAtStartPar
\sphinxstylestrong{Usage:}

\begin{sphinxVerbatim}[commandchars=\\\{\}]
\PYG{n+nf}{ResizeWindow}\PYG{p}{(}\PYG{o}{\PYGZlt{}}\PYG{n+nv}{window}\PYG{o}{\PYGZgt{}}\PYG{p}{,}\PYG{+w}{ }\PYG{o}{\PYGZlt{}}\PYG{n+nv}{width}\PYG{o}{\PYGZgt{}}\PYG{p}{,}\PYG{+w}{ }\PYG{o}{\PYGZlt{}}\PYG{n+nv}{height}\PYG{o}{\PYGZgt{}}\PYG{p}{)}
\end{sphinxVerbatim}

\sphinxAtStartPar
\sphinxstylestrong{Example:}

\begin{sphinxVerbatim}[commandchars=\\\{\}]
\PYG{n+nv}{win}\PYG{+w}{ }\PYG{o}{\PYGZlt{}\PYGZhy{}}\PYG{+w}{ }\PYG{n+nf}{MakeWindow}\PYG{p}{(}\PYG{l+s+s2}{\PYGZdq{}grey\PYGZdq{}}\PYG{p}{)}
\PYG{c+c1}{\PYGZsh{}\PYGZsh{}Start with default size, then resize}
\PYG{n+nf}{ResizeWindow}\PYG{p}{(}\PYG{n+nv}{win}\PYG{p}{,}\PYG{+w}{ }\PYG{l+m+mi}{1024}\PYG{p}{,}\PYG{+w}{ }\PYG{l+m+mi}{768}\PYG{p}{)}
\PYG{n+nf}{Draw}\PYG{p}{(}\PYG{p}{)}
\end{sphinxVerbatim}

\sphinxAtStartPar
\sphinxstylestrong{See Also:}

\sphinxAtStartPar
\sphinxcode{\sphinxupquote{MakeWindow()}}

\index{RotoZoom@\spxentry{RotoZoom}}\ignorespaces 

\subsection{RotoZoom()}
\label{\detokenize{reference/peblobjects:rotozoom}}\label{\detokenize{reference/peblobjects:index-41}}
\sphinxAtStartPar
\sphinxstyleemphasis{Rotates and zooms a graphical widget}

\sphinxAtStartPar
\sphinxstylestrong{Description:}

\sphinxAtStartPar
Rotates and zooms a widget (such as an image or label) by specified amounts. The rotation parameter specifies rotation in degrees. The xzoom and yzoom parameters specify scaling factors for horizontal and vertical dimensions (1.0 = no change, 2.0 = double size, 0.5 = half size). The smooth parameter (0 or 1) determines whether to use anti\sphinxhyphen{}aliasing for smoother appearance.

\sphinxAtStartPar
\sphinxstylestrong{Usage:}

\begin{sphinxVerbatim}[commandchars=\\\{\}]
\PYG{n+nf}{RotoZoom}\PYG{p}{(}\PYG{o}{\PYGZlt{}}\PYG{n+nv}{widget}\PYG{o}{\PYGZgt{}}\PYG{p}{,}\PYG{+w}{ }\PYG{o}{\PYGZlt{}}\PYG{n+nv}{rotation}\PYG{o}{\PYGZgt{}}\PYG{p}{,}\PYG{+w}{ }\PYG{o}{\PYGZlt{}}\PYG{n+nv}{xzoom}\PYG{o}{\PYGZgt{}}\PYG{p}{,}\PYG{+w}{ }\PYG{o}{\PYGZlt{}}\PYG{n+nv}{yzoom}\PYG{o}{\PYGZgt{}}\PYG{p}{,}\PYG{+w}{ }\PYG{o}{\PYGZlt{}}\PYG{n+nv}{smooth}\PYG{o}{\PYGZgt{}}\PYG{p}{)}
\end{sphinxVerbatim}

\sphinxAtStartPar
\sphinxstylestrong{Example:}

\begin{sphinxVerbatim}[commandchars=\\\{\}]
\PYG{n+nv}{img}\PYG{+w}{ }\PYG{o}{\PYGZlt{}\PYGZhy{}}\PYG{+w}{ }\PYG{n+nf}{MakeImage}\PYG{p}{(}\PYG{l+s+s2}{\PYGZdq{}stimulus.png\PYGZdq{}}\PYG{p}{)}
\PYG{n+nf}{AddObject}\PYG{p}{(}\PYG{n+nv}{img}\PYG{p}{,}\PYG{+w}{ }\PYG{n+nv}{win}\PYG{p}{)}
\PYG{n+nf}{Move}\PYG{p}{(}\PYG{n+nv}{img}\PYG{p}{,}\PYG{+w}{ }\PYG{l+m+mi}{320}\PYG{p}{,}\PYG{+w}{ }\PYG{l+m+mi}{240}\PYG{p}{)}

\PYG{c+c1}{\PYGZsh{}\PYGZsh{}Rotate 45 degrees, double size, with smoothing}
\PYG{n+nf}{RotoZoom}\PYG{p}{(}\PYG{n+nv}{img}\PYG{p}{,}\PYG{+w}{ }\PYG{l+m+mi}{45}\PYG{p}{,}\PYG{+w}{ }\PYG{l+m+mf}{2.0}\PYG{p}{,}\PYG{+w}{ }\PYG{l+m+mf}{2.0}\PYG{p}{,}\PYG{+w}{ }\PYG{l+m+mi}{1}\PYG{p}{)}
\PYG{n+nf}{Draw}\PYG{p}{(}\PYG{p}{)}

\PYG{c+c1}{\PYGZsh{}\PYGZsh{}Rotate 90 degrees, normal size, no smoothing}
\PYG{n+nf}{RotoZoom}\PYG{p}{(}\PYG{n+nv}{img}\PYG{p}{,}\PYG{+w}{ }\PYG{l+m+mi}{90}\PYG{p}{,}\PYG{+w}{ }\PYG{l+m+mf}{1.0}\PYG{p}{,}\PYG{+w}{ }\PYG{l+m+mf}{1.0}\PYG{p}{,}\PYG{+w}{ }\PYG{l+m+mi}{0}\PYG{p}{)}
\PYG{n+nf}{Draw}\PYG{p}{(}\PYG{p}{)}
\end{sphinxVerbatim}

\sphinxAtStartPar
\sphinxstylestrong{See Also:}

\sphinxAtStartPar
\sphinxcode{\sphinxupquote{Move()}}, \sphinxcode{\sphinxupquote{MakeImage()}}, \sphinxcode{\sphinxupquote{MakeLabel()}}

\index{SaveAudioToWaveFile@\spxentry{SaveAudioToWaveFile}}\ignorespaces 

\subsection{SaveAudioToWaveFile()}
\label{\detokenize{reference/peblobjects:saveaudiotowavefile}}\label{\detokenize{reference/peblobjects:index-42}}
\sphinxAtStartPar
\sphinxstyleemphasis{Saves buffer to a .wav file format}

\sphinxAtStartPar
\sphinxstylestrong{Description:}

\sphinxAtStartPar
Saves a buffer, recorded using the GetAudioInputBuffer, to a .wav file for later analysis or archive.

\sphinxAtStartPar
\sphinxstylestrong{Usage:}

\begin{sphinxVerbatim}[commandchars=\\\{\}]
\PYG{n+nf}{SaveAudioToWaveFile}\PYG{p}{(}\PYG{n+nv}{filename}\PYG{p}{,}\PYG{+w}{ }\PYG{n+nv}{buffer}\PYG{p}{)}
\end{sphinxVerbatim}

\sphinxAtStartPar
\sphinxstylestrong{Example:}

\begin{sphinxVerbatim}[commandchars=\\\{\}]
\PYG{n+nv+vg}{gResponseBuffer}\PYG{+w}{ }\PYG{o}{\PYGZlt{}\PYGZhy{}}\PYG{+w}{ }\PYG{n+nf}{MakeAudioInputBuffer}\PYG{p}{(}\PYG{l+m+mi}{5000}\PYG{p}{)}
\PYG{+w}{       }\PYG{n+nv}{resp0}\PYG{+w}{ }\PYG{o}{\PYGZlt{}\PYGZhy{}}\PYG{+w}{  }\PYG{n+nf}{GetVocalResponseTime}\PYG{p}{(}\PYG{n+nv+vg}{gResponseBuffer}\PYG{p}{,}\PYG{l+m+mf}{.35}\PYG{p}{,}\PYG{+w}{ }\PYG{l+m+mi}{200}\PYG{p}{)}
\PYG{+w}{      }\PYG{n+nf}{SaveAudioToWaveFile}\PYG{p}{(}\PYG{l+s+s2}{\PYGZdq{}output.wav\PYGZdq{}}\PYG{p}{,}\PYG{n+nv+vg}{gResponseBuffer}\PYG{p}{)}
\end{sphinxVerbatim}

\sphinxAtStartPar
\sphinxstylestrong{See Also:}

\sphinxAtStartPar
\sphinxcode{\sphinxupquote{GetVocalResponseTime()}}, \sphinxcode{\sphinxupquote{MakeAudioInputBuffer()}}, \sphinxcode{\sphinxupquote{RecordToBuffer()}}

\index{RecordToBuffer@\spxentry{RecordToBuffer}}\ignorespaces 

\subsection{RecordToBuffer()}
\label{\detokenize{reference/peblobjects:recordtobuffer}}\label{\detokenize{reference/peblobjects:index-43}}
\sphinxAtStartPar
\sphinxstyleemphasis{Records audio input to a pre\sphinxhyphen{}allocated buffer}

\sphinxAtStartPar
\sphinxstylestrong{Description:}

\sphinxAtStartPar
Records audio from the microphone directly into a pre\sphinxhyphen{}allocated audio buffer. This function allows precise control over recording duration and provides a synchronous recording interface. The buffer must be created first using \sphinxcode{\sphinxupquote{MakeAudioInputBuffer()}}. An optional duration parameter can specify recording time in milliseconds; if omitted, the function records for the full buffer duration. \sphinxstyleemphasis{ONLY AVAILABLE ON WINDOWS AND LINUX}.

\sphinxAtStartPar
\sphinxstylestrong{Usage:}

\begin{sphinxVerbatim}[commandchars=\\\{\}]
\PYG{n+nf}{RecordToBuffer}\PYG{p}{(}\PYG{o}{\PYGZlt{}}\PYG{n+nv}{buffer}\PYG{o}{\PYGZgt{}}\PYG{p}{)}
\PYG{n+nf}{RecordToBuffer}\PYG{p}{(}\PYG{o}{\PYGZlt{}}\PYG{n+nv}{buffer}\PYG{o}{\PYGZgt{}}\PYG{p}{,}\PYG{+w}{ }\PYG{o}{\PYGZlt{}}\PYG{n+nv}{duration\PYGZus{}ms}\PYG{o}{\PYGZgt{}}\PYG{p}{)}
\end{sphinxVerbatim}

\sphinxAtStartPar
\sphinxstylestrong{Example:}

\begin{sphinxVerbatim}[commandchars=\\\{\}]
\PYG{c+c1}{\PYGZsh{}\PYGZsh{} Record for full buffer duration (5 seconds)}
\PYG{n+nv}{buffer}\PYG{+w}{ }\PYG{o}{\PYGZlt{}\PYGZhy{}}\PYG{+w}{ }\PYG{n+nf}{MakeAudioInputBuffer}\PYG{p}{(}\PYG{l+m+mi}{5000}\PYG{p}{)}
\PYG{n+nf}{RecordToBuffer}\PYG{p}{(}\PYG{n+nv}{buffer}\PYG{p}{)}
\PYG{n+nf}{SaveAudioToWaveFile}\PYG{p}{(}\PYG{l+s+s2}{\PYGZdq{}recording.wav\PYGZdq{}}\PYG{p}{,}\PYG{+w}{ }\PYG{n+nv}{buffer}\PYG{p}{)}

\PYG{c+c1}{\PYGZsh{}\PYGZsh{} Record for specific duration (3 seconds into 5 second buffer)}
\PYG{n+nv}{buffer}\PYG{+w}{ }\PYG{o}{\PYGZlt{}\PYGZhy{}}\PYG{+w}{ }\PYG{n+nf}{MakeAudioInputBuffer}\PYG{p}{(}\PYG{l+m+mi}{5000}\PYG{p}{)}
\PYG{n+nf}{RecordToBuffer}\PYG{p}{(}\PYG{n+nv}{buffer}\PYG{p}{,}\PYG{+w}{ }\PYG{l+m+mi}{3000}\PYG{p}{)}
\PYG{n+nf}{SaveAudioToWaveFile}\PYG{p}{(}\PYG{l+s+s2}{\PYGZdq{}recording\PYGZhy{}3s.wav\PYGZdq{}}\PYG{p}{,}\PYG{+w}{ }\PYG{n+nv}{buffer}\PYG{p}{)}
\end{sphinxVerbatim}

\sphinxAtStartPar
\sphinxstylestrong{See Also:}

\sphinxAtStartPar
\sphinxcode{\sphinxupquote{MakeAudioInputBuffer()}}, \sphinxcode{\sphinxupquote{SaveAudioToWaveFile()}}, \sphinxcode{\sphinxupquote{GetVocalResponseTime()}}, \sphinxcode{\sphinxupquote{StartAudioMonitor()}}

\index{StartAudioMonitor@\spxentry{StartAudioMonitor}}\ignorespaces 

\subsection{StartAudioMonitor()}
\label{\detokenize{reference/peblobjects:startaudiomonitor}}\label{\detokenize{reference/peblobjects:index-44}}
\sphinxAtStartPar
\sphinxstyleemphasis{Starts real\sphinxhyphen{}time audio monitoring with ring buffer}

\sphinxAtStartPar
\sphinxstylestrong{Description:}

\sphinxAtStartPar
Creates and starts an audio monitoring system that continuously records audio in a ring buffer. This is useful for real\sphinxhyphen{}time audio analysis, voice key calibration, and audio level visualization. The monitor runs in the background and can be queried with \sphinxcode{\sphinxupquote{GetAudioStats()}} to retrieve recent audio statistics. The monitor must be stopped with \sphinxcode{\sphinxupquote{StopAudioMonitor()}} to free audio hardware resources. \sphinxstyleemphasis{ONLY AVAILABLE ON WINDOWS AND LINUX}.

\sphinxAtStartPar
\sphinxstylestrong{Usage:}

\begin{sphinxVerbatim}[commandchars=\\\{\}]
\PYG{n+nf}{StartAudioMonitor}\PYG{p}{(}\PYG{o}{\PYGZlt{}}\PYG{n+nv}{buffer\PYGZus{}size\PYGZus{}ms}\PYG{o}{\PYGZgt{}}\PYG{p}{)}
\end{sphinxVerbatim}

\sphinxAtStartPar
\sphinxstylestrong{Example:}

\begin{sphinxVerbatim}[commandchars=\\\{\}]
\PYG{c+c1}{\PYGZsh{}\PYGZsh{} Start monitoring with 3\PYGZhy{}second ring buffer}
\PYG{n+nv}{monitor}\PYG{+w}{ }\PYG{o}{\PYGZlt{}\PYGZhy{}}\PYG{+w}{ }\PYG{n+nf}{StartAudioMonitor}\PYG{p}{(}\PYG{l+m+mi}{3000}\PYG{p}{)}

\PYG{c+c1}{\PYGZsh{}\PYGZsh{} Continuously check audio levels}
\PYG{k}{loop}\PYG{p}{(}\PYG{n+nv}{i}\PYG{p}{,}\PYG{+w}{ }\PYG{n+nf}{Sequence}\PYG{p}{(}\PYG{l+m+mi}{1}\PYG{p}{,}\PYG{+w}{ }\PYG{l+m+mi}{100}\PYG{p}{,}\PYG{+w}{ }\PYG{l+m+mi}{1}\PYG{p}{)}\PYG{p}{)}
\PYG{p}{\PYGZob{}}
\PYG{+w}{    }\PYG{n+nf}{Wait}\PYG{p}{(}\PYG{l+m+mi}{100}\PYG{p}{)}
\PYG{+w}{    }\PYG{n+nv}{stats}\PYG{+w}{ }\PYG{o}{\PYGZlt{}\PYGZhy{}}\PYG{+w}{ }\PYG{n+nf}{GetAudioStats}\PYG{p}{(}\PYG{n+nv}{monitor}\PYG{p}{,}\PYG{+w}{ }\PYG{l+m+mi}{500}\PYG{p}{)}\PYG{+w}{  }\PYG{c+c1}{\PYGZsh{}\PYGZsh{} Last 500ms}
\PYG{+w}{    }\PYG{n+nv}{energy}\PYG{+w}{ }\PYG{o}{\PYGZlt{}\PYGZhy{}}\PYG{+w}{ }\PYG{n+nf}{First}\PYG{p}{(}\PYG{n+nv}{stats}\PYG{p}{)}
\PYG{+w}{    }\PYG{n+nf}{Print}\PYG{p}{(}\PYG{l+s+s2}{\PYGZdq{}Energy: \PYGZdq{}}\PYG{+w}{ }\PYG{o}{+}\PYG{+w}{ }\PYG{n+nv}{energy}\PYG{p}{)}
\PYG{p}{\PYGZcb{}}

\PYG{n+nf}{StopAudioMonitor}\PYG{p}{(}\PYG{n+nv}{monitor}\PYG{p}{)}
\end{sphinxVerbatim}

\sphinxAtStartPar
\sphinxstylestrong{See Also:}

\sphinxAtStartPar
\sphinxcode{\sphinxupquote{StopAudioMonitor()}}, \sphinxcode{\sphinxupquote{GetAudioStats()}}, \sphinxcode{\sphinxupquote{GetVocalResponseTime()}}

\index{StopAudioMonitor@\spxentry{StopAudioMonitor}}\ignorespaces 

\subsection{StopAudioMonitor()}
\label{\detokenize{reference/peblobjects:stopaudiomonitor}}\label{\detokenize{reference/peblobjects:index-45}}
\sphinxAtStartPar
\sphinxstyleemphasis{Stops audio monitoring and releases audio hardware}

\sphinxAtStartPar
\sphinxstylestrong{Description:}

\sphinxAtStartPar
Stops an audio monitor created by \sphinxcode{\sphinxupquote{StartAudioMonitor()}} and releases the audio hardware. This function performs complete cleanup including pausing recording, closing the SDL audio device, and clearing global audio state. It is critical to call this function before starting other audio operations like \sphinxcode{\sphinxupquote{GetVocalResponseTime()}} to ensure the audio hardware is available. \sphinxstyleemphasis{ONLY AVAILABLE ON WINDOWS AND LINUX}.

\sphinxAtStartPar
\sphinxstylestrong{Usage:}

\begin{sphinxVerbatim}[commandchars=\\\{\}]
\PYG{n+nf}{StopAudioMonitor}\PYG{p}{(}\PYG{o}{\PYGZlt{}}\PYG{n+nv}{monitor}\PYG{o}{\PYGZgt{}}\PYG{p}{)}
\end{sphinxVerbatim}

\sphinxAtStartPar
\sphinxstylestrong{Example:}

\begin{sphinxVerbatim}[commandchars=\\\{\}]
\PYG{c+c1}{\PYGZsh{}\PYGZsh{} Monitor audio for 10 seconds then stop}
\PYG{n+nv}{monitor}\PYG{+w}{ }\PYG{o}{\PYGZlt{}\PYGZhy{}}\PYG{+w}{ }\PYG{n+nf}{StartAudioMonitor}\PYG{p}{(}\PYG{l+m+mi}{2000}\PYG{p}{)}
\PYG{n+nf}{Wait}\PYG{p}{(}\PYG{l+m+mi}{10000}\PYG{p}{)}
\PYG{n+nf}{StopAudioMonitor}\PYG{p}{(}\PYG{n+nv}{monitor}\PYG{p}{)}

\PYG{c+c1}{\PYGZsh{}\PYGZsh{} Now audio hardware is free for other operations}
\PYG{n+nv}{buffer}\PYG{+w}{ }\PYG{o}{\PYGZlt{}\PYGZhy{}}\PYG{+w}{ }\PYG{n+nf}{MakeAudioInputBuffer}\PYG{p}{(}\PYG{l+m+mi}{5000}\PYG{p}{)}
\PYG{n+nv}{rt}\PYG{+w}{ }\PYG{o}{\PYGZlt{}\PYGZhy{}}\PYG{+w}{ }\PYG{n+nf}{GetVocalResponseTime}\PYG{p}{(}\PYG{n+nv}{buffer}\PYG{p}{,}\PYG{+w}{ }\PYG{l+m+mf}{0.35}\PYG{p}{,}\PYG{+w}{ }\PYG{l+m+mi}{200}\PYG{p}{)}
\end{sphinxVerbatim}

\sphinxAtStartPar
\sphinxstylestrong{See Also:}

\sphinxAtStartPar
\sphinxcode{\sphinxupquote{StartAudioMonitor()}}, \sphinxcode{\sphinxupquote{GetAudioStats()}}

\index{GetAudioStats@\spxentry{GetAudioStats}}\ignorespaces 

\subsection{GetAudioStats()}
\label{\detokenize{reference/peblobjects:getaudiostats}}\label{\detokenize{reference/peblobjects:index-46}}
\sphinxAtStartPar
\sphinxstyleemphasis{Retrieves audio statistics from monitoring buffer}

\sphinxAtStartPar
\sphinxstylestrong{Description:}

\sphinxAtStartPar
Returns audio statistics from the most recent N milliseconds of an audio monitor created by \sphinxcode{\sphinxupquote{StartAudioMonitor()}}. The function returns a list containing three values: {[}energy, power, rmssd{]}. Energy represents total signal energy, power represents average power, and rmssd (root mean square of successive differences) indicates signal variability. These statistics are useful for voice key calibration, detecting speech onset, and monitoring audio levels. \sphinxstyleemphasis{ONLY AVAILABLE ON WINDOWS AND LINUX}.

\sphinxAtStartPar
\sphinxstylestrong{Usage:}

\begin{sphinxVerbatim}[commandchars=\\\{\}]
\PYG{n+nf}{GetAudioStats}\PYG{p}{(}\PYG{o}{\PYGZlt{}}\PYG{n+nv}{monitor}\PYG{o}{\PYGZgt{}}\PYG{p}{,}\PYG{+w}{ }\PYG{o}{\PYGZlt{}}\PYG{n+nv}{window\PYGZus{}ms}\PYG{o}{\PYGZgt{}}\PYG{p}{)}
\end{sphinxVerbatim}

\sphinxAtStartPar
\sphinxstylestrong{Example:}

\begin{sphinxVerbatim}[commandchars=\\\{\}]
\PYG{c+c1}{\PYGZsh{}\PYGZsh{} Monitor and display real\PYGZhy{}time audio statistics}
\PYG{n+nv}{monitor}\PYG{+w}{ }\PYG{o}{\PYGZlt{}\PYGZhy{}}\PYG{+w}{ }\PYG{n+nf}{StartAudioMonitor}\PYG{p}{(}\PYG{l+m+mi}{5000}\PYG{p}{)}

\PYG{k}{loop}\PYG{p}{(}\PYG{n+nv}{i}\PYG{p}{,}\PYG{+w}{ }\PYG{n+nf}{Sequence}\PYG{p}{(}\PYG{l+m+mi}{1}\PYG{p}{,}\PYG{+w}{ }\PYG{l+m+mi}{50}\PYG{p}{,}\PYG{+w}{ }\PYG{l+m+mi}{1}\PYG{p}{)}\PYG{p}{)}
\PYG{p}{\PYGZob{}}
\PYG{+w}{    }\PYG{n+nf}{Wait}\PYG{p}{(}\PYG{l+m+mi}{200}\PYG{p}{)}
\PYG{+w}{    }\PYG{n+nv}{stats}\PYG{+w}{ }\PYG{o}{\PYGZlt{}\PYGZhy{}}\PYG{+w}{ }\PYG{n+nf}{GetAudioStats}\PYG{p}{(}\PYG{n+nv}{monitor}\PYG{p}{,}\PYG{+w}{ }\PYG{l+m+mi}{1000}\PYG{p}{)}\PYG{+w}{  }\PYG{c+c1}{\PYGZsh{}\PYGZsh{} Last 1 second}
\PYG{+w}{    }\PYG{n+nv}{energy}\PYG{+w}{ }\PYG{o}{\PYGZlt{}\PYGZhy{}}\PYG{+w}{ }\PYG{n+nf}{First}\PYG{p}{(}\PYG{n+nv}{stats}\PYG{p}{)}
\PYG{+w}{    }\PYG{n+nv}{power}\PYG{+w}{ }\PYG{o}{\PYGZlt{}\PYGZhy{}}\PYG{+w}{ }\PYG{n+nf}{Nth}\PYG{p}{(}\PYG{n+nv}{stats}\PYG{p}{,}\PYG{+w}{ }\PYG{l+m+mi}{2}\PYG{p}{)}
\PYG{+w}{    }\PYG{n+nv}{rmssd}\PYG{+w}{ }\PYG{o}{\PYGZlt{}\PYGZhy{}}\PYG{+w}{ }\PYG{n+nf}{Third}\PYG{p}{(}\PYG{n+nv}{stats}\PYG{p}{)}

\PYG{+w}{    }\PYG{n+nf}{Print}\PYG{p}{(}\PYG{l+s+s2}{\PYGZdq{}Energy: \PYGZdq{}}\PYG{+w}{ }\PYG{o}{+}\PYG{+w}{ }\PYG{n+nv}{energy}\PYG{+w}{ }\PYG{o}{+}\PYG{+w}{ }\PYG{l+s+s2}{\PYGZdq{} Power: \PYGZdq{}}\PYG{+w}{ }\PYG{o}{+}\PYG{+w}{ }\PYG{n+nv}{power}\PYG{+w}{ }\PYG{o}{+}\PYG{+w}{ }\PYG{l+s+s2}{\PYGZdq{} RMSSD: \PYGZdq{}}\PYG{+w}{ }\PYG{o}{+}\PYG{+w}{ }\PYG{n+nv}{rmssd}\PYG{p}{)}
\PYG{p}{\PYGZcb{}}

\PYG{n+nf}{StopAudioMonitor}\PYG{p}{(}\PYG{n+nv}{monitor}\PYG{p}{)}
\end{sphinxVerbatim}

\sphinxAtStartPar
\sphinxstylestrong{See Also:}

\sphinxAtStartPar
\sphinxcode{\sphinxupquote{StartAudioMonitor()}}, \sphinxcode{\sphinxupquote{StopAudioMonitor()}}, \sphinxcode{\sphinxupquote{GetVocalResponseTime()}}

\index{SetCursorPosition@\spxentry{SetCursorPosition}}\ignorespaces 

\subsection{SetCursorPosition()}
\label{\detokenize{reference/peblobjects:setcursorposition}}\label{\detokenize{reference/peblobjects:index-47}}
\sphinxAtStartPar
\sphinxstylestrong{Description:}

\sphinxAtStartPar
Moves the editing cursor to a specified character               position in a textbox.

\sphinxAtStartPar
\sphinxstylestrong{Usage:}

\begin{sphinxVerbatim}[commandchars=\\\{\}]
\PYG{n+nf}{SetCursorPosition}\PYG{p}{(}\PYG{o}{\PYGZlt{}}\PYG{n+nv}{textbox}\PYG{o}{\PYGZgt{}}\PYG{p}{,}\PYG{+w}{ }\PYG{o}{\PYGZlt{}}\PYG{n+nv}{integer}\PYG{o}{\PYGZgt{}}\PYG{p}{)}
\end{sphinxVerbatim}

\sphinxAtStartPar
\sphinxstylestrong{Example:}

\begin{sphinxVerbatim}[commandchars=\\\{\}]
\PYG{n+nf}{SetCursorPosition}\PYG{p}{(}\PYG{n+nv}{tb}\PYG{p}{,}\PYG{+w}{ }\PYG{l+m+mi}{23}\PYG{p}{)}
\end{sphinxVerbatim}

\sphinxAtStartPar
\sphinxstylestrong{See Also:}

\sphinxAtStartPar
\sphinxcode{\sphinxupquote{SetEditable()}}, \sphinxcode{\sphinxupquote{GetCursorPosition()}}, \sphinxcode{\sphinxupquote{SetText()}}, \sphinxcode{\sphinxupquote{GetText()}}

\index{SetEditable@\spxentry{SetEditable}}\ignorespaces 

\subsection{SetEditable()}
\label{\detokenize{reference/peblobjects:seteditable}}\label{\detokenize{reference/peblobjects:index-48}}
\sphinxAtStartPar
\sphinxstyleemphasis{Turns on or off the editing cursor}

\sphinxAtStartPar
\sphinxstylestrong{Description:}

\sphinxAtStartPar
Sets the \sphinxcode{\sphinxupquote{editable\textquotesingle{}\textquotesingle{} status of the textbox.  All   this really does is turns on or off the cursor; editing must be done   with the (currently unsupported) device function \textasciigrave{}\textasciigrave{}GetInput()}}.

\sphinxAtStartPar
\sphinxstylestrong{Usage:}

\begin{sphinxVerbatim}[commandchars=\\\{\}]
\PYG{n+nf}{SetEditable}\PYG{p}{(}\PYG{p}{)}
\end{sphinxVerbatim}

\sphinxAtStartPar
\sphinxstylestrong{Example:}

\begin{sphinxVerbatim}[commandchars=\\\{\}]
\PYG{n+nf}{SetEditable}\PYG{p}{(}\PYG{n+nv}{tb}\PYG{p}{,}\PYG{+w}{ }\PYG{l+m+mi}{0}\PYG{p}{)}
\PYG{n+nf}{SetEditable}\PYG{p}{(}\PYG{n+nv}{tb}\PYG{p}{,}\PYG{+w}{ }\PYG{l+m+mi}{1}\PYG{p}{)}
\end{sphinxVerbatim}

\sphinxAtStartPar
\sphinxstylestrong{See Also:}

\sphinxAtStartPar
\sphinxcode{\sphinxupquote{GetEditable()}}

\index{SetFont@\spxentry{SetFont}}\ignorespaces 

\subsection{SetFont()}
\label{\detokenize{reference/peblobjects:setfont}}\label{\detokenize{reference/peblobjects:index-49}}
\sphinxAtStartPar
\sphinxstyleemphasis{Changes the font of a text object}

\sphinxAtStartPar
\sphinxstylestrong{Description:}

\sphinxAtStartPar
Resets the font of a textbox or label.  Change will   not appear until the next \sphinxcode{\sphinxupquote{Draw()}} function is called.  Can be   used, for example, to change the color of a label to give richer   feedback about correctness on a trial (see example below).  Font can alse be set by assigning to the object.font property of an object.

\sphinxAtStartPar
\sphinxstylestrong{Usage:}

\begin{sphinxVerbatim}[commandchars=\\\{\}]
\PYG{n+nf}{SetFont}\PYG{p}{(}\PYG{o}{\PYGZlt{}}\PYG{n+nv}{text}\PYG{o}{\PYGZhy{}}\PYG{n+nv}{widget}\PYG{o}{\PYGZgt{}}\PYG{p}{,}\PYG{+w}{ }\PYG{o}{\PYGZlt{}}\PYG{n+nv}{font}\PYG{o}{\PYGZgt{}}\PYG{p}{)}
\end{sphinxVerbatim}

\sphinxAtStartPar
\sphinxstylestrong{Example:}

\begin{sphinxVerbatim}[commandchars=\\\{\}]
\PYG{n+nv}{fontGreen}\PYG{+w}{ }\PYG{o}{\PYGZlt{}\PYGZhy{}}\PYG{+w}{ }\PYG{n+nf}{MakeFont}\PYG{p}{(}\PYG{l+s+s2}{\PYGZdq{}vera.ttf\PYGZdq{}}\PYG{p}{,}\PYG{l+m+mi}{1}\PYG{p}{,}\PYG{l+m+mi}{22}\PYG{p}{,}
\PYG{+w}{                      }\PYG{n+nf}{MakeColor}\PYG{p}{(}\PYG{l+s+s2}{\PYGZdq{}green\PYGZdq{}}\PYG{p}{)}\PYG{p}{,}
\PYG{+w}{                      }\PYG{n+nf}{MakeColor}\PYG{p}{(}\PYG{l+s+s2}{\PYGZdq{}black\PYGZdq{}}\PYG{p}{)}\PYG{p}{,}\PYG{+w}{ }\PYG{l+m+mi}{1}\PYG{p}{)}
\PYG{n+nv}{fontRed}\PYG{+w}{   }\PYG{o}{\PYGZlt{}\PYGZhy{}}\PYG{+w}{ }\PYG{n+nf}{MakeFont}\PYG{p}{(}\PYG{l+s+s2}{\PYGZdq{}vera.ttf\PYGZdq{}}\PYG{p}{,}\PYG{l+m+mi}{1}\PYG{p}{,}\PYG{l+m+mi}{22}\PYG{p}{,}
\PYG{+w}{                      }\PYG{n+nf}{MakeColor}\PYG{p}{(}\PYG{l+s+s2}{\PYGZdq{}red\PYGZdq{}}\PYG{p}{)}\PYG{p}{,}
\PYG{+w}{                      }\PYG{n+nf}{MakeColor}\PYG{p}{(}\PYG{l+s+s2}{\PYGZdq{}black\PYGZdq{}}\PYG{p}{)}\PYG{p}{,}\PYG{+w}{ }\PYG{l+m+mi}{1}\PYG{p}{)}
\PYG{n+nv}{label}\PYG{+w}{ }\PYG{o}{\PYGZlt{}\PYGZhy{}}\PYG{+w}{ }\PYG{n+nf}{MakeLabel}\PYG{p}{(}\PYG{n+nv}{fontGreen}\PYG{p}{,}\PYG{+w}{ }\PYG{l+s+s2}{\PYGZdq{}Correct\PYGZdq{}}\PYG{p}{)}

\PYG{c+c1}{\PYGZsh{}Do trial here.}

\PYG{k}{if}\PYG{p}{(}\PYG{n+nv}{response}\PYG{+w}{ }\PYG{o}{==}\PYG{+w}{ }\PYG{l+m+mi}{1}\PYG{p}{)}
\PYG{p}{\PYGZob{}}
\PYG{n+nf}{SetText}\PYG{p}{(}\PYG{n+nv}{label}\PYG{p}{,}\PYG{+w}{ }\PYG{l+s+s2}{\PYGZdq{}CORRECT\PYGZdq{}}\PYG{p}{)}
\PYG{n+nf}{SetFont}\PYG{p}{(}\PYG{n+nv}{label}\PYG{p}{,}\PYG{+w}{ }\PYG{n+nv}{fontGreen}\PYG{p}{)}
\PYG{p}{\PYGZcb{}}\PYG{+w}{ }\PYG{k}{else}\PYG{+w}{ }\PYG{p}{\PYGZob{}}
\PYG{n+nf}{SetText}\PYG{p}{(}\PYG{n+nv}{label}\PYG{p}{,}\PYG{+w}{ }\PYG{l+s+s2}{\PYGZdq{}INCORRECT\PYGZdq{}}\PYG{p}{)}
\PYG{n+nf}{SetFont}\PYG{p}{(}\PYG{n+nv}{label}\PYG{p}{,}\PYG{+w}{ }\PYGZdq{}\PYG{n+nv}{fontRed}\PYG{p}{)}
\PYG{p}{\PYGZcb{}}
\PYG{n+nf}{Draw}\PYG{p}{(}\PYG{p}{)}
\end{sphinxVerbatim}

\sphinxAtStartPar
\sphinxstylestrong{See Also:}

\sphinxAtStartPar
\sphinxcode{\sphinxupquote{SetText()}}

\index{SetPanning@\spxentry{SetPanning}}\ignorespaces 

\subsection{SetPanning()}
\label{\detokenize{reference/peblobjects:setpanning}}\label{\detokenize{reference/peblobjects:index-50}}
\sphinxAtStartPar
\sphinxstyleemphasis{Sets volume of left and right channel.}

\sphinxAtStartPar
\sphinxstylestrong{Description:}

\sphinxAtStartPar
Sets the audio panning; the volume of the left and right audio channels.

\sphinxAtStartPar
\sphinxstylestrong{Usage:}

\begin{sphinxVerbatim}[commandchars=\\\{\}]
\PYG{n+nf}{SetPanning}\PYG{p}{(}\PYG{o}{\PYGZlt{}}\PYG{n+nv}{audio}\PYG{o}{\PYGZgt{}}\PYG{p}{,}\PYG{o}{\PYGZlt{}}\PYG{n+nv}{left}\PYG{o}{\PYGZgt{}}\PYG{p}{,}\PYG{o}{\PYGZlt{}}\PYG{n+nv}{right}\PYG{o}{\PYGZgt{}}\PYG{p}{)}
\end{sphinxVerbatim}

\sphinxAtStartPar
\sphinxstylestrong{Example:}

\begin{sphinxVerbatim}[commandchars=\\\{\}]
\PYG{n+nv}{one}\PYG{+w}{ }\PYG{o}{\PYGZlt{}\PYGZhy{}}\PYG{+w}{ }\PYG{n+nf}{LoadSound}\PYG{p}{(}\PYG{l+s+s2}{\PYGZdq{}1.wav\PYGZdq{}}\PYG{p}{)}
\PYG{+w}{   }\PYG{n+nf}{PlayForeground}\PYG{p}{(}\PYG{n+nv}{one}\PYG{p}{)}
\PYG{+w}{   }\PYG{n+nf}{SetPanning}\PYG{p}{(}\PYG{n+nv}{one}\PYG{p}{,}\PYG{l+m+mf}{1.0}\PYG{p}{,}\PYG{l+m+mf}{0.0}\PYG{p}{)}
\PYG{+w}{   }\PYG{n+nf}{PlayForeground}\PYG{p}{(}\PYG{n+nv}{one}\PYG{p}{)}
\PYG{+w}{   }\PYG{n+nf}{SetPanning}\PYG{p}{(}\PYG{n+nv}{one}\PYG{p}{,}\PYG{l+m+mf}{.5}\PYG{p}{,}\PYG{l+m+mf}{.5}\PYG{p}{)}
\PYG{+w}{   }\PYG{n+nf}{PlayForeground}\PYG{p}{(}\PYG{n+nv}{one}\PYG{p}{)}
\end{sphinxVerbatim}

\sphinxAtStartPar
\sphinxstylestrong{See Also:}

\sphinxAtStartPar
\sphinxcode{\sphinxupquote{LoadSound()}}

\index{SetPlayRepeats@\spxentry{SetPlayRepeats}}\ignorespaces 

\subsection{SetPlayRepeats()}
\label{\detokenize{reference/peblobjects:setplayrepeats}}\label{\detokenize{reference/peblobjects:index-51}}
\sphinxAtStartPar
\sphinxstyleemphasis{Sets a repeat count on a sound playback.}

\sphinxAtStartPar
\sphinxstylestrong{Description:}

\sphinxAtStartPar
Sets repetition count on an audio file.  When played back, i   will play this sound reps+1 times. If set to 0, it will play just once.  If set to \sphinxhyphen{}1, it   will repeat indefinitely.

\sphinxAtStartPar
\sphinxstylestrong{Usage:}

\begin{sphinxVerbatim}[commandchars=\\\{\}]
\PYG{n+nf}{SetPlayRepeats}\PYG{p}{(}\PYG{o}{\PYGZlt{}}\PYG{n+nv}{audio}\PYG{o}{\PYGZgt{}}\PYG{p}{,}\PYG{o}{\PYGZlt{}}\PYG{n+nv}{reps}\PYG{o}{\PYGZgt{}}\PYG{p}{)}
\end{sphinxVerbatim}

\sphinxAtStartPar
\sphinxstylestrong{Example:}

\begin{sphinxVerbatim}[commandchars=\\\{\}]
\PYG{n+nv}{one}\PYG{+w}{ }\PYG{o}{\PYGZlt{}\PYGZhy{}}\PYG{+w}{ }\PYG{n+nf}{LoadSound}\PYG{p}{(}\PYG{l+s+s2}{\PYGZdq{}1.wav\PYGZdq{}}\PYG{p}{)}
\PYG{+w}{   }\PYG{n+nf}{PlayForeground}\PYG{p}{(}\PYG{n+nv}{one}\PYG{p}{)}
\PYG{+w}{   }\PYG{n+nf}{SetPlayRepeats}\PYG{p}{(}\PYG{n+nv}{one}\PYG{p}{,}\PYG{l+m+mi}{5}\PYG{p}{)}
\PYG{+w}{   }\PYG{n+nf}{PlayForeground}\PYG{p}{(}\PYG{n+nv}{one}\PYG{p}{)}
\PYG{+w}{   }\PYG{n+nf}{SetPanning}\PYG{p}{(}\PYG{n+nv}{one}\PYG{p}{,}\PYG{o}{\PYGZhy{}}\PYG{l+m+mi}{1}\PYG{p}{)}
\PYG{+w}{   }\PYG{n+nf}{PlayBackground}\PYG{p}{(}\PYG{n+nv}{one}\PYG{p}{)}
\PYG{+w}{   }\PYG{n+nf}{Wait}\PYG{p}{(}\PYG{l+m+mi}{5000}\PYG{p}{)}
\PYG{+w}{   }\PYG{n+nf}{Stop}\PYG{p}{(}\PYG{n+nv}{one}\PYG{p}{)}
\end{sphinxVerbatim}

\sphinxAtStartPar
\sphinxstylestrong{See Also:}

\sphinxAtStartPar
\sphinxcode{\sphinxupquote{LoadSound()}}

\index{SetProperty@\spxentry{SetProperty}}\ignorespaces 

\subsection{SetProperty()}
\label{\detokenize{reference/peblobjects:setproperty}}\label{\detokenize{reference/peblobjects:index-52}}
\sphinxAtStartPar
\sphinxstyleemphasis{Sets property of an object}

\sphinxAtStartPar
\sphinxstylestrong{Description:}

\sphinxAtStartPar
Sets a a property of a custom object.   This works for custom or built\sphinxhyphen{}in objects, but new properties can only be set on custom object. This function works essentially identically to the obj.property assignment, but it allows you to create property names from input. It is used extensively for the PEBL parameter setting.

\sphinxAtStartPar
\sphinxstylestrong{Example:}

\begin{sphinxVerbatim}[commandchars=\\\{\}]
\PYG{n+nv}{obj}\PYG{+w}{ }\PYG{o}{\PYGZlt{}\PYGZhy{}}\PYG{+w}{ }\PYG{n+nf}{MakeCustomObject}\PYG{p}{(}\PYG{l+s+s2}{\PYGZdq{}myobject\PYGZdq{}}\PYG{p}{)}
\PYG{+w}{  }\PYG{n+nv}{obj.taste}\PYG{+w}{ }\PYG{o}{\PYGZlt{}\PYGZhy{}}\PYG{+w}{ }\PYG{l+s+s2}{\PYGZdq{}buttery\PYGZdq{}}
\PYG{+w}{  }\PYG{n+nv}{obj.texture}\PYG{+w}{ }\PYG{o}{\PYGZlt{}\PYGZhy{}}\PYG{+w}{ }\PYG{l+s+s2}{\PYGZdq{}creamy\PYGZdq{}}
\PYG{+w}{  }\PYG{n+nf}{SetProperty}\PYG{p}{(}\PYG{n+nv}{obj}\PYG{p}{,}\PYG{l+s+s2}{\PYGZdq{}flavor\PYGZdq{}}\PYG{p}{,}\PYG{l+s+s2}{\PYGZdq{}tasty\PYGZdq{}}\PYG{p}{)}

\PYG{+w}{  }\PYG{n+nv}{list}\PYG{+w}{ }\PYG{o}{\PYGZlt{}\PYGZhy{}}\PYG{+w}{ }\PYG{n+nf}{GetPropertyList}\PYG{p}{(}\PYG{n+nv}{obj}\PYG{p}{)}
\PYG{+w}{  }\PYG{k}{loop}\PYG{p}{(}\PYG{n+nv}{i}\PYG{p}{,}\PYG{n+nv}{list}\PYG{p}{)}
\PYG{+w}{   }\PYG{p}{\PYGZob{}}
\PYG{+w}{     }\PYG{k}{if}\PYG{p}{(}\PYG{n+nf}{PropertyExists}\PYG{p}{(}\PYG{n+nv}{obj}\PYG{p}{,}\PYG{n+nv}{i}\PYG{p}{)}
\PYG{+w}{      }\PYG{p}{\PYGZob{}}
\PYG{+w}{        }\PYG{n+nf}{Print}\PYG{p}{(}\PYG{n+nv}{i}\PYG{+w}{  }\PYG{o}{+}\PYG{+w}{ }\PYG{l+s+s2}{\PYGZdq{}:  \PYGZdq{}}\PYG{+w}{ }\PYG{o}{+}\PYG{+w}{ }\PYG{n+nf}{GetProperty}\PYG{p}{(}\PYG{n+nv}{obj}\PYG{p}{,}\PYG{n+nv}{i}\PYG{p}{)}\PYG{p}{)}
\PYG{+w}{      }\PYG{p}{\PYGZcb{}}
\PYG{+w}{   }\PYG{p}{\PYGZcb{}}
\end{sphinxVerbatim}

\sphinxAtStartPar
\sphinxstylestrong{See Also:}

\sphinxAtStartPar
\sphinxcode{\sphinxupquote{GetProperty()}}, \sphinxcode{\sphinxupquote{PropertyExists()}}, \sphinxcode{\sphinxupquote{GetPropertyList()}}, \sphinxcode{\sphinxupquote{MakeCustomObject()}}, \sphinxcode{\sphinxupquote{PrintProperties()}}

\index{SetText@\spxentry{SetText}}\ignorespaces 

\subsection{SetText()}
\label{\detokenize{reference/peblobjects:settext}}\label{\detokenize{reference/peblobjects:index-53}}
\sphinxAtStartPar
\sphinxstyleemphasis{Sets the text in a textbox or label}

\sphinxAtStartPar
\sphinxstylestrong{Description:}

\sphinxAtStartPar
Resets the text of a textbox or label.  Change will not                 appear until the next \sphinxcode{\sphinxupquote{Draw()}} function is called.  The object.text property can also be used to change text of an object, by doing: \sphinxcode{\sphinxupquote{object.text \textless{}\sphinxhyphen{} "new text"}}

\sphinxAtStartPar
\sphinxstylestrong{Usage:}

\begin{sphinxVerbatim}[commandchars=\\\{\}]
\PYG{n+nf}{SetText}\PYG{p}{(}\PYG{o}{\PYGZlt{}}\PYG{n+nv}{text}\PYG{o}{\PYGZhy{}}\PYG{n+nv}{widget}\PYG{o}{\PYGZgt{}}\PYG{p}{,}\PYG{+w}{ }\PYG{o}{\PYGZlt{}}\PYG{n+nv}{text}\PYG{o}{\PYGZgt{}}\PYG{p}{)}
\end{sphinxVerbatim}

\sphinxAtStartPar
\sphinxstylestrong{Example:}

\begin{sphinxVerbatim}[commandchars=\\\{\}]
\PYG{c+c1}{\PYGZsh{} Fixation Cross:}
\PYG{n+nv}{label}\PYG{+w}{ }\PYG{o}{\PYGZlt{}\PYGZhy{}}\PYG{+w}{ }\PYG{n+nf}{MakeLabel}\PYG{p}{(}\PYG{n+nv}{font}\PYG{p}{,}\PYG{+w}{ }\PYG{l+s+s2}{\PYGZdq{}+\PYGZdq{}}\PYG{p}{)}
\PYG{n+nf}{Draw}\PYG{p}{(}\PYG{p}{)}

\PYG{n+nf}{SetText}\PYG{p}{(}\PYG{n+nv}{label}\PYG{p}{,}\PYG{+w}{ }\PYG{l+s+s2}{\PYGZdq{}X\PYGZdq{}}\PYG{p}{)}
\PYG{n+nf}{Wait}\PYG{p}{(}\PYG{l+m+mi}{100}\PYG{p}{)}
\PYG{n+nf}{Draw}\PYG{p}{(}\PYG{p}{)}
\end{sphinxVerbatim}

\sphinxAtStartPar
\sphinxstylestrong{See Also:}

\sphinxAtStartPar
\sphinxcode{\sphinxupquote{GetText()}}, \sphinxcode{\sphinxupquote{SetFont()}}

\index{Show@\spxentry{Show}}\ignorespaces 

\subsection{Show()}
\label{\detokenize{reference/peblobjects:show}}\label{\detokenize{reference/peblobjects:index-54}}
\sphinxAtStartPar
\sphinxstyleemphasis{Shows an object}

\sphinxAtStartPar
\sphinxstylestrong{Description:}

\sphinxAtStartPar
Sets a widget to visible, once it has been added to   a parent widget.  This just changes the visibility property, it does   not make the widget appear.  The widget will not be displayed until   the \sphinxcode{\sphinxupquote{Draw()}} function is called.  The .visible property of objects can also be used to hide or show the object.

\sphinxAtStartPar
\sphinxstylestrong{Usage:}

\begin{sphinxVerbatim}[commandchars=\\\{\}]
\PYG{n+nf}{Show}\PYG{p}{(}\PYG{o}{\PYGZlt{}}\PYG{n+nv}{object}\PYG{o}{\PYGZgt{}}\PYG{p}{)}
\end{sphinxVerbatim}

\sphinxAtStartPar
\sphinxstylestrong{Example:}

\begin{sphinxVerbatim}[commandchars=\\\{\}]
\PYG{n+nv}{window}\PYG{+w}{ }\PYG{o}{\PYGZlt{}\PYGZhy{}}\PYG{+w}{ }\PYG{n+nf}{MakeWindow}\PYG{p}{(}\PYG{p}{)}
\PYG{n+nv}{image1}\PYG{+w}{  }\PYG{o}{\PYGZlt{}\PYGZhy{}}\PYG{+w}{ }\PYG{n+nf}{MakeImage}\PYG{p}{(}\PYG{l+s+s2}{\PYGZdq{}pebl.bmp\PYGZdq{}}\PYG{p}{)}
\PYG{n+nv}{image2}\PYG{+w}{  }\PYG{o}{\PYGZlt{}\PYGZhy{}}\PYG{+w}{ }\PYG{n+nf}{MakeImage}\PYG{p}{(}\PYG{l+s+s2}{\PYGZdq{}pebl.bmp\PYGZdq{}}\PYG{p}{)}
\PYG{n+nf}{AddObject}\PYG{p}{(}\PYG{n+nv}{image1}\PYG{p}{,}\PYG{+w}{ }\PYG{n+nv}{window}\PYG{p}{)}
\PYG{n+nf}{AddObject}\PYG{p}{(}\PYG{n+nv}{image2}\PYG{p}{,}\PYG{+w}{ }\PYG{n+nv}{window}\PYG{p}{)}
\PYG{n+nf}{Hide}\PYG{p}{(}\PYG{n+nv}{image2}\PYG{p}{)}
\PYG{n+nf}{Draw}\PYG{p}{(}\PYG{p}{)}
\PYG{n+nf}{Wait}\PYG{p}{(}\PYG{l+m+mi}{300}\PYG{p}{)}
\PYG{n+nf}{Show}\PYG{p}{(}\PYG{n+nv}{image2}\PYG{p}{)}
\PYG{n+nf}{Draw}\PYG{p}{(}\PYG{p}{)}
\end{sphinxVerbatim}

\sphinxAtStartPar
\sphinxstylestrong{See Also:}

\sphinxAtStartPar
\sphinxcode{\sphinxupquote{Hide()}}

\index{Square@\spxentry{Square}}\ignorespaces 

\subsection{Square()}
\label{\detokenize{reference/peblobjects:square}}\label{\detokenize{reference/peblobjects:index-55}}
\sphinxAtStartPar
\sphinxstyleemphasis{Creates square with width size centered at position x,y}

\sphinxAtStartPar
\sphinxstylestrong{Description:}

\sphinxAtStartPar
Creates a square for graphing at x,y with size   \sphinxcode{\sphinxupquote{\textless{}size\textgreater{}}}. Squares are only currently definable oriented in   horizontal/vertical directions.  A square  must be added   to a parent widget before it can be drawn; it may be added to   widgets other than a base window.  The properties of squares may be   changed by accessing their properties directly, including the FILLED   property which makes the object an outline versus a filled shape.

\sphinxAtStartPar
\sphinxstylestrong{Usage:}

\begin{sphinxVerbatim}[commandchars=\\\{\}]
\PYG{n+nf}{Ellipse}\PYG{p}{(}\PYG{o}{\PYGZlt{}}\PYG{n+nv}{x}\PYG{o}{\PYGZgt{}}\PYG{p}{,}\PYG{+w}{ }\PYG{o}{\PYGZlt{}}\PYG{n+nv}{y}\PYG{o}{\PYGZgt{}}\PYG{p}{,}\PYG{+w}{ }\PYG{o}{\PYGZlt{}}\PYG{n+nv}{size}\PYG{o}{\PYGZgt{}}\PYG{p}{,}\PYG{+w}{ }\PYG{o}{\PYGZlt{}}\PYG{n+nv}{color}\PYG{o}{\PYGZgt{}}\PYG{p}{)}
\end{sphinxVerbatim}

\sphinxAtStartPar
\sphinxstylestrong{Example:}

\begin{sphinxVerbatim}[commandchars=\\\{\}]
\PYG{n+nv}{s}\PYG{+w}{ }\PYG{o}{\PYGZlt{}\PYGZhy{}}\PYG{+w}{ }\PYG{n+nf}{Square}\PYG{p}{(}\PYG{l+m+mi}{30}\PYG{p}{,}\PYG{l+m+mi}{30}\PYG{p}{,}\PYG{l+m+mi}{20}\PYG{p}{,}\PYG{+w}{ }\PYG{n+nf}{MakeColor}\PYG{p}{(}\PYG{n+nv+vg}{green}\PYG{p}{)}\PYG{p}{)}
\PYG{+w}{  }\PYG{n+nf}{AddObject}\PYG{p}{(}\PYG{n+nv}{s}\PYG{p}{,}\PYG{+w}{ }\PYG{n+nv}{win}\PYG{p}{)}
\PYG{+w}{  }\PYG{n+nf}{Draw}\PYG{p}{(}\PYG{p}{)}
\end{sphinxVerbatim}

\sphinxAtStartPar
\sphinxstylestrong{See Also:}

\sphinxAtStartPar
\sphinxcode{\sphinxupquote{Circle()}}, \sphinxcode{\sphinxupquote{Ellipse()}}, \sphinxcode{\sphinxupquote{Rectangle()}}, \sphinxcode{\sphinxupquote{Line()}}

\index{StartPlayback@\spxentry{StartPlayback}}\ignorespaces 

\subsection{StartPlayback()}
\label{\detokenize{reference/peblobjects:startplayback}}\label{\detokenize{reference/peblobjects:index-56}}
\sphinxAtStartPar
\sphinxstyleemphasis{Initiates playback in background, updated with Wait()}

\sphinxAtStartPar
\sphinxstylestrong{Description:}

\sphinxAtStartPar
Initiates playback of a movie so that it will play in the background when a Wait() or WaitFor() function is called.  This allows one to collect a response while  playing a movie.  The movie will not actually play until the event loop is started, typically with something like Wait().

\sphinxAtStartPar
\sphinxstylestrong{Usage:}

\begin{sphinxVerbatim}[commandchars=\\\{\}]
\PYG{n+nf}{StartPlayBack}\PYG{p}{(}\PYG{n+nv}{movie}\PYG{p}{)}
\end{sphinxVerbatim}

\sphinxAtStartPar
\sphinxstylestrong{Example:}

\begin{sphinxVerbatim}[commandchars=\\\{\}]
\PYG{n+nv}{movie}\PYG{+w}{ }\PYG{o}{\PYGZlt{}\PYGZhy{}}\PYG{+w}{ }\PYG{n+nf}{LoadMovie}\PYG{p}{(}\PYG{l+s+s2}{\PYGZdq{}movie.avi\PYGZdq{}}\PYG{p}{,}\PYG{n+nv+vg}{gWin}\PYG{p}{,}\PYG{l+m+mi}{640}\PYG{p}{,}\PYG{l+m+mi}{480}\PYG{p}{)}
\PYG{+w}{   }\PYG{n+nf}{PrintProperties}\PYG{p}{(}\PYG{n+nv}{movie}\PYG{p}{)}
\PYG{+w}{   }\PYG{n+nf}{Move}\PYG{p}{(}\PYG{n+nv}{movie}\PYG{p}{,}\PYG{l+m+mi}{20}\PYG{p}{,}\PYG{l+m+mi}{20}\PYG{p}{)}
\PYG{+w}{   }\PYG{n+nf}{Draw}\PYG{p}{(}\PYG{p}{)}
\PYG{+w}{   }\PYG{n+nf}{StartPlayback}\PYG{p}{(}\PYG{n+nv}{movie}\PYG{p}{)}
\PYG{+w}{   }\PYG{n+nf}{Wait}\PYG{p}{(}\PYG{l+m+mi}{500}\PYG{p}{)}\PYG{+w}{ }\PYG{c+c1}{\PYGZsh{}Play 500 ms of the movie.}
\PYG{+w}{   }\PYG{n+nf}{PausePlayback}\PYG{p}{(}\PYG{n+nv}{movie}\PYG{p}{)}
\end{sphinxVerbatim}

\sphinxAtStartPar
\sphinxstylestrong{See Also:}

\sphinxAtStartPar
\sphinxcode{\sphinxupquote{LoadAudioFile()}}, \sphinxcode{\sphinxupquote{LoadMovie()}}, \sphinxcode{\sphinxupquote{PlayMovie()}}, \sphinxcode{\sphinxupquote{PausePlayback()}}

\index{Stop@\spxentry{Stop}}\ignorespaces 

\subsection{Stop()}
\label{\detokenize{reference/peblobjects:stop}}\label{\detokenize{reference/peblobjects:index-57}}
\sphinxAtStartPar
\sphinxstyleemphasis{Stops a sound playing in the background from playing}

\sphinxAtStartPar
\sphinxstylestrong{Description:}

\sphinxAtStartPar
Stops a sound playing in the background from   playing.  Calling \sphinxcode{\sphinxupquote{Stop()}} on a sound object that is not   playing should have no effect, but if an object is aliased,   \sphinxcode{\sphinxupquote{Stop()}} will stop the file.  Note that sounds play in a   separate thread, so interrupting the thread has a granularity up to   the duration of the thread\sphinxhyphen{}switching quantum on your computer; this   may be tens of milliseconds.

\sphinxAtStartPar
\sphinxstylestrong{Usage:}

\begin{sphinxVerbatim}[commandchars=\\\{\}]
\PYG{n+nf}{Stop}\PYG{p}{(}\PYG{o}{\PYGZlt{}}\PYG{n+nv}{sound}\PYG{o}{\PYGZhy{}}\PYG{n+nv}{object}\PYG{o}{\PYGZgt{}}\PYG{p}{)}
\end{sphinxVerbatim}

\sphinxAtStartPar
\sphinxstylestrong{Example:}

\begin{sphinxVerbatim}[commandchars=\\\{\}]
\PYG{n+nv}{buzz}\PYG{+w}{ }\PYG{o}{\PYGZlt{}\PYGZhy{}}\PYG{+w}{ }\PYG{n+nf}{LoadSound}\PYG{p}{(}\PYG{l+s+s2}{\PYGZdq{}buzz.wav\PYGZdq{}}\PYG{p}{)}
\PYG{n+nf}{PlayBackground}\PYG{p}{(}\PYG{n+nv}{buzz}\PYG{p}{)}
\PYG{n+nf}{Wait}\PYG{p}{(}\PYG{l+m+mi}{50}\PYG{p}{)}
\PYG{n+nf}{Stop}\PYG{p}{(}\PYG{n+nv}{buzz}\PYG{p}{)}
\end{sphinxVerbatim}

\sphinxAtStartPar
\sphinxstylestrong{See Also:}

\sphinxAtStartPar
\sphinxcode{\sphinxupquote{PlayForeground()}}, \sphinxcode{\sphinxupquote{PlayBackGround()}}

\index{ThickLine@\spxentry{ThickLine}}\ignorespaces 

\subsection{ThickLine()}
\label{\detokenize{reference/peblobjects:thickline}}\label{\detokenize{reference/peblobjects:index-58}}
\sphinxAtStartPar
\sphinxstyleemphasis{Creates a thick line between two points}

\sphinxAtStartPar
\sphinxstylestrong{Description:}

\sphinxAtStartPar
Makes a thick line between two coordinates. This uses the SDL\_gfx thickline primitive.

\sphinxAtStartPar
\sphinxstylestrong{Usage:}

\begin{sphinxVerbatim}[commandchars=\\\{\}]
\PYG{n+nf}{ThickLine}\PYG{p}{(}\PYG{o}{\PYGZlt{}}\PYG{n+nv}{x1}\PYG{o}{\PYGZgt{}}\PYG{p}{,}\PYG{o}{\PYGZlt{}}\PYG{n+nv}{y1}\PYG{o}{\PYGZgt{}}\PYG{p}{,}\PYG{o}{\PYGZlt{}}\PYG{n+nv}{x2}\PYG{o}{\PYGZgt{}}\PYG{p}{,}\PYG{o}{\PYGZlt{}}\PYG{n+nv}{y2}\PYG{o}{\PYGZgt{}}\PYG{p}{,}
\PYG{+w}{          }\PYG{o}{\PYGZlt{}}\PYG{n+nv}{size}\PYG{o}{\PYGZhy{}}\PYG{n+nv}{in}\PYG{o}{\PYGZhy{}}\PYG{n+nv}{pixels}\PYG{o}{\PYGZgt{}}\PYG{p}{,}\PYG{o}{\PYGZlt{}}\PYG{n+nv}{color}\PYG{o}{\PYGZgt{}}\PYG{p}{)}
\end{sphinxVerbatim}

\sphinxAtStartPar
\sphinxstylestrong{Example:}

\begin{sphinxVerbatim}[commandchars=\\\{\}]
\PYG{n+nv}{a}\PYG{+w}{ }\PYG{o}{\PYGZlt{}\PYGZhy{}}\PYG{+w}{ }\PYG{n+nf}{ThickLine}\PYG{p}{(}\PYG{l+m+mi}{10}\PYG{p}{,}\PYG{l+m+mi}{10}\PYG{p}{,}\PYG{l+m+mi}{300}\PYG{p}{,}\PYG{l+m+mi}{400}\PYG{p}{,}\PYG{l+m+mi}{20}\PYG{p}{,}
\PYG{+w}{                }\PYG{n+nf}{MakeColor}\PYG{p}{(}\PYG{l+s+s2}{\PYGZdq{}red\PYGZdq{}}\PYG{p}{)}\PYG{p}{)}
\PYG{+w}{   }\PYG{n+nf}{AddObject}\PYG{p}{(}\PYG{n+nv}{a}\PYG{p}{,}\PYG{n+nv+vg}{gWin}\PYG{p}{)}
\PYG{+w}{   }\PYG{n+nf}{Draw}\PYG{p}{(}\PYG{p}{)}
\end{sphinxVerbatim}

\sphinxAtStartPar
\sphinxstylestrong{See Also:}

\sphinxAtStartPar
\sphinxcode{\sphinxupquote{Line()}}, \sphinxcode{\sphinxupquote{Polygon()}}

\sphinxstepscope


\section{PEBLStream \sphinxhyphen{} File and Network I/O}
\label{\detokenize{reference/peblstream:peblstream-file-and-network-i-o}}\label{\detokenize{reference/peblstream::doc}}
\sphinxAtStartPar
This module contains functions for file I/O, network communication, and data streaming.

\begin{sphinxShadowBox}
\sphinxstyletopictitle{Function Index}
\begin{itemize}
\item {} 
\sphinxAtStartPar
\phantomsection\label{\detokenize{reference/peblstream:id1}}{\hyperref[\detokenize{reference/peblstream:appendfile}]{\sphinxcrossref{AppendFile()}}}

\item {} 
\sphinxAtStartPar
\phantomsection\label{\detokenize{reference/peblstream:id2}}{\hyperref[\detokenize{reference/peblstream:acceptnetworkconnection}]{\sphinxcrossref{AcceptNetworkConnection()}}}

\item {} 
\sphinxAtStartPar
\phantomsection\label{\detokenize{reference/peblstream:id3}}{\hyperref[\detokenize{reference/peblstream:checkfornetworkconnection}]{\sphinxcrossref{CheckForNetworkConnection()}}}

\item {} 
\sphinxAtStartPar
\phantomsection\label{\detokenize{reference/peblstream:id4}}{\hyperref[\detokenize{reference/peblstream:closenetworkconnection}]{\sphinxcrossref{CloseNetworkConnection()}}}

\item {} 
\sphinxAtStartPar
\phantomsection\label{\detokenize{reference/peblstream:id5}}{\hyperref[\detokenize{reference/peblstream:connecttohost}]{\sphinxcrossref{ConnectToHost()}}}

\item {} 
\sphinxAtStartPar
\phantomsection\label{\detokenize{reference/peblstream:id6}}{\hyperref[\detokenize{reference/peblstream:connecttoip}]{\sphinxcrossref{ConnectToIP()}}}

\item {} 
\sphinxAtStartPar
\phantomsection\label{\detokenize{reference/peblstream:id7}}{\hyperref[\detokenize{reference/peblstream:copyfile}]{\sphinxcrossref{CopyFile()}}}

\item {} 
\sphinxAtStartPar
\phantomsection\label{\detokenize{reference/peblstream:id8}}{\hyperref[\detokenize{reference/peblstream:endoffile}]{\sphinxcrossref{EndOfFile()}}}

\item {} 
\sphinxAtStartPar
\phantomsection\label{\detokenize{reference/peblstream:id9}}{\hyperref[\detokenize{reference/peblstream:endofline}]{\sphinxcrossref{EndOfLine()}}}

\item {} 
\sphinxAtStartPar
\phantomsection\label{\detokenize{reference/peblstream:id10}}{\hyperref[\detokenize{reference/peblstream:fileclose}]{\sphinxcrossref{FileClose()}}}

\item {} 
\sphinxAtStartPar
\phantomsection\label{\detokenize{reference/peblstream:id11}}{\hyperref[\detokenize{reference/peblstream:fileopenappend}]{\sphinxcrossref{FileOpenAppend()}}}

\item {} 
\sphinxAtStartPar
\phantomsection\label{\detokenize{reference/peblstream:id12}}{\hyperref[\detokenize{reference/peblstream:fileopenoverwrite}]{\sphinxcrossref{FileOpenOverwrite()}}}

\item {} 
\sphinxAtStartPar
\phantomsection\label{\detokenize{reference/peblstream:id13}}{\hyperref[\detokenize{reference/peblstream:fileopenread}]{\sphinxcrossref{FileOpenRead()}}}

\item {} 
\sphinxAtStartPar
\phantomsection\label{\detokenize{reference/peblstream:id14}}{\hyperref[\detokenize{reference/peblstream:fileopenwrite}]{\sphinxcrossref{FileOpenWrite()}}}

\item {} 
\sphinxAtStartPar
\phantomsection\label{\detokenize{reference/peblstream:id15}}{\hyperref[\detokenize{reference/peblstream:fileprint}]{\sphinxcrossref{FilePrint()}}}

\item {} 
\sphinxAtStartPar
\phantomsection\label{\detokenize{reference/peblstream:id16}}{\hyperref[\detokenize{reference/peblstream:filereadcharacter}]{\sphinxcrossref{FileReadCharacter()}}}

\item {} 
\sphinxAtStartPar
\phantomsection\label{\detokenize{reference/peblstream:id17}}{\hyperref[\detokenize{reference/peblstream:filereadline}]{\sphinxcrossref{FileReadLine()}}}

\item {} 
\sphinxAtStartPar
\phantomsection\label{\detokenize{reference/peblstream:id18}}{\hyperref[\detokenize{reference/peblstream:filereadlist}]{\sphinxcrossref{FileReadList()}}}

\item {} 
\sphinxAtStartPar
\phantomsection\label{\detokenize{reference/peblstream:id19}}{\hyperref[\detokenize{reference/peblstream:filereadtable}]{\sphinxcrossref{FileReadTable()}}}

\item {} 
\sphinxAtStartPar
\phantomsection\label{\detokenize{reference/peblstream:id20}}{\hyperref[\detokenize{reference/peblstream:filereadtext}]{\sphinxcrossref{FileReadText()}}}

\item {} 
\sphinxAtStartPar
\phantomsection\label{\detokenize{reference/peblstream:id21}}{\hyperref[\detokenize{reference/peblstream:filereadword}]{\sphinxcrossref{FileReadWord()}}}

\item {} 
\sphinxAtStartPar
\phantomsection\label{\detokenize{reference/peblstream:id22}}{\hyperref[\detokenize{reference/peblstream:getdata}]{\sphinxcrossref{GetData()}}}

\item {} 
\sphinxAtStartPar
\phantomsection\label{\detokenize{reference/peblstream:id23}}{\hyperref[\detokenize{reference/peblstream:getmyipaddress}]{\sphinxcrossref{GetMyIPAddress()}}}

\item {} 
\sphinxAtStartPar
\phantomsection\label{\detokenize{reference/peblstream:id24}}{\hyperref[\detokenize{reference/peblstream:getpportstate}]{\sphinxcrossref{GetPPortState()}}}

\item {} 
\sphinxAtStartPar
\phantomsection\label{\detokenize{reference/peblstream:id25}}{\hyperref[\detokenize{reference/peblstream:opencomport}]{\sphinxcrossref{OpenCOMPort()}}}

\item {} 
\sphinxAtStartPar
\phantomsection\label{\detokenize{reference/peblstream:id26}}{\hyperref[\detokenize{reference/peblstream:opennetworklistener}]{\sphinxcrossref{OpenNetworkListener()}}}

\item {} 
\sphinxAtStartPar
\phantomsection\label{\detokenize{reference/peblstream:id27}}{\hyperref[\detokenize{reference/peblstream:openpport}]{\sphinxcrossref{OpenPPort()}}}

\item {} 
\sphinxAtStartPar
\phantomsection\label{\detokenize{reference/peblstream:id28}}{\hyperref[\detokenize{reference/peblstream:parsejson}]{\sphinxcrossref{ParseJSON()}}}

\item {} 
\sphinxAtStartPar
\phantomsection\label{\detokenize{reference/peblstream:id29}}{\hyperref[\detokenize{reference/peblstream:print}]{\sphinxcrossref{Print()}}}

\item {} 
\sphinxAtStartPar
\phantomsection\label{\detokenize{reference/peblstream:id30}}{\hyperref[\detokenize{reference/peblstream:senddata}]{\sphinxcrossref{SendData()}}}

\item {} 
\sphinxAtStartPar
\phantomsection\label{\detokenize{reference/peblstream:id31}}{\hyperref[\detokenize{reference/peblstream:setnetworkport}]{\sphinxcrossref{SetNetworkPort()}}}

\item {} 
\sphinxAtStartPar
\phantomsection\label{\detokenize{reference/peblstream:id32}}{\hyperref[\detokenize{reference/peblstream:setpportmode}]{\sphinxcrossref{SetPPortMode()}}}

\item {} 
\sphinxAtStartPar
\phantomsection\label{\detokenize{reference/peblstream:id33}}{\hyperref[\detokenize{reference/peblstream:setpportstate}]{\sphinxcrossref{SetPPortState()}}}

\item {} 
\sphinxAtStartPar
\phantomsection\label{\detokenize{reference/peblstream:id34}}{\hyperref[\detokenize{reference/peblstream:waitfornetworkconnection}]{\sphinxcrossref{WaitForNetworkConnection()}}}

\item {} 
\sphinxAtStartPar
\phantomsection\label{\detokenize{reference/peblstream:id35}}{\hyperref[\detokenize{reference/peblstream:writepng}]{\sphinxcrossref{WritePNG()}}}

\item {} 
\sphinxAtStartPar
\phantomsection\label{\detokenize{reference/peblstream:id36}}{\hyperref[\detokenize{reference/peblstream:comportgetbyte}]{\sphinxcrossref{COMPortGetByte()}}}

\item {} 
\sphinxAtStartPar
\phantomsection\label{\detokenize{reference/peblstream:id37}}{\hyperref[\detokenize{reference/peblstream:comportsendbyte}]{\sphinxcrossref{COMPortSendByte()}}}

\item {} 
\sphinxAtStartPar
\phantomsection\label{\detokenize{reference/peblstream:id38}}{\hyperref[\detokenize{reference/peblstream:gethttpfile}]{\sphinxcrossref{GetHTTPFile()}}}

\item {} 
\sphinxAtStartPar
\phantomsection\label{\detokenize{reference/peblstream:id39}}{\hyperref[\detokenize{reference/peblstream:gethttptext}]{\sphinxcrossref{GetHTTPText()}}}

\item {} 
\sphinxAtStartPar
\phantomsection\label{\detokenize{reference/peblstream:id40}}{\hyperref[\detokenize{reference/peblstream:posthttp}]{\sphinxcrossref{PostHTTP()}}}

\item {} 
\sphinxAtStartPar
\phantomsection\label{\detokenize{reference/peblstream:id41}}{\hyperref[\detokenize{reference/peblstream:posthttpfile}]{\sphinxcrossref{PostHTTPFile()}}}

\end{itemize}
\end{sphinxShadowBox}

\index{AppendFile@\spxentry{AppendFile}}\ignorespaces 

\subsection{AppendFile()}
\label{\detokenize{reference/peblstream:appendfile}}\label{\detokenize{reference/peblstream:index-0}}
\sphinxAtStartPar
\sphinxstyleemphasis{Appends a file2 to file1}

\sphinxAtStartPar
\sphinxstylestrong{Description:}

\sphinxAtStartPar
Appends onto the end of \sphinxcode{\sphinxupquote{\textless{}file1\textgreater{}}} the contents of \sphinxcode{\sphinxupquote{\textless{}file2\textgreater{}}}.  Useful for compiling pooled data at the end of an experiment.

\sphinxAtStartPar
\sphinxstylestrong{Usage:}

\begin{sphinxVerbatim}[commandchars=\\\{\}]
\PYG{n+nf}{AppendFile}\PYG{p}{(}\PYG{o}{\PYGZlt{}}\PYG{n+nv}{file1}\PYG{o}{\PYGZgt{}}\PYG{p}{,}\PYG{+w}{ }\PYG{o}{\PYGZlt{}}\PYG{n+nv}{file2}\PYG{o}{\PYGZgt{}}\PYG{p}{)}
\end{sphinxVerbatim}

\sphinxAtStartPar
\sphinxstylestrong{See Also:}

\sphinxAtStartPar
\sphinxcode{\sphinxupquote{FileOpenWrite()}} , \sphinxcode{\sphinxupquote{FileOpenAppend()}}

\index{AcceptNetworkConnection@\spxentry{AcceptNetworkConnection}}\ignorespaces 

\subsection{AcceptNetworkConnection()}
\label{\detokenize{reference/peblstream:acceptnetworkconnection}}\label{\detokenize{reference/peblstream:index-1}}
\sphinxAtStartPar
\sphinxstyleemphasis{Accepts an incoming network connection on a listening port}

\sphinxAtStartPar
\sphinxstylestrong{Description:}

\sphinxAtStartPar
Accepts an incoming TCP/IP connection on a network listener that was opened using \sphinxcode{\sphinxupquote{OpenNetworkListener()}}. Returns a network connection object that can be used to send and receive data. This is typically used after \sphinxcode{\sphinxupquote{CheckForNetworkConnection()}} confirms a connection is available.

\sphinxAtStartPar
\sphinxstylestrong{Usage:}

\begin{sphinxVerbatim}[commandchars=\\\{\}]
\PYG{n+nf}{AcceptNetworkConnection}\PYG{p}{(}\PYG{o}{\PYGZlt{}}\PYG{n+nv}{listener}\PYG{o}{\PYGZgt{}}\PYG{p}{,}\PYG{+w}{ }\PYG{o}{\PYGZlt{}}\PYG{n+nv}{port}\PYG{o}{\PYGZgt{}}\PYG{p}{)}
\end{sphinxVerbatim}

\sphinxAtStartPar
\sphinxstylestrong{Example:}

\begin{sphinxVerbatim}[commandchars=\\\{\}]
\PYG{n+nv}{listener}\PYG{+w}{ }\PYG{o}{\PYGZlt{}\PYGZhy{}}\PYG{+w}{ }\PYG{n+nf}{OpenNetworkListener}\PYG{p}{(}\PYG{l+m+mi}{4444}\PYG{p}{)}
\PYG{k}{if}\PYG{p}{(}\PYG{n+nf}{CheckForNetworkConnection}\PYG{p}{(}\PYG{n+nv}{listener}\PYG{p}{)}\PYG{p}{)}
\PYG{p}{\PYGZob{}}
\PYG{+w}{   }\PYG{n+nv}{connection}\PYG{+w}{ }\PYG{o}{\PYGZlt{}\PYGZhy{}}\PYG{+w}{ }\PYG{n+nf}{AcceptNetworkConnection}\PYG{p}{(}\PYG{n+nv}{listener}\PYG{p}{,}\PYG{+w}{ }\PYG{l+m+mi}{4444}\PYG{p}{)}
\PYG{+w}{   }\PYG{n+nf}{SendData}\PYG{p}{(}\PYG{n+nv}{connection}\PYG{p}{,}\PYG{+w}{ }\PYG{l+s+s2}{\PYGZdq{}Hello client!\PYGZdq{}}\PYG{p}{)}
\PYG{+w}{   }\PYG{n+nf}{CloseNetworkConnection}\PYG{p}{(}\PYG{n+nv}{connection}\PYG{p}{)}
\PYG{p}{\PYGZcb{}}
\end{sphinxVerbatim}

\sphinxAtStartPar
\sphinxstylestrong{See Also:}

\sphinxAtStartPar
\sphinxcode{\sphinxupquote{OpenNetworkListener()}}, \sphinxcode{\sphinxupquote{CheckForNetworkConnection()}}, \sphinxcode{\sphinxupquote{WaitForNetworkConnection()}}, \sphinxcode{\sphinxupquote{CloseNetworkConnection()}}

\index{CheckForNetworkConnection@\spxentry{CheckForNetworkConnection}}\ignorespaces 

\subsection{CheckForNetworkConnection()}
\label{\detokenize{reference/peblstream:checkfornetworkconnection}}\label{\detokenize{reference/peblstream:index-2}}
\sphinxAtStartPar
\sphinxstylestrong{Description:}

\sphinxAtStartPar
Checks to see if there is an incoming TCP/IP connection on a network that is opened using \sphinxcode{\sphinxupquote{OpenNetworkListener}}.  This is an alternative to the \sphinxcode{\sphinxupquote{WaitForNetworkConnection}} function that allows more flexibility (and allows updating the during waiting for the connection).

\sphinxAtStartPar
\sphinxstylestrong{Usage:}

\begin{sphinxVerbatim}[commandchars=\\\{\}]
\PYG{n+nv}{net}\PYG{+w}{ }\PYG{o}{\PYGZlt{}\PYGZhy{}}\PYG{+w}{ }\PYG{n+nf}{CheckForNetwokConnection}\PYG{p}{(}\PYG{n+nv}{network}\PYG{p}{)}
\end{sphinxVerbatim}

\sphinxAtStartPar
\sphinxstylestrong{Example:}

\begin{sphinxVerbatim}[commandchars=\\\{\}]
\PYG{n+nv}{network}\PYG{+w}{ }\PYG{o}{\PYGZlt{}\PYGZhy{}}\PYG{+w}{      }\PYG{n+nf}{OpenNetworkListener}\PYG{p}{(}\PYG{l+m+mi}{4444}\PYG{p}{)}
\PYG{+w}{  }\PYG{n+nv}{time}\PYG{+w}{ }\PYG{o}{\PYGZlt{}\PYGZhy{}}\PYG{+w}{ }\PYG{n+nf}{GetTime}\PYG{p}{(}\PYG{p}{)}
\PYG{+w}{  }\PYG{k}{while}\PYG{p}{(}\PYG{k}{not}\PYG{+w}{ }\PYG{n+nv}{connected}\PYG{+w}{ }\PYG{k}{and}\PYG{+w}{ }\PYG{p}{(}\PYG{n+nf}{GetTime}\PYG{p}{(}\PYG{p}{)}\PYG{+w}{ }\PYG{o}{\PYGZlt{}}\PYG{+w}{ }\PYG{n+nv}{time}\PYG{+w}{ }\PYG{o}{+}\PYG{+w}{ }\PYG{l+m+mi}{5000}\PYG{p}{)}\PYG{p}{)}
\PYG{+w}{   }\PYG{p}{\PYGZob{}}
\PYG{+w}{      }\PYG{n+nv}{connected}\PYG{+w}{ }\PYG{o}{\PYGZlt{}\PYGZhy{}}\PYG{+w}{ }\PYG{n+nf}{CheckForNetwokConnection}\PYG{p}{(}\PYG{n+nv}{network}\PYG{p}{)}
\PYG{+w}{   }\PYG{p}{\PYGZcb{}}
\end{sphinxVerbatim}

\sphinxAtStartPar
\sphinxstylestrong{See Also:}

\sphinxAtStartPar
\sphinxcode{\sphinxupquote{OpenNetworkListener()}}, \sphinxcode{\sphinxupquote{Getdata()}}, \sphinxcode{\sphinxupquote{WaitForNetworkConnection()}}, \sphinxcode{\sphinxupquote{CloseNetwork()}}

\index{CloseNetworkConnection@\spxentry{CloseNetworkConnection}}\ignorespaces 

\subsection{CloseNetworkConnection()}
\label{\detokenize{reference/peblstream:closenetworkconnection}}\label{\detokenize{reference/peblstream:index-3}}
\sphinxAtStartPar
\sphinxstylestrong{Description:}

\sphinxAtStartPar
Closes network connection

\sphinxAtStartPar
\sphinxstylestrong{Usage:}

\begin{sphinxVerbatim}[commandchars=\\\{\}]
\PYG{n+nf}{CloseNetwork}\PYG{p}{(}\PYG{o}{\PYGZlt{}}\PYG{n+nv}{network}\PYG{o}{\PYGZgt{}}\PYG{p}{)}
\end{sphinxVerbatim}

\sphinxAtStartPar
\sphinxstylestrong{Example:}

\begin{sphinxVerbatim}[commandchars=\\\{\}]
\PYG{n+nv}{net}\PYG{+w}{ }\PYG{o}{\PYGZlt{}\PYGZhy{}}\PYG{+w}{ }\PYG{n+nf}{WaitForNetworkConnection}\PYG{p}{(}\PYG{l+s+s2}{\PYGZdq{}localhost\PYGZdq{}}\PYG{p}{,}\PYG{l+m+mi}{1234}\PYG{p}{)}
\PYG{n+nf}{SendData}\PYG{p}{(}\PYG{n+nv}{net}\PYG{p}{,}\PYG{l+s+s2}{\PYGZdq{}Watson, come here. I need you.\PYGZdq{}}\PYG{p}{)}
\PYG{n+nf}{CloseNetworkConnection}\PYG{p}{(}\PYG{n+nv}{net}\PYG{p}{)}
\end{sphinxVerbatim}

\sphinxAtStartPar
\sphinxstylestrong{See Also:}

\sphinxAtStartPar
\sphinxcode{\sphinxupquote{ConnectToIP()}}, \sphinxcode{\sphinxupquote{ConnectToHost()}},  \sphinxcode{\sphinxupquote{WaitForNetworkConnection()}}, \sphinxcode{\sphinxupquote{GetData()}},  \sphinxcode{\sphinxupquote{SendData()}}, \sphinxcode{\sphinxupquote{ConvertIPString()}}

\index{ConnectToHost@\spxentry{ConnectToHost}}\ignorespaces 

\subsection{ConnectToHost()}
\label{\detokenize{reference/peblstream:connecttohost}}\label{\detokenize{reference/peblstream:index-4}}
\sphinxAtStartPar
\sphinxstyleemphasis{Connects to a port on another computer, returning network object.}

\sphinxAtStartPar
\sphinxstylestrong{Description:}

\sphinxAtStartPar
Connects to a host computer waiting for a   connection on \textless{}port\textgreater{}, returning a network object that can be used to   communicate.  Host is a text hostname, like \sphinxcode{\sphinxupquote{"myname.indiana.edu"}}, or   use \sphinxcode{\sphinxupquote{"localhost"}} to specify your current computer.

\sphinxAtStartPar
\sphinxstylestrong{Usage:}

\begin{sphinxVerbatim}[commandchars=\\\{\}]
\PYG{n+nf}{ConnectToHost}\PYG{p}{(}\PYG{o}{\PYGZlt{}}\PYG{n+nv}{hostname}\PYG{o}{\PYGZgt{}}\PYG{p}{,}\PYG{o}{\PYGZlt{}}\PYG{n+nv}{port}\PYG{o}{\PYGZgt{}}\PYG{p}{)}
\end{sphinxVerbatim}

\sphinxAtStartPar
\sphinxstylestrong{See Also:}

\sphinxAtStartPar
\sphinxcode{\sphinxupquote{ConnectToIP()}}, \sphinxcode{\sphinxupquote{GetData()}},  \sphinxcode{\sphinxupquote{WaitForNetworkConnection()}}, \sphinxcode{\sphinxupquote{SendData()}}, \sphinxcode{\sphinxupquote{ConvertIPString()}}, \sphinxcode{\sphinxupquote{CloseNetworkConnection()}}

\index{ConnectToIP@\spxentry{ConnectToIP}}\ignorespaces 

\subsection{ConnectToIP()}
\label{\detokenize{reference/peblstream:connecttoip}}\label{\detokenize{reference/peblstream:index-5}}
\sphinxAtStartPar
\sphinxstyleemphasis{Connects to a port on another computer, returning network object.}

\sphinxAtStartPar
\sphinxstylestrong{Description:}

\sphinxAtStartPar
Connects to a host computer waiting for a   connection on \sphinxcode{\sphinxupquote{\textless{}port\textgreater{}}}, returning a network object that can be used to   communicate.  \sphinxcode{\sphinxupquote{\textless{}ip\textgreater{}}} is a numeric ip address, which must be   created with the \sphinxcode{\sphinxupquote{ConvertIPString(ip)}} function.

\sphinxAtStartPar
\sphinxstylestrong{Usage:}

\begin{sphinxVerbatim}[commandchars=\\\{\}]
\PYG{n+nf}{ConnectToIP}\PYG{p}{(}\PYG{o}{\PYGZlt{}}\PYG{n+nv}{ip}\PYG{o}{\PYGZgt{}}\PYG{p}{,}\PYG{o}{\PYGZlt{}}\PYG{n+nv}{port}\PYG{o}{\PYGZgt{}}\PYG{p}{)}
\end{sphinxVerbatim}

\sphinxAtStartPar
\sphinxstylestrong{See Also:}

\sphinxAtStartPar
\sphinxcode{\sphinxupquote{ConnectToHost()}}, \sphinxcode{\sphinxupquote{GetData()}}, \sphinxcode{\sphinxupquote{WaitForNetworkConnection()}}, \sphinxcode{\sphinxupquote{SendData()}}, \sphinxcode{\sphinxupquote{ConvertIPString()}}, \sphinxcode{\sphinxupquote{CloseNetworkConnection()}}

\index{CopyFile@\spxentry{CopyFile}}\ignorespaces 

\subsection{CopyFile()}
\label{\detokenize{reference/peblstream:copyfile}}\label{\detokenize{reference/peblstream:index-6}}
\sphinxAtStartPar
\sphinxstyleemphasis{Makes a copy of a file}

\sphinxAtStartPar
\sphinxstylestrong{Description:}

\sphinxAtStartPar
This makes a copy of a specified file, by  Copying the contents of one file to another.         This makes the copy byte\sphinxhyphen{}by\sphinxhyphen{}byte (so should work for binary data).  It is probably better to use a systemcall function to make a copy of an entire file at once. This is likely to be slower and possibly error\sphinxhyphen{}prone (i.e., permissions and other file properties may not copy.), but it is a useful cross\sphinxhyphen{}platform solution to creating a new file based on others.  It copies by name from the current working directory.

\sphinxAtStartPar
\sphinxstylestrong{Example:}

\begin{sphinxVerbatim}[commandchars=\\\{\}]
\PYG{n+nv}{base}\PYG{+w}{ }\PYG{o}{\PYGZlt{}\PYGZhy{}}\PYG{+w}{ }\PYG{l+s+s2}{\PYGZdq{}template.txt\PYGZdq{}}
\PYG{n+nf}{CopyFile}\PYG{p}{(}\PYG{n+nv}{base}\PYG{p}{,}\PYG{l+s+s2}{\PYGZdq{}newfile.txt\PYGZdq{}}\PYG{p}{)}
\end{sphinxVerbatim}

\sphinxAtStartPar
\sphinxstylestrong{See Also:}

\sphinxAtStartPar
\sphinxcode{\sphinxupquote{DeleteFile()}}, \sphinxcode{\sphinxupquote{AppendFile()}} , \sphinxcode{\sphinxupquote{FileExists()}}

\index{EndOfFile@\spxentry{EndOfFile}}\ignorespaces 

\subsection{EndOfFile()}
\label{\detokenize{reference/peblstream:endoffile}}\label{\detokenize{reference/peblstream:index-7}}
\sphinxAtStartPar
\sphinxstyleemphasis{Returns true if at the end of a file}

\sphinxAtStartPar
\sphinxstylestrong{Description:}

\sphinxAtStartPar
Returns true if at the end of a file.

\sphinxAtStartPar
\sphinxstylestrong{Usage:}

\begin{sphinxVerbatim}[commandchars=\\\{\}]
\PYG{n+nf}{EndOfFile}\PYG{p}{(}\PYG{o}{\PYGZlt{}}\PYG{n+nv}{filestream}\PYG{o}{\PYGZgt{}}\PYG{p}{)}
\end{sphinxVerbatim}

\sphinxAtStartPar
\sphinxstylestrong{Example:}

\begin{sphinxVerbatim}[commandchars=\\\{\}]
\PYG{k}{while}\PYG{p}{(}\PYG{k}{not}\PYG{+w}{ }\PYG{n+nf}{EndOfFile}\PYG{p}{(}\PYG{n+nv}{fstream}\PYG{p}{)}\PYG{p}{)}
\PYG{p}{\PYGZob{}}
\PYG{+w}{ }\PYG{n+nf}{Print}\PYG{p}{(}\PYG{n+nf}{FileReadLine}\PYG{p}{(}\PYG{n+nv}{fstream}\PYG{p}{)}\PYG{p}{)}
\PYG{p}{\PYGZcb{}}
\end{sphinxVerbatim}

\index{EndOfLine@\spxentry{EndOfLine}}\ignorespaces 

\subsection{EndOfLine()}
\label{\detokenize{reference/peblstream:endofline}}\label{\detokenize{reference/peblstream:index-8}}
\sphinxAtStartPar
\sphinxstyleemphasis{Returns true if at end of line}

\sphinxAtStartPar
\sphinxstylestrong{Description:}

\sphinxAtStartPar
Returns true if at end of line.

\sphinxAtStartPar
\sphinxstylestrong{Usage:}

\begin{sphinxVerbatim}[commandchars=\\\{\}]
\PYG{n+nf}{EndOfLine}\PYG{p}{(}\PYG{o}{\PYGZlt{}}\PYG{n+nv}{filestream}\PYG{o}{\PYGZgt{}}\PYG{p}{)}
\end{sphinxVerbatim}

\index{FileClose@\spxentry{FileClose}}\ignorespaces 

\subsection{FileClose()}
\label{\detokenize{reference/peblstream:fileclose}}\label{\detokenize{reference/peblstream:index-9}}
\sphinxAtStartPar
\sphinxstyleemphasis{Closes a filestream variable. Pass the variable name, not the filename}

\sphinxAtStartPar
\sphinxstylestrong{Description:}

\sphinxAtStartPar
Closes a filestream  variable.  Be sure to              pass the variable name, not the filename.

\sphinxAtStartPar
\sphinxstylestrong{Usage:}

\begin{sphinxVerbatim}[commandchars=\\\{\}]
\PYG{n+nf}{FileClose}\PYG{p}{(}\PYG{o}{\PYGZlt{}}\PYG{n+nv}{filestream}\PYG{o}{\PYGZgt{}}\PYG{p}{)}
\end{sphinxVerbatim}

\sphinxAtStartPar
\sphinxstylestrong{Example:}

\begin{sphinxVerbatim}[commandchars=\\\{\}]
\PYG{n+nv}{x}\PYG{+w}{ }\PYG{o}{\PYGZlt{}\PYGZhy{}}\PYG{+w}{ }\PYG{n+nf}{FileOpenRead}\PYG{p}{(}\PYG{l+s+s2}{\PYGZdq{}file.txt\PYGZdq{}}\PYG{p}{)}
\PYG{c+c1}{\PYGZsh{} Do relevant stuff here.}
\PYG{n+nf}{FileClose}\PYG{p}{(}\PYG{n+nv}{x}\PYG{p}{)}
\end{sphinxVerbatim}

\sphinxAtStartPar
\sphinxstylestrong{See Also:}

\sphinxAtStartPar
\sphinxcode{\sphinxupquote{FileOpenAppend()}}, \sphinxcode{\sphinxupquote{FileOpenRead()}}, \sphinxcode{\sphinxupquote{FileOpenWrite()}}

\index{FileOpenAppend@\spxentry{FileOpenAppend}}\ignorespaces 

\subsection{FileOpenAppend()}
\label{\detokenize{reference/peblstream:fileopenappend}}\label{\detokenize{reference/peblstream:index-10}}
\sphinxAtStartPar
\sphinxstyleemphasis{Opens a filename, returning a stream that can be used for writing info. Appends if the file already exists, opens if file does not}

\sphinxAtStartPar
\sphinxstylestrong{Description:}

\sphinxAtStartPar
Opens a filename, returning a stream that can be   used for writing information.  Appends if the file already exists.

\sphinxAtStartPar
\sphinxstylestrong{Usage:}

\begin{sphinxVerbatim}[commandchars=\\\{\}]
\PYG{n+nf}{FileOpenAppend}\PYG{p}{(}\PYG{o}{\PYGZlt{}}\PYG{n+nv}{filename}\PYG{o}{\PYGZgt{}}\PYG{p}{)}
\end{sphinxVerbatim}

\sphinxAtStartPar
\sphinxstylestrong{See Also:}

\sphinxAtStartPar
\sphinxcode{\sphinxupquote{FileClose()}}, \sphinxcode{\sphinxupquote{FileOpenRead()}}, \sphinxcode{\sphinxupquote{FileOpenWrite()}},  \sphinxcode{\sphinxupquote{FileOpenOverWrite()}}

\index{FileOpenOverwrite@\spxentry{FileOpenOverwrite}}\ignorespaces 

\subsection{FileOpenOverwrite()}
\label{\detokenize{reference/peblstream:fileopenoverwrite}}\label{\detokenize{reference/peblstream:index-11}}
\sphinxAtStartPar
\sphinxstyleemphasis{Opens a filename, returning a stream that can be used for writing information. Overwrites if file already exists}

\sphinxAtStartPar
\sphinxstylestrong{Description:}

\sphinxAtStartPar
Opens a filename, returning a stream that can be   used for writing information.  Overwrites if file already exists.   This function should not be used for opening data files; instead,   use FileOpenWrite, which saves to a backup file if the specified   file already exists.

\sphinxAtStartPar
\sphinxstylestrong{Usage:}

\begin{sphinxVerbatim}[commandchars=\\\{\}]
\PYG{n+nf}{FileOpenOverWrite}\PYG{p}{(}\PYG{o}{\PYGZlt{}}\PYG{n+nv}{filename}\PYG{o}{\PYGZgt{}}\PYG{p}{)}
\end{sphinxVerbatim}

\sphinxAtStartPar
\sphinxstylestrong{See Also:}

\sphinxAtStartPar
\sphinxcode{\sphinxupquote{FileClose()}}, \sphinxcode{\sphinxupquote{FileOpenAppend()}}, \sphinxcode{\sphinxupquote{FileOpenRead()}}  \sphinxcode{\sphinxupquote{FileOpenWrite()}}

\index{FileOpenRead@\spxentry{FileOpenRead}}\ignorespaces 

\subsection{FileOpenRead()}
\label{\detokenize{reference/peblstream:fileopenread}}\label{\detokenize{reference/peblstream:index-12}}
\sphinxAtStartPar
\sphinxstyleemphasis{Opens a filename, returning a stream to be used for reading information}

\sphinxAtStartPar
\sphinxstylestrong{Description:}

\sphinxAtStartPar
Opens a filename, returning  a stream to be used                for reading information.

\sphinxAtStartPar
\sphinxstylestrong{Usage:}

\begin{sphinxVerbatim}[commandchars=\\\{\}]
\PYG{n+nf}{FileOpenRead}\PYG{p}{(}\PYG{o}{\PYGZlt{}}\PYG{n+nv}{filename}\PYG{o}{\PYGZgt{}}\PYG{p}{)}
\end{sphinxVerbatim}

\sphinxAtStartPar
\sphinxstylestrong{See Also:}

\sphinxAtStartPar
\sphinxcode{\sphinxupquote{FileClose()}}, \sphinxcode{\sphinxupquote{FileOpenAppend()}}, \sphinxcode{\sphinxupquote{FileOpenWrite()}}, \sphinxcode{\sphinxupquote{FileOpenOverWrite()}}

\index{FileOpenWrite@\spxentry{FileOpenWrite}}\ignorespaces 

\subsection{FileOpenWrite()}
\label{\detokenize{reference/peblstream:fileopenwrite}}\label{\detokenize{reference/peblstream:index-13}}
\sphinxAtStartPar
\sphinxstyleemphasis{Opens a filename, returning a stream that can be used for writing information. Creates new file if file already exists}

\sphinxAtStartPar
\sphinxstylestrong{Description:}

\sphinxAtStartPar
Opens a filename, returning a stream that can be   used for writing information.  If the specified filename exists, it   won’t overwrite that file.  Instead, it will create a related   filename, appending a \sphinxhyphen{}integer before the filename extension.

\sphinxAtStartPar
\sphinxstylestrong{Usage:}

\begin{sphinxVerbatim}[commandchars=\\\{\}]
\PYG{n+nf}{FileOpenWrite}\PYG{p}{(}\PYG{o}{\PYGZlt{}}\PYG{n+nv}{filename}\PYG{o}{\PYGZgt{}}\PYG{p}{)}
\end{sphinxVerbatim}

\sphinxAtStartPar
\sphinxstylestrong{See Also:}

\sphinxAtStartPar
\sphinxcode{\sphinxupquote{FileClose()}}, \sphinxcode{\sphinxupquote{FileOpenAppend()}}, \sphinxcode{\sphinxupquote{FileOpenRead()}}, \sphinxcode{\sphinxupquote{FileOpenOverWrite()}}

\index{FilePrint@\spxentry{FilePrint}}\ignorespaces 

\subsection{FilePrint()}
\label{\detokenize{reference/peblstream:fileprint}}\label{\detokenize{reference/peblstream:index-14}}
\sphinxAtStartPar
\sphinxstylestrong{Description:}

\sphinxAtStartPar
Like \sphinxcode{\sphinxupquote{Print\_}}, but to a file.  Prints a   string to a file,   without appending a newline character.  Returns a   copy of the string it prints.

\sphinxAtStartPar
\sphinxstylestrong{Usage:}

\begin{sphinxVerbatim}[commandchars=\\\{\}]
\PYG{n+nf}{FilePrint\PYGZus{}}\PYG{p}{(}\PYG{o}{\PYGZlt{}}\PYG{n+nv}{filestream}\PYG{o}{\PYGZgt{}}\PYG{p}{,}\PYG{+w}{ }\PYG{o}{\PYGZlt{}}\PYG{n+nv}{value}\PYG{o}{\PYGZgt{}}\PYG{p}{)}
\end{sphinxVerbatim}

\sphinxAtStartPar
\sphinxstylestrong{Example:}

\begin{sphinxVerbatim}[commandchars=\\\{\}]
\PYG{n+nf}{FilePrint\PYGZus{}}\PYG{p}{(}\PYG{n+nv}{fstream}\PYG{p}{,}\PYG{+w}{ }\PYG{l+s+s2}{\PYGZdq{}This line doesn\PYGZsq{}t end.\PYGZdq{}}\PYG{p}{)}
\end{sphinxVerbatim}

\sphinxAtStartPar
\sphinxstylestrong{See Also:}

\sphinxAtStartPar
\sphinxcode{\sphinxupquote{Print\_()}}, \sphinxcode{\sphinxupquote{FilePrint()}}

\index{FileReadCharacter@\spxentry{FileReadCharacter}}\ignorespaces 

\subsection{FileReadCharacter()}
\label{\detokenize{reference/peblstream:filereadcharacter}}\label{\detokenize{reference/peblstream:index-15}}
\sphinxAtStartPar
\sphinxstyleemphasis{Reads and returns a single character from a filestream}

\sphinxAtStartPar
\sphinxstylestrong{Description:}

\sphinxAtStartPar
Reads and returns a single character from a filestream.

\sphinxAtStartPar
\sphinxstylestrong{Usage:}

\begin{sphinxVerbatim}[commandchars=\\\{\}]
\PYG{n+nf}{FileReadCharacter}\PYG{p}{(}\PYG{o}{\PYGZlt{}}\PYG{n+nv}{filestream}\PYG{o}{\PYGZgt{}}\PYG{p}{)}
\end{sphinxVerbatim}

\sphinxAtStartPar
\sphinxstylestrong{See Also:}

\sphinxAtStartPar
\sphinxcode{\sphinxupquote{FileReadList()}}, \sphinxcode{\sphinxupquote{FileReadTable()}}    \sphinxcode{\sphinxupquote{FileReadLine()}},      \sphinxcode{\sphinxupquote{FileReadText()}},         \sphinxcode{\sphinxupquote{FileReadWord()}},

\index{FileReadLine@\spxentry{FileReadLine}}\ignorespaces 

\subsection{FileReadLine()}
\label{\detokenize{reference/peblstream:filereadline}}\label{\detokenize{reference/peblstream:index-16}}
\sphinxAtStartPar
\sphinxstyleemphasis{Reads and returns a line from a file; all characters up until the next newline or the end of the file}

\sphinxAtStartPar
\sphinxstylestrong{Description:}

\sphinxAtStartPar
Reads and returns a line from a file; all characters up                 until the next newline or the end of the file.

\sphinxAtStartPar
\sphinxstylestrong{Usage:}

\begin{sphinxVerbatim}[commandchars=\\\{\}]
\PYG{n+nf}{FileReadLine}\PYG{p}{(}\PYG{o}{\PYGZlt{}}\PYG{n+nv}{filestream}\PYG{o}{\PYGZgt{}}\PYG{p}{)}
\end{sphinxVerbatim}

\sphinxAtStartPar
\sphinxstylestrong{See Also:}

\sphinxAtStartPar
\sphinxcode{\sphinxupquote{FileReadCharacter()}}, \sphinxcode{\sphinxupquote{FileReadList()}}, \sphinxcode{\sphinxupquote{FileReadTable()}}, \sphinxcode{\sphinxupquote{FileReadText()}}, \sphinxcode{\sphinxupquote{FileReadWord()}}

\index{FileReadList@\spxentry{FileReadList}}\ignorespaces 

\subsection{FileReadList()}
\label{\detokenize{reference/peblstream:filereadlist}}\label{\detokenize{reference/peblstream:index-17}}
\sphinxAtStartPar
\sphinxstyleemphasis{Given a filename, will open it, read in all the items into a list (one item per line), and close the file afterwards}

\sphinxAtStartPar
\sphinxstylestrong{Description:}

\sphinxAtStartPar
Given a filename, will open it, read in all the   items into a list (one item per line), and close the file afterward.   Ignores blank lines or lines starting with \sphinxcode{\sphinxupquote{\#}}.  Useful with a   number of pre\sphinxhyphen{}defined data files stored in \sphinxcode{\sphinxupquote{media/text/}}.  See   Section\textasciitilde{}ref\{sec:media\}: Provided Media Files.

\sphinxAtStartPar
\sphinxstylestrong{Usage:}

\begin{sphinxVerbatim}[commandchars=\\\{\}]
\PYG{n+nf}{FileReadList}\PYG{p}{(}\PYG{o}{\PYGZlt{}}\PYG{n+nv}{filename}\PYG{o}{\PYGZgt{}}\PYG{p}{)}
\end{sphinxVerbatim}

\sphinxAtStartPar
\sphinxstylestrong{Example:}

\begin{sphinxVerbatim}[commandchars=\\\{\}]
\PYG{n+nf}{FileReadList}\PYG{p}{(}\PYG{l+s+s2}{\PYGZdq{}data.txt\PYGZdq{}}\PYG{p}{)}
\end{sphinxVerbatim}

\sphinxAtStartPar
\sphinxstylestrong{See Also:}

\sphinxAtStartPar
\sphinxcode{\sphinxupquote{FileReadCharacter()}}, \sphinxcode{\sphinxupquote{FileReadTable()}}    \sphinxcode{\sphinxupquote{FileReadLine()}},         \sphinxcode{\sphinxupquote{FileReadText()}},         \sphinxcode{\sphinxupquote{FileReadWord()}},

\index{FileReadTable@\spxentry{FileReadTable}}\ignorespaces 

\subsection{FileReadTable()}
\label{\detokenize{reference/peblstream:filereadtable}}\label{\detokenize{reference/peblstream:index-18}}
\sphinxAtStartPar
\sphinxstylestrong{Description:}

\sphinxAtStartPar
Reads a table directly from a file. Data in file should                 separated by spaces.  Reads each line onto a sublist,           with space\sphinxhyphen{}separated tokens as items in sublist.  Ignores               blank lines or lines beginning with \sphinxcode{\sphinxupquote{\#}}. Optionally,          specify a token separator other than space.

\sphinxAtStartPar
\sphinxstylestrong{Usage:}

\begin{sphinxVerbatim}[commandchars=\\\{\}]
\PYG{n+nf}{FileReadTable}\PYG{p}{(}\PYG{o}{\PYGZlt{}}\PYG{n+nv}{filename}\PYG{o}{\PYGZgt{}}\PYG{p}{,}\PYG{+w}{ }\PYG{o}{\PYGZlt{}}\PYG{n+nv}{optional}\PYG{o}{\PYGZhy{}}\PYG{n+nv}{separator}\PYG{o}{\PYGZgt{}}\PYG{p}{)}
\end{sphinxVerbatim}

\sphinxAtStartPar
\sphinxstylestrong{Example:}

\begin{sphinxVerbatim}[commandchars=\\\{\}]
\PYG{n+nv}{a}\PYG{+w}{ }\PYG{o}{\PYGZlt{}\PYGZhy{}}\PYG{+w}{ }\PYG{n+nf}{FileReadTable}\PYG{p}{(}\PYG{l+s+s2}{\PYGZdq{}data.txt\PYGZdq{}}\PYG{p}{)}
\end{sphinxVerbatim}

\sphinxAtStartPar
\sphinxstylestrong{See Also:}

\sphinxAtStartPar
\sphinxcode{\sphinxupquote{FileReadCharacter()}}, \sphinxcode{\sphinxupquote{FileReadList()}}, \sphinxcode{\sphinxupquote{FileReadLine()}}, \sphinxcode{\sphinxupquote{FileReadText()}}, \sphinxcode{\sphinxupquote{FileReadWord()}}

\index{FileReadText@\spxentry{FileReadText}}\ignorespaces 

\subsection{FileReadText()}
\label{\detokenize{reference/peblstream:filereadtext}}\label{\detokenize{reference/peblstream:index-19}}
\sphinxAtStartPar
\sphinxstyleemphasis{Reads all of the text in the file into a variable}

\sphinxAtStartPar
\sphinxstylestrong{Description:}

\sphinxAtStartPar
Returns all of the text from a file, ignoring any lines                 beginning with \sphinxcode{\sphinxupquote{\#}}. Opens and closes the file transparently.

\sphinxAtStartPar
\sphinxstylestrong{Usage:}

\begin{sphinxVerbatim}[commandchars=\\\{\}]
\PYG{n+nf}{FileReadText}\PYG{p}{(}\PYG{o}{\PYGZlt{}}\PYG{n+nv}{filename}\PYG{o}{\PYGZgt{}}\PYG{p}{)}
\end{sphinxVerbatim}

\sphinxAtStartPar
\sphinxstylestrong{Example:}

\begin{sphinxVerbatim}[commandchars=\\\{\}]
\PYG{n+nv}{instructions}\PYG{+w}{ }\PYG{o}{\PYGZlt{}\PYGZhy{}}\PYG{+w}{ }\PYG{n+nf}{FileReadText}\PYG{p}{(}\PYG{l+s+s2}{\PYGZdq{}instructions.txt\PYGZdq{}}\PYG{p}{)}
\end{sphinxVerbatim}

\sphinxAtStartPar
\sphinxstylestrong{See Also:}

\sphinxAtStartPar
\sphinxcode{\sphinxupquote{FileReadCharacter()}}, \sphinxcode{\sphinxupquote{FileReadList()}}, \sphinxcode{\sphinxupquote{FileReadTable()}}, \sphinxcode{\sphinxupquote{FileReadLine()}}, \sphinxcode{\sphinxupquote{FileReadWord()}}

\index{FileReadWord@\spxentry{FileReadWord}}\ignorespaces 

\subsection{FileReadWord()}
\label{\detokenize{reference/peblstream:filereadword}}\label{\detokenize{reference/peblstream:index-20}}
\sphinxAtStartPar
\sphinxstylestrong{Description:}

\sphinxAtStartPar
Reads and returns  a \sphinxtitleref{word’ from a file; the next               connected stream of characters not including a \textasciigrave{}}’ ‘\textasciigrave{}\textasciigrave{}          or a newline. Will not read newline characters.

\sphinxAtStartPar
\sphinxstylestrong{Usage:}

\begin{sphinxVerbatim}[commandchars=\\\{\}]
\PYG{n+nf}{FileReadWord}\PYG{p}{(}\PYG{o}{\PYGZlt{}}\PYG{n+nv}{filestream}\PYG{o}{\PYGZgt{}}\PYG{p}{)}
\end{sphinxVerbatim}

\sphinxAtStartPar
\sphinxstylestrong{See Also:}

\sphinxAtStartPar
\sphinxcode{\sphinxupquote{FileReadCharacter()}}, \sphinxcode{\sphinxupquote{FileReadList()}}, \sphinxcode{\sphinxupquote{FileReadTable()}}, \sphinxcode{\sphinxupquote{FileReadLine()}}, \sphinxcode{\sphinxupquote{FileReadText()}}

\index{GetData@\spxentry{GetData}}\ignorespaces 

\subsection{GetData()}
\label{\detokenize{reference/peblstream:getdata}}\label{\detokenize{reference/peblstream:index-21}}
\sphinxAtStartPar
\sphinxstyleemphasis{return a string from network connection}

\sphinxAtStartPar
\sphinxstylestrong{Description:}

\sphinxAtStartPar
Gets Data from network connection.  Example of   usage in demo/nim.pbl.

\sphinxAtStartPar
\sphinxstylestrong{Usage:}

\begin{sphinxVerbatim}[commandchars=\\\{\}]
\PYG{n+nv}{val}\PYG{+w}{ }\PYG{o}{\PYGZlt{}\PYGZhy{}}\PYG{+w}{ }\PYG{n+nf}{GetData}\PYG{p}{(}\PYG{o}{\PYGZlt{}}\PYG{n+nv}{network}\PYG{o}{\PYGZgt{}}\PYG{p}{,}\PYG{o}{\PYGZlt{}}\PYG{n+nv}{size}\PYG{o}{\PYGZgt{}}\PYG{p}{)}
\end{sphinxVerbatim}

\sphinxAtStartPar
\sphinxstylestrong{See Also:}

\sphinxAtStartPar
\sphinxcode{\sphinxupquote{ConnectToIP()}}, \sphinxcode{\sphinxupquote{ConnectToHost()}}, \sphinxcode{\sphinxupquote{WaitForNetworkConnection()}},    \sphinxcode{\sphinxupquote{SendData()}}, \sphinxcode{\sphinxupquote{ConvertIPString()}}, \sphinxcode{\sphinxupquote{CloseNetworkConnection()}}

\index{GetMyIPAddress@\spxentry{GetMyIPAddress}}\ignorespaces 

\subsection{GetMyIPAddress()}
\label{\detokenize{reference/peblstream:getmyipaddress}}\label{\detokenize{reference/peblstream:index-22}}
\sphinxAtStartPar
\sphinxstyleemphasis{Returns the local IP address of this computer}

\sphinxAtStartPar
\sphinxstylestrong{Description:}

\sphinxAtStartPar
Returns the IP address of the current computer as a formatted string. The returned address is suitable for use in networking functions. If multiple network interfaces are present, it typically returns the primary interface address.

\sphinxAtStartPar
\sphinxstylestrong{Usage:}

\begin{sphinxVerbatim}[commandchars=\\\{\}]
\PYG{n+nf}{GetMyIPAddress}\PYG{p}{(}\PYG{o}{\PYGZlt{}}\PYG{n+nv}{interface\PYGZus{}number}\PYG{o}{\PYGZgt{}}\PYG{p}{)}
\end{sphinxVerbatim}

\sphinxAtStartPar
\sphinxstylestrong{Example:}

\begin{sphinxVerbatim}[commandchars=\\\{\}]
\PYG{n+nv}{myIP}\PYG{+w}{ }\PYG{o}{\PYGZlt{}\PYGZhy{}}\PYG{+w}{ }\PYG{n+nf}{GetMyIPAddress}\PYG{p}{(}\PYG{l+m+mi}{0}\PYG{p}{)}
\PYG{n+nf}{Print}\PYG{p}{(}\PYG{l+s+s2}{\PYGZdq{}My IP address is: \PYGZdq{}}\PYG{+w}{ }\PYG{o}{+}\PYG{+w}{ }\PYG{n+nv}{myIP}\PYG{p}{)}
\end{sphinxVerbatim}

\sphinxAtStartPar
\sphinxstylestrong{See Also:}

\sphinxAtStartPar
\sphinxcode{\sphinxupquote{ConnectToHost()}}, \sphinxcode{\sphinxupquote{ConnectToIP()}}, \sphinxcode{\sphinxupquote{OpenNetworkListener()}}

\index{GetPPortState@\spxentry{GetPPortState}}\ignorespaces 

\subsection{GetPPortState()}
\label{\detokenize{reference/peblstream:getpportstate}}\label{\detokenize{reference/peblstream:index-23}}
\sphinxAtStartPar
\sphinxstyleemphasis{Gets state of parallel port data bits}

\sphinxAtStartPar
\sphinxstylestrong{Description:}

\sphinxAtStartPar
Gets the parallel port state, as a list of 8 ‘bits’ (1s or 0s).

\sphinxAtStartPar
\sphinxstylestrong{See Also:}

\sphinxAtStartPar
\sphinxcode{\sphinxupquote{COMPortGetByte()}}, \sphinxcode{\sphinxupquote{COMPortSendByte()}}, \sphinxcode{\sphinxupquote{OpenPPort()}} \sphinxcode{\sphinxupquote{OpenCOMPort()}}, \sphinxcode{\sphinxupquote{SetPPortMode()}}, \sphinxcode{\sphinxupquote{GetPPortState()}}

\index{OpenCOMPort@\spxentry{OpenCOMPort}}\ignorespaces 

\subsection{OpenCOMPort()}
\label{\detokenize{reference/peblstream:opencomport}}\label{\detokenize{reference/peblstream:index-24}}
\sphinxAtStartPar
\sphinxstyleemphasis{Opens a serial (com) port}

\sphinxAtStartPar
\sphinxstylestrong{Description:}

\sphinxAtStartPar
This opens a COM/Serial port, and is used by many usb devices for communication.  The general process is to use OpenComPort to create the port, and then send and receive text strings from that port.  These are sent one byte at a time. The mode argument is a 3\sphinxhyphen{}character string that specifies aspects of the mode (see  Teunis van Beelen’s rs232 library at \sphinxurl{http://www.teuniz.net/RS-232/}. The first character is the data bits (5,6,7 or 8), parity (N=none, E=even, O=odd), and the third is the stop bit (1 or 2 bits).   Within the demodirectory, there is some basic code for communicating with the cedrus response box that uses these functions.  In addition, that script provide a NumToASCII() function that can be useful in translating numbers to strings to communicate with a device.

\sphinxAtStartPar
\sphinxstylestrong{Example:}

\begin{sphinxVerbatim}[commandchars=\\\{\}]
\PYG{n+nv}{port}\PYG{+w}{ }\PYG{o}{\PYGZlt{}\PYGZhy{}}\PYG{+w}{ }\PYG{n+nf}{OpenCOMPort}\PYG{p}{(}\PYG{l+m+mi}{16}\PYG{p}{,}\PYG{l+m+mi}{9600}\PYG{p}{,}\PYG{l+s+s2}{\PYGZdq{}8N1\PYGZdq{}}\PYG{p}{)}
\PYG{+w}{  }\PYG{n+nf}{Print}\PYG{p}{(}\PYG{+w}{ }\PYG{n+nf}{ComPortGetByte}\PYG{p}{(}\PYG{n+nv}{port}\PYG{p}{)}\PYG{p}{)}
\end{sphinxVerbatim}

\sphinxAtStartPar
\sphinxstylestrong{See Also:}

\sphinxAtStartPar
\sphinxcode{\sphinxupquote{COMPortGetByte()}}, \sphinxcode{\sphinxupquote{COMPortSendByte()}}, \sphinxcode{\sphinxupquote{OpenPPort()}}, \sphinxcode{\sphinxupquote{SetPPortMode()}}, \sphinxcode{\sphinxupquote{GetPPortMode()}}

\index{OpenNetworkListener@\spxentry{OpenNetworkListener}}\ignorespaces 

\subsection{OpenNetworkListener()}
\label{\detokenize{reference/peblstream:opennetworklistener}}\label{\detokenize{reference/peblstream:index-25}}
\sphinxAtStartPar
\sphinxstyleemphasis{Opens a port for listening}

\sphinxAtStartPar
\sphinxstylestrong{Description:}

\sphinxAtStartPar
Creates a network object that listens on a particular port, and is able to accept incoming connections. You can the nuse \sphinxcode{\sphinxupquote{CheckForNetworkConnections}} to accept incoming connections.   This is an alternative to the \sphinxcode{\sphinxupquote{WaitForNetworkConnection}} function that allows more flexibility (and allows updating the during waiting for the connection).

\sphinxAtStartPar
\sphinxstylestrong{Usage:}

\begin{sphinxVerbatim}[commandchars=\\\{\}]
\PYG{n+nv}{net}\PYG{+w}{ }\PYG{o}{\PYGZlt{}\PYGZhy{}}\PYG{+w}{ }\PYG{n+nf}{OpennetworkListener}\PYG{p}{(}\PYG{n+nv}{port}\PYG{p}{)}
\end{sphinxVerbatim}

\sphinxAtStartPar
\sphinxstylestrong{Example:}

\begin{sphinxVerbatim}[commandchars=\\\{\}]
\PYG{n+nv}{network}\PYG{+w}{ }\PYG{o}{\PYGZlt{}\PYGZhy{}}\PYG{+w}{      }\PYG{n+nf}{OpenNetworkListener}\PYG{p}{(}\PYG{l+m+mi}{4444}\PYG{p}{)}
\PYG{+w}{  }\PYG{n+nv}{time}\PYG{+w}{ }\PYG{o}{\PYGZlt{}\PYGZhy{}}\PYG{+w}{ }\PYG{n+nf}{GetTime}\PYG{p}{(}\PYG{p}{)}
\PYG{+w}{  }\PYG{k}{while}\PYG{p}{(}\PYG{k}{not}\PYG{+w}{ }\PYG{n+nv}{connected}\PYG{+w}{ }\PYG{k}{and}\PYG{+w}{ }\PYG{p}{(}\PYG{n+nf}{GetTime}\PYG{p}{(}\PYG{p}{)}\PYG{+w}{ }\PYG{o}{\PYGZlt{}}\PYG{+w}{ }\PYG{n+nv}{time}\PYG{+w}{ }\PYG{o}{+}\PYG{+w}{ }\PYG{l+m+mi}{5000}\PYG{p}{)}\PYG{p}{)}
\PYG{+w}{   }\PYG{p}{\PYGZob{}}
\PYG{+w}{      }\PYG{n+nv}{connected}\PYG{+w}{ }\PYG{o}{\PYGZlt{}\PYGZhy{}}\PYG{+w}{ }\PYG{n+nf}{CheckForNetwokConnection}\PYG{p}{(}\PYG{n+nv}{network}\PYG{p}{)}
\PYG{+w}{   }\PYG{p}{\PYGZcb{}}
\end{sphinxVerbatim}

\sphinxAtStartPar
\sphinxstylestrong{See Also:}

\sphinxAtStartPar
\sphinxcode{\sphinxupquote{CheckForNetworkConnection()}}, \sphinxcode{\sphinxupquote{Getdata()}}, \sphinxcode{\sphinxupquote{WaitForNetworkConnection()}}, \sphinxcode{\sphinxupquote{CloseNetwork()}}

\index{OpenPPort@\spxentry{OpenPPort}}\ignorespaces 

\subsection{OpenPPort()}
\label{\detokenize{reference/peblstream:openpport}}\label{\detokenize{reference/peblstream:index-26}}
\sphinxAtStartPar
\sphinxstyleemphasis{Opens parallel port}

\sphinxAtStartPar
\sphinxstylestrong{Description:}

\sphinxAtStartPar
Opens a Parallel  port, returning an object that can be used for parallel port communications.

\sphinxAtStartPar
\sphinxstylestrong{See Also:}

\sphinxAtStartPar
\sphinxcode{\sphinxupquote{COMPortGetByte()}}, \sphinxcode{\sphinxupquote{COMPortSendByte()}}, \sphinxcode{\sphinxupquote{OpenCOMPort()}}, \sphinxcode{\sphinxupquote{SetPPortMode()}}, \sphinxcode{\sphinxupquote{GetPPortMode()}}

\index{ParseJSON@\spxentry{ParseJSON}}\ignorespaces 

\subsection{ParseJSON()}
\label{\detokenize{reference/peblstream:parsejson}}\label{\detokenize{reference/peblstream:index-27}}
\sphinxAtStartPar
\sphinxstyleemphasis{Parses a JSON string into PEBL data structures}

\sphinxAtStartPar
\sphinxstylestrong{Description:}

\sphinxAtStartPar
Parses a JSON\sphinxhyphen{}formatted string and converts it into PEBL data structures (lists and custom objects). JSON objects become PEBL custom objects with properties, and JSON arrays become PEBL lists. This is useful for working with web APIs and configuration files.

\sphinxAtStartPar
\sphinxstylestrong{Usage:}

\begin{sphinxVerbatim}[commandchars=\\\{\}]
\PYG{n+nf}{ParseJSON}\PYG{p}{(}\PYG{o}{\PYGZlt{}}\PYG{n+nv}{json\PYGZus{}string}\PYG{o}{\PYGZgt{}}\PYG{p}{)}
\end{sphinxVerbatim}

\sphinxAtStartPar
\sphinxstylestrong{Example:}

\begin{sphinxVerbatim}[commandchars=\\\{\}]
\PYG{n+nv}{jsonString}\PYG{+w}{ }\PYG{o}{\PYGZlt{}\PYGZhy{}}\PYG{+w}{ }\PYG{n+nf}{GetHTTPText}\PYG{p}{(}\PYG{l+s+s2}{\PYGZdq{}https://api.example.com/data\PYGZdq{}}\PYG{p}{,}\PYG{+w}{ }\PYG{l+s+s2}{\PYGZdq{}\PYGZdq{}}\PYG{p}{,}\PYG{+w}{ }\PYG{l+s+s2}{\PYGZdq{}\PYGZdq{}}\PYG{p}{)}
\PYG{n+nv}{data}\PYG{+w}{ }\PYG{o}{\PYGZlt{}\PYGZhy{}}\PYG{+w}{ }\PYG{n+nf}{ParseJSON}\PYG{p}{(}\PYG{n+nv}{jsonString}\PYG{p}{)}

\PYG{c+c1}{\PYGZsh{}\PYGZsh{}Access parsed data}
\PYG{n+nf}{Print}\PYG{p}{(}\PYG{n+nv}{data.name}\PYG{p}{)}
\PYG{n+nf}{Print}\PYG{p}{(}\PYG{n+nv}{data.values}\PYG{p}{)}
\end{sphinxVerbatim}

\sphinxAtStartPar
\sphinxstylestrong{See Also:}

\sphinxAtStartPar
\sphinxcode{\sphinxupquote{GetHTTPText()}}, \sphinxcode{\sphinxupquote{JSONText()}}, \sphinxcode{\sphinxupquote{PostHTTP()}}, \sphinxcode{\sphinxupquote{MakeCustomObject()}}

\index{Print@\spxentry{Print}}\ignorespaces 

\subsection{Print()}
\label{\detokenize{reference/peblstream:print}}\label{\detokenize{reference/peblstream:index-28}}
\sphinxAtStartPar
\sphinxstylestrong{Description:}

\sphinxAtStartPar
Prints \sphinxcode{\sphinxupquote{\textless{}value\textgreater{}}} to stdout; doesn’t append a newline afterwards.

\sphinxAtStartPar
\sphinxstylestrong{Usage:}

\begin{sphinxVerbatim}[commandchars=\\\{\}]
\PYG{n+nf}{Print\PYGZus{}}\PYG{p}{(}\PYG{o}{\PYGZlt{}}\PYG{n+nv}{value}\PYG{o}{\PYGZgt{}}\PYG{p}{)}
\end{sphinxVerbatim}

\sphinxAtStartPar
\sphinxstylestrong{Example:}

\begin{sphinxVerbatim}[commandchars=\\\{\}]
\PYG{n+nf}{Print\PYGZus{}}\PYG{p}{(}\PYG{l+s+s2}{\PYGZdq{}This line\PYGZdq{}}\PYG{p}{)}
\PYG{n+nf}{Print\PYGZus{}}\PYG{p}{(}\PYG{l+s+s2}{\PYGZdq{} \PYGZdq{}}\PYG{p}{)}
\PYG{n+nf}{Print\PYGZus{}}\PYG{p}{(}\PYG{l+s+s2}{\PYGZdq{}and\PYGZdq{}}\PYG{p}{)}
\PYG{n+nf}{Print\PYGZus{}}\PYG{p}{(}\PYG{l+s+s2}{\PYGZdq{} \PYGZdq{}}\PYG{p}{)}
\PYG{n+nf}{Print}\PYG{p}{(}\PYG{l+s+s2}{\PYGZdq{}Another line\PYGZdq{}}\PYG{p}{)}
\PYG{c+c1}{\PYGZsh{} prints out: \PYGZsq{}This line and Another line\PYGZsq{}}
\end{sphinxVerbatim}

\sphinxAtStartPar
\sphinxstylestrong{See Also:}

\sphinxAtStartPar
\sphinxcode{\sphinxupquote{Print()}}, \sphinxcode{\sphinxupquote{FilePrint()}}

\index{SendData@\spxentry{SendData}}\ignorespaces 

\subsection{SendData()}
\label{\detokenize{reference/peblstream:senddata}}\label{\detokenize{reference/peblstream:index-29}}
\sphinxAtStartPar
\sphinxstyleemphasis{Sends a data string over connection.}

\sphinxAtStartPar
\sphinxstylestrong{Description:}

\sphinxAtStartPar
Sends data on network connection.  Example of   usage in demo/nim.pbl. You can only send text data.

\sphinxAtStartPar
\sphinxstylestrong{Usage:}

\begin{sphinxVerbatim}[commandchars=\\\{\}]
\PYG{n+nf}{SendData}\PYG{p}{(}\PYG{o}{\PYGZlt{}}\PYG{n+nv}{network}\PYG{o}{\PYGZgt{}}\PYG{p}{,}\PYG{o}{\PYGZlt{}}\PYG{n+nv}{data\PYGZus{}as\PYGZus{}string}\PYG{o}{\PYGZgt{}}\PYG{p}{)}
\end{sphinxVerbatim}

\sphinxAtStartPar
\sphinxstylestrong{See Also:}

\sphinxAtStartPar
\sphinxcode{\sphinxupquote{ConnectToIP()}}, \sphinxcode{\sphinxupquote{ConnectToHost()}}, \sphinxcode{\sphinxupquote{WaitForNetworkConnection()}}, \sphinxcode{\sphinxupquote{GetData()}}, \sphinxcode{\sphinxupquote{ConvertIPString()}}, \sphinxcode{\sphinxupquote{CloseNetworkConnection()}}

\index{SetNetworkPort@\spxentry{SetNetworkPort}}\ignorespaces 

\subsection{SetNetworkPort()}
\label{\detokenize{reference/peblstream:setnetworkport}}\label{\detokenize{reference/peblstream:index-30}}
\sphinxAtStartPar
\sphinxstyleemphasis{Configures the default network port for connections}

\sphinxAtStartPar
\sphinxstylestrong{Description:}

\sphinxAtStartPar
Sets the default network port number to be used for subsequent network operations. This allows you to configure the port once rather than specifying it for each connection. The port number should be between 1024 and 65535 for user applications.

\sphinxAtStartPar
\sphinxstylestrong{Usage:}

\begin{sphinxVerbatim}[commandchars=\\\{\}]
\PYG{n+nf}{SetNetworkPort}\PYG{p}{(}\PYG{o}{\PYGZlt{}}\PYG{n+nv}{port}\PYG{o}{\PYGZgt{}}\PYG{p}{)}
\end{sphinxVerbatim}

\sphinxAtStartPar
\sphinxstylestrong{Example:}

\begin{sphinxVerbatim}[commandchars=\\\{\}]
\PYG{n+nf}{SetNetworkPort}\PYG{p}{(}\PYG{l+m+mi}{8080}\PYG{p}{)}
\PYG{n+nv}{listener}\PYG{+w}{ }\PYG{o}{\PYGZlt{}\PYGZhy{}}\PYG{+w}{ }\PYG{n+nf}{OpenNetworkListener}\PYG{p}{(}\PYG{l+m+mi}{8080}\PYG{p}{)}
\end{sphinxVerbatim}

\sphinxAtStartPar
\sphinxstylestrong{See Also:}

\sphinxAtStartPar
\sphinxcode{\sphinxupquote{OpenNetworkListener()}}, \sphinxcode{\sphinxupquote{ConnectToHost()}}, \sphinxcode{\sphinxupquote{ConnectToIP()}}

\index{SetPPortMode@\spxentry{SetPPortMode}}\ignorespaces 

\subsection{SetPPortMode()}
\label{\detokenize{reference/peblstream:setpportmode}}\label{\detokenize{reference/peblstream:index-31}}
\sphinxAtStartPar
\sphinxstyleemphasis{Sets parallel port mode (input/output)}

\sphinxAtStartPar
\sphinxstylestrong{Description:}

\sphinxAtStartPar
Sets a parallel port mode, either “\textless{}input\textgreater{}” or “\textless{}output\textgreater{}”.

\sphinxAtStartPar
\sphinxstylestrong{See Also:}

\sphinxAtStartPar
\sphinxcode{\sphinxupquote{COMPortGetByte()}}, \sphinxcode{\sphinxupquote{COMPortSendByte()}}, \sphinxcode{\sphinxupquote{OpenPPort()}} \sphinxcode{\sphinxupquote{OpenCOMPort()}}, \sphinxcode{\sphinxupquote{SetPPortMode()}}, \sphinxcode{\sphinxupquote{GetPPortState()}}

\index{SetPPortState@\spxentry{SetPPortState}}\ignorespaces 

\subsection{SetPPortState()}
\label{\detokenize{reference/peblstream:setpportstate}}\label{\detokenize{reference/peblstream:index-32}}
\sphinxAtStartPar
\sphinxstyleemphasis{Sets parallel port state}

\sphinxAtStartPar
\sphinxstylestrong{Description:}

\sphinxAtStartPar
Sets a parallel port state, using a list of 8 ‘bits’ (1s or 0s).

\sphinxAtStartPar
\sphinxstylestrong{See Also:}

\sphinxAtStartPar
\sphinxcode{\sphinxupquote{COMPortGetByte()}}, \sphinxcode{\sphinxupquote{COMPortSendByte()}}, \sphinxcode{\sphinxupquote{OpenPPort()}} \sphinxcode{\sphinxupquote{OpenCOMPort()}}, \sphinxcode{\sphinxupquote{SetPPortMode()}}, \sphinxcode{\sphinxupquote{GetPPortState()}}

\index{WaitForNetworkConnection@\spxentry{WaitForNetworkConnection}}\ignorespaces 

\subsection{WaitForNetworkConnection()}
\label{\detokenize{reference/peblstream:waitfornetworkconnection}}\label{\detokenize{reference/peblstream:index-33}}
\sphinxAtStartPar
\sphinxstylestrong{Description:}

\sphinxAtStartPar
Listens on a port, waiting until another computer or process   connects. Return a network object that can be used for communication.

\sphinxAtStartPar
\sphinxstylestrong{Usage:}

\begin{sphinxVerbatim}[commandchars=\\\{\}]
\PYG{n+nf}{WaitForNetworkConnection}\PYG{p}{(}\PYG{o}{\PYGZlt{}}\PYG{n+nv}{port}\PYG{o}{\PYGZgt{}}\PYG{p}{)}
\end{sphinxVerbatim}

\sphinxAtStartPar
\sphinxstylestrong{See Also:}

\sphinxAtStartPar
\sphinxcode{\sphinxupquote{ConnectToHost()}}, \sphinxcode{\sphinxupquote{ConnectToIP()}}, \sphinxcode{\sphinxupquote{GetData()}}, \sphinxcode{\sphinxupquote{WaitForNetworkConnection()}},    \sphinxcode{\sphinxupquote{SendData()}}, \sphinxcode{\sphinxupquote{ConvertIPString()}}, \sphinxcode{\sphinxupquote{CloseNetworkConnection()}}

\index{WritePNG@\spxentry{WritePNG}}\ignorespaces 

\subsection{WritePNG()}
\label{\detokenize{reference/peblstream:writepng}}\label{\detokenize{reference/peblstream:index-34}}
\sphinxAtStartPar
\sphinxstyleemphasis{Makes a .png from a window or object}

\sphinxAtStartPar
\sphinxstylestrong{Description:}

\sphinxAtStartPar
WritePNG() creates a graphic file of the screen or   a widget on the screen.  It can also be given an arbitrary widget.   For the most part, widgets added to other widgets will be captured   fine, but sometimes polygons and shapes added to other widgets may   not appear in the output png.

\sphinxAtStartPar
\sphinxstylestrong{Usage:}

\begin{sphinxVerbatim}[commandchars=\\\{\}]
\PYG{n+nv}{x}\PYG{+w}{ }\PYG{o}{\PYGZlt{}\PYGZhy{}}\PYG{+w}{  }\PYG{n+nf}{WritePNG}\PYG{p}{(}\PYG{l+s+s2}{\PYGZdq{}screen1.png\PYGZdq{}}\PYG{p}{,}\PYG{n+nv+vg}{gWin}\PYG{p}{)}

\PYG{+w}{  }\PYG{c+c1}{\PYGZsh{}\PYGZsh{} Use like this to create an animated screencast}
\PYG{+w}{   }\PYG{k}{define}\PYG{+w}{ }\PYG{n+nf}{DrawMe}\PYG{p}{(}\PYG{p}{)}
\PYG{+w}{    }\PYG{p}{\PYGZob{}}
\PYG{+w}{      }\PYG{n+nv}{pname}\PYG{+w}{ }\PYG{o}{\PYGZlt{}\PYGZhy{}}\PYG{+w}{ }\PYG{l+s+s2}{\PYGZdq{}fileout\PYGZdq{}}\PYG{o}{+}\PYG{n+nf}{ZeroPad}\PYG{p}{(}\PYG{n+nv+vg}{gid}\PYG{p}{,}\PYG{l+m+mi}{5}\PYG{p}{)}\PYG{o}{+}\PYG{l+s+s2}{\PYGZdq{}.png\PYGZdq{}}
\PYG{+w}{      }\PYG{n+nf}{Draw}\PYG{p}{(}\PYG{p}{)}
\PYG{+w}{      }\PYG{n+nf}{WritePNG}\PYG{p}{(}\PYG{n+nv}{pname}\PYG{p}{,}\PYG{n+nv+vg}{gWin}\PYG{p}{)}
\PYG{+w}{    }\PYG{p}{\PYGZcb{}}

\PYG{+w}{   }\PYG{k}{define}\PYG{+w}{ }\PYG{n+nf}{Start}\PYG{p}{(}\PYG{n+nv}{p}\PYG{p}{)}
\PYG{+w}{   }\PYG{p}{\PYGZob{}}
\PYG{+w}{     }\PYG{n+nv+vg}{gid}\PYG{+w}{ }\PYG{o}{\PYGZlt{}\PYGZhy{}}\PYG{+w}{ }\PYG{l+m+mi}{1}
\PYG{+w}{     }\PYG{n+nv+vg}{gWin}\PYG{+w}{ }\PYG{o}{\PYGZlt{}\PYGZhy{}}\PYG{+w}{ }\PYG{n+nf}{MakeWindow}\PYG{p}{(}\PYG{p}{)}
\PYG{+w}{     }\PYG{n+nv}{img}\PYG{+w}{ }\PYG{o}{\PYGZlt{}\PYGZhy{}}\PYG{+w}{ }\PYG{n+nf}{MakeImage}\PYG{p}{(}\PYG{l+s+s2}{\PYGZdq{}pebl.png\PYGZdq{}}\PYG{p}{)}
\PYG{+w}{     }\PYG{n+nf}{AddObject}\PYG{p}{(}\PYG{n+nv}{img}\PYG{p}{,}\PYG{n+nv+vg}{gWin}\PYG{p}{)}
\PYG{+w}{     }\PYG{k}{while}\PYG{p}{(}\PYG{n+nv+vg}{gid}\PYG{+w}{ }\PYG{o}{\PYGZlt{}}\PYG{+w}{ }\PYG{l+m+mi}{100}\PYG{p}{)}
\PYG{+w}{      }\PYG{p}{\PYGZob{}}
\PYG{+w}{         }\PYG{n+nf}{Move}\PYG{p}{(}\PYG{n+nv}{img}\PYG{p}{,}\PYG{n+nf}{RandomDiscrete}\PYG{p}{(}\PYG{l+m+mi}{800}\PYG{p}{)}\PYG{p}{,}
\PYG{+w}{                  }\PYG{n+nf}{RandomDiscrete}\PYG{p}{(}\PYG{l+m+mi}{600}\PYG{p}{)}\PYG{p}{)}

\PYG{+w}{         }\PYG{n+nf}{DrawMe}\PYG{p}{(}\PYG{p}{)}
\PYG{+w}{         }\PYG{n+nv+vg}{gid}\PYG{+w}{ }\PYG{o}{\PYGZlt{}\PYGZhy{}}\PYG{+w}{ }\PYG{n+nv+vg}{gid}\PYG{+w}{ }\PYG{o}{+}\PYG{+w}{ }\PYG{l+m+mi}{1}
\PYG{+w}{      }\PYG{p}{\PYGZcb{}}

\PYG{+w}{   }\PYG{p}{\PYGZcb{}}
\end{sphinxVerbatim}

\sphinxAtStartPar
\sphinxstylestrong{See Also:}

\sphinxAtStartPar
\sphinxcode{\sphinxupquote{FileWriteTable()}}

\index{COMPortGetByte@\spxentry{COMPortGetByte}}\ignorespaces 

\subsection{COMPortGetByte()}
\label{\detokenize{reference/peblstream:comportgetbyte}}\label{\detokenize{reference/peblstream:index-35}}
\sphinxAtStartPar
\sphinxstyleemphasis{Gets a byte from the COM port}

\sphinxAtStartPar
\sphinxstylestrong{Description:}

\sphinxAtStartPar
Reads a single byte from an open COM/serial port. Returns the byte value as an integer (0\sphinxhyphen{}255). If no data is available, it returns \sphinxhyphen{}1. This is used for serial communication with external devices.

\sphinxAtStartPar
\sphinxstylestrong{Usage:}

\begin{sphinxVerbatim}[commandchars=\\\{\}]
\PYG{n+nf}{COMPortGetByte}\PYG{p}{(}\PYG{o}{\PYGZlt{}}\PYG{n+nv}{port}\PYG{o}{\PYGZgt{}}\PYG{p}{)}
\end{sphinxVerbatim}

\sphinxAtStartPar
\sphinxstylestrong{Example:}

\begin{sphinxVerbatim}[commandchars=\\\{\}]
\PYG{n+nv}{port}\PYG{+w}{ }\PYG{o}{\PYGZlt{}\PYGZhy{}}\PYG{+w}{ }\PYG{n+nf}{OpenCOMPort}\PYG{p}{(}\PYG{l+m+mi}{1}\PYG{p}{,}\PYG{+w}{ }\PYG{l+m+mi}{9600}\PYG{p}{,}\PYG{+w}{ }\PYG{l+s+s2}{\PYGZdq{}8N1\PYGZdq{}}\PYG{p}{)}
\PYG{n+nv}{byte}\PYG{+w}{ }\PYG{o}{\PYGZlt{}\PYGZhy{}}\PYG{+w}{ }\PYG{n+nf}{COMPortGetByte}\PYG{p}{(}\PYG{n+nv}{port}\PYG{p}{)}
\PYG{k}{if}\PYG{p}{(}\PYG{n+nv}{byte}\PYG{+w}{ }\PYG{o}{\PYGZgt{}=}\PYG{+w}{ }\PYG{l+m+mi}{0}\PYG{p}{)}
\PYG{p}{\PYGZob{}}
\PYG{+w}{   }\PYG{n+nf}{Print}\PYG{p}{(}\PYG{l+s+s2}{\PYGZdq{}Received byte: \PYGZdq{}}\PYG{+w}{ }\PYG{o}{+}\PYG{+w}{ }\PYG{n+nv}{byte}\PYG{p}{)}
\PYG{p}{\PYGZcb{}}
\end{sphinxVerbatim}

\sphinxAtStartPar
\sphinxstylestrong{See Also:}

\sphinxAtStartPar
\sphinxcode{\sphinxupquote{COMPortSendByte()}}, \sphinxcode{\sphinxupquote{OpenCOMPort()}}

\index{COMPortSendByte@\spxentry{COMPortSendByte}}\ignorespaces 

\subsection{COMPortSendByte()}
\label{\detokenize{reference/peblstream:comportsendbyte}}\label{\detokenize{reference/peblstream:index-36}}
\sphinxAtStartPar
\sphinxstyleemphasis{Sends a byte to the COM port}

\sphinxAtStartPar
\sphinxstylestrong{Description:}

\sphinxAtStartPar
Sends a single byte to an open COM/serial port. The byte should be an integer value between 0 and 255. This is used for serial communication with external devices.

\sphinxAtStartPar
\sphinxstylestrong{Usage:}

\begin{sphinxVerbatim}[commandchars=\\\{\}]
\PYG{n+nf}{COMPortSendByte}\PYG{p}{(}\PYG{o}{\PYGZlt{}}\PYG{n+nv}{port}\PYG{o}{\PYGZgt{}}\PYG{p}{,}\PYG{+w}{ }\PYG{o}{\PYGZlt{}}\PYG{n+nv}{byte}\PYG{o}{\PYGZgt{}}\PYG{p}{)}
\end{sphinxVerbatim}

\sphinxAtStartPar
\sphinxstylestrong{Example:}

\begin{sphinxVerbatim}[commandchars=\\\{\}]
\PYG{n+nv}{port}\PYG{+w}{ }\PYG{o}{\PYGZlt{}\PYGZhy{}}\PYG{+w}{ }\PYG{n+nf}{OpenCOMPort}\PYG{p}{(}\PYG{l+m+mi}{1}\PYG{p}{,}\PYG{+w}{ }\PYG{l+m+mi}{9600}\PYG{p}{,}\PYG{+w}{ }\PYG{l+s+s2}{\PYGZdq{}8N1\PYGZdq{}}\PYG{p}{)}
\PYG{c+c1}{\PYGZsh{}\PYGZsh{}Send ASCII \PYGZsq{}A\PYGZsq{} (65)}
\PYG{n+nf}{COMPortSendByte}\PYG{p}{(}\PYG{n+nv}{port}\PYG{p}{,}\PYG{+w}{ }\PYG{l+m+mi}{65}\PYG{p}{)}
\end{sphinxVerbatim}

\sphinxAtStartPar
\sphinxstylestrong{See Also:}

\sphinxAtStartPar
\sphinxcode{\sphinxupquote{COMPortGetByte()}}, \sphinxcode{\sphinxupquote{OpenCOMPort()}}

\index{GetHTTPFile@\spxentry{GetHTTPFile}}\ignorespaces 

\subsection{GetHTTPFile()}
\label{\detokenize{reference/peblstream:gethttpfile}}\label{\detokenize{reference/peblstream:index-37}}
\sphinxAtStartPar
\sphinxstyleemphasis{Downloads a file from a web server via HTTP}

\sphinxAtStartPar
\sphinxstylestrong{Description:}

\sphinxAtStartPar
Fetches a file from a web server using HTTP and saves it to a local file. Supports HTTP and HTTPS protocols. Useful for downloading stimuli, configuration files, or data from web servers during experiments.

\sphinxAtStartPar
\sphinxstylestrong{Usage:}

\begin{sphinxVerbatim}[commandchars=\\\{\}]
\PYG{n+nf}{GetHTTPFile}\PYG{p}{(}\PYG{o}{\PYGZlt{}}\PYG{n+nv}{url}\PYG{o}{\PYGZgt{}}\PYG{p}{,}\PYG{+w}{ }\PYG{o}{\PYGZlt{}}\PYG{n+nv}{output\PYGZus{}filename}\PYG{o}{\PYGZgt{}}\PYG{p}{,}\PYG{+w}{ }\PYG{o}{\PYGZlt{}}\PYG{n+nv}{username}\PYG{o}{\PYGZgt{}}\PYG{p}{,}\PYG{+w}{ }\PYG{o}{\PYGZlt{}}\PYG{n+nv}{password}\PYG{o}{\PYGZgt{}}\PYG{p}{)}
\end{sphinxVerbatim}

\sphinxAtStartPar
\sphinxstylestrong{Example:}

\begin{sphinxVerbatim}[commandchars=\\\{\}]
\PYG{c+c1}{\PYGZsh{}\PYGZsh{}Download without authentication}
\PYG{n+nv}{success}\PYG{+w}{ }\PYG{o}{\PYGZlt{}\PYGZhy{}}\PYG{+w}{ }\PYG{n+nf}{GetHTTPFile}\PYG{p}{(}\PYG{l+s+s2}{\PYGZdq{}http://example.com/data.csv\PYGZdq{}}\PYG{p}{,}\PYG{+w}{ }\PYG{l+s+s2}{\PYGZdq{}local\PYGZus{}data.csv\PYGZdq{}}\PYG{p}{,}\PYG{+w}{ }\PYG{l+s+s2}{\PYGZdq{}\PYGZdq{}}\PYG{p}{,}\PYG{+w}{ }\PYG{l+s+s2}{\PYGZdq{}\PYGZdq{}}\PYG{p}{)}

\PYG{c+c1}{\PYGZsh{}\PYGZsh{}Download with authentication}
\PYG{n+nv}{success}\PYG{+w}{ }\PYG{o}{\PYGZlt{}\PYGZhy{}}\PYG{+w}{ }\PYG{n+nf}{GetHTTPFile}\PYG{p}{(}\PYG{l+s+s2}{\PYGZdq{}https://secure.example.com/file.zip\PYGZdq{}}\PYG{p}{,}\PYG{+w}{ }\PYG{l+s+s2}{\PYGZdq{}download.zip\PYGZdq{}}\PYG{p}{,}\PYG{+w}{ }\PYG{l+s+s2}{\PYGZdq{}user\PYGZdq{}}\PYG{p}{,}\PYG{+w}{ }\PYG{l+s+s2}{\PYGZdq{}pass\PYGZdq{}}\PYG{p}{)}
\end{sphinxVerbatim}

\sphinxAtStartPar
\sphinxstylestrong{See Also:}

\sphinxAtStartPar
\sphinxcode{\sphinxupquote{GetHTTPText()}}, \sphinxcode{\sphinxupquote{PostHTTP()}}, \sphinxcode{\sphinxupquote{PostHTTPFile()}}

\index{GetHTTPText@\spxentry{GetHTTPText}}\ignorespaces 

\subsection{GetHTTPText()}
\label{\detokenize{reference/peblstream:gethttptext}}\label{\detokenize{reference/peblstream:index-38}}
\sphinxAtStartPar
\sphinxstyleemphasis{Retrieves text content from a web server via HTTP}

\sphinxAtStartPar
\sphinxstylestrong{Description:}

\sphinxAtStartPar
Fetches content from a web server using HTTP and returns it as a text string. Supports HTTP and HTTPS protocols. Useful for retrieving instructions, configurations, or data from web services during experiments.

\sphinxAtStartPar
\sphinxstylestrong{Usage:}

\begin{sphinxVerbatim}[commandchars=\\\{\}]
\PYG{n+nf}{GetHTTPText}\PYG{p}{(}\PYG{o}{\PYGZlt{}}\PYG{n+nv}{url}\PYG{o}{\PYGZgt{}}\PYG{p}{,}\PYG{+w}{ }\PYG{o}{\PYGZlt{}}\PYG{n+nv}{username}\PYG{o}{\PYGZgt{}}\PYG{p}{,}\PYG{+w}{ }\PYG{o}{\PYGZlt{}}\PYG{n+nv}{password}\PYG{o}{\PYGZgt{}}\PYG{p}{)}
\end{sphinxVerbatim}

\sphinxAtStartPar
\sphinxstylestrong{Example:}

\begin{sphinxVerbatim}[commandchars=\\\{\}]
\PYG{c+c1}{\PYGZsh{}\PYGZsh{}Fetch text without authentication}
\PYG{n+nv}{text}\PYG{+w}{ }\PYG{o}{\PYGZlt{}\PYGZhy{}}\PYG{+w}{ }\PYG{n+nf}{GetHTTPText}\PYG{p}{(}\PYG{l+s+s2}{\PYGZdq{}http://example.com/instructions.txt\PYGZdq{}}\PYG{p}{,}\PYG{+w}{ }\PYG{l+s+s2}{\PYGZdq{}\PYGZdq{}}\PYG{p}{,}\PYG{+w}{ }\PYG{l+s+s2}{\PYGZdq{}\PYGZdq{}}\PYG{p}{)}
\PYG{n+nf}{Print}\PYG{p}{(}\PYG{n+nv}{text}\PYG{p}{)}

\PYG{c+c1}{\PYGZsh{}\PYGZsh{}Fetch JSON data with authentication}
\PYG{n+nv}{jsonData}\PYG{+w}{ }\PYG{o}{\PYGZlt{}\PYGZhy{}}\PYG{+w}{ }\PYG{n+nf}{GetHTTPText}\PYG{p}{(}\PYG{l+s+s2}{\PYGZdq{}https://api.example.com/config\PYGZdq{}}\PYG{p}{,}\PYG{+w}{ }\PYG{l+s+s2}{\PYGZdq{}user\PYGZdq{}}\PYG{p}{,}\PYG{+w}{ }\PYG{l+s+s2}{\PYGZdq{}pass\PYGZdq{}}\PYG{p}{)}
\PYG{n+nv}{config}\PYG{+w}{ }\PYG{o}{\PYGZlt{}\PYGZhy{}}\PYG{+w}{ }\PYG{n+nf}{ParseJSON}\PYG{p}{(}\PYG{n+nv}{jsonData}\PYG{p}{)}
\end{sphinxVerbatim}

\sphinxAtStartPar
\sphinxstylestrong{See Also:}

\sphinxAtStartPar
\sphinxcode{\sphinxupquote{GetHTTPFile()}}, \sphinxcode{\sphinxupquote{PostHTTP()}}, \sphinxcode{\sphinxupquote{ParseJSON()}}

\index{PostHTTP@\spxentry{PostHTTP}}\ignorespaces 

\subsection{PostHTTP()}
\label{\detokenize{reference/peblstream:posthttp}}\label{\detokenize{reference/peblstream:index-39}}
\sphinxAtStartPar
\sphinxstyleemphasis{Sends data to a web server via HTTP POST}

\sphinxAtStartPar
\sphinxstylestrong{Description:}

\sphinxAtStartPar
Sends data to a web server using the HTTP POST method. Returns the server’s response as a string. This is useful for submitting experimental data to web servers, interacting with web APIs, or logging data remotely.

\sphinxAtStartPar
\sphinxstylestrong{Usage:}

\begin{sphinxVerbatim}[commandchars=\\\{\}]
\PYG{n+nf}{PostHTTP}\PYG{p}{(}\PYG{o}{\PYGZlt{}}\PYG{n+nv}{url}\PYG{o}{\PYGZgt{}}\PYG{p}{,}\PYG{+w}{ }\PYG{o}{\PYGZlt{}}\PYG{n+nv}{post\PYGZus{}data}\PYG{o}{\PYGZgt{}}\PYG{p}{,}\PYG{+w}{ }\PYG{o}{\PYGZlt{}}\PYG{n+nv}{username}\PYG{o}{\PYGZgt{}}\PYG{p}{,}\PYG{+w}{ }\PYG{o}{\PYGZlt{}}\PYG{n+nv}{password}\PYG{o}{\PYGZgt{}}\PYG{p}{,}\PYG{+w}{ }\PYG{o}{\PYGZlt{}}\PYG{n+nv}{content\PYGZus{}type}\PYG{o}{\PYGZgt{}}\PYG{p}{)}
\end{sphinxVerbatim}

\sphinxAtStartPar
\sphinxstylestrong{Example:}

\begin{sphinxVerbatim}[commandchars=\\\{\}]
\PYG{c+c1}{\PYGZsh{}\PYGZsh{}Post form data}
\PYG{n+nv}{postData}\PYG{+w}{ }\PYG{o}{\PYGZlt{}\PYGZhy{}}\PYG{+w}{ }\PYG{l+s+s2}{\PYGZdq{}subjectID=123\PYGZam{}condition=A\PYGZam{}score=85\PYGZdq{}}
\PYG{n+nv}{response}\PYG{+w}{ }\PYG{o}{\PYGZlt{}\PYGZhy{}}\PYG{+w}{ }\PYG{n+nf}{PostHTTP}\PYG{p}{(}\PYG{l+s+s2}{\PYGZdq{}http://example.com/submit\PYGZdq{}}\PYG{p}{,}\PYG{+w}{ }\PYG{n+nv}{postData}\PYG{p}{,}\PYG{+w}{ }\PYG{l+s+s2}{\PYGZdq{}\PYGZdq{}}\PYG{p}{,}\PYG{+w}{ }\PYG{l+s+s2}{\PYGZdq{}\PYGZdq{}}\PYG{p}{,}\PYG{+w}{ }\PYG{l+s+s2}{\PYGZdq{}application/x\PYGZhy{}www\PYGZhy{}form\PYGZhy{}urlencoded\PYGZdq{}}\PYG{p}{)}

\PYG{c+c1}{\PYGZsh{}\PYGZsh{}Post JSON data}
\PYG{n+nv}{jsonData}\PYG{+w}{ }\PYG{o}{\PYGZlt{}\PYGZhy{}}\PYG{+w}{ }\PYG{l+s+s2}{\PYGZdq{}\PYGZob{}\PYGZdq{}}\PYG{n+nv}{subject}\PYG{l+s+s2}{\PYGZdq{}:123,\PYGZdq{}}\PYG{n+nv}{rt}\PYG{l+s+s2}{\PYGZdq{}:450\PYGZcb{}\PYGZdq{}}
\PYG{n+nv}{response}\PYG{+w}{ }\PYG{o}{\PYGZlt{}\PYGZhy{}}\PYG{+w}{ }\PYG{n+nf}{PostHTTP}\PYG{p}{(}\PYG{l+s+s2}{\PYGZdq{}https://api.example.com/data\PYGZdq{}}\PYG{p}{,}\PYG{+w}{ }\PYG{n+nv}{jsonData}\PYG{p}{,}\PYG{+w}{ }\PYG{l+s+s2}{\PYGZdq{}user\PYGZdq{}}\PYG{p}{,}\PYG{+w}{ }\PYG{l+s+s2}{\PYGZdq{}pass\PYGZdq{}}\PYG{p}{,}\PYG{+w}{ }\PYG{l+s+s2}{\PYGZdq{}application/json\PYGZdq{}}\PYG{p}{)}
\end{sphinxVerbatim}

\sphinxAtStartPar
\sphinxstylestrong{See Also:}

\sphinxAtStartPar
\sphinxcode{\sphinxupquote{PostHTTPFile()}}, \sphinxcode{\sphinxupquote{GetHTTPText()}}, \sphinxcode{\sphinxupquote{GetHTTPFile()}}

\index{PostHTTPFile@\spxentry{PostHTTPFile}}\ignorespaces 

\subsection{PostHTTPFile()}
\label{\detokenize{reference/peblstream:posthttpfile}}\label{\detokenize{reference/peblstream:index-40}}
\sphinxAtStartPar
\sphinxstyleemphasis{Uploads a file to a web server via HTTP POST}

\sphinxAtStartPar
\sphinxstylestrong{Description:}

\sphinxAtStartPar
Uploads a file to a web server using HTTP POST multipart/form\sphinxhyphen{}data encoding. This is useful for uploading experimental data files, log files, or other content to a web server for storage or processing.

\sphinxAtStartPar
\sphinxstylestrong{Usage:}

\begin{sphinxVerbatim}[commandchars=\\\{\}]
\PYG{n+nf}{PostHTTPFile}\PYG{p}{(}\PYG{o}{\PYGZlt{}}\PYG{n+nv}{url}\PYG{o}{\PYGZgt{}}\PYG{p}{,}\PYG{+w}{ }\PYG{o}{\PYGZlt{}}\PYG{n+nv}{filename}\PYG{o}{\PYGZgt{}}\PYG{p}{,}\PYG{+w}{ }\PYG{o}{\PYGZlt{}}\PYG{n+nv}{fieldname}\PYG{o}{\PYGZgt{}}\PYG{p}{,}\PYG{+w}{ }\PYG{o}{\PYGZlt{}}\PYG{n+nv}{username}\PYG{o}{\PYGZgt{}}\PYG{p}{,}\PYG{+w}{ }\PYG{o}{\PYGZlt{}}\PYG{n+nv}{password}\PYG{o}{\PYGZgt{}}\PYG{p}{,}\PYG{+w}{ }\PYG{o}{\PYGZlt{}}\PYG{n+nv}{additional\PYGZus{}fields}\PYG{o}{\PYGZgt{}}\PYG{p}{)}
\end{sphinxVerbatim}

\sphinxAtStartPar
\sphinxstylestrong{Example:}

\begin{sphinxVerbatim}[commandchars=\\\{\}]
\PYG{c+c1}{\PYGZsh{}\PYGZsh{}Upload data file}
\PYG{n+nv}{response}\PYG{+w}{ }\PYG{o}{\PYGZlt{}\PYGZhy{}}\PYG{+w}{ }\PYG{n+nf}{PostHTTPFile}\PYG{p}{(}\PYG{l+s+s2}{\PYGZdq{}http://example.com/upload\PYGZdq{}}\PYG{p}{,}\PYG{+w}{ }\PYG{l+s+s2}{\PYGZdq{}data.csv\PYGZdq{}}\PYG{p}{,}\PYG{+w}{ }\PYG{l+s+s2}{\PYGZdq{}datafile\PYGZdq{}}\PYG{p}{,}\PYG{+w}{ }\PYG{l+s+s2}{\PYGZdq{}\PYGZdq{}}\PYG{p}{,}\PYG{+w}{ }\PYG{l+s+s2}{\PYGZdq{}\PYGZdq{}}\PYG{p}{,}\PYG{+w}{ }\PYG{l+s+s2}{\PYGZdq{}\PYGZdq{}}\PYG{p}{)}

\PYG{c+c1}{\PYGZsh{}\PYGZsh{}Upload with additional form fields}
\PYG{n+nv}{additionalData}\PYG{+w}{ }\PYG{o}{\PYGZlt{}\PYGZhy{}}\PYG{+w}{ }\PYG{l+s+s2}{\PYGZdq{}subjectID=123\PYGZam{}session=1\PYGZdq{}}
\PYG{n+nv}{response}\PYG{+w}{ }\PYG{o}{\PYGZlt{}\PYGZhy{}}\PYG{+w}{ }\PYG{n+nf}{PostHTTPFile}\PYG{p}{(}\PYG{l+s+s2}{\PYGZdq{}https://secure.example.com/upload\PYGZdq{}}\PYG{p}{,}\PYG{+w}{ }\PYG{l+s+s2}{\PYGZdq{}results.txt\PYGZdq{}}\PYG{p}{,}\PYG{+w}{ }\PYG{l+s+s2}{\PYGZdq{}file\PYGZdq{}}\PYG{p}{,}\PYG{+w}{ }\PYG{l+s+s2}{\PYGZdq{}user\PYGZdq{}}\PYG{p}{,}\PYG{+w}{ }\PYG{l+s+s2}{\PYGZdq{}pass\PYGZdq{}}\PYG{p}{,}\PYG{+w}{ }\PYG{n+nv}{additionalData}\PYG{p}{)}
\end{sphinxVerbatim}

\sphinxAtStartPar
\sphinxstylestrong{See Also:}

\sphinxAtStartPar
\sphinxcode{\sphinxupquote{PostHTTP()}}, \sphinxcode{\sphinxupquote{GetHTTPFile()}}, \sphinxcode{\sphinxupquote{FileOpenRead()}}

\sphinxstepscope


\section{PEBLString \sphinxhyphen{} String Manipulation}
\label{\detokenize{reference/peblstring:peblstring-string-manipulation}}\label{\detokenize{reference/peblstring::doc}}
\sphinxAtStartPar
This module contains functions for string manipulation and formatting.

\begin{sphinxShadowBox}
\sphinxstyletopictitle{Function Index}
\begin{itemize}
\item {} 
\sphinxAtStartPar
\phantomsection\label{\detokenize{reference/peblstring:id1}}{\hyperref[\detokenize{reference/peblstring:copytoclipboard}]{\sphinxcrossref{CopyToClipboard()}}}

\item {} 
\sphinxAtStartPar
\phantomsection\label{\detokenize{reference/peblstring:id2}}{\hyperref[\detokenize{reference/peblstring:findinstring}]{\sphinxcrossref{FindInString()}}}

\item {} 
\sphinxAtStartPar
\phantomsection\label{\detokenize{reference/peblstring:id3}}{\hyperref[\detokenize{reference/peblstring:lowercase}]{\sphinxcrossref{Lowercase()}}}

\item {} 
\sphinxAtStartPar
\phantomsection\label{\detokenize{reference/peblstring:id4}}{\hyperref[\detokenize{reference/peblstring:splitstring}]{\sphinxcrossref{SplitString()}}}

\item {} 
\sphinxAtStartPar
\phantomsection\label{\detokenize{reference/peblstring:id5}}{\hyperref[\detokenize{reference/peblstring:stringlength}]{\sphinxcrossref{StringLength()}}}

\item {} 
\sphinxAtStartPar
\phantomsection\label{\detokenize{reference/peblstring:id6}}{\hyperref[\detokenize{reference/peblstring:substring}]{\sphinxcrossref{SubString()}}}

\item {} 
\sphinxAtStartPar
\phantomsection\label{\detokenize{reference/peblstring:id7}}{\hyperref[\detokenize{reference/peblstring:toascii}]{\sphinxcrossref{ToASCII()}}}

\item {} 
\sphinxAtStartPar
\phantomsection\label{\detokenize{reference/peblstring:id8}}{\hyperref[\detokenize{reference/peblstring:uppercase}]{\sphinxcrossref{Uppercase()}}}

\item {} 
\sphinxAtStartPar
\phantomsection\label{\detokenize{reference/peblstring:id9}}{\hyperref[\detokenize{reference/peblstring:detecttextscript}]{\sphinxcrossref{DetectTextScript()}}}

\item {} 
\sphinxAtStartPar
\phantomsection\label{\detokenize{reference/peblstring:id10}}{\hyperref[\detokenize{reference/peblstring:isrtl}]{\sphinxcrossref{IsRTL()}}}

\item {} 
\sphinxAtStartPar
\phantomsection\label{\detokenize{reference/peblstring:id11}}{\hyperref[\detokenize{reference/peblstring:getfontfortext}]{\sphinxcrossref{GetFontForText()}}}

\item {} 
\sphinxAtStartPar
\phantomsection\label{\detokenize{reference/peblstring:id12}}{\hyperref[\detokenize{reference/peblstring:getsystemlocale}]{\sphinxcrossref{GetSystemLocale()}}}

\item {} 
\sphinxAtStartPar
\phantomsection\label{\detokenize{reference/peblstring:id13}}{\hyperref[\detokenize{reference/peblstring:issystemlocalertl}]{\sphinxcrossref{IsSystemLocaleRTL()}}}

\end{itemize}
\end{sphinxShadowBox}

\index{CopyToClipboard@\spxentry{CopyToClipboard}}\ignorespaces 

\subsection{CopyToClipboard()}
\label{\detokenize{reference/peblstring:copytoclipboard}}\label{\detokenize{reference/peblstring:index-0}}
\sphinxAtStartPar
\sphinxstyleemphasis{Puts argument in system clipboard.}

\sphinxAtStartPar
\sphinxstylestrong{Description:}

\sphinxAtStartPar
Puts text into the the system clipboard, so that it can be accessed either by another program or by the \sphinxcode{\sphinxupquote{Copyfromclipboard}} function. Note that, possibly depending on platform, text copied into the clipboard by PEBL may not stay there after PEBL exits.

\sphinxAtStartPar
\sphinxstylestrong{Example:}

\begin{sphinxVerbatim}[commandchars=\\\{\}]
\PYG{n+nv}{text}\PYG{+w}{ }\PYG{o}{\PYGZlt{}\PYGZhy{}}\PYG{+w}{ }\PYG{n+nf}{GetInput}\PYG{p}{(}\PYG{n+nv}{textbox}\PYG{p}{,}\PYG{l+s+s2}{\PYGZdq{}\PYGZlt{}enter\PYGZgt{}\PYGZdq{}}\PYG{p}{)}
\PYG{n+nf}{CopyToClipboard}\PYG{p}{(}\PYG{n+nv}{text}\PYG{p}{)}
\PYG{n+nf}{MessageBox}\PYG{p}{(}\PYG{l+s+s2}{\PYGZdq{}Text : \PYGZdq{}}\PYG{+w}{ }\PYG{o}{+}\PYG{+w}{ }\PYG{n+nv}{text}\PYG{+w}{ }\PYG{o}{+}\PYG{+w}{ }\PYG{l+s+s2}{\PYGZdq{} copied to clipboard\PYGZdq{}}\PYG{p}{,}\PYG{n+nv+vg}{gWin}\PYG{p}{)}
\end{sphinxVerbatim}

\sphinxAtStartPar
\sphinxstylestrong{See Also:}

\sphinxAtStartPar
\sphinxcode{\sphinxupquote{CopyFromClipboard()}}

\index{FindInString@\spxentry{FindInString}}\ignorespaces 

\subsection{FindInString()}
\label{\detokenize{reference/peblstring:findinstring}}\label{\detokenize{reference/peblstring:index-1}}
\sphinxAtStartPar
\sphinxstylestrong{Description:}

\sphinxAtStartPar
Finds a token in a string, returning the position (starting at a particular position).

\sphinxAtStartPar
\sphinxstylestrong{Usage:}

\begin{sphinxVerbatim}[commandchars=\\\{\}]
\PYG{n+nf}{FindInString}\PYG{p}{(}\PYG{o}{\PYGZlt{}}\PYG{n+nv}{basestring}\PYG{o}{\PYGZgt{}}\PYG{p}{,}\PYG{o}{\PYGZlt{}}\PYG{n+nv}{searchstring}\PYG{o}{\PYGZgt{}}\PYG{p}{,}\PYG{o}{\PYGZlt{}}\PYG{n+nv}{startingpos}\PYG{o}{\PYGZgt{}}\PYG{p}{)}
\end{sphinxVerbatim}

\sphinxAtStartPar
\sphinxstylestrong{Example:}

\begin{sphinxVerbatim}[commandchars=\\\{\}]
\PYG{n+nf}{FindInString}\PYG{p}{(}\PYG{l+s+s2}{\PYGZdq{}about\PYGZdq{}}\PYG{p}{,}\PYG{l+s+s2}{\PYGZdq{}bo\PYGZdq{}}\PYG{p}{,}\PYG{l+m+mi}{1}\PYG{p}{)}\PYG{+w}{         }\PYG{c+c1}{\PYGZsh{} == 2}
\PYG{n+nf}{FindInString}\PYG{p}{(}\PYG{l+s+s2}{\PYGZdq{}banana\PYGZdq{}}\PYG{p}{,}\PYG{l+s+s2}{\PYGZdq{}na\PYGZdq{}}\PYG{p}{,}\PYG{l+m+mi}{1}\PYG{p}{)}\PYG{+w}{        }\PYG{c+c1}{\PYGZsh{} == 3}
\PYG{n+nf}{FindInString}\PYG{p}{(}\PYG{l+s+s2}{\PYGZdq{}banana\PYGZdq{}}\PYG{p}{,}\PYG{l+s+s2}{\PYGZdq{}na\PYGZdq{}}\PYG{p}{,}\PYG{l+m+mi}{4}\PYG{p}{)}\PYG{+w}{        }\PYG{c+c1}{\PYGZsh{} == 5}
\end{sphinxVerbatim}

\sphinxAtStartPar
\sphinxstylestrong{See Also:}

\sphinxAtStartPar
\sphinxcode{\sphinxupquote{SplitString()}}

\index{Lowercase@\spxentry{Lowercase}}\ignorespaces 

\subsection{Lowercase()}
\label{\detokenize{reference/peblstring:lowercase}}\label{\detokenize{reference/peblstring:index-2}}
\sphinxAtStartPar
\sphinxstyleemphasis{Returns lowercased string}

\sphinxAtStartPar
\sphinxstylestrong{Description:}

\sphinxAtStartPar
Changes a string to lowercase.  Useful for testing user                 input against a stored value, to ensure case differences                are not detected.

\sphinxAtStartPar
\sphinxstylestrong{Usage:}

\begin{sphinxVerbatim}[commandchars=\\\{\}]
\PYG{n+nf}{Lowercase}\PYG{p}{(}\PYG{o}{\PYGZlt{}}\PYG{n+nv}{string}\PYG{o}{\PYGZgt{}}\PYG{p}{)}
\end{sphinxVerbatim}

\sphinxAtStartPar
\sphinxstylestrong{Example:}

\begin{sphinxVerbatim}[commandchars=\\\{\}]
\PYG{n+nf}{Lowercase}\PYG{p}{(}\PYG{l+s+s2}{\PYGZdq{}POtaTo\PYGZdq{}}\PYG{p}{)}\PYG{+w}{  }\PYG{c+c1}{\PYGZsh{} == \PYGZdq{}potato\PYGZdq{}}
\end{sphinxVerbatim}

\sphinxAtStartPar
\sphinxstylestrong{See Also:}

\sphinxAtStartPar
\sphinxcode{\sphinxupquote{Uppercase()}}

\index{SplitString@\spxentry{SplitString}}\ignorespaces 

\subsection{SplitString()}
\label{\detokenize{reference/peblstring:splitstring}}\label{\detokenize{reference/peblstring:index-3}}
\sphinxAtStartPar
\sphinxstylestrong{Description:}

\sphinxAtStartPar
Splits a string into tokens. \sphinxcode{\sphinxupquote{\textless{}split\textgreater{}}} must be a string. If           \sphinxcode{\sphinxupquote{\textless{}split\textgreater{}}} is not found in \sphinxcode{\sphinxupquote{\textless{}string\textgreater{}}}, a list containing the entire                  string is returned; if split is equal to \sphinxcode{\sphinxupquote{""}}, the each letter                in the string is placed into a different item in the list.  Only the first character of \sphinxcode{\sphinxupquote{\textless{}split\textgreater{}}} is used.  IF you need a multicharacter split, you can use \sphinxcode{\sphinxupquote{\textless{}SplitStringSlow\textgreater{}}}, which can handle multi\sphinxhyphen{}character splits but is relatively slower. This should not matter for short strings, but if you are using splitstring on long files, it could make a difference.

\sphinxAtStartPar
\sphinxstylestrong{Usage:}

\begin{sphinxVerbatim}[commandchars=\\\{\}]
\PYG{n+nf}{SplitString}\PYG{p}{(}\PYG{o}{\PYGZlt{}}\PYG{n+nv}{string}\PYG{o}{\PYGZgt{}}\PYG{p}{,}\PYG{+w}{ }\PYG{o}{\PYGZlt{}}\PYG{n+nv}{split}\PYG{o}{\PYGZgt{}}\PYG{p}{)}
\end{sphinxVerbatim}

\sphinxAtStartPar
\sphinxstylestrong{Example:}

\begin{sphinxVerbatim}[commandchars=\\\{\}]
\PYG{n+nf}{SplitString}\PYG{p}{(}\PYG{l+s+s2}{\PYGZdq{}Everybody Loves a Clown\PYGZdq{}}\PYG{p}{,}\PYG{+w}{ }\PYG{l+s+s2}{\PYGZdq{} \PYGZdq{}}\PYG{p}{)}
\PYG{c+c1}{\PYGZsh{} Produces [\PYGZdq{}Everybody\PYGZdq{}, \PYGZdq{}Loves\PYGZdq{}, \PYGZdq{}a\PYGZdq{}, \PYGZdq{}Clown\PYGZdq{}]}
\end{sphinxVerbatim}

\sphinxAtStartPar
\sphinxstylestrong{See Also:}

\sphinxAtStartPar
\sphinxcode{\sphinxupquote{FindInString()}}, \sphinxcode{\sphinxupquote{ReplaceChar()}}, \sphinxcode{\sphinxupquote{SplitStringSlow()}}

\index{StringLength@\spxentry{StringLength}}\ignorespaces 

\subsection{StringLength()}
\label{\detokenize{reference/peblstring:stringlength}}\label{\detokenize{reference/peblstring:index-4}}
\sphinxAtStartPar
\sphinxstyleemphasis{Returns the length of a string}

\sphinxAtStartPar
\sphinxstylestrong{Description:}

\sphinxAtStartPar
Determines the length of a string, in characters.

\sphinxAtStartPar
\sphinxstylestrong{Usage:}

\begin{sphinxVerbatim}[commandchars=\\\{\}]
\PYG{n+nf}{StringLength}\PYG{p}{(}\PYG{o}{\PYGZlt{}}\PYG{n+nv}{string}\PYG{o}{\PYGZgt{}}\PYG{p}{)}
\end{sphinxVerbatim}

\sphinxAtStartPar
\sphinxstylestrong{Example:}

\begin{sphinxVerbatim}[commandchars=\\\{\}]
\PYG{n+nf}{StringLength}\PYG{p}{(}\PYG{l+s+s2}{\PYGZdq{}absolute\PYGZdq{}}\PYG{p}{)}\PYG{+w}{     }\PYG{c+c1}{\PYGZsh{} == 8}
\PYG{n+nf}{StringLength}\PYG{p}{(}\PYG{l+s+s2}{\PYGZdq{}   spaces   \PYGZdq{}}\PYG{p}{)}\PYG{+w}{ }\PYG{c+c1}{\PYGZsh{} == 12}
\PYG{n+nf}{StringLength}\PYG{p}{(}\PYG{l+s+s2}{\PYGZdq{}\PYGZdq{}}\PYG{p}{)}\PYG{+w}{             }\PYG{c+c1}{\PYGZsh{} == 0}
\end{sphinxVerbatim}

\sphinxAtStartPar
\sphinxstylestrong{See Also:}

\sphinxAtStartPar
\sphinxcode{\sphinxupquote{Length()}}, \sphinxcode{\sphinxupquote{SubString()}}

\index{SubString@\spxentry{SubString}}\ignorespaces 

\subsection{SubString()}
\label{\detokenize{reference/peblstring:substring}}\label{\detokenize{reference/peblstring:index-5}}
\sphinxAtStartPar
\sphinxstyleemphasis{Returns a substring}

\sphinxAtStartPar
\sphinxstylestrong{Description:}

\sphinxAtStartPar
Extracts a substring from a longer string.

\sphinxAtStartPar
\sphinxstylestrong{Usage:}

\begin{sphinxVerbatim}[commandchars=\\\{\}]
\PYG{n+nf}{SubString}\PYG{p}{(}\PYG{o}{\PYGZlt{}}\PYG{n+nv}{string}\PYG{o}{\PYGZgt{}}\PYG{p}{,}\PYG{o}{\PYGZlt{}}\PYG{n+nv}{position}\PYG{o}{\PYGZgt{}}\PYG{p}{,}\PYG{o}{\PYGZlt{}}\PYG{n+nv}{length}\PYG{o}{\PYGZgt{}}\PYG{p}{)}
\end{sphinxVerbatim}

\sphinxAtStartPar
\sphinxstylestrong{Example:}

\begin{sphinxVerbatim}[commandchars=\\\{\}]
\PYG{n+nf}{SubString}\PYG{p}{(}\PYG{l+s+s2}{\PYGZdq{}abcdefghijklmnop\PYGZdq{}}\PYG{p}{,}\PYG{l+m+mi}{3}\PYG{p}{,}\PYG{l+m+mi}{5}\PYG{p}{)}\PYG{+w}{    }\PYG{c+c1}{\PYGZsh{} == \PYGZdq{}cdefg\PYGZdq{}}
\end{sphinxVerbatim}

\sphinxAtStartPar
\sphinxstylestrong{See Also:}

\sphinxAtStartPar
\sphinxcode{\sphinxupquote{StringLength()}}, \sphinxcode{\sphinxupquote{FindInString()}}

\index{ToASCII@\spxentry{ToASCII}}\ignorespaces 

\subsection{ToASCII()}
\label{\detokenize{reference/peblstring:toascii}}\label{\detokenize{reference/peblstring:index-6}}
\sphinxAtStartPar
\sphinxstyleemphasis{Converts an ASCII code to a character}

\sphinxAtStartPar
\sphinxstylestrong{Description:}

\sphinxAtStartPar
Converts an integer ASCII code to its corresponding single\sphinxhyphen{}character string. This is useful for creating special characters or control characters from their numeric codes.

\sphinxAtStartPar
\sphinxstylestrong{Usage:}

\begin{sphinxVerbatim}[commandchars=\\\{\}]
\PYG{n+nf}{ToASCII}\PYG{p}{(}\PYG{o}{\PYGZlt{}}\PYG{n+nv}{ascii\PYGZus{}code}\PYG{o}{\PYGZgt{}}\PYG{p}{)}
\end{sphinxVerbatim}

\sphinxAtStartPar
\sphinxstylestrong{Example:}

\begin{sphinxVerbatim}[commandchars=\\\{\}]
\PYG{n+nf}{ToASCII}\PYG{p}{(}\PYG{l+m+mi}{65}\PYG{p}{)}\PYG{+w}{    }\PYG{c+c1}{\PYGZsh{} == \PYGZdq{}A\PYGZdq{}}
\PYG{n+nf}{ToASCII}\PYG{p}{(}\PYG{l+m+mi}{97}\PYG{p}{)}\PYG{+w}{    }\PYG{c+c1}{\PYGZsh{} == \PYGZdq{}a\PYGZdq{}}
\PYG{n+nf}{ToASCII}\PYG{p}{(}\PYG{l+m+mi}{32}\PYG{p}{)}\PYG{+w}{    }\PYG{c+c1}{\PYGZsh{} == \PYGZdq{} \PYGZdq{} (space)}
\PYG{n+nf}{ToASCII}\PYG{p}{(}\PYG{l+m+mi}{10}\PYG{p}{)}\PYG{+w}{    }\PYG{c+c1}{\PYGZsh{} == newline character}
\end{sphinxVerbatim}

\sphinxAtStartPar
\sphinxstylestrong{See Also:}

\sphinxAtStartPar
\sphinxcode{\sphinxupquote{StringLength()}}, \sphinxcode{\sphinxupquote{Uppercase()}}, \sphinxcode{\sphinxupquote{Lowercase()}}

\index{Uppercase@\spxentry{Uppercase}}\ignorespaces 

\subsection{Uppercase()}
\label{\detokenize{reference/peblstring:uppercase}}\label{\detokenize{reference/peblstring:index-7}}
\sphinxAtStartPar
\sphinxstyleemphasis{Returns uppercased string}

\sphinxAtStartPar
\sphinxstylestrong{Description:}

\sphinxAtStartPar
Changes a string to uppercase.  Useful for testing user                 input against a stored value, to ensure case differences                are not detected.

\sphinxAtStartPar
\sphinxstylestrong{Usage:}

\begin{sphinxVerbatim}[commandchars=\\\{\}]
\PYG{n+nf}{Uppercase}\PYG{p}{(}\PYG{o}{\PYGZlt{}}\PYG{n+nv}{string}\PYG{o}{\PYGZgt{}}\PYG{p}{)}
\end{sphinxVerbatim}

\sphinxAtStartPar
\sphinxstylestrong{Example:}

\begin{sphinxVerbatim}[commandchars=\\\{\}]
\PYG{n+nf}{Uppercase}\PYG{p}{(}\PYG{l+s+s2}{\PYGZdq{}POtaTo\PYGZdq{}}\PYG{p}{)}\PYG{+w}{  }\PYG{c+c1}{\PYGZsh{} == \PYGZdq{}POTATO\PYGZdq{}}
\end{sphinxVerbatim}

\sphinxAtStartPar
\sphinxstylestrong{See Also:}

\sphinxAtStartPar
\sphinxcode{\sphinxupquote{Lowercase()}}

\index{DetectTextScript@\spxentry{DetectTextScript}}\ignorespaces 

\subsection{DetectTextScript()}
\label{\detokenize{reference/peblstring:detecttextscript}}\label{\detokenize{reference/peblstring:index-8}}
\sphinxAtStartPar
\sphinxstyleemphasis{Detects the Unicode script of text}

\sphinxAtStartPar
\sphinxstylestrong{Description:}

\sphinxAtStartPar
Analyzes text and returns the ISO 15924 four\sphinxhyphen{}letter script code identifying the writing system used. Returns codes such as “Arab” (Arabic), “Hebr” (Hebrew), “Hani” (Han/Chinese), “Thai” (Thai), “Cyrl” (Cyrillic), etc. Returns an empty string for Latin script or when the script cannot be determined. This is useful for automatic font selection, determining text directionality, and providing appropriate text rendering for international content.

\sphinxAtStartPar
\sphinxstylestrong{Usage:}

\begin{sphinxVerbatim}[commandchars=\\\{\}]
\PYG{n+nf}{DetectTextScript}\PYG{p}{(}\PYG{o}{\PYGZlt{}}\PYG{n+nv}{text}\PYG{o}{\PYGZgt{}}\PYG{p}{)}
\end{sphinxVerbatim}

\sphinxAtStartPar
\sphinxstylestrong{Example:}

\begin{sphinxVerbatim}[commandchars=\\\{\}]
\PYG{c+c1}{\PYGZsh{}\PYGZsh{} Detect scripts in different languages}
\PYG{n+nf}{Print}\PYG{p}{(}\PYG{n+nf}{DetectTextScript}\PYG{p}{(}\PYG{l+s+s2}{\PYGZdq{}Hello\PYGZdq{}}\PYG{p}{)}\PYG{p}{)}\PYG{+w}{           }\PYG{c+c1}{\PYGZsh{}\PYGZsh{} \PYGZdq{}\PYGZdq{} (Latin)}
\PYG{c+c1}{\PYGZsh{}\PYGZsh{} Arabic text \PYGZdq{}marhaba\PYGZdq{} (hello):}
\PYG{n+nf}{Print}\PYG{p}{(}\PYG{n+nf}{DetectTextScript}\PYG{p}{(}\PYG{n+nv}{arabicText}\PYG{p}{)}\PYG{p}{)}\PYG{+w}{        }\PYG{c+c1}{\PYGZsh{}\PYGZsh{} \PYGZdq{}Arab\PYGZdq{} (Arabic)}
\PYG{c+c1}{\PYGZsh{}\PYGZsh{} Hebrew text \PYGZdq{}shalom\PYGZdq{} (hello):}
\PYG{n+nf}{Print}\PYG{p}{(}\PYG{n+nf}{DetectTextScript}\PYG{p}{(}\PYG{n+nv}{hebrewText}\PYG{p}{)}\PYG{p}{)}\PYG{+w}{        }\PYG{c+c1}{\PYGZsh{}\PYGZsh{} \PYGZdq{}Hebr\PYGZdq{} (Hebrew)}
\PYG{c+c1}{\PYGZsh{}\PYGZsh{} Chinese text \PYGZdq{}nihao\PYGZdq{} (hello):}
\PYG{n+nf}{Print}\PYG{p}{(}\PYG{n+nf}{DetectTextScript}\PYG{p}{(}\PYG{n+nv}{chineseText}\PYG{p}{)}\PYG{p}{)}\PYG{+w}{       }\PYG{c+c1}{\PYGZsh{}\PYGZsh{} \PYGZdq{}Hani\PYGZdq{} (Chinese)}
\PYG{c+c1}{\PYGZsh{}\PYGZsh{} Thai text \PYGZdq{}sawasdee\PYGZdq{} (hello):}
\PYG{n+nf}{Print}\PYG{p}{(}\PYG{n+nf}{DetectTextScript}\PYG{p}{(}\PYG{n+nv}{thaiText}\PYG{p}{)}\PYG{p}{)}\PYG{+w}{          }\PYG{c+c1}{\PYGZsh{}\PYGZsh{} \PYGZdq{}Thai\PYGZdq{} (Thai)}
\PYG{c+c1}{\PYGZsh{}\PYGZsh{} Cyrillic text \PYGZdq{}privet\PYGZdq{} (hello):}
\PYG{n+nf}{Print}\PYG{p}{(}\PYG{n+nf}{DetectTextScript}\PYG{p}{(}\PYG{n+nv}{cyrillicText}\PYG{p}{)}\PYG{p}{)}\PYG{+w}{      }\PYG{c+c1}{\PYGZsh{}\PYGZsh{} \PYGZdq{}Cyrl\PYGZdq{} (Cyrillic)}
\end{sphinxVerbatim}

\sphinxAtStartPar
\sphinxstylestrong{See Also:}

\sphinxAtStartPar
\sphinxcode{\sphinxupquote{IsRTL()}}, \sphinxcode{\sphinxupquote{GetFontForText()}}, \sphinxcode{\sphinxupquote{GetSystemLocale()}}

\index{IsRTL@\spxentry{IsRTL}}\ignorespaces 

\subsection{IsRTL()}
\label{\detokenize{reference/peblstring:isrtl}}\label{\detokenize{reference/peblstring:index-9}}
\sphinxAtStartPar
\sphinxstyleemphasis{Determines if text or script is right\sphinxhyphen{}to\sphinxhyphen{}left}

\sphinxAtStartPar
\sphinxstylestrong{Description:}

\sphinxAtStartPar
Tests whether the input is right\sphinxhyphen{}to\sphinxhyphen{}left (RTL) text. The function accepts either actual text (which will be analyzed to detect its script) or a four\sphinxhyphen{}letter ISO 15924 script code. Returns 1 for RTL scripts (Arabic, Hebrew, etc.) and 0 for LTR scripts. This is essential for proper text layout, UI mirroring, and ensuring correct text directionality in international experiments.

\sphinxAtStartPar
\sphinxstylestrong{Usage:}

\begin{sphinxVerbatim}[commandchars=\\\{\}]
\PYG{n+nf}{IsRTL}\PYG{p}{(}\PYG{o}{\PYGZlt{}}\PYG{n+nv}{text\PYGZus{}or\PYGZus{}script\PYGZus{}code}\PYG{o}{\PYGZgt{}}\PYG{p}{)}
\end{sphinxVerbatim}

\sphinxAtStartPar
\sphinxstylestrong{Example:}

\begin{sphinxVerbatim}[commandchars=\\\{\}]
\PYG{c+c1}{\PYGZsh{}\PYGZsh{} Test with actual text}
\PYG{n+nf}{Print}\PYG{p}{(}\PYG{n+nf}{IsRTL}\PYG{p}{(}\PYG{l+s+s2}{\PYGZdq{}Hello\PYGZdq{}}\PYG{p}{)}\PYG{p}{)}\PYG{+w}{                }\PYG{c+c1}{\PYGZsh{}\PYGZsh{} 0 (LTR)}
\PYG{c+c1}{\PYGZsh{}\PYGZsh{} Arabic text \PYGZdq{}marhaba\PYGZdq{}:}
\PYG{n+nf}{Print}\PYG{p}{(}\PYG{n+nf}{IsRTL}\PYG{p}{(}\PYG{n+nv}{arabicText}\PYG{p}{)}\PYG{p}{)}\PYG{+w}{             }\PYG{c+c1}{\PYGZsh{}\PYGZsh{} 1 (RTL \PYGZhy{} Arabic)}
\PYG{c+c1}{\PYGZsh{}\PYGZsh{} Hebrew text \PYGZdq{}shalom\PYGZdq{}:}
\PYG{n+nf}{Print}\PYG{p}{(}\PYG{n+nf}{IsRTL}\PYG{p}{(}\PYG{n+nv}{hebrewText}\PYG{p}{)}\PYG{p}{)}\PYG{+w}{             }\PYG{c+c1}{\PYGZsh{}\PYGZsh{} 1 (RTL \PYGZhy{} Hebrew)}

\PYG{c+c1}{\PYGZsh{}\PYGZsh{} Test with script codes}
\PYG{n+nf}{Print}\PYG{p}{(}\PYG{n+nf}{IsRTL}\PYG{p}{(}\PYG{l+s+s2}{\PYGZdq{}Arab\PYGZdq{}}\PYG{p}{)}\PYG{p}{)}\PYG{+w}{                 }\PYG{c+c1}{\PYGZsh{}\PYGZsh{} 1 (Arabic script is RTL)}
\PYG{n+nf}{Print}\PYG{p}{(}\PYG{n+nf}{IsRTL}\PYG{p}{(}\PYG{l+s+s2}{\PYGZdq{}Hebr\PYGZdq{}}\PYG{p}{)}\PYG{p}{)}\PYG{+w}{                 }\PYG{c+c1}{\PYGZsh{}\PYGZsh{} 1 (Hebrew script is RTL)}
\PYG{n+nf}{Print}\PYG{p}{(}\PYG{n+nf}{IsRTL}\PYG{p}{(}\PYG{l+s+s2}{\PYGZdq{}Latn\PYGZdq{}}\PYG{p}{)}\PYG{p}{)}\PYG{+w}{                 }\PYG{c+c1}{\PYGZsh{}\PYGZsh{} 0 (Latin script is LTR)}

\PYG{c+c1}{\PYGZsh{}\PYGZsh{} Use for text justification}
\PYG{n+nv}{text}\PYG{+w}{ }\PYG{o}{\PYGZlt{}\PYGZhy{}}\PYG{+w}{ }\PYG{n+nf}{GetInput}\PYG{p}{(}\PYG{n+nv}{textbox}\PYG{p}{,}\PYG{+w}{ }\PYG{l+s+s2}{\PYGZdq{}\PYGZlt{}return\PYGZgt{}\PYGZdq{}}\PYG{p}{)}
\PYG{k}{if}\PYG{p}{(}\PYG{n+nf}{IsRTL}\PYG{p}{(}\PYG{n+nv}{text}\PYG{p}{)}\PYG{p}{)}
\PYG{p}{\PYGZob{}}
\PYG{+w}{    }\PYG{n+nv}{textbox.hjustify}\PYG{+w}{ }\PYG{o}{\PYGZlt{}\PYGZhy{}}\PYG{+w}{ }\PYG{l+s+s2}{\PYGZdq{}right\PYGZdq{}}
\PYG{p}{\PYGZcb{}}
\end{sphinxVerbatim}

\sphinxAtStartPar
\sphinxstylestrong{See Also:}

\sphinxAtStartPar
\sphinxcode{\sphinxupquote{DetectTextScript()}}, \sphinxcode{\sphinxupquote{IsSystemLocaleRTL()}}, \sphinxcode{\sphinxupquote{GetFontForText()}}

\index{GetFontForText@\spxentry{GetFontForText}}\ignorespaces 

\subsection{GetFontForText()}
\label{\detokenize{reference/peblstring:getfontfortext}}\label{\detokenize{reference/peblstring:index-10}}
\sphinxAtStartPar
\sphinxstyleemphasis{Returns appropriate font filename for text based on detected script}

\sphinxAtStartPar
\sphinxstylestrong{Description:}

\sphinxAtStartPar
Automatically selects an appropriate font for the given text by detecting its Unicode script. This ensures proper rendering of international text by choosing fonts that support the necessary character ranges. The optional font\_type parameter specifies the font style: 0 for sans\sphinxhyphen{}serif (default), 1 for monospace, or 2 for serif. Returns a font filename suitable for use with \sphinxcode{\sphinxupquote{MakeFont()}}. The function uses the DejaVu and Noto font families which provide extensive Unicode coverage.

\sphinxAtStartPar
\sphinxstylestrong{Usage:}

\begin{sphinxVerbatim}[commandchars=\\\{\}]
\PYG{n+nf}{GetFontForText}\PYG{p}{(}\PYG{o}{\PYGZlt{}}\PYG{n+nv}{text}\PYG{o}{\PYGZgt{}}\PYG{p}{)}
\PYG{n+nf}{GetFontForText}\PYG{p}{(}\PYG{o}{\PYGZlt{}}\PYG{n+nv}{text}\PYG{o}{\PYGZgt{}}\PYG{p}{,}\PYG{+w}{ }\PYG{o}{\PYGZlt{}}\PYG{n+nv}{font\PYGZus{}type}\PYG{o}{\PYGZgt{}}\PYG{p}{)}
\end{sphinxVerbatim}

\sphinxAtStartPar
\sphinxstylestrong{Example:}

\begin{sphinxVerbatim}[commandchars=\\\{\}]
\PYG{c+c1}{\PYGZsh{}\PYGZsh{} Automatic font selection for different languages}
\PYG{c+c1}{\PYGZsh{}\PYGZsh{} arabicText would contain Arabic text \PYGZdq{}marhaba bik\PYGZdq{}}
\PYG{n+nv}{fontFile}\PYG{+w}{ }\PYG{o}{\PYGZlt{}\PYGZhy{}}\PYG{+w}{ }\PYG{n+nf}{GetFontForText}\PYG{p}{(}\PYG{n+nv}{arabicText}\PYG{p}{)}
\PYG{n+nv}{font}\PYG{+w}{ }\PYG{o}{\PYGZlt{}\PYGZhy{}}\PYG{+w}{ }\PYG{n+nf}{MakeFont}\PYG{p}{(}\PYG{n+nv}{fontFile}\PYG{p}{,}\PYG{+w}{ }\PYG{l+m+mi}{0}\PYG{p}{,}\PYG{+w}{ }\PYG{l+m+mi}{24}\PYG{p}{,}\PYG{+w}{ }\PYG{n+nf}{MakeColor}\PYG{p}{(}\PYG{l+s+s2}{\PYGZdq{}black\PYGZdq{}}\PYG{p}{)}\PYG{p}{,}\PYG{+w}{ }\PYG{n+nf}{MakeColor}\PYG{p}{(}\PYG{l+s+s2}{\PYGZdq{}white\PYGZdq{}}\PYG{p}{)}\PYG{p}{,}\PYG{+w}{ }\PYG{l+m+mi}{1}\PYG{p}{)}
\PYG{n+nv}{label}\PYG{+w}{ }\PYG{o}{\PYGZlt{}\PYGZhy{}}\PYG{+w}{ }\PYG{n+nf}{MakeLabel}\PYG{p}{(}\PYG{n+nv}{arabicText}\PYG{p}{,}\PYG{+w}{ }\PYG{n+nv}{font}\PYG{p}{)}

\PYG{c+c1}{\PYGZsh{}\PYGZsh{} Select monospace font for code display}
\PYG{c+c1}{\PYGZsh{}\PYGZsh{} hebrewCode would contain Hebrew text}
\PYG{n+nv}{monoFont}\PYG{+w}{ }\PYG{o}{\PYGZlt{}\PYGZhy{}}\PYG{+w}{ }\PYG{n+nf}{GetFontForText}\PYG{p}{(}\PYG{n+nv}{hebrewCode}\PYG{p}{,}\PYG{+w}{ }\PYG{l+m+mi}{1}\PYG{p}{)}\PYG{+w}{  }\PYG{c+c1}{\PYGZsh{}\PYGZsh{} 1 = monospace}

\PYG{c+c1}{\PYGZsh{}\PYGZsh{} Select serif font for Thai text}
\PYG{c+c1}{\PYGZsh{}\PYGZsh{} thaiText would contain Thai text \PYGZdq{}sawasdee\PYGZdq{}}
\PYG{n+nv}{serifFont}\PYG{+w}{ }\PYG{o}{\PYGZlt{}\PYGZhy{}}\PYG{+w}{ }\PYG{n+nf}{GetFontForText}\PYG{p}{(}\PYG{n+nv}{thaiText}\PYG{p}{,}\PYG{+w}{ }\PYG{l+m+mi}{2}\PYG{p}{)}\PYG{+w}{   }\PYG{c+c1}{\PYGZsh{}\PYGZsh{} 2 = serif}
\end{sphinxVerbatim}

\sphinxAtStartPar
\sphinxstylestrong{See Also:}

\sphinxAtStartPar
\sphinxcode{\sphinxupquote{DetectTextScript()}}, \sphinxcode{\sphinxupquote{MakeFont()}}, \sphinxcode{\sphinxupquote{IsRTL()}}

\index{GetSystemLocale@\spxentry{GetSystemLocale}}\ignorespaces 

\subsection{GetSystemLocale()}
\label{\detokenize{reference/peblstring:getsystemlocale}}\label{\detokenize{reference/peblstring:index-11}}
\sphinxAtStartPar
\sphinxstyleemphasis{Retrieves the operating system’s locale setting}

\sphinxAtStartPar
\sphinxstylestrong{Description:}

\sphinxAtStartPar
Returns the current system locale as configured in the operating system. The locale string typically follows the format “language\_COUNTRY” (e.g., “en\_US”, “zh\_CN”, “ar\_SA”) or may be just a language code (e.g., “ar”, “he”). Returns an empty string if locale detection fails. This is useful for automatically adapting experiment interfaces to the user’s language and regional settings.

\sphinxAtStartPar
\sphinxstylestrong{Usage:}

\begin{sphinxVerbatim}[commandchars=\\\{\}]
\PYG{n+nf}{GetSystemLocale}\PYG{p}{(}\PYG{p}{)}
\end{sphinxVerbatim}

\sphinxAtStartPar
\sphinxstylestrong{Example:}

\begin{sphinxVerbatim}[commandchars=\\\{\}]
\PYG{c+c1}{\PYGZsh{}\PYGZsh{} Detect system locale and adapt interface}
\PYG{n+nv}{locale}\PYG{+w}{ }\PYG{o}{\PYGZlt{}\PYGZhy{}}\PYG{+w}{ }\PYG{n+nf}{GetSystemLocale}\PYG{p}{(}\PYG{p}{)}
\PYG{n+nf}{Print}\PYG{p}{(}\PYG{l+s+s2}{\PYGZdq{}System locale: \PYGZdq{}}\PYG{+w}{ }\PYG{o}{+}\PYG{+w}{ }\PYG{n+nv}{locale}\PYG{p}{)}

\PYG{k}{if}\PYG{p}{(}\PYG{n+nf}{SubString}\PYG{p}{(}\PYG{n+nv}{locale}\PYG{p}{,}\PYG{+w}{ }\PYG{l+m+mi}{1}\PYG{p}{,}\PYG{+w}{ }\PYG{l+m+mi}{2}\PYG{p}{)}\PYG{+w}{ }\PYG{o}{==}\PYG{+w}{ }\PYG{l+s+s2}{\PYGZdq{}ar\PYGZdq{}}\PYG{p}{)}
\PYG{p}{\PYGZob{}}
\PYG{+w}{    }\PYG{c+c1}{\PYGZsh{}\PYGZsh{} Arabic locale detected}
\PYG{+w}{    }\PYG{n+nv+vg}{gLanguage}\PYG{+w}{ }\PYG{o}{\PYGZlt{}\PYGZhy{}}\PYG{+w}{ }\PYG{l+s+s2}{\PYGZdq{}ar\PYGZdq{}}
\PYG{p}{\PYGZcb{}}\PYG{+w}{ }\PYG{k}{elseif}\PYG{p}{(}\PYG{n+nf}{SubString}\PYG{p}{(}\PYG{n+nv}{locale}\PYG{p}{,}\PYG{+w}{ }\PYG{l+m+mi}{1}\PYG{p}{,}\PYG{+w}{ }\PYG{l+m+mi}{2}\PYG{p}{)}\PYG{+w}{ }\PYG{o}{==}\PYG{+w}{ }\PYG{l+s+s2}{\PYGZdq{}he\PYGZdq{}}\PYG{p}{)}
\PYG{p}{\PYGZob{}}
\PYG{+w}{    }\PYG{c+c1}{\PYGZsh{}\PYGZsh{} Hebrew locale detected}
\PYG{+w}{    }\PYG{n+nv+vg}{gLanguage}\PYG{+w}{ }\PYG{o}{\PYGZlt{}\PYGZhy{}}\PYG{+w}{ }\PYG{l+s+s2}{\PYGZdq{}he\PYGZdq{}}
\PYG{p}{\PYGZcb{}}\PYG{+w}{ }\PYG{k}{else}\PYG{+w}{ }\PYG{p}{\PYGZob{}}
\PYG{+w}{    }\PYG{c+c1}{\PYGZsh{}\PYGZsh{} Default to English}
\PYG{+w}{    }\PYG{n+nv+vg}{gLanguage}\PYG{+w}{ }\PYG{o}{\PYGZlt{}\PYGZhy{}}\PYG{+w}{ }\PYG{l+s+s2}{\PYGZdq{}en\PYGZdq{}}
\PYG{p}{\PYGZcb{}}

\PYG{c+c1}{\PYGZsh{}\PYGZsh{} Possible outputs: \PYGZdq{}en\PYGZus{}US\PYGZdq{}, \PYGZdq{}ar\PYGZus{}SA\PYGZdq{}, \PYGZdq{}he\PYGZus{}IL\PYGZdq{}, \PYGZdq{}zh\PYGZus{}CN\PYGZdq{}, \PYGZdq{}es\PYGZus{}MX\PYGZdq{}, etc.}
\end{sphinxVerbatim}

\sphinxAtStartPar
\sphinxstylestrong{See Also:}

\sphinxAtStartPar
\sphinxcode{\sphinxupquote{IsSystemLocaleRTL()}}, \sphinxcode{\sphinxupquote{DetectTextScript()}}, \sphinxcode{\sphinxupquote{GetFontForText()}}

\index{IsSystemLocaleRTL@\spxentry{IsSystemLocaleRTL}}\ignorespaces 

\subsection{IsSystemLocaleRTL()}
\label{\detokenize{reference/peblstring:issystemlocalertl}}\label{\detokenize{reference/peblstring:index-12}}
\sphinxAtStartPar
\sphinxstyleemphasis{Checks if the system locale uses right\sphinxhyphen{}to\sphinxhyphen{}left text}

\sphinxAtStartPar
\sphinxstylestrong{Description:}

\sphinxAtStartPar
Determines whether the operating system’s current locale setting is for a right\sphinxhyphen{}to\sphinxhyphen{}left (RTL) language (Arabic or Hebrew). Returns 1 if the system locale is RTL, 0 if it is LTR. This is particularly useful for setting default text justification and UI layout before any text input occurs, ensuring that the interface matches the user’s language expectations.

\sphinxAtStartPar
\sphinxstylestrong{Usage:}

\begin{sphinxVerbatim}[commandchars=\\\{\}]
\PYG{n+nf}{IsSystemLocaleRTL}\PYG{p}{(}\PYG{p}{)}
\end{sphinxVerbatim}

\sphinxAtStartPar
\sphinxstylestrong{Example:}

\begin{sphinxVerbatim}[commandchars=\\\{\}]
\PYG{c+c1}{\PYGZsh{}\PYGZsh{} Set default text justification based on system locale}
\PYG{k}{if}\PYG{p}{(}\PYG{n+nf}{IsSystemLocaleRTL}\PYG{p}{(}\PYG{p}{)}\PYG{p}{)}
\PYG{p}{\PYGZob{}}
\PYG{+w}{    }\PYG{c+c1}{\PYGZsh{}\PYGZsh{} System is configured for RTL language (Arabic/Hebrew)}
\PYG{+w}{    }\PYG{n+nv}{defaultJustify}\PYG{+w}{ }\PYG{o}{\PYGZlt{}\PYGZhy{}}\PYG{+w}{ }\PYG{l+s+s2}{\PYGZdq{}right\PYGZdq{}}
\PYG{+w}{    }\PYG{n+nf}{Print}\PYG{p}{(}\PYG{l+s+s2}{\PYGZdq{}RTL locale detected\PYGZdq{}}\PYG{p}{)}
\PYG{p}{\PYGZcb{}}\PYG{+w}{ }\PYG{k}{else}\PYG{+w}{ }\PYG{p}{\PYGZob{}}
\PYG{+w}{    }\PYG{c+c1}{\PYGZsh{}\PYGZsh{} System is configured for LTR language}
\PYG{+w}{    }\PYG{n+nv}{defaultJustify}\PYG{+w}{ }\PYG{o}{\PYGZlt{}\PYGZhy{}}\PYG{+w}{ }\PYG{l+s+s2}{\PYGZdq{}left\PYGZdq{}}
\PYG{+w}{    }\PYG{n+nf}{Print}\PYG{p}{(}\PYG{l+s+s2}{\PYGZdq{}LTR locale detected\PYGZdq{}}\PYG{p}{)}
\PYG{p}{\PYGZcb{}}

\PYG{c+c1}{\PYGZsh{}\PYGZsh{} Create textbox with appropriate default alignment}
\PYG{n+nv}{textbox}\PYG{+w}{ }\PYG{o}{\PYGZlt{}\PYGZhy{}}\PYG{+w}{ }\PYG{n+nf}{MakeTextBox}\PYG{p}{(}\PYG{l+s+s2}{\PYGZdq{}\PYGZdq{}}\PYG{p}{,}\PYG{+w}{ }\PYG{n+nv}{font}\PYG{p}{,}\PYG{+w}{ }\PYG{l+m+mi}{400}\PYG{p}{,}\PYG{+w}{ }\PYG{l+m+mi}{100}\PYG{p}{)}
\PYG{n+nv}{textbox.hjustify}\PYG{+w}{ }\PYG{o}{\PYGZlt{}\PYGZhy{}}\PYG{+w}{ }\PYG{n+nv}{defaultJustify}
\end{sphinxVerbatim}

\sphinxAtStartPar
\sphinxstylestrong{See Also:}

\sphinxAtStartPar
\sphinxcode{\sphinxupquote{GetSystemLocale()}}, \sphinxcode{\sphinxupquote{IsRTL()}}, \sphinxcode{\sphinxupquote{DetectTextScript()}}

\sphinxstepscope


\section{Design Library \sphinxhyphen{} Experimental Design}
\label{\detokenize{reference/design:design-library-experimental-design}}\label{\detokenize{reference/design::doc}}
\sphinxAtStartPar
This library contains functions for experimental design, including Latin squares, counterbalancing, and design matrices.

\begin{sphinxShadowBox}
\sphinxstyletopictitle{Function Index}
\begin{itemize}
\item {} 
\sphinxAtStartPar
\phantomsection\label{\detokenize{reference/design:id3}}{\hyperref[\detokenize{reference/design:choosen}]{\sphinxcrossref{ChooseN()}}}

\item {} 
\sphinxAtStartPar
\phantomsection\label{\detokenize{reference/design:id4}}{\hyperref[\detokenize{reference/design:designbalancedsampling}]{\sphinxcrossref{DesignBalancedSampling()}}}

\item {} 
\sphinxAtStartPar
\phantomsection\label{\detokenize{reference/design:id5}}{\hyperref[\detokenize{reference/design:designgrecolatinsquare}]{\sphinxcrossref{DesignGrecoLatinSquare()}}}

\item {} 
\sphinxAtStartPar
\phantomsection\label{\detokenize{reference/design:id6}}{\hyperref[\detokenize{reference/design:designlatinsquare}]{\sphinxcrossref{DesignLatinSquare()}}}

\item {} 
\sphinxAtStartPar
\phantomsection\label{\detokenize{reference/design:id7}}{\hyperref[\detokenize{reference/design:extractlistitems}]{\sphinxcrossref{ExtractListItems()}}}

\item {} 
\sphinxAtStartPar
\phantomsection\label{\detokenize{reference/design:id8}}{\hyperref[\detokenize{reference/design:flatten}]{\sphinxcrossref{Flatten()}}}

\item {} 
\sphinxAtStartPar
\phantomsection\label{\detokenize{reference/design:id9}}{\hyperref[\detokenize{reference/design:flattenn}]{\sphinxcrossref{FlattenN()}}}

\item {} 
\sphinxAtStartPar
\phantomsection\label{\detokenize{reference/design:id10}}{\hyperref[\detokenize{reference/design:foldlist}]{\sphinxcrossref{FoldList()}}}

\item {} 
\sphinxAtStartPar
\phantomsection\label{\detokenize{reference/design:id11}}{\hyperref[\detokenize{reference/design:insert}]{\sphinxcrossref{Insert()}}}

\item {} 
\sphinxAtStartPar
\phantomsection\label{\detokenize{reference/design:id12}}{\hyperref[\detokenize{reference/design:latinsquare}]{\sphinxcrossref{LatinSquare()}}}

\item {} 
\sphinxAtStartPar
\phantomsection\label{\detokenize{reference/design:id13}}{\hyperref[\detokenize{reference/design:levels}]{\sphinxcrossref{Levels()}}}

\item {} 
\sphinxAtStartPar
\phantomsection\label{\detokenize{reference/design:id14}}{\hyperref[\detokenize{reference/design:listby}]{\sphinxcrossref{ListBy()}}}

\item {} 
\sphinxAtStartPar
\phantomsection\label{\detokenize{reference/design:id15}}{\hyperref[\detokenize{reference/design:removesubset}]{\sphinxcrossref{RemoveSubset()}}}

\item {} 
\sphinxAtStartPar
\phantomsection\label{\detokenize{reference/design:id16}}{\hyperref[\detokenize{reference/design:replace}]{\sphinxcrossref{Replace()}}}

\item {} 
\sphinxAtStartPar
\phantomsection\label{\detokenize{reference/design:id17}}{\hyperref[\detokenize{reference/design:rest}]{\sphinxcrossref{Rest()}}}

\item {} 
\sphinxAtStartPar
\phantomsection\label{\detokenize{reference/design:id18}}{\hyperref[\detokenize{reference/design:sample}]{\sphinxcrossref{Sample()}}}

\item {} 
\sphinxAtStartPar
\phantomsection\label{\detokenize{reference/design:id19}}{\hyperref[\detokenize{reference/design:samplen}]{\sphinxcrossref{SampleN()}}}

\item {} 
\sphinxAtStartPar
\phantomsection\label{\detokenize{reference/design:id20}}{\hyperref[\detokenize{reference/design:samplenwithreplacement}]{\sphinxcrossref{SampleNWithReplacement()}}}

\item {} 
\sphinxAtStartPar
\phantomsection\label{\detokenize{reference/design:id21}}{\hyperref[\detokenize{reference/design:shufflerepeat}]{\sphinxcrossref{ShuffleRepeat()}}}

\item {} 
\sphinxAtStartPar
\phantomsection\label{\detokenize{reference/design:id22}}{\hyperref[\detokenize{reference/design:shufflewithoutadjacents}]{\sphinxcrossref{ShuffleWithoutAdjacents()}}}

\item {} 
\sphinxAtStartPar
\phantomsection\label{\detokenize{reference/design:id23}}{\hyperref[\detokenize{reference/design:subset}]{\sphinxcrossref{Subset()}}}

\item {} 
\sphinxAtStartPar
\phantomsection\label{\detokenize{reference/design:id24}}{\hyperref[\detokenize{reference/design:functions-pending-documentation}]{\sphinxcrossref{Functions Pending Documentation}}}

\item {} 
\sphinxAtStartPar
\phantomsection\label{\detokenize{reference/design:id25}}{\hyperref[\detokenize{reference/design:findtoken}]{\sphinxcrossref{FindToken()}}}

\item {} 
\sphinxAtStartPar
\phantomsection\label{\detokenize{reference/design:id26}}{\hyperref[\detokenize{reference/design:reverse}]{\sphinxcrossref{Reverse()}}}

\end{itemize}
\end{sphinxShadowBox}

\index{ChooseN@\spxentry{ChooseN}}\ignorespaces 

\subsection{ChooseN()}
\label{\detokenize{reference/design:choosen}}\label{\detokenize{reference/design:index-0}}
\sphinxAtStartPar
\sphinxstylestrong{Description:}

\sphinxAtStartPar
Samples \sphinxcode{\sphinxupquote{\textless{}number\textgreater{}}} items from list, returning   a list in the original order. Items are sampled without replacement, so   once an item is chosen it will not be chosen again. If   \sphinxcode{\sphinxupquote{\textless{}number\textgreater{}}} is larger than the length of the list, the entire   list is returned in order.  It differs from \sphinxcode{\sphinxupquote{SampleN}} in that   \sphinxcode{\sphinxupquote{ChooseN}} returns items in the order they appeared in the   originial list, but \sphinxcode{\sphinxupquote{SampleN}} is shuffled.

\sphinxAtStartPar
\sphinxstylestrong{Usage:}

\begin{sphinxVerbatim}[commandchars=\\\{\}]
\PYG{k}{define}\PYG{+w}{ }\PYG{n+nf}{ChooseN}\PYG{p}{(}\PYG{p}{.}\PYG{p}{.}\PYG{p}{.}\PYG{p}{)}
\end{sphinxVerbatim}

\sphinxAtStartPar
\sphinxstylestrong{Example:}

\begin{sphinxVerbatim}[commandchars=\\\{\}]
\PYG{c+c1}{\PYGZsh{} Returns 5 numbers}
\PYG{n+nf}{ChooseN}\PYG{p}{(}\PYG{p}{[}\PYG{l+m+mi}{1}\PYG{p}{,}\PYG{l+m+mi}{1}\PYG{p}{,}\PYG{l+m+mi}{1}\PYG{p}{,}\PYG{l+m+mi}{2}\PYG{p}{,}\PYG{l+m+mi}{2}\PYG{p}{]}\PYG{p}{,}\PYG{+w}{ }\PYG{l+m+mi}{5}\PYG{p}{)}

\PYG{c+c1}{\PYGZsh{} Returns 3 numbers from 1 and 7:}
\PYG{n+nf}{ChooseN}\PYG{p}{(}\PYG{p}{[}\PYG{l+m+mi}{1}\PYG{p}{,}\PYG{l+m+mi}{2}\PYG{p}{,}\PYG{l+m+mi}{3}\PYG{p}{,}\PYG{l+m+mi}{4}\PYG{p}{,}\PYG{l+m+mi}{5}\PYG{p}{,}\PYG{l+m+mi}{6}\PYG{p}{,}\PYG{l+m+mi}{7}\PYG{p}{]}\PYG{p}{,}\PYG{+w}{ }\PYG{l+m+mi}{3}\PYG{p}{)}
\end{sphinxVerbatim}

\sphinxAtStartPar
\sphinxstylestrong{See Also:}

\sphinxAtStartPar
\sphinxcode{\sphinxupquote{SampleN()}}, \sphinxcode{\sphinxupquote{SampleNWithReplacement()}}, \sphinxcode{\sphinxupquote{Subset()}}

\index{DesignBalancedSampling@\spxentry{DesignBalancedSampling}}\ignorespaces 

\subsection{DesignBalancedSampling()}
\label{\detokenize{reference/design:designbalancedsampling}}\label{\detokenize{reference/design:index-1}}
\sphinxAtStartPar
\sphinxstylestrong{Description:}

\sphinxAtStartPar
Samples elements \sphinxcode{\sphinxupquote{roughly\textquotesingle{}\textquotesingle{} equally.                   This function returns a list of repeated samples from           \textasciigrave{}\textasciigrave{}\textless{}treatment\_list\textgreater{}}}, such that each element in \sphinxcode{\sphinxupquote{\textless{}treatment\_list\textgreater{}}}            appears approximately equally.  Each element from               \sphinxcode{\sphinxupquote{\textless{}treatment\_list\textgreater{}}} is sampled once without replacement before                 all elements are returned to the mix and sampling is repeated.                  If there are no repeated items in \sphinxcode{\sphinxupquote{\textless{}list\textgreater{}}}, there will be no                  consecutive repeats in the output.  The last repeat\sphinxhyphen{}sampling            will be truncated so that a \sphinxcode{\sphinxupquote{\textless{}length\textgreater{}}}\sphinxhyphen{}size list is returned.                 If you don’t want the repeated epochs this function provides,           Shuffle() the results.

\sphinxAtStartPar
\sphinxstylestrong{Usage:}

\begin{sphinxVerbatim}[commandchars=\\\{\}]
\PYG{k}{define}\PYG{+w}{ }\PYG{n+nf}{DesignBalancedSampling}\PYG{p}{(}\PYG{p}{.}\PYG{p}{.}\PYG{p}{.}\PYG{p}{)}
\end{sphinxVerbatim}

\sphinxAtStartPar
\sphinxstylestrong{Example:}

\begin{sphinxVerbatim}[commandchars=\\\{\}]
\PYG{n+nf}{DesignBalancedSampling}\PYG{p}{(}\PYG{p}{[}\PYG{l+m+mi}{1}\PYG{p}{,}\PYG{l+m+mi}{2}\PYG{p}{,}\PYG{l+m+mi}{3}\PYG{p}{,}\PYG{l+m+mi}{4}\PYG{p}{,}\PYG{l+m+mi}{5}\PYG{p}{]}\PYG{p}{,}\PYG{l+m+mi}{12}\PYG{p}{)}
\PYG{c+c1}{\PYGZsh{}\PYGZsh{} e.g., produces something like:}
\PYG{c+c1}{\PYGZsh{}\PYGZsh{}    [5,3,1,4,2, 3,1,5,2,4, 3,1 ]}
\end{sphinxVerbatim}

\sphinxAtStartPar
\sphinxstylestrong{See Also:}
\begin{description}
\sphinxlineitem{\sphinxcode{\sphinxupquote{CrossFactorWithoutDuplicates()}},}\begin{description}
\sphinxlineitem{\sphinxcode{\sphinxupquote{Shuffle()}}, \sphinxcode{\sphinxupquote{DesignFullCounterBalance()}},}
\sphinxAtStartPar
\sphinxcode{\sphinxupquote{DesignGrecoLatinSquare()}}, \sphinxcode{\sphinxupquote{DesignLatinSquare()}}, \sphinxcode{\sphinxupquote{Repeat()}},
\sphinxcode{\sphinxupquote{RepeatList()}}, \sphinxcode{\sphinxupquote{LatinSquare()}}

\end{description}

\end{description}

\index{DesignGrecoLatinSquare@\spxentry{DesignGrecoLatinSquare}}\ignorespaces 

\subsection{DesignGrecoLatinSquare()}
\label{\detokenize{reference/design:designgrecolatinsquare}}\label{\detokenize{reference/design:index-2}}
\sphinxAtStartPar
\sphinxstylestrong{Description:}

\sphinxAtStartPar
This will return a list of lists formed by rotating   through each element of the {\color{red}\bfseries{}\textasciigrave{}\textasciigrave{}}\textless{}treatment\_list\textgreater{}\textasciigrave{}\textasciigrave{}s, making a list   containing all element of the list, according to a greco\sphinxhyphen{}latin   square.  All lists must be of the same length.

\sphinxAtStartPar
\sphinxstylestrong{Usage:}

\begin{sphinxVerbatim}[commandchars=\\\{\}]
\PYG{k}{define}\PYG{+w}{ }\PYG{n+nf}{DesignGrecoLatinSquare}\PYG{p}{(}\PYG{p}{.}\PYG{p}{.}\PYG{p}{.}\PYG{p}{)}
\end{sphinxVerbatim}

\sphinxAtStartPar
\sphinxstylestrong{Example:}

\begin{sphinxVerbatim}[commandchars=\\\{\}]
\PYG{n+nv}{x}\PYG{+w}{ }\PYG{o}{\PYGZlt{}\PYGZhy{}}\PYG{+w}{ }\PYG{p}{[}\PYG{l+s+s2}{\PYGZdq{}a\PYGZdq{}}\PYG{p}{,}\PYG{l+s+s2}{\PYGZdq{}b\PYGZdq{}}\PYG{p}{,}\PYG{l+s+s2}{\PYGZdq{}c\PYGZdq{}}\PYG{p}{]}
\PYG{n+nv}{y}\PYG{+w}{ }\PYG{o}{\PYGZlt{}\PYGZhy{}}\PYG{+w}{ }\PYG{p}{[}\PYG{l+s+s2}{\PYGZdq{}p\PYGZdq{}}\PYG{p}{,}\PYG{l+s+s2}{\PYGZdq{}q\PYGZdq{}}\PYG{p}{,}\PYG{l+s+s2}{\PYGZdq{}r\PYGZdq{}}\PYG{p}{]}
\PYG{n+nv}{z}\PYG{+w}{ }\PYG{o}{\PYGZlt{}\PYGZhy{}}\PYG{+w}{ }\PYG{p}{[}\PYG{l+s+s2}{\PYGZdq{}x\PYGZdq{}}\PYG{p}{,}\PYG{l+s+s2}{\PYGZdq{}y\PYGZdq{}}\PYG{p}{,}\PYG{l+s+s2}{\PYGZdq{}z\PYGZdq{}}\PYG{p}{]}
\PYG{n+nf}{Print}\PYG{p}{(}\PYG{n+nf}{DesignGrecoLatinSquare}\PYG{p}{(}\PYG{n+nv}{x}\PYG{p}{,}\PYG{n+nv}{y}\PYG{p}{,}\PYG{n+nv}{z}\PYG{p}{)}\PYG{p}{)}
\PYG{c+c1}{\PYGZsh{} produces:          [[[a, p, x], [b, q, y], [c, r, z]],}
\PYG{c+c1}{\PYGZsh{}               [[a, q, z], [b, r, x], [c, p, y]],}
\PYG{c+c1}{\PYGZsh{}               [[a, r, y], [b, p, z], [c, q, x]]]}
\end{sphinxVerbatim}

\sphinxAtStartPar
\sphinxstylestrong{See Also:}
\begin{description}
\sphinxlineitem{\sphinxcode{\sphinxupquote{CrossFactorWithoutDuplicates()}}, \sphinxcode{\sphinxupquote{LatinSquare()}}, \sphinxcode{\sphinxupquote{DesignFullCounterBalance()}}, \sphinxcode{\sphinxupquote{DesignBalancedSampling()}}, \sphinxcode{\sphinxupquote{DesignLatinSquare()}}, \sphinxcode{\sphinxupquote{Repeat()}}, \sphinxcode{\sphinxupquote{RepeatList()}},}
\sphinxAtStartPar
\sphinxcode{\sphinxupquote{Shuffle()}}

\end{description}

\index{DesignLatinSquare@\spxentry{DesignLatinSquare}}\ignorespaces 

\subsection{DesignLatinSquare()}
\label{\detokenize{reference/design:designlatinsquare}}\label{\detokenize{reference/design:index-3}}
\sphinxAtStartPar
\sphinxstyleemphasis{Simple latin square}

\sphinxAtStartPar
\sphinxstylestrong{Description:}

\sphinxAtStartPar
This returns return a list of lists formed by   rotating through each element of \sphinxcode{\sphinxupquote{\textless{}treatment\_list\textgreater{}}}, making a   list containing all element of the list. Has no side effect on input   lists.

\sphinxAtStartPar
\sphinxstylestrong{Usage:}

\begin{sphinxVerbatim}[commandchars=\\\{\}]
\PYG{k}{define}\PYG{+w}{ }\PYG{n+nf}{DesignLatinSquare}\PYG{p}{(}\PYG{p}{.}\PYG{p}{.}\PYG{p}{.}\PYG{p}{)}
\end{sphinxVerbatim}

\sphinxAtStartPar
\sphinxstylestrong{Example:}

\begin{sphinxVerbatim}[commandchars=\\\{\}]
\PYG{n+nv}{order}\PYG{+w}{ }\PYG{o}{\PYGZlt{}\PYGZhy{}}\PYG{+w}{ }\PYG{p}{[}\PYG{l+m+mi}{1}\PYG{p}{,}\PYG{l+m+mi}{2}\PYG{p}{,}\PYG{l+m+mi}{3}\PYG{p}{]}
\PYG{n+nv}{treatment}\PYG{+w}{ }\PYG{o}{\PYGZlt{}\PYGZhy{}}\PYG{+w}{ }\PYG{p}{[}\PYG{l+s+s2}{\PYGZdq{}A\PYGZdq{}}\PYG{p}{,}\PYG{l+s+s2}{\PYGZdq{}B\PYGZdq{}}\PYG{p}{,}\PYG{l+s+s2}{\PYGZdq{}C\PYGZdq{}}\PYG{p}{]}
\PYG{n+nv}{design}\PYG{+w}{ }\PYG{o}{\PYGZlt{}\PYGZhy{}}\PYG{+w}{ }\PYG{n+nf}{DesignLatinSquare}\PYG{p}{(}\PYG{n+nv}{order}\PYG{p}{,}\PYG{n+nv}{treatment}\PYG{p}{)}
\PYG{c+c1}{\PYGZsh{} produces: [[[1, A], [2, B], [3, C]],}
\PYG{c+c1}{\PYGZsh{}            [[1, B], [2, C], [3, A]],}
\PYG{c+c1}{\PYGZsh{}            [[1, C], [2, A], [3, B]]]}
\end{sphinxVerbatim}

\sphinxAtStartPar
\sphinxstylestrong{See Also:}
\begin{description}
\sphinxlineitem{\sphinxcode{\sphinxupquote{CrossFactorWithoutDuplicates()}},}
\sphinxAtStartPar
\sphinxcode{\sphinxupquote{DesignFullCounterBalance()}}, \sphinxcode{\sphinxupquote{DesignBalancedSampling()}},
\sphinxcode{\sphinxupquote{DesignGrecoLatinSquare()}}, \sphinxcode{\sphinxupquote{Repeat()}}, \sphinxcode{\sphinxupquote{LatinSquare()}}
\sphinxcode{\sphinxupquote{RepeatList()}}, \sphinxcode{\sphinxupquote{Shuffle()}}, \sphinxcode{\sphinxupquote{Rotate()}}

\end{description}

\index{ExtractListItems@\spxentry{ExtractListItems}}\ignorespaces 

\subsection{ExtractListItems()}
\label{\detokenize{reference/design:extractlistitems}}\label{\detokenize{reference/design:index-4}}
\sphinxAtStartPar
\sphinxstyleemphasis{Gets a subset of items from a list}

\sphinxAtStartPar
\sphinxstylestrong{Description:}

\sphinxAtStartPar
Extracts items from a list, forming a new list.                 The list \sphinxcode{\sphinxupquote{\textless{}items\textgreater{}}} are the integers representing the          indices that should be extracted.

\sphinxAtStartPar
\sphinxstylestrong{Usage:}

\begin{sphinxVerbatim}[commandchars=\\\{\}]
\PYG{k}{define}\PYG{+w}{ }\PYG{n+nf}{ExtractListItems}\PYG{p}{(}\PYG{p}{.}\PYG{p}{.}\PYG{p}{.}\PYG{p}{)}
\end{sphinxVerbatim}

\sphinxAtStartPar
\sphinxstylestrong{Example:}

\begin{sphinxVerbatim}[commandchars=\\\{\}]
\PYG{n+nv}{myList}\PYG{+w}{ }\PYG{o}{\PYGZlt{}\PYGZhy{}}\PYG{+w}{ }\PYG{n+nf}{Sequence}\PYG{p}{(}\PYG{l+m+mi}{101}\PYG{p}{,}\PYG{+w}{ }\PYG{l+m+mi}{110}\PYG{p}{,}\PYG{+w}{ }\PYG{l+m+mi}{1}\PYG{p}{)}
\PYG{n+nf}{ExtractListItems}\PYG{p}{(}\PYG{n+nv}{myList}\PYG{p}{,}\PYG{+w}{ }\PYG{p}{[}\PYG{l+m+mi}{2}\PYG{p}{,}\PYG{l+m+mi}{4}\PYG{p}{,}\PYG{l+m+mi}{5}\PYG{p}{,}\PYG{l+m+mi}{1}\PYG{p}{,}\PYG{l+m+mi}{4}\PYG{p}{]}\PYG{p}{)}
\PYG{c+c1}{\PYGZsh{} produces [102, 104, 105, 101, 104]}
\end{sphinxVerbatim}

\sphinxAtStartPar
\sphinxstylestrong{See Also:}

\sphinxAtStartPar
\sphinxcode{\sphinxupquote{Subset()}}, \sphinxcode{\sphinxupquote{SubList()}}, \sphinxcode{\sphinxupquote{SampleN()}}, \sphinxcode{\sphinxupquote{Filter()}}

\index{Flatten@\spxentry{Flatten}}\ignorespaces 

\subsection{Flatten()}
\label{\detokenize{reference/design:flatten}}\label{\detokenize{reference/design:index-5}}
\sphinxAtStartPar
\sphinxstyleemphasis{Flattens a nested list completely}

\sphinxAtStartPar
\sphinxstylestrong{Description:}

\sphinxAtStartPar
Flattens nested list \sphinxcode{\sphinxupquote{\textless{}list\textgreater{}}} to a single flat list.

\sphinxAtStartPar
\sphinxstylestrong{Usage:}

\begin{sphinxVerbatim}[commandchars=\\\{\}]
\PYG{k}{define}\PYG{+w}{ }\PYG{n+nf}{Flatten}\PYG{p}{(}\PYG{p}{.}\PYG{p}{.}\PYG{p}{.}\PYG{p}{)}
\end{sphinxVerbatim}

\sphinxAtStartPar
\sphinxstylestrong{Example:}

\begin{sphinxVerbatim}[commandchars=\\\{\}]
\PYG{n+nf}{Flatten}\PYG{p}{(}\PYG{p}{[}\PYG{l+m+mi}{1}\PYG{p}{,}\PYG{l+m+mi}{2}\PYG{p}{,}\PYG{p}{[}\PYG{l+m+mi}{3}\PYG{p}{,}\PYG{l+m+mi}{4}\PYG{p}{]}\PYG{p}{,}\PYG{p}{[}\PYG{l+m+mi}{5}\PYG{p}{,}\PYG{p}{[}\PYG{l+m+mi}{6}\PYG{p}{,}\PYG{l+m+mi}{7}\PYG{p}{]}\PYG{p}{,}\PYG{l+m+mi}{8}\PYG{p}{]}\PYG{p}{,}\PYG{p}{[}\PYG{l+m+mi}{9}\PYG{p}{]}\PYG{p}{]}\PYG{p}{)}\PYG{+w}{ }\PYG{c+c1}{\PYGZsh{} == [1,2,3,4,5,6,7,8,9]}
\PYG{n+nf}{Flatten}\PYG{p}{(}\PYG{p}{[}\PYG{l+m+mi}{1}\PYG{p}{,}\PYG{l+m+mi}{2}\PYG{p}{,}\PYG{p}{[}\PYG{l+m+mi}{3}\PYG{p}{,}\PYG{l+m+mi}{4}\PYG{p}{]}\PYG{p}{,}\PYG{p}{[}\PYG{l+m+mi}{5}\PYG{p}{,}\PYG{p}{[}\PYG{l+m+mi}{6}\PYG{p}{,}\PYG{l+m+mi}{7}\PYG{p}{]}\PYG{p}{,}\PYG{l+m+mi}{8}\PYG{p}{]}\PYG{p}{,}\PYG{p}{[}\PYG{l+m+mi}{9}\PYG{p}{]}\PYG{p}{]}\PYG{p}{)}\PYG{+w}{ }\PYG{c+c1}{\PYGZsh{} == [1,2,3,4,5,6,7,8,9]}
\end{sphinxVerbatim}

\sphinxAtStartPar
\sphinxstylestrong{See Also:}

\sphinxAtStartPar
\sphinxcode{\sphinxupquote{FlattenN()}}, \sphinxcode{\sphinxupquote{FoldList()}}

\index{FlattenN@\spxentry{FlattenN}}\ignorespaces 

\subsection{FlattenN()}
\label{\detokenize{reference/design:flattenn}}\label{\detokenize{reference/design:index-6}}
\sphinxAtStartPar
\sphinxstyleemphasis{Flattens n levels of a nested list}

\sphinxAtStartPar
\sphinxstylestrong{Description:}

\sphinxAtStartPar
Flattens \sphinxcode{\sphinxupquote{\textless{}n\textgreater{}}} levels of nested list \sphinxcode{\sphinxupquote{\textless{}list\textgreater{}}}.

\sphinxAtStartPar
\sphinxstylestrong{Usage:}

\begin{sphinxVerbatim}[commandchars=\\\{\}]
\PYG{k}{define}\PYG{+w}{ }\PYG{n+nf}{FlattenN}\PYG{p}{(}\PYG{p}{.}\PYG{p}{.}\PYG{p}{.}\PYG{p}{)}
\end{sphinxVerbatim}

\sphinxAtStartPar
\sphinxstylestrong{Example:}

\begin{sphinxVerbatim}[commandchars=\\\{\}]
\PYG{n+nf}{Flatten}\PYG{p}{(}\PYG{p}{[}\PYG{l+m+mi}{1}\PYG{p}{,}\PYG{l+m+mi}{2}\PYG{p}{,}\PYG{p}{[}\PYG{l+m+mi}{3}\PYG{p}{,}\PYG{l+m+mi}{4}\PYG{p}{]}\PYG{p}{,}\PYG{p}{[}\PYG{l+m+mi}{5}\PYG{p}{,}\PYG{p}{[}\PYG{l+m+mi}{6}\PYG{p}{,}\PYG{l+m+mi}{7}\PYG{p}{]}\PYG{p}{,}\PYG{l+m+mi}{8}\PYG{p}{]}\PYG{p}{,}\PYG{p}{[}\PYG{l+m+mi}{9}\PYG{p}{]}\PYG{p}{]}\PYG{p}{,}\PYG{l+m+mi}{1}\PYG{p}{)}
\PYG{c+c1}{\PYGZsh{} == [1,2,3,4,5,[6,7],8,9]}
\end{sphinxVerbatim}

\sphinxAtStartPar
\sphinxstylestrong{See Also:}

\sphinxAtStartPar
\sphinxcode{\sphinxupquote{Flatten()}}, \sphinxcode{\sphinxupquote{FoldList()}}

\index{FoldList@\spxentry{FoldList}}\ignorespaces 

\subsection{FoldList()}
\label{\detokenize{reference/design:foldlist}}\label{\detokenize{reference/design:index-7}}
\sphinxAtStartPar
\sphinxstyleemphasis{Folds list into length\sphinxhyphen{}n sublists.}

\sphinxAtStartPar
\sphinxstylestrong{Description:}

\sphinxAtStartPar
Folds a list into equal\sphinxhyphen{}length sublists.

\sphinxAtStartPar
\sphinxstylestrong{Usage:}

\begin{sphinxVerbatim}[commandchars=\\\{\}]
\PYG{k}{define}\PYG{+w}{ }\PYG{n+nf}{FoldList}\PYG{p}{(}\PYG{p}{.}\PYG{p}{.}\PYG{p}{.}\PYG{p}{)}
\end{sphinxVerbatim}

\sphinxAtStartPar
\sphinxstylestrong{Example:}

\begin{sphinxVerbatim}[commandchars=\\\{\}]
\PYG{n+nf}{FoldList}\PYG{p}{(}\PYG{p}{[}\PYG{l+m+mi}{1}\PYG{p}{,}\PYG{l+m+mi}{2}\PYG{p}{,}\PYG{l+m+mi}{3}\PYG{p}{,}\PYG{l+m+mi}{4}\PYG{p}{,}\PYG{l+m+mi}{5}\PYG{p}{,}\PYG{l+m+mi}{6}\PYG{p}{,}\PYG{l+m+mi}{7}\PYG{p}{,}\PYG{l+m+mi}{8}\PYG{p}{]}\PYG{p}{,}\PYG{l+m+mi}{2}\PYG{p}{)}\PYG{+w}{        }\PYG{c+c1}{\PYGZsh{} == [[1,2],[3,4],[5,6],[7,8]]}
\end{sphinxVerbatim}

\sphinxAtStartPar
\sphinxstylestrong{See Also:}

\sphinxAtStartPar
\sphinxcode{\sphinxupquote{FlattenN()}}, \sphinxcode{\sphinxupquote{Flatten()}}

\index{Insert@\spxentry{Insert}}\ignorespaces 

\subsection{Insert()}
\label{\detokenize{reference/design:insert}}\label{\detokenize{reference/design:index-8}}
\sphinxAtStartPar
\sphinxstylestrong{Description:}

\sphinxAtStartPar
Inserts an element into a list at a specified   position, returning the new list. The original list in unchanged.

\sphinxAtStartPar
\sphinxstylestrong{Usage:}

\begin{sphinxVerbatim}[commandchars=\\\{\}]
\PYG{k}{define}\PYG{+w}{ }\PYG{n+nf}{Insert}\PYG{p}{(}\PYG{p}{.}\PYG{p}{.}\PYG{p}{.}\PYG{p}{)}
\end{sphinxVerbatim}

\sphinxAtStartPar
\sphinxstylestrong{Example:}

\begin{sphinxVerbatim}[commandchars=\\\{\}]
\PYG{n+nv}{x}\PYG{+w}{ }\PYG{o}{\PYGZlt{}\PYGZhy{}}\PYG{+w}{ }\PYG{p}{[}\PYG{l+m+mi}{1}\PYG{p}{,}\PYG{l+m+mi}{2}\PYG{p}{,}\PYG{l+m+mi}{3}\PYG{p}{,}\PYG{l+m+mi}{5}\PYG{p}{]}
\PYG{n+nv}{y}\PYG{+w}{ }\PYG{o}{\PYGZlt{}\PYGZhy{}}\PYG{+w}{ }\PYG{n+nf}{Insert}\PYG{p}{(}\PYG{n+nv}{x}\PYG{p}{,}\PYG{l+m+mi}{1}\PYG{p}{,}\PYG{l+m+mi}{4}\PYG{p}{)}
\PYG{c+c1}{\PYGZsh{}\PYGZsh{}y== [1,2,3,1,5]}
\end{sphinxVerbatim}

\sphinxAtStartPar
\sphinxstylestrong{See Also:}

\sphinxAtStartPar
\sphinxcode{\sphinxupquote{List()}}, \sphinxcode{\sphinxupquote{Merge}}, \sphinxcode{\sphinxupquote{Append}}

\index{LatinSquare@\spxentry{LatinSquare}}\ignorespaces 

\subsection{LatinSquare()}
\label{\detokenize{reference/design:latinsquare}}\label{\detokenize{reference/design:index-9}}
\sphinxAtStartPar
\sphinxstyleemphasis{A simple latin square constructor}

\sphinxAtStartPar
\sphinxstylestrong{Description:}

\sphinxAtStartPar
Quick and dirty latin square, taking on just one   list argument.

\sphinxAtStartPar
\sphinxstylestrong{Usage:}

\begin{sphinxVerbatim}[commandchars=\\\{\}]
\PYG{k}{define}\PYG{+w}{ }\PYG{n+nf}{LatinSquare}\PYG{p}{(}\PYG{p}{.}\PYG{p}{.}\PYG{p}{.}\PYG{p}{)}
\end{sphinxVerbatim}

\sphinxAtStartPar
\sphinxstylestrong{Example:}

\begin{sphinxVerbatim}[commandchars=\\\{\}]
\PYG{n+nf}{Print}\PYG{p}{(}\PYG{n+nf}{LatinSquare}\PYG{p}{(}\PYG{p}{[}\PYG{l+m+mi}{11}\PYG{p}{,}\PYG{l+m+mi}{12}\PYG{p}{,}\PYG{l+m+mi}{13}\PYG{p}{,}\PYG{l+m+mi}{14}\PYG{p}{,}\PYG{l+m+mi}{15}\PYG{p}{,}\PYG{l+m+mi}{16}\PYG{p}{]}\PYG{p}{)}\PYG{p}{)}
\PYG{c+c1}{\PYGZsh{} Output:}
\PYG{c+c1}{\PYGZsh{}[[11, 12, 13, 14, 15, 16]}
\PYG{c+c1}{\PYGZsh{}, [12, 13, 14, 15, 16, 11]}
\PYG{c+c1}{\PYGZsh{}, [13, 14, 15, 16, 11, 12]}
\PYG{c+c1}{\PYGZsh{}, [14, 15, 16, 11, 12, 13]}
\PYG{c+c1}{\PYGZsh{}, [15, 16, 11, 12, 13, 14]}
\PYG{c+c1}{\PYGZsh{}, [16, 11, 12, 13, 14, 15]}
\PYG{c+c1}{\PYGZsh{}]}
\end{sphinxVerbatim}

\sphinxAtStartPar
\sphinxstylestrong{See Also:}
\begin{description}
\sphinxlineitem{\sphinxcode{\sphinxupquote{DesignFullCounterBalance()}},}
\sphinxAtStartPar
\sphinxcode{\sphinxupquote{DesignBalancedSampling()}}, \sphinxcode{\sphinxupquote{DesignGrecoLatinSquare()}},
\sphinxcode{\sphinxupquote{DesignLatinSquare()}}, \sphinxcode{\sphinxupquote{Repeat()}}, \sphinxcode{\sphinxupquote{RepeatList()}},
\sphinxcode{\sphinxupquote{Shuffle()}}

\end{description}

\index{Levels@\spxentry{Levels}}\ignorespaces 

\subsection{Levels()}
\label{\detokenize{reference/design:levels}}\label{\detokenize{reference/design:index-10}}
\sphinxAtStartPar
\sphinxstyleemphasis{Returns a sorted list of unique elements in list.}

\sphinxAtStartPar
\sphinxstylestrong{Description:}

\sphinxAtStartPar
Returns sorted list of unique elements of a list.

\sphinxAtStartPar
\sphinxstylestrong{Usage:}

\begin{sphinxVerbatim}[commandchars=\\\{\}]
\PYG{k}{define}\PYG{+w}{ }\PYG{n+nf}{Levels}\PYG{p}{(}\PYG{p}{.}\PYG{p}{.}\PYG{p}{.}\PYG{p}{)}
\end{sphinxVerbatim}

\sphinxAtStartPar
\sphinxstylestrong{Example:}

\begin{sphinxVerbatim}[commandchars=\\\{\}]
\PYG{n+nf}{Levels}\PYG{p}{(}\PYG{p}{[}\PYG{l+m+mi}{1}\PYG{p}{,}\PYG{l+m+mi}{3}\PYG{p}{,}\PYG{l+m+mi}{55}\PYG{p}{,}\PYG{l+m+mi}{1}\PYG{p}{,}\PYG{l+m+mi}{5}\PYG{p}{,}\PYG{l+m+mi}{1}\PYG{p}{,}\PYG{l+m+mi}{5}\PYG{p}{]}\PYG{p}{)}\PYG{+w}{     }\PYG{c+c1}{\PYGZsh{} == [1,3,5,55]}
\end{sphinxVerbatim}

\sphinxAtStartPar
\sphinxstylestrong{See Also:}

\sphinxAtStartPar
\sphinxcode{\sphinxupquote{Match()}}, \sphinxcode{\sphinxupquote{Filter()}}, \sphinxcode{\sphinxupquote{Sort()}}

\index{ListBy@\spxentry{ListBy}}\ignorespaces 

\subsection{ListBy()}
\label{\detokenize{reference/design:listby}}\label{\detokenize{reference/design:index-11}}
\sphinxAtStartPar
\sphinxstyleemphasis{Segments a list into sublist by the values of a second list}

\sphinxAtStartPar
\sphinxstylestrong{Description:}

\sphinxAtStartPar
organizes a list into sublists, based on the   elements of a second list.  It returns a list of two entities: (1) a   condition list, describing what values were aggregated across; (2)   the nested list elements.  The length of each element should be the same.  Together with Match and Filter, ListBy is useful for aggregating data across blocks and conditions for immediate feedback.

\sphinxAtStartPar
\sphinxstylestrong{Usage:}

\begin{sphinxVerbatim}[commandchars=\\\{\}]
\PYG{k}{define}\PYG{+w}{ }\PYG{n+nf}{ListBy}\PYG{p}{(}\PYG{p}{.}\PYG{p}{.}\PYG{p}{.}\PYG{p}{)}
\end{sphinxVerbatim}

\sphinxAtStartPar
\sphinxstylestrong{Example:}

\begin{sphinxVerbatim}[commandchars=\\\{\}]
\PYG{+w}{     }\PYG{n+nv}{a}\PYG{+w}{ }\PYG{o}{\PYGZlt{}\PYGZhy{}}\PYG{+w}{ }\PYG{n+nf}{Sequence}\PYG{p}{(}\PYG{l+m+mi}{1}\PYG{p}{,}\PYG{l+m+mi}{10}\PYG{p}{,}\PYG{l+m+mi}{1}\PYG{p}{)}
\PYG{+w}{    }\PYG{n+nv}{b}\PYG{+w}{ }\PYG{o}{\PYGZlt{}\PYGZhy{}}\PYG{+w}{ }\PYG{n+nf}{RepeatList}\PYG{p}{(}\PYG{p}{[}\PYG{l+m+mi}{1}\PYG{p}{,}\PYG{l+m+mi}{2}\PYG{p}{]}\PYG{p}{,}\PYG{l+m+mi}{5}\PYG{p}{)}
\PYG{+w}{    }\PYG{n+nv}{x}\PYG{+w}{ }\PYG{o}{\PYGZlt{}\PYGZhy{}}\PYG{+w}{ }\PYG{n+nf}{ListBy}\PYG{p}{(}\PYG{n+nv}{a}\PYG{p}{,}\PYG{n+nv}{b}\PYG{p}{)}
\PYG{+w}{    }\PYG{n+nf}{Print}\PYG{p}{(}\PYG{n+nv}{x}\PYG{p}{)}
\PYG{c+c1}{\PYGZsh{}[[1, 2],}
\PYG{c+c1}{\PYGZsh{}  [[1, 3, 5, 7, 9],}
\PYG{c+c1}{\PYGZsh{}   [2, 4, 6, 8, 10]]}
\PYG{c+c1}{\PYGZsh{}]}

\PYG{+w}{    }\PYG{n+nf}{Print}\PYG{p}{(}\PYG{n+nf}{ListBy}\PYG{p}{(}\PYG{n+nv}{b}\PYG{p}{,}\PYG{n+nv}{a}\PYG{p}{)}\PYG{p}{)}
\PYG{c+c1}{\PYGZsh{}[[1, 2, 3, 4, 5, 6, 7, 8, 9, 10],}
\PYG{c+c1}{\PYGZsh{} [[1], [2], [1], [2], [1], [2], [1], [2], [1], [2]]]}
\end{sphinxVerbatim}

\sphinxAtStartPar
\sphinxstylestrong{See Also:}

\sphinxAtStartPar
\sphinxcode{\sphinxupquote{List()}}, \sphinxcode{\sphinxupquote{{[} {]}}}, \sphinxcode{\sphinxupquote{Merge()}}, \sphinxcode{\sphinxupquote{Append()}}

\index{RemoveSubset@\spxentry{RemoveSubset}}\ignorespaces 

\subsection{RemoveSubset()}
\label{\detokenize{reference/design:removesubset}}\label{\detokenize{reference/design:index-12}}
\sphinxAtStartPar
\sphinxstylestrong{Description:}

\sphinxAtStartPar
Removes a subset of elements from a list. Creates a new list, and does not affect the original

\sphinxAtStartPar
\sphinxstylestrong{Usage:}

\begin{sphinxVerbatim}[commandchars=\\\{\}]
\PYG{k}{define}\PYG{+w}{ }\PYG{n+nf}{RemoveSubset}\PYG{p}{(}\PYG{p}{.}\PYG{p}{.}\PYG{p}{.}\PYG{p}{)}
\end{sphinxVerbatim}

\sphinxAtStartPar
\sphinxstylestrong{Example:}

\begin{sphinxVerbatim}[commandchars=\\\{\}]
\PYG{n+nv}{list1}\PYG{+w}{ }\PYG{o}{\PYGZlt{}\PYGZhy{}}\PYG{+w}{ }\PYG{p}{[}\PYG{l+m+mi}{1}\PYG{p}{,}\PYG{l+m+mi}{2}\PYG{p}{,}\PYG{l+m+mi}{2}\PYG{p}{,}\PYG{l+m+mi}{4}\PYG{p}{,}\PYG{l+m+mi}{5}\PYG{p}{]}
\PYG{n+nv}{list2}\PYG{+w}{ }\PYG{o}{\PYGZlt{}\PYGZhy{}}\PYG{+w}{ }\PYG{n+nf}{RemoveSubset}\PYG{p}{(}\PYG{n+nv}{list1}\PYG{p}{,}\PYG{p}{[}\PYG{l+m+mi}{2}\PYG{p}{,}\PYG{l+m+mi}{3}\PYG{p}{]}\PYG{p}{)}
\PYG{n+nf}{Print}\PYG{p}{(}\PYG{n+nv}{list1}\PYG{p}{)}\PYG{+w}{ }\PYG{c+c1}{\PYGZsh{}[1,2,2,4,5]}
\PYG{n+nf}{Print}\PYG{p}{(}\PYG{n+nv}{list2}\PYG{p}{)}\PYG{+w}{ }\PYG{c+c1}{\PYGZsh{}[1,4,5]}
\end{sphinxVerbatim}

\sphinxAtStartPar
\sphinxstylestrong{See Also:}

\sphinxAtStartPar
\sphinxcode{\sphinxupquote{Merge()}}, \sphinxcode{\sphinxupquote{Insert()}}, \sphinxcode{\sphinxupquote{Rest()}}

\index{Replace@\spxentry{Replace}}\ignorespaces 

\subsection{Replace()}
\label{\detokenize{reference/design:replace}}\label{\detokenize{reference/design:index-13}}
\sphinxAtStartPar
\sphinxstyleemphasis{Replaces items in a data structure}

\sphinxAtStartPar
\sphinxstylestrong{Description:}

\sphinxAtStartPar
Creates a copy of a (possibly nested) list in which             items matching some list are replaced for other items.                  \sphinxcode{\sphinxupquote{\textless{}template\textgreater{}}} can be any data structure, and can be nested.            \sphinxcode{\sphinxupquote{\textless{}replacementList\textgreater{}}} is a list containing two\sphinxhyphen{}item list pairs:                 the to\sphinxhyphen{}be\sphinxhyphen{}replaced item and to what it should be transformed. Note: replacement searches the entire \sphinxcode{\sphinxupquote{\textless{}replacementList\textgreater{}}} for           matches.  If multiple keys are identical, the item will be              replaced with the last item that matches.

\sphinxAtStartPar
\sphinxstylestrong{Usage:}

\begin{sphinxVerbatim}[commandchars=\\\{\}]
\PYG{k}{define}\PYG{+w}{ }\PYG{n+nf}{Replace}\PYG{p}{(}\PYG{p}{.}\PYG{p}{.}\PYG{p}{.}\PYG{p}{)}
\end{sphinxVerbatim}

\sphinxAtStartPar
\sphinxstylestrong{Example:}

\begin{sphinxVerbatim}[commandchars=\\\{\}]
\PYG{n+nv}{x}\PYG{+w}{ }\PYG{o}{\PYGZlt{}\PYGZhy{}}\PYG{+w}{ }\PYG{p}{[}\PYG{l+s+s2}{\PYGZdq{}a\PYGZdq{}}\PYG{p}{,}\PYG{l+s+s2}{\PYGZdq{}b\PYGZdq{}}\PYG{p}{,}\PYG{l+s+s2}{\PYGZdq{}c\PYGZdq{}}\PYG{p}{,}\PYG{l+s+s2}{\PYGZdq{}x\PYGZdq{}}\PYG{p}{]}
\PYG{n+nv}{rep}\PYG{+w}{ }\PYG{o}{\PYGZlt{}\PYGZhy{}}\PYG{+w}{ }\PYG{p}{[}\PYG{p}{[}\PYG{l+s+s2}{\PYGZdq{}a\PYGZdq{}}\PYG{p}{,}\PYG{l+s+s2}{\PYGZdq{}A\PYGZdq{}}\PYG{p}{]}\PYG{p}{,}\PYG{p}{[}\PYG{l+s+s2}{\PYGZdq{}b\PYGZdq{}}\PYG{p}{,}\PYG{l+s+s2}{\PYGZdq{}B\PYGZdq{}}\PYG{p}{]}\PYG{p}{,}\PYG{p}{[}\PYG{l+s+s2}{\PYGZdq{}x\PYGZdq{}}\PYG{p}{,}\PYG{l+s+s2}{\PYGZdq{}D\PYGZdq{}}\PYG{p}{]}\PYG{p}{]}
\PYG{n+nf}{Print}\PYG{p}{(}\PYG{n+nf}{Replace}\PYG{p}{(}\PYG{n+nv}{x}\PYG{p}{,}\PYG{n+nv}{rep}\PYG{p}{)}\PYG{p}{)}
\PYG{c+c1}{\PYGZsh{} Result:  [A, B, c, D]}
\end{sphinxVerbatim}

\sphinxAtStartPar
\sphinxstylestrong{See Also:}

\sphinxAtStartPar
\sphinxcode{\sphinxupquote{ReplaceChar()}}

\index{Rest@\spxentry{Rest}}\ignorespaces 

\subsection{Rest()}
\label{\detokenize{reference/design:rest}}\label{\detokenize{reference/design:index-14}}
\sphinxAtStartPar
\sphinxstyleemphasis{Returns a list minus its first element}

\sphinxAtStartPar
\sphinxstylestrong{Description:}

\sphinxAtStartPar
Returns the ‘rest’ of a list; a list minus its   first element.  If the list is empty or has a single member, it will   return an empty list {[}{]}.  This is a very common function in LISP.

\sphinxAtStartPar
\sphinxstylestrong{Usage:}

\begin{sphinxVerbatim}[commandchars=\\\{\}]
\PYG{k}{define}\PYG{+w}{ }\PYG{n+nf}{Rest}\PYG{p}{(}\PYG{p}{.}\PYG{p}{.}\PYG{p}{.}\PYG{p}{)}
\end{sphinxVerbatim}

\sphinxAtStartPar
\sphinxstylestrong{Example:}

\begin{sphinxVerbatim}[commandchars=\\\{\}]
\PYG{n+nv}{x}\PYG{+w}{ }\PYG{o}{\PYGZlt{}\PYGZhy{}}\PYG{+w}{ }\PYG{n+nf}{Sequence}\PYG{p}{(}\PYG{l+m+mi}{1}\PYG{p}{,}\PYG{l+m+mi}{5}\PYG{p}{,}\PYG{l+m+mi}{1}\PYG{p}{)}
\PYG{n+nv}{y}\PYG{+w}{ }\PYG{o}{\PYGZlt{}\PYGZhy{}}\PYG{+w}{ }\PYG{n+nf}{Rest}\PYG{p}{(}\PYG{n+nv}{x}\PYG{p}{)}
\PYG{n+nf}{Print}\PYG{p}{(}\PYG{n+nv}{rep}\PYG{p}{)}
\PYG{c+c1}{\PYGZsh{} Result:  [2,3,4,5]}
\end{sphinxVerbatim}

\sphinxAtStartPar
\sphinxstylestrong{See Also:}

\sphinxAtStartPar
\sphinxcode{\sphinxupquote{Insert()}}

\index{Sample@\spxentry{Sample}}\ignorespaces 

\subsection{Sample()}
\label{\detokenize{reference/design:sample}}\label{\detokenize{reference/design:index-15}}
\sphinxAtStartPar
\sphinxstylestrong{Description:}

\sphinxAtStartPar
Samples a single item from a list, returning it.   It is a bit more convenient at times than ShuffleN(list,1), which   returns a list of length 1.  Implemented as First(ShuffleN(list,1))

\sphinxAtStartPar
\sphinxstylestrong{Usage:}

\begin{sphinxVerbatim}[commandchars=\\\{\}]
\PYG{k}{define}\PYG{+w}{ }\PYG{n+nf}{Sample}\PYG{p}{(}\PYG{p}{.}\PYG{p}{.}\PYG{p}{.}\PYG{p}{)}
\end{sphinxVerbatim}

\sphinxAtStartPar
\sphinxstylestrong{Example:}

\begin{sphinxVerbatim}[commandchars=\\\{\}]
\PYG{n+nf}{Sample}\PYG{p}{(}\PYG{p}{[}\PYG{l+m+mi}{1}\PYG{p}{,}\PYG{l+m+mi}{1}\PYG{p}{,}\PYG{l+m+mi}{1}\PYG{p}{,}\PYG{l+m+mi}{2}\PYG{p}{,}\PYG{l+m+mi}{2}\PYG{p}{]}\PYG{p}{)}\PYG{+w}{     }\PYG{c+c1}{\PYGZsh{} Returns a single number}
\PYG{n+nf}{Sample}\PYG{p}{(}\PYG{p}{[}\PYG{l+m+mi}{1}\PYG{p}{,}\PYG{l+m+mi}{2}\PYG{p}{,}\PYG{l+m+mi}{3}\PYG{p}{,}\PYG{l+m+mi}{4}\PYG{p}{,}\PYG{l+m+mi}{5}\PYG{p}{,}\PYG{l+m+mi}{6}\PYG{p}{,}\PYG{l+m+mi}{7}\PYG{p}{]}\PYG{p}{)}\PYG{+w}{ }\PYG{c+c1}{\PYGZsh{} Returns a single number}
\end{sphinxVerbatim}

\sphinxAtStartPar
\sphinxstylestrong{See Also:}

\sphinxAtStartPar
\sphinxcode{\sphinxupquote{SeedRNG()}}, \sphinxcode{\sphinxupquote{Sample()}}, \sphinxcode{\sphinxupquote{ChooseN()}}, \sphinxcode{\sphinxupquote{SampleNWithReplacement()}}, \sphinxcode{\sphinxupquote{Subset()}}

\index{SampleN@\spxentry{SampleN}}\ignorespaces 

\subsection{SampleN()}
\label{\detokenize{reference/design:samplen}}\label{\detokenize{reference/design:index-16}}
\sphinxAtStartPar
\sphinxstylestrong{Description:}

\sphinxAtStartPar
Samples \sphinxcode{\sphinxupquote{\textless{}number\textgreater{}}} items from list, returning   a randomly\sphinxhyphen{} ordered list. Items are sampled without replacement, so   once an item is chosen it will not be chosen again. If   \sphinxcode{\sphinxupquote{\textless{}number\textgreater{}}} is larger than the length of the list, the entire   list is returned shuffled.  It differs from \sphinxcode{\sphinxupquote{ChooseN}} in that   \sphinxcode{\sphinxupquote{ChooseN}} returns items in the order they appeared in the   originial list.  It is implemented as \sphinxcode{\sphinxupquote{Shuffle(ChooseN())}}.

\sphinxAtStartPar
\sphinxstylestrong{Usage:}

\begin{sphinxVerbatim}[commandchars=\\\{\}]
\PYG{k}{define}\PYG{+w}{ }\PYG{n+nf}{SampleN}\PYG{p}{(}\PYG{p}{.}\PYG{p}{.}\PYG{p}{.}\PYG{p}{)}
\end{sphinxVerbatim}

\sphinxAtStartPar
\sphinxstylestrong{Example:}

\begin{sphinxVerbatim}[commandchars=\\\{\}]
\PYG{n+nf}{SampleN}\PYG{p}{(}\PYG{p}{[}\PYG{l+m+mi}{1}\PYG{p}{,}\PYG{l+m+mi}{1}\PYG{p}{,}\PYG{l+m+mi}{1}\PYG{p}{,}\PYG{l+m+mi}{2}\PYG{p}{,}\PYG{l+m+mi}{2}\PYG{p}{]}\PYG{p}{,}\PYG{+w}{ }\PYG{l+m+mi}{5}\PYG{p}{)}\PYG{+w}{     }\PYG{c+c1}{\PYGZsh{} Returns 5 numbers}
\PYG{n+nf}{SampleN}\PYG{p}{(}\PYG{p}{[}\PYG{l+m+mi}{1}\PYG{p}{,}\PYG{l+m+mi}{2}\PYG{p}{,}\PYG{l+m+mi}{3}\PYG{p}{,}\PYG{l+m+mi}{4}\PYG{p}{,}\PYG{l+m+mi}{5}\PYG{p}{,}\PYG{l+m+mi}{6}\PYG{p}{,}\PYG{l+m+mi}{7}\PYG{p}{]}\PYG{p}{,}\PYG{+w}{ }\PYG{l+m+mi}{3}\PYG{p}{)}\PYG{+w}{ }\PYG{c+c1}{\PYGZsh{} Returns 3 numbers}
\end{sphinxVerbatim}

\sphinxAtStartPar
\sphinxstylestrong{See Also:}

\sphinxAtStartPar
\sphinxcode{\sphinxupquote{ChooseN()}}, \sphinxcode{\sphinxupquote{SampleNWithReplacement()}}, \sphinxcode{\sphinxupquote{Subset()}}

\index{SampleNWithReplacement@\spxentry{SampleNWithReplacement}}\ignorespaces 

\subsection{SampleNWithReplacement()}
\label{\detokenize{reference/design:samplenwithreplacement}}\label{\detokenize{reference/design:index-17}}
\sphinxAtStartPar
\sphinxstylestrong{Description:}

\sphinxAtStartPar
\sphinxcode{\sphinxupquote{SampleNWithReplacement}} samples   \sphinxcode{\sphinxupquote{\textless{}number\textgreater{}}} items from \sphinxcode{\sphinxupquote{\textless{}list\textgreater{}}}, replacing after each draw   so that items can be sampled again.  \sphinxcode{\sphinxupquote{\textless{}number\textgreater{}}} can be larger   than the length of the list. It has no side effects on its   arguments.

\sphinxAtStartPar
\sphinxstylestrong{Usage:}

\begin{sphinxVerbatim}[commandchars=\\\{\}]
\PYG{k}{define}\PYG{+w}{ }\PYG{n+nf}{SampleNWithReplacement}\PYG{p}{(}\PYG{p}{.}\PYG{p}{.}\PYG{p}{.}\PYG{p}{)}
\end{sphinxVerbatim}

\sphinxAtStartPar
\sphinxstylestrong{Example:}

\begin{sphinxVerbatim}[commandchars=\\\{\}]
\PYG{n+nv}{x}\PYG{+w}{ }\PYG{o}{\PYGZlt{}\PYGZhy{}}\PYG{+w}{ }\PYG{n+nf}{Sequence}\PYG{p}{(}\PYG{l+m+mi}{1}\PYG{o}{:}\PYG{l+m+mi}{100}\PYG{p}{,}\PYG{l+m+mi}{1}\PYG{p}{)}
\PYG{n+nf}{SampleNWithReplacement}\PYG{p}{(}\PYG{n+nv}{x}\PYG{p}{,}\PYG{+w}{ }\PYG{l+m+mi}{10}\PYG{p}{)}
\PYG{c+c1}{\PYGZsh{} Produces 10 numbers between 1 and 100, possibly}
\PYG{c+c1}{\PYGZsh{} repeating some.}
\end{sphinxVerbatim}

\sphinxAtStartPar
\sphinxstylestrong{See Also:}

\sphinxAtStartPar
\sphinxcode{\sphinxupquote{SampleN()}}, \sphinxcode{\sphinxupquote{ChooseN()}}, \sphinxcode{\sphinxupquote{Subset()}}

\index{ShuffleRepeat@\spxentry{ShuffleRepeat}}\ignorespaces 

\subsection{ShuffleRepeat()}
\label{\detokenize{reference/design:shufflerepeat}}\label{\detokenize{reference/design:index-18}}
\sphinxAtStartPar
\sphinxstylestrong{Description:}

\sphinxAtStartPar
Randomly shuffles  \sphinxcode{\sphinxupquote{\textless{}list\textgreater{}}}, repeating \sphinxcode{\sphinxupquote{\textless{}n\textgreater{}}} times.  Shuffles  each iteration of the list separately, so you are guaranteed to go  through all elements of the list before you get another.  Returns a nested list.

\sphinxAtStartPar
\sphinxstylestrong{Usage:}

\begin{sphinxVerbatim}[commandchars=\\\{\}]
\PYG{k}{define}\PYG{+w}{ }\PYG{n+nf}{ShuffleRepeat}\PYG{p}{(}\PYG{p}{.}\PYG{p}{.}\PYG{p}{.}\PYG{p}{)}
\end{sphinxVerbatim}

\sphinxAtStartPar
\sphinxstylestrong{Example:}

\begin{sphinxVerbatim}[commandchars=\\\{\}]
\PYG{n+nf}{Print}\PYG{p}{(}\PYG{n+nf}{ShuffleRepeat}\PYG{p}{(}\PYG{p}{[}\PYG{l+m+mi}{1}\PYG{p}{,}\PYG{l+m+mi}{2}\PYG{p}{,}\PYG{l+m+mi}{3}\PYG{p}{,}\PYG{l+m+mi}{4}\PYG{p}{,}\PYG{l+m+mi}{5}\PYG{p}{]}\PYG{p}{)}\PYG{p}{,}\PYG{l+m+mi}{3}\PYG{p}{)}
\PYG{c+c1}{\PYGZsh{}\PYGZsh{}  Results might be anything, like:}
\PYG{c+c1}{\PYGZsh{}\PYGZsh{} [[5,3,2,1,4], [3,2,5,1,4], [1,4,5,3,2]]}


T\PYG{n+nv}{ypically}\PYG{p}{,}\PYG{+w}{ }\PYG{n+nv}{you}\PYG{+w}{ }\PYG{n+nv}{will}\PYG{+w}{ }\PYG{n+nv}{want}\PYG{+w}{ }\PYG{n+nv}{to}\PYG{+w}{ }\PYG{n+nv}{flatten}\PYG{+w}{ }\PYG{n+nv}{before}\PYG{+w}{ }\PYG{n+nv}{using}\PYG{o}{:}

\PYG{n+nv}{list}\PYG{+w}{ }\PYG{o}{\PYGZlt{}\PYGZhy{}}\PYG{+w}{  }\PYG{n+nf}{Flatten}\PYG{p}{(}\PYG{n+nf}{ShuffleRepeat}\PYG{p}{(}\PYG{p}{[}\PYG{l+m+mi}{1}\PYG{p}{,}\PYG{l+m+mi}{2}\PYG{p}{,}\PYG{l+m+mi}{3}\PYG{p}{]}\PYG{p}{,}\PYG{+w}{ }\PYG{l+m+mi}{5}\PYG{p}{)}\PYG{p}{)}
\end{sphinxVerbatim}

\sphinxAtStartPar
\sphinxstylestrong{See Also:}
\begin{description}
\sphinxlineitem{\sphinxcode{\sphinxupquote{Sort()}}, \sphinxcode{\sphinxupquote{SortBy()}}, \sphinxcode{\sphinxupquote{ShuffleRepeat()}},}
\sphinxAtStartPar
\sphinxcode{\sphinxupquote{ShuffleWithoutAdjacents()}}

\end{description}

\index{ShuffleWithoutAdjacents@\spxentry{ShuffleWithoutAdjacents}}\ignorespaces 

\subsection{ShuffleWithoutAdjacents()}
\label{\detokenize{reference/design:shufflewithoutadjacents}}\label{\detokenize{reference/design:index-19}}
\sphinxAtStartPar
\sphinxstylestrong{Description:}

\sphinxAtStartPar
Randomly shuffles  \sphinxcode{\sphinxupquote{\textless{}nested\sphinxhyphen{}list\textgreater{}}}, attempting to   create a list where the nested elements do not appear adjacently in   the new list. Returns a list that is flattened one level. It will   always return a shuffled list, but it is not guaranteed to return   one that has the non\sphinxhyphen{}adjecent structure specified, because this is   sometimes impossible or very difficult to do randomly.  Given small   enough non\sphinxhyphen{}adjacent constraints with enough fillers, it should be   able to find something satisfactory.

\sphinxAtStartPar
\sphinxstylestrong{Usage:}

\begin{sphinxVerbatim}[commandchars=\\\{\}]
\PYG{k}{define}\PYG{+w}{ }\PYG{n+nf}{ShuffleWithoutAdjacents}\PYG{p}{(}\PYG{p}{.}\PYG{p}{.}\PYG{p}{.}\PYG{p}{)}
\end{sphinxVerbatim}

\sphinxAtStartPar
\sphinxstylestrong{Example:}

\begin{sphinxVerbatim}[commandchars=\\\{\}]
\PYG{n+nf}{Print}\PYG{p}{(}\PYG{n+nf}{ShuffleWithoutAdjacents}\PYG{p}{(}\PYG{p}{[}\PYG{p}{[}\PYG{l+m+mi}{1}\PYG{p}{,}\PYG{l+m+mi}{2}\PYG{p}{,}\PYG{l+m+mi}{3}\PYG{p}{]}\PYG{p}{,}
\PYG{+w}{                               }\PYG{p}{[}\PYG{l+m+mi}{4}\PYG{p}{,}\PYG{l+m+mi}{5}\PYG{p}{,}\PYG{l+m+mi}{6}\PYG{p}{]}\PYG{p}{,}
\PYG{+w}{                               }\PYG{p}{[}\PYG{l+m+mi}{7}\PYG{p}{,}\PYG{l+m+mi}{8}\PYG{p}{,}\PYG{l+m+mi}{9}\PYG{p}{]}\PYG{p}{]}\PYG{p}{)}
\PYG{c+c1}{\PYGZsh{}\PYGZsh{} Example Output:}
\PYG{c+c1}{\PYGZsh{}\PYGZsh{} [8, 5, 2, 7, 4, 1, 6, 9, 3]}
\PYG{c+c1}{\PYGZsh{}\PYGZsh{} [7, 4, 8, 1, 9, 2, 5, 3, 6]}

\PYG{c+c1}{\PYGZsh{}\PYGZsh{} Non\PYGZhy{}nested items are shuffled without constraint}
\PYG{n+nf}{Print}\PYG{p}{(}\PYG{n+nf}{ShuffleWithoutAdjacents}\PYG{p}{(}\PYG{p}{[}\PYG{p}{[}\PYG{l+m+mi}{1}\PYG{p}{,}\PYG{l+m+mi}{2}\PYG{p}{,}\PYG{l+m+mi}{3}\PYG{p}{]}\PYG{p}{,}
\PYG{+w}{                              }\PYG{l+m+mi}{11}\PYG{p}{,}\PYG{l+m+mi}{12}\PYG{p}{,}\PYG{l+m+mi}{13}\PYG{p}{,}\PYG{l+m+mi}{14}\PYG{p}{,}\PYG{l+m+mi}{15}\PYG{p}{,}\PYG{l+m+mi}{16}\PYG{p}{]}\PYG{p}{)}\PYG{p}{)}
\PYG{c+c1}{\PYGZsh{}\PYGZsh{} output: [13, 11, 2, 14, 3, 15, 1, 16, 12]}
\PYG{c+c1}{\PYGZsh{}\PYGZsh{}         [13, 12, 2, 16, 15, 11, 1, 14, 3]}
\PYG{c+c1}{\PYGZsh{}\PYGZsh{}         [11, 1, 15, 2, 12, 16, 14, 13, 3]}

\PYG{c+c1}{\PYGZsh{}\PYGZsh{} Sometimes the constraints cannot be satisfied.}
\PYG{c+c1}{\PYGZsh{}\PYGZsh{} 9 will always appear in position 2}
\PYG{n+nf}{Print}\PYG{p}{(}\PYG{n+nf}{ShuffleWithoutAdjacents}\PYG{p}{(}\PYG{p}{[}\PYG{p}{[}\PYG{l+m+mi}{1}\PYG{p}{,}\PYG{l+m+mi}{2}\PYG{p}{,}\PYG{l+m+mi}{3}\PYG{p}{]}\PYG{p}{,}\PYG{+w}{ }\PYG{l+m+mi}{9}\PYG{p}{]}\PYG{p}{)}
\PYG{c+c1}{\PYGZsh{}\PYGZsh{} output: [3, 9, 1, 2]}
\PYG{c+c1}{\PYGZsh{}\PYGZsh{}         [2, 9, 3, 1]}
\PYG{c+c1}{\PYGZsh{}\PYGZsh{}         [3, 9, 2, 1]}
\end{sphinxVerbatim}

\sphinxAtStartPar
\sphinxstylestrong{See Also:}
\begin{description}
\sphinxlineitem{\sphinxcode{\sphinxupquote{Shuffle()}}, \sphinxcode{\sphinxupquote{Sort()}}, \sphinxcode{\sphinxupquote{SortBy()}},}
\sphinxAtStartPar
\sphinxcode{\sphinxupquote{ShuffleRepeat()}}, \sphinxcode{\sphinxupquote{ShuffleWithoutAdjacents()}}

\end{description}

\index{Subset@\spxentry{Subset}}\ignorespaces 

\subsection{Subset()}
\label{\detokenize{reference/design:subset}}\label{\detokenize{reference/design:index-20}}
\sphinxAtStartPar
\sphinxstyleemphasis{returns a subset of items from a list}

\sphinxAtStartPar
\sphinxstylestrong{Description:}

\sphinxAtStartPar
Extracts a subset of items from another list,   returning a new list that includes items from the original list only   once and in their original orders.  Item indices in the second   argument that do not exist in the first argument are ignored.  It   has no side effects on its arguments.

\sphinxAtStartPar
\sphinxstylestrong{Usage:}

\begin{sphinxVerbatim}[commandchars=\\\{\}]
\PYG{k}{define}\PYG{+w}{ }\PYG{n+nf}{Subset}\PYG{p}{(}\PYG{p}{.}\PYG{p}{.}\PYG{p}{.}\PYG{p}{)}
\end{sphinxVerbatim}

\sphinxAtStartPar
\sphinxstylestrong{Example:}

\begin{sphinxVerbatim}[commandchars=\\\{\}]
\PYG{n+nf}{Subset}\PYG{p}{(}\PYG{p}{[}\PYG{l+m+mi}{1}\PYG{p}{,}\PYG{l+m+mi}{2}\PYG{p}{,}\PYG{l+m+mi}{3}\PYG{p}{,}\PYG{l+m+mi}{4}\PYG{p}{,}\PYG{l+m+mi}{5}\PYG{p}{,}\PYG{l+m+mi}{6}\PYG{p}{]}\PYG{p}{,}\PYG{p}{[}\PYG{l+m+mi}{5}\PYG{p}{,}\PYG{l+m+mi}{3}\PYG{p}{,}\PYG{l+m+mi}{1}\PYG{p}{,}\PYG{l+m+mi}{1}\PYG{p}{]}\PYG{p}{)}\PYG{+w}{      }\PYG{c+c1}{\PYGZsh{} == [1,3,5]}
\PYG{n+nf}{Subset}\PYG{p}{(}\PYG{p}{[}\PYG{l+m+mi}{1}\PYG{p}{,}\PYG{l+m+mi}{2}\PYG{p}{,}\PYG{l+m+mi}{3}\PYG{p}{,}\PYG{l+m+mi}{4}\PYG{p}{,}\PYG{l+m+mi}{5}\PYG{p}{]}\PYG{p}{,}\PYG{+w}{ }\PYG{p}{[}\PYG{l+m+mi}{23}\PYG{p}{,}\PYG{l+m+mi}{4}\PYG{p}{,}\PYG{l+m+mi}{2}\PYG{p}{]}\PYG{p}{)}\PYG{+w}{                }\PYG{c+c1}{\PYGZsh{} == [2,4]}
\end{sphinxVerbatim}

\sphinxAtStartPar
\sphinxstylestrong{See Also:}

\sphinxAtStartPar
\sphinxcode{\sphinxupquote{SubList()}}, \sphinxcode{\sphinxupquote{ExtractItems()}}, \sphinxcode{\sphinxupquote{SampleN()}}


\subsection{Functions Pending Documentation}
\label{\detokenize{reference/design:functions-pending-documentation}}
\index{FindToken@\spxentry{FindToken}}\ignorespaces 

\subsection{FindToken()}
\label{\detokenize{reference/design:findtoken}}\label{\detokenize{reference/design:index-21}}
\sphinxAtStartPar
\sphinxstyleemphasis{Recursively searches for a token in a nested list structure}

\sphinxAtStartPar
\sphinxstylestrong{Description:}

\sphinxAtStartPar
Searches recursively through a possibly nested list to find a specific token (value). Returns the index (1\sphinxhyphen{}based) of the first occurrence of the token found. If the token is in a nested sublist, it searches that sublist recursively. Returns 0 if the token is not found. Useful for searching complex nested data structures.

\sphinxAtStartPar
\sphinxstylestrong{Usage:}

\begin{sphinxVerbatim}[commandchars=\\\{\}]
\PYG{k}{define}\PYG{+w}{ }\PYG{n+nf}{FindToken}\PYG{p}{(}\PYG{n+nv}{token}\PYG{p}{,}\PYG{+w}{ }\PYG{n+nv}{nestedlist}\PYG{p}{)}
\end{sphinxVerbatim}

\sphinxAtStartPar
\sphinxstylestrong{Example:}

\begin{sphinxVerbatim}[commandchars=\\\{\}]
\PYG{c+c1}{\PYGZsh{} Search in a flat list}
\PYG{n+nv}{data}\PYG{+w}{ }\PYG{o}{\PYGZlt{}\PYGZhy{}}\PYG{+w}{ }\PYG{p}{[}\PYG{l+s+s2}{\PYGZdq{}apple\PYGZdq{}}\PYG{p}{,}\PYG{+w}{ }\PYG{l+s+s2}{\PYGZdq{}banana\PYGZdq{}}\PYG{p}{,}\PYG{+w}{ }\PYG{l+s+s2}{\PYGZdq{}cherry\PYGZdq{}}\PYG{p}{]}
\PYG{n+nv}{index}\PYG{+w}{ }\PYG{o}{\PYGZlt{}\PYGZhy{}}\PYG{+w}{ }\PYG{n+nf}{FindToken}\PYG{p}{(}\PYG{l+s+s2}{\PYGZdq{}banana\PYGZdq{}}\PYG{p}{,}\PYG{+w}{ }\PYG{n+nv}{data}\PYG{p}{)}
\PYG{n+nf}{Print}\PYG{p}{(}\PYG{n+nv}{index}\PYG{p}{)}
\PYG{c+c1}{\PYGZsh{} Result: 2}

\PYG{c+c1}{\PYGZsh{} Search in nested list}
\PYG{n+nv}{nested}\PYG{+w}{ }\PYG{o}{\PYGZlt{}\PYGZhy{}}\PYG{+w}{ }\PYG{p}{[}\PYG{p}{[}\PYG{l+s+s2}{\PYGZdq{}a\PYGZdq{}}\PYG{p}{,}\PYG{+w}{ }\PYG{l+s+s2}{\PYGZdq{}b\PYGZdq{}}\PYG{p}{]}\PYG{p}{,}\PYG{+w}{ }\PYG{p}{[}\PYG{l+s+s2}{\PYGZdq{}c\PYGZdq{}}\PYG{p}{,}\PYG{+w}{ }\PYG{l+s+s2}{\PYGZdq{}d\PYGZdq{}}\PYG{p}{]}\PYG{p}{,}\PYG{+w}{ }\PYG{p}{[}\PYG{l+s+s2}{\PYGZdq{}e\PYGZdq{}}\PYG{p}{,}\PYG{+w}{ }\PYG{l+s+s2}{\PYGZdq{}f\PYGZdq{}}\PYG{p}{]}\PYG{p}{]}
\PYG{n+nv}{index}\PYG{+w}{ }\PYG{o}{\PYGZlt{}\PYGZhy{}}\PYG{+w}{ }\PYG{n+nf}{FindToken}\PYG{p}{(}\PYG{l+s+s2}{\PYGZdq{}d\PYGZdq{}}\PYG{p}{,}\PYG{+w}{ }\PYG{n+nv}{nested}\PYG{p}{)}
\PYG{n+nf}{Print}\PYG{p}{(}\PYG{n+nv}{index}\PYG{p}{)}
\PYG{c+c1}{\PYGZsh{} Result: 2 (found in second sublist)}

\PYG{c+c1}{\PYGZsh{} Token not found}
\PYG{n+nv}{index}\PYG{+w}{ }\PYG{o}{\PYGZlt{}\PYGZhy{}}\PYG{+w}{ }\PYG{n+nf}{FindToken}\PYG{p}{(}\PYG{l+s+s2}{\PYGZdq{}z\PYGZdq{}}\PYG{p}{,}\PYG{+w}{ }\PYG{n+nv}{data}\PYG{p}{)}
\PYG{n+nf}{Print}\PYG{p}{(}\PYG{n+nv}{index}\PYG{p}{)}
\PYG{c+c1}{\PYGZsh{} Result: 0}
\end{sphinxVerbatim}

\sphinxAtStartPar
\sphinxstylestrong{See Also:}

\sphinxAtStartPar
\sphinxcode{\sphinxupquote{IsMember()}}, \sphinxcode{\sphinxupquote{Match()}}, \sphinxcode{\sphinxupquote{Filter()}}

\index{Reverse@\spxentry{Reverse}}\ignorespaces 

\subsection{Reverse()}
\label{\detokenize{reference/design:reverse}}\label{\detokenize{reference/design:index-22}}
\sphinxAtStartPar
\sphinxstyleemphasis{Reverses the order of elements in a list}

\sphinxAtStartPar
\sphinxstylestrong{Description:}

\sphinxAtStartPar
Returns a new list containing all elements from the input list in reverse order. The original list is unchanged. This function is useful for reversing presentation order, creating backwards sequences, or implementing stack\sphinxhyphen{}like data structures.

\sphinxAtStartPar
\sphinxstylestrong{Usage:}

\begin{sphinxVerbatim}[commandchars=\\\{\}]
\PYG{k}{define}\PYG{+w}{ }\PYG{n+nf}{Reverse}\PYG{p}{(}\PYG{n+nv}{list}\PYG{p}{)}
\end{sphinxVerbatim}

\sphinxAtStartPar
\sphinxstylestrong{Example:}

\begin{sphinxVerbatim}[commandchars=\\\{\}]
\PYG{n+nv}{x}\PYG{+w}{ }\PYG{o}{\PYGZlt{}\PYGZhy{}}\PYG{+w}{ }\PYG{p}{[}\PYG{l+m+mi}{1}\PYG{p}{,}\PYG{+w}{ }\PYG{l+m+mi}{2}\PYG{p}{,}\PYG{+w}{ }\PYG{l+m+mi}{3}\PYG{p}{,}\PYG{+w}{ }\PYG{l+m+mi}{4}\PYG{p}{,}\PYG{+w}{ }\PYG{l+m+mi}{5}\PYG{p}{]}
\PYG{n+nv}{y}\PYG{+w}{ }\PYG{o}{\PYGZlt{}\PYGZhy{}}\PYG{+w}{ }\PYG{n+nf}{Reverse}\PYG{p}{(}\PYG{n+nv}{x}\PYG{p}{)}
\PYG{n+nf}{Print}\PYG{p}{(}\PYG{n+nv}{y}\PYG{p}{)}
\PYG{c+c1}{\PYGZsh{} Result: [5, 4, 3, 2, 1]}

\PYG{n+nv}{words}\PYG{+w}{ }\PYG{o}{\PYGZlt{}\PYGZhy{}}\PYG{+w}{ }\PYG{p}{[}\PYG{l+s+s2}{\PYGZdq{}first\PYGZdq{}}\PYG{p}{,}\PYG{+w}{ }\PYG{l+s+s2}{\PYGZdq{}second\PYGZdq{}}\PYG{p}{,}\PYG{+w}{ }\PYG{l+s+s2}{\PYGZdq{}third\PYGZdq{}}\PYG{p}{]}
\PYG{n+nf}{Print}\PYG{p}{(}\PYG{n+nf}{Reverse}\PYG{p}{(}\PYG{n+nv}{words}\PYG{p}{)}\PYG{p}{)}
\PYG{c+c1}{\PYGZsh{} Result: [third, second, first]}
\end{sphinxVerbatim}

\sphinxAtStartPar
\sphinxstylestrong{See Also:}

\sphinxAtStartPar
\sphinxcode{\sphinxupquote{Rotate()}}, \sphinxcode{\sphinxupquote{Shuffle()}}, \sphinxcode{\sphinxupquote{Sort()}}

\sphinxstepscope


\section{Graphics Library \sphinxhyphen{} Advanced Graphics}
\label{\detokenize{reference/graphics:graphics-library-advanced-graphics}}\label{\detokenize{reference/graphics::doc}}
\sphinxAtStartPar
This library contains advanced graphics functions for creating complex visual stimuli and shapes.

\begin{sphinxShadowBox}
\sphinxstyletopictitle{Function Index}
\begin{itemize}
\item {} 
\sphinxAtStartPar
\phantomsection\label{\detokenize{reference/graphics:id7}}{\hyperref[\detokenize{reference/graphics:blocke}]{\sphinxcrossref{BlockE()}}}

\item {} 
\sphinxAtStartPar
\phantomsection\label{\detokenize{reference/graphics:id8}}{\hyperref[\detokenize{reference/graphics:convexhull}]{\sphinxcrossref{ConvexHull()}}}

\item {} 
\sphinxAtStartPar
\phantomsection\label{\detokenize{reference/graphics:id9}}{\hyperref[\detokenize{reference/graphics:getangle}]{\sphinxcrossref{GetAngle()}}}

\item {} 
\sphinxAtStartPar
\phantomsection\label{\detokenize{reference/graphics:id10}}{\hyperref[\detokenize{reference/graphics:getangle3}]{\sphinxcrossref{GetAngle3()}}}

\item {} 
\sphinxAtStartPar
\phantomsection\label{\detokenize{reference/graphics:id11}}{\hyperref[\detokenize{reference/graphics:kaniszapolygon}]{\sphinxcrossref{KaniszaPolygon()}}}

\item {} 
\sphinxAtStartPar
\phantomsection\label{\detokenize{reference/graphics:id12}}{\hyperref[\detokenize{reference/graphics:kaniszasquare}]{\sphinxcrossref{KaniszaSquare()}}}

\item {} 
\sphinxAtStartPar
\phantomsection\label{\detokenize{reference/graphics:id13}}{\hyperref[\detokenize{reference/graphics:layoutgrid}]{\sphinxcrossref{LayoutGrid()}}}

\item {} 
\sphinxAtStartPar
\phantomsection\label{\detokenize{reference/graphics:id14}}{\hyperref[\detokenize{reference/graphics:makeattneave}]{\sphinxcrossref{MakeAttneave()}}}

\item {} 
\sphinxAtStartPar
\phantomsection\label{\detokenize{reference/graphics:id15}}{\hyperref[\detokenize{reference/graphics:makegabor}]{\sphinxcrossref{MakeGabor()}}}

\item {} 
\sphinxAtStartPar
\phantomsection\label{\detokenize{reference/graphics:id16}}{\hyperref[\detokenize{reference/graphics:makegraph}]{\sphinxcrossref{MakeGraph()}}}

\item {} 
\sphinxAtStartPar
\phantomsection\label{\detokenize{reference/graphics:id17}}{\hyperref[\detokenize{reference/graphics:makengonpoints}]{\sphinxcrossref{MakeNGonPoints()}}}

\item {} 
\sphinxAtStartPar
\phantomsection\label{\detokenize{reference/graphics:id18}}{\hyperref[\detokenize{reference/graphics:makestarpoints}]{\sphinxcrossref{MakeStarPoints()}}}

\item {} 
\sphinxAtStartPar
\phantomsection\label{\detokenize{reference/graphics:id19}}{\hyperref[\detokenize{reference/graphics:nonoverlaplayout}]{\sphinxcrossref{NonOverlapLayout()}}}

\item {} 
\sphinxAtStartPar
\phantomsection\label{\detokenize{reference/graphics:id20}}{\hyperref[\detokenize{reference/graphics:plus}]{\sphinxcrossref{Plus()}}}

\item {} 
\sphinxAtStartPar
\phantomsection\label{\detokenize{reference/graphics:id21}}{\hyperref[\detokenize{reference/graphics:reflectpoints}]{\sphinxcrossref{ReflectPoints()}}}

\item {} 
\sphinxAtStartPar
\phantomsection\label{\detokenize{reference/graphics:id22}}{\hyperref[\detokenize{reference/graphics:resetcanvas}]{\sphinxcrossref{ResetCanvas()}}}

\item {} 
\sphinxAtStartPar
\phantomsection\label{\detokenize{reference/graphics:id23}}{\hyperref[\detokenize{reference/graphics:rgbtohsv}]{\sphinxcrossref{RGBtoHSV()}}}

\item {} 
\sphinxAtStartPar
\phantomsection\label{\detokenize{reference/graphics:id24}}{\hyperref[\detokenize{reference/graphics:rotatepoints}]{\sphinxcrossref{RotatePoints()}}}

\item {} 
\sphinxAtStartPar
\phantomsection\label{\detokenize{reference/graphics:id25}}{\hyperref[\detokenize{reference/graphics:segmentsintersect}]{\sphinxcrossref{SegmentsIntersect()}}}

\item {} 
\sphinxAtStartPar
\phantomsection\label{\detokenize{reference/graphics:id26}}{\hyperref[\detokenize{reference/graphics:toright}]{\sphinxcrossref{ToRight()}}}

\item {} 
\sphinxAtStartPar
\phantomsection\label{\detokenize{reference/graphics:id27}}{\hyperref[\detokenize{reference/graphics:zoompoints}]{\sphinxcrossref{ZoomPoints()}}}

\item {} 
\sphinxAtStartPar
\phantomsection\label{\detokenize{reference/graphics:id28}}{\hyperref[\detokenize{reference/graphics:functions-pending-documentation}]{\sphinxcrossref{Functions Pending Documentation}}}

\item {} 
\sphinxAtStartPar
\phantomsection\label{\detokenize{reference/graphics:id29}}{\hyperref[\detokenize{reference/graphics:getmindist}]{\sphinxcrossref{GetMinDist()}}}

\item {} 
\sphinxAtStartPar
\phantomsection\label{\detokenize{reference/graphics:id30}}{\hyperref[\detokenize{reference/graphics:hideobject}]{\sphinxcrossref{HideObject()}}}

\item {} 
\sphinxAtStartPar
\phantomsection\label{\detokenize{reference/graphics:id31}}{\hyperref[\detokenize{reference/graphics:landoltring}]{\sphinxcrossref{LandoltRing()}}}

\item {} 
\sphinxAtStartPar
\phantomsection\label{\detokenize{reference/graphics:id32}}{\hyperref[\detokenize{reference/graphics:maketable}]{\sphinxcrossref{MakeTable()}}}

\item {} 
\sphinxAtStartPar
\phantomsection\label{\detokenize{reference/graphics:id33}}{\hyperref[\detokenize{reference/graphics:showobject}]{\sphinxcrossref{ShowObject()}}}

\item {} 
\sphinxAtStartPar
\phantomsection\label{\detokenize{reference/graphics:id34}}{\hyperref[\detokenize{reference/graphics:functions-under-investigation}]{\sphinxcrossref{Functions Under Investigation}}}

\item {} 
\sphinxAtStartPar
\phantomsection\label{\detokenize{reference/graphics:id35}}{\hyperref[\detokenize{reference/graphics:thickline2}]{\sphinxcrossref{ThickLine2()}}}

\end{itemize}
\end{sphinxShadowBox}

\index{BlockE@\spxentry{BlockE}}\ignorespaces 

\subsection{BlockE()}
\label{\detokenize{reference/graphics:blocke}}\label{\detokenize{reference/graphics:index-0}}
\sphinxAtStartPar
\sphinxstyleemphasis{Creates a block E as a useable polygon which can be added to a window directly.}

\sphinxAtStartPar
\sphinxstylestrong{Description:}

\sphinxAtStartPar
Creates a polygon in the shape of a  block E, pointing in one of four directions. Arguments include position in window.
\begin{itemize}
\item {} 
\sphinxAtStartPar
\sphinxcode{\sphinxupquote{\textless{}x\textgreater{}}} and \sphinxcode{\sphinxupquote{\textless{}y\textgreater{}}} is the position of the center

\item {} 
\sphinxAtStartPar
\sphinxcode{\sphinxupquote{\textless{}h\textgreater{}}} and \sphinxcode{\sphinxupquote{\textless{}w\textgreater{}}} or the size of the E in pixels

\item {} 
\sphinxAtStartPar
\sphinxcode{\sphinxupquote{\textless{}thickness\textgreater{}}} thickness of the E

\item {} 
\sphinxAtStartPar
\sphinxcode{\sphinxupquote{\textless{}direction\textgreater{}}} specifies which way the E points:  1=right,   2=down, 3=left, 4=up.

\item {} 
\sphinxAtStartPar
\sphinxcode{\sphinxupquote{\textless{}color\textgreater{}}} is a color object (not just the name)

\sphinxAtStartPar
Like other drawn objects, the Block E must then be added to the window to appear.

\end{itemize}

\sphinxAtStartPar
\sphinxstylestrong{Usage:}

\begin{sphinxVerbatim}[commandchars=\\\{\}]
\PYG{k}{define}\PYG{+w}{ }\PYG{n+nf}{BlockE}\PYG{p}{(}\PYG{p}{.}\PYG{p}{.}\PYG{p}{.}\PYG{p}{)}
\end{sphinxVerbatim}

\sphinxAtStartPar
\sphinxstylestrong{Example:}

\begin{sphinxVerbatim}[commandchars=\\\{\}]
\PYG{n+nv}{win}\PYG{+w}{ }\PYG{o}{\PYGZlt{}\PYGZhy{}}\PYG{+w}{ }\PYG{n+nf}{MakeWindow}\PYG{p}{(}\PYG{p}{)}
\PYG{n+nv}{e1}\PYG{+w}{ }\PYG{o}{\PYGZlt{}\PYGZhy{}}\PYG{+w}{ }\PYG{n+nf}{BlockE}\PYG{p}{(}\PYG{l+m+mi}{100}\PYG{p}{,}\PYG{l+m+mi}{100}\PYG{p}{,}\PYG{l+m+mi}{40}\PYG{p}{,}\PYG{l+m+mi}{80}\PYG{p}{,}\PYG{l+m+mi}{10}\PYG{p}{,}\PYG{l+m+mi}{1}\PYG{p}{,}\PYG{n+nf}{MakeColor}\PYG{p}{(}\PYG{l+s+s2}{\PYGZdq{}black\PYGZdq{}}\PYG{p}{)}\PYG{p}{)}
\PYG{n+nf}{AddObject}\PYG{p}{(}\PYG{n+nv}{e1}\PYG{p}{,}\PYG{n+nv}{win}\PYG{p}{)}
\PYG{n+nf}{Draw}\PYG{p}{(}\PYG{p}{)}
\end{sphinxVerbatim}

\sphinxAtStartPar
\sphinxstylestrong{See Also:}

\sphinxAtStartPar
\sphinxcode{\sphinxupquote{Plus()}}, \sphinxcode{\sphinxupquote{Polygon()}}, \sphinxcode{\sphinxupquote{MakeStarPoints()}},
\sphinxcode{\sphinxupquote{MakeNGonPoints()}}

\index{ConvexHull@\spxentry{ConvexHull}}\ignorespaces 

\subsection{ConvexHull()}
\label{\detokenize{reference/graphics:convexhull}}\label{\detokenize{reference/graphics:index-1}}
\sphinxAtStartPar
\sphinxstyleemphasis{Returns a convex subset of points for a set}

\sphinxAtStartPar
\sphinxstylestrong{Description:}

\sphinxAtStartPar
Computes the convex hull of a set of {[}x,y{]}   points. It returns a set of points that forms the convex hull, with   the first and last point identical.  A convex hull is the set of   outermost points, such that a polygon connecting just those points   will encompass all other points, and such that no angle is acute.   It is used in MakeAttneave.

\sphinxAtStartPar
\sphinxstylestrong{Usage:}

\begin{sphinxVerbatim}[commandchars=\\\{\}]
\PYG{k}{define}\PYG{+w}{ }\PYG{n+nf}{ConvexHull}\PYG{p}{(}\PYG{p}{.}\PYG{p}{.}\PYG{p}{.}\PYG{p}{)}
\end{sphinxVerbatim}

\sphinxAtStartPar
\sphinxstylestrong{Example:}

\begin{sphinxVerbatim}[commandchars=\\\{\}]
\PYG{n+nv}{pts}\PYG{+w}{ }\PYG{o}{\PYGZlt{}\PYGZhy{}}\PYG{+w}{ }\PYG{p}{[}\PYG{p}{[}\PYG{l+m+mf}{0.579081}\PYG{p}{,}\PYG{+w}{ }\PYG{l+m+mf}{0.0327737}\PYG{p}{]}\PYG{p}{,}
\PYG{+w}{         }\PYG{p}{[}\PYG{l+m+mf}{0.0536094}\PYG{p}{,}\PYG{+w}{ }\PYG{l+m+mf}{0.378258}\PYG{p}{]}\PYG{p}{,}
\PYG{+w}{         }\PYG{p}{[}\PYG{l+m+mf}{0.239628}\PYG{p}{,}\PYG{+w}{ }\PYG{l+m+mf}{0.187751}\PYG{p}{]}\PYG{p}{,}
\PYG{+w}{         }\PYG{p}{[}\PYG{l+m+mf}{0.940625}\PYG{p}{,}\PYG{+w}{ }\PYG{l+m+mf}{0.26526}\PYG{p}{]}\PYG{p}{,}
\PYG{+w}{         }\PYG{p}{[}\PYG{l+m+mf}{0.508748}\PYG{p}{,}\PYG{+w}{ }\PYG{l+m+mf}{0.840846}\PYG{p}{]}\PYG{p}{,}
\PYG{+w}{         }\PYG{p}{[}\PYG{l+m+mf}{0.352604}\PYG{p}{,}\PYG{+w}{ }\PYG{l+m+mf}{0.200193}\PYG{p}{]}\PYG{p}{,}
\PYG{+w}{         }\PYG{p}{[}\PYG{l+m+mf}{0.38684}\PYG{p}{,}\PYG{+w}{ }\PYG{l+m+mf}{0.212413}\PYG{p}{]}\PYG{p}{,}
\PYG{+w}{         }\PYG{p}{[}\PYG{l+m+mf}{0.00114761}\PYG{p}{,}\PYG{+w}{ }\PYG{l+m+mf}{0.768165}\PYG{p}{]}\PYG{p}{,}
\PYG{+w}{         }\PYG{p}{[}\PYG{l+m+mf}{0.432963}\PYG{p}{,}\PYG{+w}{ }\PYG{l+m+mf}{0.629412}\PYG{p}{]}\PYG{p}{]}
\PYG{+w}{  }\PYG{n+nf}{Print}\PYG{p}{(}\PYG{n+nf}{ConvexHull}\PYG{p}{(}\PYG{n+nv}{pts}\PYG{p}{)}\PYG{p}{)}



\PYG{n+nv}{output}\PYG{o}{:}

\PYG{p}{[}\PYG{p}{[}\PYG{l+m+mf}{0.940625}\PYG{p}{,}\PYG{+w}{ }\PYG{l+m+mf}{0.26526}\PYG{p}{]}
\PYG{p}{,}\PYG{+w}{ }\PYG{p}{[}\PYG{l+m+mf}{0.508748}\PYG{p}{,}\PYG{+w}{ }\PYG{l+m+mf}{0.840846}\PYG{p}{]}
\PYG{p}{,}\PYG{+w}{ }\PYG{p}{[}\PYG{l+m+mf}{0.00114761}\PYG{p}{,}\PYG{+w}{ }\PYG{l+m+mf}{0.768165}\PYG{p}{]}
\PYG{p}{,}\PYG{+w}{ }\PYG{p}{[}\PYG{l+m+mf}{0.0536094}\PYG{p}{,}\PYG{+w}{ }\PYG{l+m+mf}{0.378258}\PYG{p}{]}
\PYG{p}{,}\PYG{+w}{ }\PYG{p}{[}\PYG{l+m+mf}{0.239628}\PYG{p}{,}\PYG{+w}{ }\PYG{l+m+mf}{0.187751}\PYG{p}{]}
\PYG{p}{,}\PYG{+w}{ }\PYG{p}{[}\PYG{l+m+mf}{0.579081}\PYG{p}{,}\PYG{+w}{ }\PYG{l+m+mf}{0.0327737}\PYG{p}{]}
\PYG{p}{,}\PYG{+w}{ }\PYG{p}{[}\PYG{l+m+mf}{0.940625}\PYG{p}{,}\PYG{+w}{ }\PYG{l+m+mf}{0.26526}\PYG{p}{]}
\end{sphinxVerbatim}

\sphinxAtStartPar
\sphinxstylestrong{See Also:}

\sphinxAtStartPar
\sphinxcode{\sphinxupquote{MakeAttneave}},

\index{GetAngle@\spxentry{GetAngle}}\ignorespaces 

\subsection{GetAngle()}
\label{\detokenize{reference/graphics:getangle}}\label{\detokenize{reference/graphics:index-2}}
\sphinxAtStartPar
\sphinxstyleemphasis{Returns the angle in degrees of a vector.}

\sphinxAtStartPar
\sphinxstylestrong{Description:}

\sphinxAtStartPar
Gets  an angle (in degrees) from (0,0) of an x,y coordinate

\sphinxAtStartPar
\sphinxstylestrong{Usage:}

\begin{sphinxVerbatim}[commandchars=\\\{\}]
\PYG{k}{define}\PYG{+w}{ }\PYG{n+nf}{GetAngle}\PYG{p}{(}\PYG{p}{.}\PYG{p}{.}\PYG{p}{.}\PYG{p}{)}
\end{sphinxVerbatim}

\sphinxAtStartPar
\sphinxstylestrong{Example:}

\begin{sphinxVerbatim}[commandchars=\\\{\}]
\PYG{c+c1}{\PYGZsh{}\PYGZsh{}point sprite in the direction of a click}
\PYG{n+nv}{sprite}\PYG{+w}{ }\PYG{o}{\PYGZlt{}\PYGZhy{}}\PYG{+w}{ }\PYG{n+nf}{LoadImage}\PYG{p}{(}\PYG{l+s+s2}{\PYGZdq{}car.png\PYGZdq{}}\PYG{p}{)}
\PYG{n+nf}{AddObject}\PYG{p}{(}\PYG{n+nv}{sprite}\PYG{p}{,}\PYG{n+nv+vg}{gWin}\PYG{p}{)}
\PYG{n+nf}{Move}\PYG{p}{(}\PYG{n+nv}{sprite}\PYG{p}{,}\PYG{l+m+mi}{300}\PYG{p}{,}\PYG{l+m+mi}{300}\PYG{p}{)}
\PYG{n+nv}{xy}\PYG{+w}{ }\PYG{o}{\PYGZlt{}\PYGZhy{}}\PYG{+w}{ }\PYG{n+nf}{WaitForDownClick}\PYG{p}{(}\PYG{p}{)}
\PYG{n+nv}{newangle}\PYG{+w}{ }\PYG{o}{\PYGZlt{}\PYGZhy{}}\PYG{+w}{ }\PYG{n+nf}{GetAngle}\PYG{p}{(}\PYG{n+nf}{First}\PYG{p}{(}\PYG{n+nv}{xy}\PYG{p}{)}\PYG{o}{\PYGZhy{}}\PYG{l+m+mi}{300}\PYG{p}{,}\PYG{n+nf}{Second}\PYG{p}{(}\PYG{n+nv}{xy}\PYG{p}{)}\PYG{o}{\PYGZhy{}}\PYG{l+m+mi}{300}\PYG{p}{)}
\PYG{n+nv}{sprite.rotation}\PYG{+w}{ }\PYG{o}{\PYGZlt{}\PYGZhy{}}\PYG{+w}{ }\PYG{n+nv}{newangle}
\PYG{n+nf}{Draw}\PYG{p}{(}\PYG{p}{)}
\end{sphinxVerbatim}

\sphinxAtStartPar
\sphinxstylestrong{See Also:}

\sphinxAtStartPar
\sphinxcode{\sphinxupquote{DegtoRad}}, \sphinxcode{\sphinxupquote{RadToDeg}}

\index{GetAngle3@\spxentry{GetAngle3}}\ignorespaces 

\subsection{GetAngle3()}
\label{\detokenize{reference/graphics:getangle3}}\label{\detokenize{reference/graphics:index-3}}
\sphinxAtStartPar
\sphinxstyleemphasis{Gets angle abc.}

\sphinxAtStartPar
\sphinxstylestrong{Description:}

\sphinxAtStartPar
Gets  an angle (in radians) of abc.

\sphinxAtStartPar
\sphinxstylestrong{Usage:}

\begin{sphinxVerbatim}[commandchars=\\\{\}]
\PYG{k}{define}\PYG{+w}{ }\PYG{n+nf}{GetAngle3}\PYG{p}{(}\PYG{p}{.}\PYG{p}{.}\PYG{p}{.}\PYG{p}{)}
\end{sphinxVerbatim}

\sphinxAtStartPar
\sphinxstylestrong{Example:}

\begin{sphinxVerbatim}[commandchars=\\\{\}]
\PYG{+w}{ }\PYG{n+nv}{a}\PYG{+w}{ }\PYG{o}{\PYGZlt{}\PYGZhy{}}\PYG{+w}{ }\PYG{p}{[}\PYG{l+m+mf}{0.579081}\PYG{p}{,}\PYG{+w}{ }\PYG{l+m+mf}{0.0327737}\PYG{p}{]}
\PYG{+w}{ }\PYG{n+nv}{b}\PYG{+w}{ }\PYG{o}{\PYGZlt{}\PYGZhy{}}\PYG{+w}{ }\PYG{p}{[}\PYG{l+m+mf}{0.0536094}\PYG{p}{,}\PYG{+w}{ }\PYG{l+m+mf}{0.378258}\PYG{p}{]}
\PYG{+w}{ }\PYG{n+nv}{c}\PYG{+w}{ }\PYG{o}{\PYGZlt{}\PYGZhy{}}\PYG{+w}{ }\PYG{p}{[}\PYG{l+m+mf}{0.239628}\PYG{p}{,}\PYG{+w}{ }\PYG{l+m+mf}{0.187751}\PYG{p}{]}

\PYG{n+nf}{Print}\PYG{p}{(}\PYG{n+nf}{GetAngle3}\PYG{p}{(}\PYG{n+nv}{a}\PYG{p}{,}\PYG{n+nv}{b}\PYG{p}{,}\PYG{n+nv}{c}\PYG{p}{)}\PYG{p}{)}\PYG{+w}{ }\PYG{c+c1}{\PYGZsh{}\PYGZsh{} .2157}
\end{sphinxVerbatim}

\sphinxAtStartPar
\sphinxstylestrong{See Also:}

\sphinxAtStartPar
\sphinxcode{\sphinxupquote{DegtoRad}}, \sphinxcode{\sphinxupquote{RadToDeg}}, \sphinxcode{\sphinxupquote{GetAngle}}, \sphinxcode{\sphinxupquote{ToRight}}

\index{KaniszaPolygon@\spxentry{KaniszaPolygon}}\ignorespaces 

\subsection{KaniszaPolygon()}
\label{\detokenize{reference/graphics:kaniszapolygon}}\label{\detokenize{reference/graphics:index-4}}
\sphinxAtStartPar
\sphinxstylestrong{Description:}

\sphinxAtStartPar
Creates generic polygon, defined only by with {\color{red}\bfseries{}\textasciigrave{}\textasciigrave{}}pac\sphinxhyphen{}man’’ circles at specified vertices.

\sphinxAtStartPar
\sphinxstylestrong{Usage:}

\begin{sphinxVerbatim}[commandchars=\\\{\}]
\PYG{k}{define}\PYG{+w}{ }\PYG{n+nf}{KaniszaPolygon}\PYG{p}{(}\PYG{p}{.}\PYG{p}{.}\PYG{p}{.}\PYG{p}{)}
\end{sphinxVerbatim}

\sphinxAtStartPar
\sphinxstylestrong{Example:}

\begin{sphinxVerbatim}[commandchars=\\\{\}]
F\PYG{n+nv}{or}\PYG{+w}{ }\PYG{n+nv}{detailed}\PYG{+w}{ }\PYG{n+nv}{usage}\PYG{+w}{ }\PYG{n+nv}{example}\PYG{p}{,}\PYG{+w}{ }\PYG{n+nv}{see}\PYG{o}{:}\PYG{+w}{ }`\PYG{n+nv}{http}\PYG{o}{:}\PYG{o}{/}\PYG{o}{/}\PYG{n+nv}{peblblog.blogspot}\PYG{p}{.}\PYG{n+nv}{com}\PYG{o}{/}\PYG{l+m+mi}{2010}\PYG{o}{/}\PYG{l+m+mi}{11}\PYG{o}{/}\PYG{n+nv}{kanizsa}\PYG{o}{\PYGZhy{}}\PYG{n+nv}{shapes.html}`
P\PYG{n+nv}{art}\PYG{+w}{ }\PYG{n+nv}{of}\PYG{+w}{ }\PYG{n+nv}{a}\PYG{+w}{ }\PYG{n+nv}{script}\PYG{+w}{ }\PYG{n+nv}{using}\PYG{+w}{ }K\PYG{n+nv}{aniszaPolygon}\PYG{o}{:}

\PYG{+w}{   }\PYG{c+c1}{\PYGZsh{}Specify the xy points}
\PYG{+w}{   }\PYG{n+nv}{xys}\PYG{+w}{ }\PYG{o}{\PYGZlt{}\PYGZhy{}}\PYG{+w}{ }\PYG{p}{[}\PYG{p}{[}\PYG{l+m+mi}{10}\PYG{p}{,}\PYG{l+m+mi}{10}\PYG{p}{]}\PYG{p}{,}\PYG{p}{[}\PYG{l+m+mi}{10}\PYG{p}{,}\PYG{l+m+mi}{50}\PYG{p}{]}\PYG{p}{,}\PYG{p}{[}\PYG{l+m+mi}{130}\PYG{p}{,}\PYG{l+m+mi}{60}\PYG{p}{]}\PYG{p}{,}\PYG{p}{[}\PYG{l+m+mi}{100}\PYG{p}{,}\PYG{l+m+mi}{100}\PYG{p}{]}\PYG{p}{,}\PYG{p}{[}\PYG{l+m+mi}{150}\PYG{p}{,}\PYG{l+m+mi}{100}\PYG{p}{]}\PYG{p}{,}
\PYG{+w}{           }\PYG{p}{[}\PYG{l+m+mi}{150}\PYG{p}{,}\PYG{l+m+mi}{20}\PYG{p}{]}\PYG{p}{,}\PYG{p}{[}\PYG{l+m+mi}{80}\PYG{p}{,}\PYG{o}{\PYGZhy{}}\PYG{l+m+mi}{10}\PYG{p}{]}\PYG{p}{,}\PYG{p}{[}\PYG{l+m+mi}{45}\PYG{p}{,}\PYG{l+m+mi}{10}\PYG{p}{]}\PYG{p}{]}

\PYG{+w}{    }\PYG{c+c1}{\PYGZsh{}Specify which vertices to show (do all)}
\PYG{+w}{    }\PYG{n+nv}{show}\PYG{+w}{ }\PYG{o}{\PYGZlt{}\PYGZhy{}}\PYG{+w}{ }\PYG{p}{[}\PYG{l+m+mi}{1}\PYG{p}{,}\PYG{l+m+mi}{1}\PYG{p}{,}\PYG{l+m+mi}{1}\PYG{p}{,}\PYG{l+m+mi}{1}\PYG{p}{,}\PYG{l+m+mi}{1}\PYG{p}{,}\PYG{l+m+mi}{1}\PYG{p}{,}\PYG{l+m+mi}{1}\PYG{p}{,}\PYG{l+m+mi}{1}\PYG{p}{]}

\PYG{+w}{    }\PYG{c+c1}{\PYGZsh{}Make one, showing the line}
\PYG{+w}{    }\PYG{n+nv}{x}\PYG{+w}{ }\PYG{o}{\PYGZlt{}\PYGZhy{}}\PYG{+w}{  }\PYG{n+nf}{KaniszaPolygon}\PYG{p}{(}\PYG{n+nv}{xys}\PYG{p}{,}\PYG{n+nv}{show}\PYG{p}{,}\PYG{l+m+mi}{10}\PYG{p}{,}\PYG{n+nv}{fg}\PYG{p}{,}\PYG{n+nv}{bg}\PYG{p}{,}\PYG{l+m+mi}{1}\PYG{p}{)}
\PYG{+w}{    }\PYG{n+nf}{AddObject}\PYG{p}{(}\PYG{n+nv}{x}\PYG{p}{,}\PYG{n+nv+vg}{gWin}\PYG{p}{)}\PYG{p}{;}\PYG{+w}{   }\PYG{n+nf}{Move}\PYG{p}{(}\PYG{n+nv}{x}\PYG{p}{,}\PYG{l+m+mi}{200}\PYG{p}{,}\PYG{l+m+mi}{200}\PYG{p}{)}

\PYG{+w}{    }\PYG{c+c1}{\PYGZsh{}Make a second, not showing the line}
\PYG{+w}{    }\PYG{n+nv}{x2}\PYG{+w}{ }\PYG{o}{\PYGZlt{}\PYGZhy{}}\PYG{+w}{  }\PYG{n+nf}{KaniszaPolygon}\PYG{p}{(}\PYG{n+nv}{xys}\PYG{p}{,}\PYG{n+nv}{show}\PYG{p}{,}\PYG{l+m+mi}{10}\PYG{p}{,}\PYG{n+nv}{fg}\PYG{p}{,}\PYG{n+nv}{bg}\PYG{p}{,}\PYG{l+m+mi}{0}\PYG{p}{)}
\PYG{+w}{    }\PYG{n+nf}{AddObject}\PYG{p}{(}\PYG{n+nv}{x2}\PYG{p}{,}\PYG{n+nv+vg}{gWin}\PYG{p}{)}\PYG{p}{;}\PYG{+w}{   }\PYG{n+nf}{Move}\PYG{p}{(}\PYG{n+nv}{x2}\PYG{p}{,}\PYG{l+m+mi}{400}\PYG{p}{,}\PYG{l+m+mi}{200}\PYG{p}{)}

\PYG{+w}{    }\PYG{c+c1}{\PYGZsh{}Make a third, only showing some vertices:}
\PYG{+w}{    }\PYG{n+nv}{x3}\PYG{+w}{ }\PYG{o}{\PYGZlt{}\PYGZhy{}}\PYG{+w}{  }\PYG{n+nf}{KaniszaPolygon}\PYG{p}{(}\PYG{n+nv}{xys}\PYG{p}{,}\PYG{p}{[}\PYG{l+m+mi}{1}\PYG{p}{,}\PYG{l+m+mi}{1}\PYG{p}{,}\PYG{l+m+mi}{1}\PYG{p}{,}\PYG{l+m+mi}{1}\PYG{p}{,}\PYG{l+m+mi}{1}\PYG{p}{,}\PYG{l+m+mi}{0}\PYG{p}{,}\PYG{l+m+mi}{0}\PYG{p}{,}\PYG{l+m+mi}{1}\PYG{p}{]}\PYG{p}{,}\PYG{l+m+mi}{10}\PYG{p}{,}\PYG{n+nv}{fg}\PYG{p}{,}\PYG{n+nv}{bg}\PYG{p}{,}\PYG{l+m+mi}{0}\PYG{p}{)}
\PYG{+w}{    }\PYG{n+nf}{AddObject}\PYG{p}{(}\PYG{n+nv}{x3}\PYG{p}{,}\PYG{n+nv+vg}{gWin}\PYG{p}{)}\PYG{p}{;}\PYG{+w}{  }\PYG{n+nf}{Move}\PYG{p}{(}\PYG{n+nv}{x3}\PYG{p}{,}\PYG{l+m+mi}{600}\PYG{p}{,}\PYG{l+m+mi}{200}\PYG{p}{)}
\end{sphinxVerbatim}

\sphinxAtStartPar
\sphinxstylestrong{See Also:}

\sphinxAtStartPar
\sphinxcode{\sphinxupquote{Polygon()}}, \sphinxcode{\sphinxupquote{KaneszaSquare()}}

\index{KaniszaSquare@\spxentry{KaniszaSquare}}\ignorespaces 

\subsection{KaniszaSquare()}
\label{\detokenize{reference/graphics:kaniszasquare}}\label{\detokenize{reference/graphics:index-5}}
\sphinxAtStartPar
\sphinxstylestrong{Description:}

\sphinxAtStartPar
Creates generic Kanesza Square, one defined only by with {\color{red}\bfseries{}\textasciigrave{}\textasciigrave{}}pac\sphinxhyphen{}man’’ circles at its vertices:

\sphinxAtStartPar
\sphinxstylestrong{Usage:}

\begin{sphinxVerbatim}[commandchars=\\\{\}]
\PYG{k}{define}\PYG{+w}{ }\PYG{n+nf}{KaniszaSquare}\PYG{p}{(}\PYG{p}{.}\PYG{p}{.}\PYG{p}{.}\PYG{p}{)}
\end{sphinxVerbatim}

\sphinxAtStartPar
\sphinxstylestrong{Example:}

\begin{sphinxVerbatim}[commandchars=\\\{\}]
F\PYG{n+nv}{or}\PYG{+w}{ }\PYG{n+nv}{detailed}\PYG{+w}{ }\PYG{n+nv}{usage}\PYG{+w}{ }\PYG{n+nv}{example}\PYG{p}{,}\PYG{+w}{ }\PYG{n+nv}{see}
`\PYG{n+nv}{http}\PYG{o}{:}\PYG{o}{/}\PYG{o}{/}\PYG{n+nv}{peblblog.blogspot}\PYG{p}{.}\PYG{n+nv}{com}\PYG{o}{/}\PYG{l+m+mi}{2010}\PYG{o}{/}\PYG{l+m+mi}{11}\PYG{o}{/}\PYG{n+nv}{kanizsa}\PYG{o}{\PYGZhy{}}\PYG{n+nv}{shapes.html}`



\PYG{+w}{   }\PYG{n+nv+vg}{gWin}\PYG{+w}{ }\PYG{o}{\PYGZlt{}\PYGZhy{}}\PYG{+w}{ }\PYG{n+nf}{MakeWindow}\PYG{p}{(}\PYG{p}{)}
\PYG{+w}{   }\PYG{n+nv}{square}\PYG{+w}{ }\PYG{o}{\PYGZlt{}\PYGZhy{}}\PYG{+w}{ }\PYG{n+nf}{KaniszaSquare}\PYG{p}{(}\PYG{l+m+mi}{150}\PYG{p}{,}\PYG{l+m+mi}{20}\PYG{p}{,}\PYG{n+nf}{MakeColor}\PYG{p}{(}\PYG{l+s+s2}{\PYGZdq{}red\PYGZdq{}}\PYG{p}{)}\PYG{p}{,}
\PYG{+w}{                                  }\PYG{n+nf}{MakeColor}\PYG{p}{(}\PYG{l+s+s2}{\PYGZdq{}green\PYGZdq{}}\PYG{p}{)}\PYG{p}{)}
\PYG{+w}{   }\PYG{n+nf}{AddObject}\PYG{p}{(}\PYG{n+nv}{square}\PYG{p}{,}\PYG{n+nv+vg}{gWin}\PYG{p}{)}
\PYG{+w}{   }\PYG{n+nf}{Move}\PYG{p}{(}\PYG{n+nv}{square}\PYG{p}{,}\PYG{l+m+mi}{200}\PYG{p}{,}\PYG{l+m+mi}{200}\PYG{p}{)}
\PYG{+w}{   }\PYG{n+nf}{Draw}\PYG{p}{(}\PYG{p}{)}
\PYG{+w}{   }\PYG{n+nf}{WaitForAnyKeyPress}\PYG{p}{(}\PYG{p}{)}
\end{sphinxVerbatim}

\sphinxAtStartPar
\sphinxstylestrong{See Also:}

\sphinxAtStartPar
\sphinxcode{\sphinxupquote{Polygon()}}, \sphinxcode{\sphinxupquote{KaneszaPolygon()}}

\index{LayoutGrid@\spxentry{LayoutGrid}}\ignorespaces 

\subsection{LayoutGrid()}
\label{\detokenize{reference/graphics:layoutgrid}}\label{\detokenize{reference/graphics:index-6}}
\sphinxAtStartPar
\sphinxstyleemphasis{Creates {[}x,y{]} pairs in a grid for graphical layout}

\sphinxAtStartPar
\sphinxstylestrong{Description:}

\sphinxAtStartPar
Creates a grid of x,y points in a range, that are  spaced in a specified number of rows and columns.  Furthermore, you can specify whether they are vertical or horizontally laid out.

\sphinxAtStartPar
\sphinxstylestrong{Usage:}

\begin{sphinxVerbatim}[commandchars=\\\{\}]
\PYG{k}{define}\PYG{+w}{ }\PYG{n+nf}{LayoutGrid}\PYG{p}{(}\PYG{p}{.}\PYG{p}{.}\PYG{p}{.}\PYG{p}{)}
\end{sphinxVerbatim}

\sphinxAtStartPar
\sphinxstylestrong{Example:}

\begin{sphinxVerbatim}[commandchars=\\\{\}]
E\PYG{n+nv}{xample}\PYG{+w}{ }PEBL\PYG{+w}{ }P\PYG{n+nv}{rogram}\PYG{+w}{ }\PYG{n+nv}{using}\PYG{+w}{ }N\PYG{n+nv}{onoverlapLayout}\PYG{o}{:}

\PYG{k}{define}\PYG{+w}{ }\PYG{n+nf}{Start}\PYG{p}{(}\PYG{n+nv}{p}\PYG{p}{)}
\PYG{p}{\PYGZob{}}
\PYG{+w}{   }\PYG{n+nv+vg}{gWin}\PYG{+w}{ }\PYG{o}{\PYGZlt{}\PYGZhy{}}\PYG{+w}{ }\PYG{n+nf}{MakeWindow}\PYG{p}{(}\PYG{p}{)}
\PYG{+w}{   }\PYG{n+nv+vg}{gVideoWidth}\PYG{+w}{ }\PYG{o}{\PYGZlt{}\PYGZhy{}}\PYG{+w}{ }\PYG{l+m+mi}{800}
\PYG{+w}{   }\PYG{n+nv+vg}{gVideoHeight}\PYG{+w}{ }\PYG{o}{\PYGZlt{}\PYGZhy{}}\PYG{+w}{ }\PYG{l+m+mi}{300}

\PYG{+w}{   }\PYG{n+nv}{lab1}\PYG{+w}{ }\PYG{o}{\PYGZlt{}\PYGZhy{}}\PYG{+w}{ }\PYG{n+nf}{EasyLabel}\PYG{p}{(}\PYG{l+s+s2}{\PYGZdq{}LayoutGrid, horizontal\PYGZdq{}}\PYG{p}{,}
\PYG{+w}{                     }\PYG{l+m+mi}{200}\PYG{p}{,}\PYG{l+m+mi}{25}\PYG{p}{,}\PYG{n+nv+vg}{gWin}\PYG{p}{,}\PYG{l+m+mi}{24}\PYG{p}{)}
\PYG{+w}{   }\PYG{n+nv}{lab2}\PYG{+w}{ }\PYG{o}{\PYGZlt{}\PYGZhy{}}\PYG{+w}{ }\PYG{n+nf}{EasyLabel}\PYG{p}{(}\PYG{l+s+s2}{\PYGZdq{}LayoutGrid, vertical\PYGZdq{}}\PYG{p}{,}
\PYG{+w}{                     }\PYG{l+m+mi}{600}\PYG{p}{,}\PYG{l+m+mi}{25}\PYG{p}{,}\PYG{n+nv+vg}{gWin}\PYG{p}{,}\PYG{l+m+mi}{24}\PYG{p}{)}
\PYG{+w}{   }\PYG{n+nv}{nums}\PYG{+w}{ }\PYG{o}{\PYGZlt{}\PYGZhy{}}\PYG{+w}{ }\PYG{n+nf}{Sequence}\PYG{p}{(}\PYG{l+m+mi}{1}\PYG{p}{,}\PYG{l+m+mi}{20}\PYG{p}{,}\PYG{l+m+mi}{1}\PYG{p}{)}
\PYG{+w}{   }\PYG{n+nv}{stim1}\PYG{+w}{ }\PYG{o}{\PYGZlt{}\PYGZhy{}}\PYG{+w}{ }\PYG{p}{[}\PYG{p}{]}
\PYG{+w}{   }\PYG{n+nv}{stim2}\PYG{+w}{ }\PYG{o}{\PYGZlt{}\PYGZhy{}}\PYG{+w}{ }\PYG{p}{[}\PYG{p}{]}

\PYG{+w}{   }\PYG{n+nv}{font}\PYG{+w}{ }\PYG{o}{\PYGZlt{}\PYGZhy{}}\PYG{+w}{ }\PYG{n+nf}{MakeFont}\PYG{p}{(}\PYG{n+nv+vg}{gPeblBaseFont}\PYG{p}{,}\PYG{l+m+mi}{0}\PYG{p}{,}\PYG{l+m+mi}{25}\PYG{p}{,}
\PYG{+w}{              }\PYG{n+nf}{MakeColor}\PYG{p}{(}\PYG{l+s+s2}{\PYGZdq{}black\PYGZdq{}}\PYG{p}{)}\PYG{p}{,}\PYG{n+nf}{MakeColor}\PYG{p}{(}\PYG{l+s+s2}{\PYGZdq{}white\PYGZdq{}}\PYG{p}{)}\PYG{p}{,}\PYG{l+m+mi}{0}\PYG{p}{)}
\PYG{+w}{   }\PYG{k}{loop}\PYG{p}{(}\PYG{n+nv}{i}\PYG{p}{,}\PYG{n+nv}{nums}\PYG{p}{)}
\PYG{+w}{   }\PYG{p}{\PYGZob{}}
\PYG{+w}{     }\PYG{n+nv}{stim1}\PYG{+w}{ }\PYG{o}{\PYGZlt{}\PYGZhy{}}\PYG{+w}{ }\PYG{n+nf}{Append}\PYG{p}{(}\PYG{n+nv}{stim1}\PYG{p}{,}\PYG{n+nf}{MakeLabel}\PYG{p}{(}\PYG{n+nv}{i}\PYG{o}{+}\PYG{l+s+s2}{\PYGZdq{}\PYGZdq{}}\PYG{p}{,}\PYG{n+nv}{font}\PYG{p}{)}\PYG{p}{)}
\PYG{+w}{     }\PYG{n+nv}{stim2}\PYG{+w}{ }\PYG{o}{\PYGZlt{}\PYGZhy{}}\PYG{+w}{ }\PYG{n+nf}{Append}\PYG{p}{(}\PYG{n+nv}{stim2}\PYG{p}{,}\PYG{n+nf}{MakeLabel}\PYG{p}{(}\PYG{n+nv}{i}\PYG{o}{+}\PYG{l+s+s2}{\PYGZdq{}\PYGZdq{}}\PYG{p}{,}\PYG{n+nv}{font}\PYG{p}{)}\PYG{p}{)}
\PYG{+w}{    }\PYG{p}{\PYGZcb{}}

\PYG{+w}{  }\PYG{n+nv}{layout1}\PYG{+w}{ }\PYG{o}{\PYGZlt{}\PYGZhy{}}\PYG{+w}{ }\PYG{n+nf}{LayoutGrid}\PYG{p}{(}\PYG{l+m+mi}{50}\PYG{p}{,}\PYG{n+nv+vg}{gVideoWidth}\PYG{o}{/}\PYG{l+m+mi}{2}\PYG{o}{\PYGZhy{}}\PYG{l+m+mi}{50}\PYG{p}{,}
\PYG{+w}{                       }\PYG{l+m+mi}{50}\PYG{p}{,}\PYG{n+nv+vg}{gVideoHeight}\PYG{o}{\PYGZhy{}}\PYG{l+m+mi}{50}\PYG{p}{,}\PYG{l+m+mi}{5}\PYG{p}{,}\PYG{l+m+mi}{4}\PYG{p}{,}\PYG{l+m+mi}{0}\PYG{p}{)}
\PYG{+w}{  }\PYG{n+nv}{layout2}\PYG{+w}{ }\PYG{o}{\PYGZlt{}\PYGZhy{}}\PYG{+w}{ }\PYG{n+nf}{LayoutGrid}\PYG{p}{(}\PYG{n+nv+vg}{gVideoWidth}\PYG{o}{/}\PYG{l+m+mi}{2}\PYG{o}{+}\PYG{l+m+mi}{50}\PYG{p}{,}\PYG{n+nv+vg}{gVideoWidth}\PYG{o}{\PYGZhy{}}\PYG{l+m+mi}{50}\PYG{p}{,}
\PYG{+w}{                       }\PYG{l+m+mi}{50}\PYG{p}{,}\PYG{n+nv+vg}{gVideoHeight}\PYG{o}{\PYGZhy{}}\PYG{l+m+mi}{50}\PYG{p}{,}\PYG{l+m+mi}{5}\PYG{p}{,}\PYG{l+m+mi}{4}\PYG{p}{,}\PYG{l+m+mi}{1}\PYG{p}{)}


\PYG{+w}{  }\PYG{c+c1}{\PYGZsh{}\PYGZsh{}Now, layout the stuff.}

\PYG{+w}{  }\PYG{k}{loop}\PYG{p}{(}\PYG{n+nv}{i}\PYG{p}{,}\PYG{n+nf}{Transpose}\PYG{p}{(}\PYG{p}{[}\PYG{n+nv}{stim1}\PYG{p}{,}\PYG{n+nv}{layout1}\PYG{p}{]}\PYG{p}{)}\PYG{p}{)}
\PYG{+w}{   }\PYG{p}{\PYGZob{}}
\PYG{+w}{      }\PYG{n+nv}{obj}\PYG{+w}{ }\PYG{o}{\PYGZlt{}\PYGZhy{}}\PYG{+w}{ }\PYG{n+nf}{First}\PYG{p}{(}\PYG{n+nv}{i}\PYG{p}{)}
\PYG{+w}{      }\PYG{n+nv}{xy}\PYG{+w}{ }\PYG{o}{\PYGZlt{}\PYGZhy{}}\PYG{+w}{ }\PYG{n+nf}{Second}\PYG{p}{(}\PYG{n+nv}{i}\PYG{p}{)}
\PYG{+w}{      }\PYG{n+nf}{AddObject}\PYG{p}{(}\PYG{n+nv}{obj}\PYG{p}{,}\PYG{n+nv+vg}{gWin}\PYG{p}{)}
\PYG{+w}{      }\PYG{n+nf}{Move}\PYG{p}{(}\PYG{n+nv}{obj}\PYG{p}{,}\PYG{+w}{ }\PYG{n+nf}{First}\PYG{p}{(}\PYG{n+nv}{xy}\PYG{p}{)}\PYG{p}{,}\PYG{n+nf}{Second}\PYG{p}{(}\PYG{n+nv}{xy}\PYG{p}{)}\PYG{p}{)}
\PYG{+w}{   }\PYG{p}{\PYGZcb{}}

\PYG{+w}{  }\PYG{k}{loop}\PYG{p}{(}\PYG{n+nv}{i}\PYG{p}{,}\PYG{n+nf}{Transpose}\PYG{p}{(}\PYG{p}{[}\PYG{n+nv}{stim2}\PYG{p}{,}\PYG{n+nv}{layout2}\PYG{p}{]}\PYG{p}{)}\PYG{p}{)}
\PYG{+w}{   }\PYG{p}{\PYGZob{}}
\PYG{+w}{      }\PYG{n+nv}{obj}\PYG{+w}{ }\PYG{o}{\PYGZlt{}\PYGZhy{}}\PYG{+w}{ }\PYG{n+nf}{First}\PYG{p}{(}\PYG{n+nv}{i}\PYG{p}{)}
\PYG{+w}{      }\PYG{n+nv}{xy}\PYG{+w}{ }\PYG{o}{\PYGZlt{}\PYGZhy{}}\PYG{+w}{ }\PYG{n+nf}{Second}\PYG{p}{(}\PYG{n+nv}{i}\PYG{p}{)}
\PYG{+w}{      }\PYG{n+nf}{AddObject}\PYG{p}{(}\PYG{n+nv}{obj}\PYG{p}{,}\PYG{n+nv+vg}{gWin}\PYG{p}{)}
\PYG{+w}{      }\PYG{n+nf}{Move}\PYG{p}{(}\PYG{n+nv}{obj}\PYG{p}{,}\PYG{+w}{ }\PYG{n+nf}{First}\PYG{p}{(}\PYG{n+nv}{xy}\PYG{p}{)}\PYG{p}{,}\PYG{n+nf}{Second}\PYG{p}{(}\PYG{n+nv}{xy}\PYG{p}{)}\PYG{p}{)}
\PYG{+w}{   }\PYG{p}{\PYGZcb{}}

\PYG{+w}{  }\PYG{n+nf}{Draw}\PYG{p}{(}\PYG{p}{)}
\PYG{+w}{  }\PYG{n+nf}{WaitForAnyKeyPress}\PYG{p}{(}\PYG{p}{)}
\PYG{p}{\PYGZcb{}}


T\PYG{n+nv}{he}\PYG{+w}{ }\PYG{n+nv}{output}\PYG{+w}{ }\PYG{n+nv}{of}\PYG{+w}{ }\PYG{n+nv}{the}\PYG{+w}{ }\PYG{n+nv}{above}\PYG{+w}{ }\PYG{n+nv}{program}\PYG{+w}{ }\PYG{n+nv}{is}\PYG{+w}{ }\PYG{n+nv}{shown}\PYG{+w}{ }\PYG{n+nv}{below}\PYG{p}{.}\PYG{+w}{  }E\PYG{n+nv}{ven}\PYG{+w}{ }\PYG{n+nv}{for}\PYG{+w}{ }\PYG{n+nv}{the}\PYG{+w}{ }\PYG{n+nv}{left}\PYG{+w}{ }\PYG{n+nv}{configuration}\PYG{p}{,}\PYG{+w}{ }\PYG{n+nv}{which}\PYG{+w}{ }\PYG{n+nv}{is}\PYG{+w}{ }\PYG{n+nv}{too}\PYG{+w}{ }\PYG{n+nv}{compact}\PYG{+w}{ }\PYG{p}{(}\PYG{k}{and}\PYG{+w}{ }\PYG{n+nv}{which}\PYG{+w}{ }\PYG{n+nv}{takes}\PYG{+w}{ }\PYG{n+nv}{a}\PYG{+w}{ }\PYG{n+nv}{couple}\PYG{+w}{ }\PYG{n+nv}{seconds}\PYG{+w}{ }\PYG{n+nv}{to}\PYG{+w}{ }\PYG{n+nv}{run}\PYG{p}{)}\PYG{p}{,}\PYG{+w}{ }\PYG{n+nv}{the}\PYG{+w}{ }\PYG{n+nv}{targets}\PYG{+w}{ }\PYG{n+nv}{are}\PYG{+w}{ }\PYG{n+nv}{fairly}\PYG{+w}{ }\PYG{n+nv}{well}\PYG{+w}{ }\PYG{n+nv}{distributed}\PYG{p}{.}
\end{sphinxVerbatim}

\sphinxAtStartPar
\sphinxstylestrong{See Also:}

\sphinxAtStartPar
\sphinxcode{\sphinxupquote{NonOverlapLayout()}}

\index{MakeAttneave@\spxentry{MakeAttneave}}\ignorespaces 

\subsection{MakeAttneave()}
\label{\detokenize{reference/graphics:makeattneave}}\label{\detokenize{reference/graphics:index-7}}
\sphinxAtStartPar
\sphinxstyleemphasis{Makes a complex \textasciigrave{}\textasciigrave{}Attneave’’ polygon}

\sphinxAtStartPar
\sphinxstylestrong{Description:}

\sphinxAtStartPar
Makes a random ‘Attneave’ figure((Collin, C. A., \& Mcmullen, P. A. (2002). Using Matlab to generate  families of similar Attneave shapes. Behavior Research Methods  Instruments and Computers, 34(1), 55\sphinxhyphen{}68.).). An Attneave figure is a complex polygon that can be used as a stimulus in a number of situations.  It returns a sequence of points for use in Polygon().  \{\} MakeAttneave uses ConvexHull,  InsertAttneavePointRandom() and ValidateAttneaveShape(), found in Graphics.pbl.  Override these to change constraints such as  minimum/maximum side lengths, angles, complexity, etc.  MakeAttneave uses a sampling\sphinxhyphen{}and\sphinxhyphen{}rejection scheme to create in\sphinxhyphen{}bounds shapes.  Thus, if you specify impossible or nearly\sphinxhyphen{}impossible constraints, the time necessary to create shapes may be very long or infinite.   The arguments to MakeAttneave are:
\begin{itemize}
\item {} 
\sphinxAtStartPar
size: size, in pixels, of a circle from which points are   sampled in a uniform distribution.

\item {} 
\sphinxAtStartPar
numpoints: number of points in the polygon.

\item {} 
\sphinxAtStartPar
minangle: smallest angle acceptable (in degrees).

\item {} 
\sphinxAtStartPar
maxangle: largest angle acceptable  (in degrees).

\end{itemize}

\sphinxAtStartPar
\sphinxstylestrong{Usage:}

\begin{sphinxVerbatim}[commandchars=\\\{\}]
\PYG{k}{define}\PYG{+w}{ }\PYG{n+nf}{MakeAttneave}\PYG{p}{(}\PYG{p}{.}\PYG{p}{.}\PYG{p}{.}\PYG{p}{)}
\end{sphinxVerbatim}

\sphinxAtStartPar
\sphinxstylestrong{Example:}

\begin{sphinxVerbatim}[commandchars=\\\{\}]
\PYG{n+nv+vg}{gWin}\PYG{+w}{ }\PYG{o}{\PYGZlt{}\PYGZhy{}}\PYG{+w}{ }\PYG{n+nf}{MakeWindow}\PYG{p}{(}\PYG{p}{)}
\PYG{n+nv}{shape}\PYG{+w}{ }\PYG{o}{\PYGZlt{}\PYGZhy{}}\PYG{+w}{ }\PYG{n+nf}{MakeAttneave}\PYG{p}{(}\PYG{l+m+mi}{100}\PYG{p}{,}\PYG{l+m+mi}{5}\PYG{o}{+}\PYG{n+nf}{RandomDiscrete}\PYG{p}{(}\PYG{l+m+mi}{5}\PYG{p}{)}\PYG{p}{,}\PYG{l+m+mi}{5}\PYG{p}{,}\PYG{l+m+mi}{170}\PYG{p}{)}
\PYG{n+nv}{pts}\PYG{+w}{ }\PYG{o}{\PYGZlt{}\PYGZhy{}}\PYG{+w}{ }\PYG{n+nf}{Transpose}\PYG{p}{(}\PYG{n+nv}{shape}\PYG{p}{)}
\PYG{n+nv}{poly}\PYG{+w}{ }\PYG{o}{\PYGZlt{}\PYGZhy{}}\PYG{+w}{ }\PYG{n+nf}{Polygon}\PYG{p}{(}\PYG{l+m+mi}{200}\PYG{p}{,}\PYG{l+m+mi}{200}\PYG{p}{,}\PYG{n+nf}{First}\PYG{p}{(}\PYG{n+nv}{pts}\PYG{p}{)}\PYG{p}{,}\PYG{n+nf}{Second}\PYG{p}{(}\PYG{n+nv}{pts}\PYG{p}{)}\PYG{p}{,}
\PYG{+w}{                }\PYG{n+nf}{MakeColor}\PYG{p}{(}\PYG{l+s+s2}{\PYGZdq{}blue\PYGZdq{}}\PYG{p}{)}\PYG{p}{,}\PYG{l+m+mi}{1}\PYG{p}{)}
\PYG{n+nf}{AddObject}\PYG{p}{(}\PYG{n+nv}{poly}\PYG{p}{,}\PYG{n+nv+vg}{gWin}\PYG{p}{)}
\PYG{n+nf}{Draw}\PYG{p}{(}\PYG{p}{)}
\PYG{n+nf}{WaitForAnyKeyPress}\PYG{p}{(}\PYG{p}{)}
\end{sphinxVerbatim}

\sphinxAtStartPar
\sphinxstylestrong{See Also:}

\sphinxAtStartPar
\sphinxcode{\sphinxupquote{MakeImage()}}, \sphinxcode{\sphinxupquote{Polygon()}}, \sphinxcode{\sphinxupquote{Square()}}

\index{MakeGabor@\spxentry{MakeGabor}}\ignorespaces 

\subsection{MakeGabor()}
\label{\detokenize{reference/graphics:makegabor}}\label{\detokenize{reference/graphics:index-8}}
\sphinxAtStartPar
\sphinxstyleemphasis{Creates a ‘gabor patch’ with specified parameters}

\sphinxAtStartPar
\sphinxstylestrong{Description:}

\sphinxAtStartPar
Creates a greyscale gabor patch, with seven variables:
\begin{itemize}
\item {} 
\sphinxAtStartPar
size (in pixels) of square the patch is drawn on

\item {} 
\sphinxAtStartPar
freq: frequency of grating (number of wavelengths in size)

\item {} 
\sphinxAtStartPar
sd: standard deviation, in pixels, of gaussian window

\item {} 
\sphinxAtStartPar
angle: angle of rotation of grating, in radians

\item {} 
\sphinxAtStartPar
phase: phase offset of grating (in radians)

\item {} 
\sphinxAtStartPar
bglev: number between 0 and 255 indicating background color in greyscale.

\sphinxAtStartPar
\{ \}

\end{itemize}

\sphinxAtStartPar
\sphinxstylestrong{Usage:}

\begin{sphinxVerbatim}[commandchars=\\\{\}]
\PYG{k}{define}\PYG{+w}{ }\PYG{n+nf}{MakeGabor}\PYG{p}{(}\PYG{p}{.}\PYG{p}{.}\PYG{p}{.}\PYG{p}{)}
\end{sphinxVerbatim}

\sphinxAtStartPar
\sphinxstylestrong{Example:}

\begin{sphinxVerbatim}[commandchars=\\\{\}]
\PYG{n+nv}{win}\PYG{+w}{ }\PYG{o}{\PYGZlt{}\PYGZhy{}}\PYG{+w}{ }\PYG{n+nf}{MakeWindow}\PYG{p}{(}\PYG{p}{)}
\PYG{n+nv}{patch}\PYG{+w}{ }\PYG{o}{\PYGZlt{}\PYGZhy{}}\PYG{+w}{ }\PYG{n+nf}{MakeGabor}\PYG{p}{(}\PYG{l+m+mi}{80}\PYG{p}{,}\PYG{+w}{ }\PYG{l+m+mi}{0}\PYG{p}{,}\PYG{l+m+mi}{10}\PYG{p}{,}\PYG{l+m+mi}{0}\PYG{p}{,}\PYG{l+m+mi}{0}\PYG{p}{,}\PYG{l+m+mi}{100}\PYG{p}{)}
\PYG{n+nf}{AddObject}\PYG{p}{(}\PYG{n+nv}{patch}\PYG{p}{,}\PYG{n+nv}{win}\PYG{p}{)}
\PYG{n+nf}{Move}\PYG{p}{(}\PYG{n+nv}{patch}\PYG{p}{,}\PYG{l+m+mi}{200}\PYG{p}{,}\PYG{l+m+mi}{200}\PYG{p}{)}
\PYG{n+nf}{Draw}\PYG{p}{(}\PYG{p}{)}
\end{sphinxVerbatim}

\sphinxAtStartPar
\sphinxstylestrong{See Also:}

\sphinxAtStartPar
\sphinxcode{\sphinxupquote{MakeAttneave()}}, \sphinxcode{\sphinxupquote{SetPixel()}}, \sphinxcode{\sphinxupquote{MakeCanvas()}}

\index{MakeGraph@\spxentry{MakeGraph}}\ignorespaces 

\subsection{MakeGraph()}
\label{\detokenize{reference/graphics:makegraph}}\label{\detokenize{reference/graphics:index-9}}
\sphinxAtStartPar
\sphinxstylestrong{Description:}

\sphinxAtStartPar
Creates a simple bargraph that can be added to/moved on a window..

\sphinxAtStartPar
\sphinxstylestrong{Usage:}

\begin{sphinxVerbatim}[commandchars=\\\{\}]
\PYG{k}{define}\PYG{+w}{ }\PYG{n+nf}{MakeGraph}\PYG{p}{(}\PYG{p}{.}\PYG{p}{.}\PYG{p}{.}\PYG{p}{)}
\end{sphinxVerbatim}

\index{MakeNGonPoints@\spxentry{MakeNGonPoints}}\ignorespaces 

\subsection{MakeNGonPoints()}
\label{\detokenize{reference/graphics:makengonpoints}}\label{\detokenize{reference/graphics:index-10}}
\sphinxAtStartPar
\sphinxstyleemphasis{Creates points for a polygon, which can then be fed to Polygon}

\sphinxAtStartPar
\sphinxstylestrong{Description:}

\sphinxAtStartPar
Creates a set of points that form a regular n\sphinxhyphen{}gon.  It can be transformed with functions like \sphinxcode{\sphinxupquote{RotatePoints}}, or it can be  used to create a graphical object with \sphinxcode{\sphinxupquote{Polygon}}.  Note: \sphinxcode{\sphinxupquote{MakeNGonPoints}} returns a list like:

\begin{sphinxVerbatim}[commandchars=\\\{\}]
  [[x1, x2, x3,...],[y1,y2,y3,...]],

while Polygon() takes the X and Y lists independently.
\end{sphinxVerbatim}

\sphinxAtStartPar
\sphinxstylestrong{Usage:}

\begin{sphinxVerbatim}[commandchars=\\\{\}]
\PYG{k}{define}\PYG{+w}{ }\PYG{n+nf}{MakeNGonPoints}\PYG{p}{(}\PYG{p}{.}\PYG{p}{.}\PYG{p}{.}\PYG{p}{)}
\end{sphinxVerbatim}

\sphinxAtStartPar
\sphinxstylestrong{Example:}

\begin{sphinxVerbatim}[commandchars=\\\{\}]
\PYG{n+nv}{window}\PYG{+w}{ }\PYG{o}{\PYGZlt{}\PYGZhy{}}\PYG{+w}{ }\PYG{n+nf}{MakeWindow}\PYG{p}{(}\PYG{p}{)}
\PYG{n+nv}{ngonp}\PYG{+w}{ }\PYG{o}{\PYGZlt{}\PYGZhy{}}\PYG{+w}{ }\PYG{n+nf}{MakeNGonPoints}\PYG{p}{(}\PYG{l+m+mi}{50}\PYG{p}{,}\PYG{l+m+mi}{10}\PYG{p}{)}
\PYG{n+nv}{ngon}\PYG{+w}{ }\PYG{o}{\PYGZlt{}\PYGZhy{}}\PYG{+w}{ }\PYG{n+nf}{Polygon}\PYG{p}{(}\PYG{l+m+mi}{200}\PYG{p}{,}\PYG{l+m+mi}{200}\PYG{p}{,}\PYG{n+nf}{First}\PYG{p}{(}\PYG{n+nv}{ngonp}\PYG{p}{)}\PYG{p}{,}\PYG{n+nf}{Nth}\PYG{p}{(}\PYG{n+nv}{ngonp}\PYG{p}{,}\PYG{l+m+mi}{2}\PYG{p}{)}\PYG{p}{,}
\PYG{+w}{                }\PYG{n+nf}{MakeColor}\PYG{p}{(}\PYG{l+s+s2}{\PYGZdq{}red\PYGZdq{}}\PYG{p}{)}\PYG{p}{,}\PYG{l+m+mi}{1}\PYG{p}{)}
\PYG{n+nf}{AddObject}\PYG{p}{(}\PYG{n+nv}{ngon}\PYG{p}{,}\PYG{n+nv}{window}\PYG{p}{)}
\PYG{n+nf}{Draw}\PYG{p}{(}\PYG{p}{)}
\end{sphinxVerbatim}

\sphinxAtStartPar
\sphinxstylestrong{See Also:}

\sphinxAtStartPar
\sphinxcode{\sphinxupquote{MakeStarPoints}}, \sphinxcode{\sphinxupquote{Polygon}}, \sphinxcode{\sphinxupquote{RotatePoints}}, \sphinxcode{\sphinxupquote{ZoomPoints}}

\index{MakeStarPoints@\spxentry{MakeStarPoints}}\ignorespaces 

\subsection{MakeStarPoints()}
\label{\detokenize{reference/graphics:makestarpoints}}\label{\detokenize{reference/graphics:index-11}}
\sphinxAtStartPar
\sphinxstyleemphasis{Creates points for a star, which can then be fed to Polygon}

\sphinxAtStartPar
\sphinxstylestrong{Description:}

\sphinxAtStartPar
Creates a set of points that form a regular star.  It can be transformed with functions like \sphinxcode{\sphinxupquote{RotatePoints}}, or it can be  used to create a graphical object with \sphinxcode{\sphinxupquote{Polygon}}.  Note: \sphinxcode{\sphinxupquote{MakeStarPoints}} returns a list:

\begin{sphinxVerbatim}[commandchars=\\\{\}]
  [[x1, x2, x3,...],[y1,y2,y3,...]],

while ``Polygon()`` takes the X and Y lists independently.
\end{sphinxVerbatim}

\sphinxAtStartPar
\sphinxstylestrong{Usage:}

\begin{sphinxVerbatim}[commandchars=\\\{\}]
\PYG{k}{define}\PYG{+w}{ }\PYG{n+nf}{MakeStarPoints}\PYG{p}{(}\PYG{p}{.}\PYG{p}{.}\PYG{p}{.}\PYG{p}{)}
\end{sphinxVerbatim}

\sphinxAtStartPar
\sphinxstylestrong{Example:}

\begin{sphinxVerbatim}[commandchars=\\\{\}]
\PYG{n+nv}{window}\PYG{+w}{ }\PYG{o}{\PYGZlt{}\PYGZhy{}}\PYG{+w}{ }\PYG{n+nf}{MakeWindow}\PYG{p}{(}\PYG{p}{)}
\PYG{n+nv}{sp}\PYG{+w}{ }\PYG{o}{\PYGZlt{}\PYGZhy{}}\PYG{+w}{ }\PYG{n+nf}{MakeStarPoints}\PYG{p}{(}\PYG{l+m+mi}{50}\PYG{p}{,}\PYG{l+m+mi}{20}\PYG{p}{,}\PYG{l+m+mi}{10}\PYG{p}{)}
\PYG{n+nv}{star}\PYG{+w}{ }\PYG{o}{\PYGZlt{}\PYGZhy{}}\PYG{+w}{ }\PYG{n+nf}{Polygon}\PYG{p}{(}\PYG{l+m+mi}{200}\PYG{p}{,}\PYG{l+m+mi}{200}\PYG{p}{,}\PYG{n+nf}{First}\PYG{p}{(}\PYG{n+nv}{sp}\PYG{p}{)}\PYG{p}{,}\PYG{n+nf}{Nth}\PYG{p}{(}\PYG{n+nv}{sp}\PYG{p}{,}\PYG{l+m+mi}{2}\PYG{p}{)}\PYG{p}{,}
\PYG{+w}{                }\PYG{n+nf}{MakeColor}\PYG{p}{(}\PYG{l+s+s2}{\PYGZdq{}red\PYGZdq{}}\PYG{p}{)}\PYG{p}{,}\PYG{l+m+mi}{1}\PYG{p}{)}
\PYG{n+nf}{AddObject}\PYG{p}{(}\PYG{n+nv}{star}\PYG{p}{,}\PYG{n+nv}{window}\PYG{p}{)}
\PYG{n+nf}{Draw}\PYG{p}{(}\PYG{p}{)}
\end{sphinxVerbatim}

\sphinxAtStartPar
\sphinxstylestrong{See Also:}

\sphinxAtStartPar
\sphinxcode{\sphinxupquote{MakeNGonPoints}}, \sphinxcode{\sphinxupquote{Polygon}}, \sphinxcode{\sphinxupquote{RotatePoints}}, \sphinxcode{\sphinxupquote{ZoomPoints}}

\index{NonOverlapLayout@\spxentry{NonOverlapLayout}}\ignorespaces 

\subsection{NonOverlapLayout()}
\label{\detokenize{reference/graphics:nonoverlaplayout}}\label{\detokenize{reference/graphics:index-12}}
\sphinxAtStartPar
\sphinxstyleemphasis{Creates a set of num points that don’t overlap, but fails gracefully}

\sphinxAtStartPar
\sphinxstylestrong{Description:}

\sphinxAtStartPar
Creates a set of num points in a xy range, that have a (soft) minimum tolerance of ‘tol’ between points.  That is, to the extent possible, the returned points will have a minumum distance between them of \sphinxcode{\sphinxupquote{\textless{}tol\textgreater{}}}.  This may not be possible or be very difficult, and so after a limited number of attempts (by default, 100), the algorithm will return the current configuration, which may have some violations of the minimum tolerance rule, but it will usually be fairly good.    The algorithm works by initializing with a random set of points, then computing a pairwise distance matrix between all points, finding the closest two points, and resampling one of them until its minumum distance is larger than the current.  Thus, each internal iteration uniformly improves (or keeps the configuration the same), and the worst points are reconfigured first, so that even if a configuration that does not satisfy the constraints, it will usually be very close.  Internally, the function (located in pebl\sphinxhyphen{}lib/Graphics.pbl) has a variable that controls how many steps are taken, called {\color{red}\bfseries{}\textasciigrave{}\textasciigrave{}}limit’’, which is set to 100.  For very compacted or very large iterations, this limit can be increased by editing the file or making a copy of the function.    The function usually returns fairly quickly, so it can often be used real\sphinxhyphen{}time between trials.  However, for complex enough configurations, it can take on the order of seconds; furthermore, more  complex configurations might take longer than less complex configurations, which could represent a potential confound (if more complex stimuli have longer ISIs).  Users should thus consider creating the configurations when the test is initialized, or created prior to the study and then saved out to a file for later use.   newpage

\sphinxAtStartPar
\sphinxstylestrong{Usage:}

\begin{sphinxVerbatim}[commandchars=\\\{\}]
\PYG{k}{define}\PYG{+w}{ }\PYG{n+nf}{NonOverlapLayout}\PYG{p}{(}\PYG{p}{.}\PYG{p}{.}\PYG{p}{.}\PYG{p}{)}
\end{sphinxVerbatim}

\sphinxAtStartPar
\sphinxstylestrong{Example:}

\begin{sphinxVerbatim}[commandchars=\\\{\}]
E\PYG{n+nv}{xample}\PYG{+w}{ }PEBL\PYG{+w}{ }P\PYG{n+nv}{rogram}\PYG{+w}{ }\PYG{n+nv}{using}\PYG{+w}{ }N\PYG{n+nv}{onoverlapLayout}\PYG{o}{:}

\PYG{k}{define}\PYG{+w}{ }\PYG{n+nf}{Start}\PYG{p}{(}\PYG{n+nv}{p}\PYG{p}{)}
\PYG{+w}{ }\PYG{p}{\PYGZob{}}
\PYG{+w}{   }\PYG{n+nv}{win}\PYG{+w}{ }\PYG{o}{\PYGZlt{}\PYGZhy{}}\PYG{+w}{ }\PYG{n+nf}{MakeWindow}\PYG{p}{(}\PYG{p}{)}
\PYG{+w}{   }\PYG{c+c1}{\PYGZsh{}\PYGZsh{} Make 25 points in a square in the middle}
\PYG{+w}{   }\PYG{c+c1}{\PYGZsh{}\PYGZsh{} of the screen, a minimum of 50 pixels apart.}
\PYG{+w}{   }\PYG{c+c1}{\PYGZsh{}\PYGZsh{} This is too compact, but it will be OK.}

\PYG{+w}{   }\PYG{n+nv}{points}\PYG{+w}{ }\PYG{o}{\PYGZlt{}\PYGZhy{}}\PYG{+w}{ }\PYG{n+nf}{NonOverlapLayout}\PYG{p}{(}\PYG{l+m+mi}{100}\PYG{p}{,}\PYG{l+m+mi}{300}\PYG{p}{,}\PYG{l+m+mi}{200}\PYG{p}{,}\PYG{l+m+mi}{400}\PYG{p}{,}\PYG{l+m+mi}{50}\PYG{p}{,}\PYG{l+m+mi}{25}\PYG{p}{)}
\PYG{+w}{   }\PYG{n+nv}{circs}\PYG{+w}{ }\PYG{o}{\PYGZlt{}\PYGZhy{}}\PYG{+w}{ }\PYG{p}{[}\PYG{p}{]}
\PYG{+w}{   }\PYG{c+c1}{\PYGZsh{}\PYGZsh{}This should non\PYGZhy{}overlapping circles of radius 25}
\PYG{+w}{   }\PYG{k}{loop}\PYG{p}{(}\PYG{n+nv}{i}\PYG{p}{,}\PYG{n+nv}{points}\PYG{p}{)}
\PYG{+w}{    }\PYG{p}{\PYGZob{}}
\PYG{+w}{       }\PYG{n+nv}{tmp}\PYG{+w}{ }\PYG{o}{\PYGZlt{}\PYGZhy{}}\PYG{+w}{ }\PYG{n+nf}{Circle}\PYG{p}{(}\PYG{n+nf}{First}\PYG{p}{(}\PYG{n+nv}{i}\PYG{p}{)}\PYG{p}{,}\PYG{n+nf}{Second}\PYG{p}{(}\PYG{n+nv}{i}\PYG{p}{)}\PYG{p}{,}\PYG{l+m+mi}{25}\PYG{p}{,}
\PYG{+w}{                     }\PYG{n+nf}{MakeColor}\PYG{p}{(}\PYG{l+s+s2}{\PYGZdq{}blue\PYGZdq{}}\PYG{p}{)}\PYG{p}{,}\PYG{l+m+mi}{0}\PYG{p}{)}
\PYG{+w}{       }\PYG{n+nf}{AddObject}\PYG{p}{(}\PYG{n+nv}{tmp}\PYG{p}{,}\PYG{n+nv}{win}\PYG{p}{)}
\PYG{+w}{       }\PYG{n+nv}{circs}\PYG{+w}{ }\PYG{o}{\PYGZlt{}\PYGZhy{}}\PYG{+w}{ }\PYG{n+nf}{Append}\PYG{p}{(}\PYG{n+nv}{circs}\PYG{p}{,}\PYG{n+nv}{tmp}\PYG{p}{)}
\PYG{+w}{    }\PYG{p}{\PYGZcb{}}


\PYG{+w}{   }\PYG{n+nv}{rect1}\PYG{+w}{ }\PYG{o}{\PYGZlt{}\PYGZhy{}}\PYG{+w}{ }\PYG{n+nf}{Square}\PYG{p}{(}\PYG{l+m+mi}{200}\PYG{p}{,}\PYG{l+m+mi}{300}\PYG{p}{,}\PYG{l+m+mi}{200}\PYG{p}{,}\PYG{n+nf}{MakeColor}\PYG{p}{(}\PYG{l+s+s2}{\PYGZdq{}black\PYGZdq{}}\PYG{p}{)}\PYG{p}{,}\PYG{l+m+mi}{0}\PYG{p}{)}
\PYG{+w}{   }\PYG{n+nv}{rect2}\PYG{+w}{ }\PYG{o}{\PYGZlt{}\PYGZhy{}}\PYG{+w}{ }\PYG{n+nf}{Square}\PYG{p}{(}\PYG{l+m+mi}{600}\PYG{p}{,}\PYG{l+m+mi}{300}\PYG{p}{,}\PYG{l+m+mi}{200}\PYG{p}{,}\PYG{n+nf}{MakeColor}\PYG{p}{(}\PYG{l+s+s2}{\PYGZdq{}black\PYGZdq{}}\PYG{p}{)}\PYG{p}{,}\PYG{l+m+mi}{0}\PYG{p}{)}

\PYG{+w}{   }\PYG{n+nf}{AddObject}\PYG{p}{(}\PYG{n+nv}{rect1}\PYG{p}{,}\PYG{n+nv}{win}\PYG{p}{)}
\PYG{+w}{   }\PYG{n+nf}{AddObject}\PYG{p}{(}\PYG{n+nv}{rect2}\PYG{p}{,}\PYG{n+nv}{win}\PYG{p}{)}
\PYG{+w}{   }\PYG{c+c1}{\PYGZsh{}\PYGZsh{}Reduce the tolerance: this one should be bettter}
\PYG{+w}{   }\PYG{n+nv}{points}\PYG{+w}{ }\PYG{o}{\PYGZlt{}\PYGZhy{}}\PYG{+w}{ }\PYG{n+nf}{NonOverlapLayout}\PYG{p}{(}\PYG{l+m+mi}{500}\PYG{p}{,}\PYG{l+m+mi}{700}\PYG{p}{,}\PYG{l+m+mi}{200}\PYG{p}{,}\PYG{l+m+mi}{400}\PYG{p}{,}\PYG{l+m+mi}{50}\PYG{p}{,}\PYG{l+m+mi}{15}\PYG{p}{)}


\PYG{+w}{   }\PYG{c+c1}{\PYGZsh{}\PYGZsh{}This should non\PYGZhy{}overlapping circles of radius 15}
\PYG{+w}{   }\PYG{k}{loop}\PYG{p}{(}\PYG{n+nv}{i}\PYG{p}{,}\PYG{n+nv}{points}\PYG{p}{)}
\PYG{+w}{    }\PYG{p}{\PYGZob{}}
\PYG{+w}{       }\PYG{n+nv}{tmp}\PYG{+w}{ }\PYG{o}{\PYGZlt{}\PYGZhy{}}\PYG{+w}{ }\PYG{n+nf}{Circle}\PYG{p}{(}\PYG{n+nf}{First}\PYG{p}{(}\PYG{n+nv}{i}\PYG{p}{)}\PYG{p}{,}\PYG{n+nf}{Second}\PYG{p}{(}\PYG{n+nv}{i}\PYG{p}{)}\PYG{p}{,}
\PYG{+w}{                     }\PYG{l+m+mi}{15}\PYG{p}{,}\PYG{n+nf}{MakeColor}\PYG{p}{(}\PYG{l+s+s2}{\PYGZdq{}blue\PYGZdq{}}\PYG{p}{)}\PYG{p}{,}\PYG{l+m+mi}{0}\PYG{p}{)}
\PYG{+w}{       }\PYG{n+nf}{AddObject}\PYG{p}{(}\PYG{n+nv}{tmp}\PYG{p}{,}\PYG{n+nv}{win}\PYG{p}{)}
\PYG{+w}{        }\PYG{n+nv}{circs}\PYG{+w}{ }\PYG{o}{\PYGZlt{}\PYGZhy{}}\PYG{+w}{ }\PYG{n+nf}{Append}\PYG{p}{(}\PYG{n+nv}{circs}\PYG{p}{,}\PYG{n+nv}{tmp}\PYG{p}{)}
\PYG{+w}{    }\PYG{p}{\PYGZcb{}}
\PYG{+w}{   }\PYG{n+nf}{Draw}\PYG{p}{(}\PYG{p}{)}
\PYG{+w}{   }\PYG{n+nf}{WaitForAnyKeyPress}\PYG{p}{(}\PYG{p}{)}

\PYG{p}{\PYGZcb{}}

T\PYG{n+nv}{he}\PYG{+w}{ }\PYG{n+nv}{output}\PYG{+w}{ }\PYG{n+nv}{of}\PYG{+w}{ }\PYG{n+nv}{the}\PYG{+w}{ }\PYG{n+nv}{above}\PYG{+w}{ }\PYG{n+nv}{program}\PYG{+w}{ }\PYG{n+nv}{is}\PYG{+w}{ }\PYG{n+nv}{shown}\PYG{+w}{ }\PYG{n+nv}{below}\PYG{p}{.}\PYG{+w}{  }E\PYG{n+nv}{ven}\PYG{+w}{ }\PYG{n+nv}{for}\PYG{+w}{ }\PYG{n+nv}{the}\PYG{+w}{ }\PYG{n+nv}{left}\PYG{+w}{ }\PYG{n+nv}{configuration}\PYG{p}{,}\PYG{+w}{ }\PYG{n+nv}{which}\PYG{+w}{ }\PYG{n+nv}{is}\PYG{+w}{ }\PYG{n+nv}{too}\PYG{+w}{ }\PYG{n+nv}{compact}\PYG{+w}{ }\PYG{p}{(}\PYG{k}{and}\PYG{+w}{ }\PYG{n+nv}{which}\PYG{+w}{ }\PYG{n+nv}{takes}\PYG{+w}{ }\PYG{n+nv}{a}\PYG{+w}{ }\PYG{n+nv}{couple}\PYG{+w}{ }\PYG{n+nv}{seconds}\PYG{+w}{ }\PYG{n+nv}{to}\PYG{+w}{ }\PYG{n+nv}{run}\PYG{p}{)}\PYG{p}{,}\PYG{+w}{ }\PYG{n+nv}{the}\PYG{+w}{ }\PYG{n+nv}{targets}\PYG{+w}{ }\PYG{n+nv}{are}\PYG{+w}{ }\PYG{n+nv}{fairly}\PYG{+w}{ }\PYG{n+nv}{well}\PYG{+w}{ }\PYG{n+nv}{distributed}\PYG{p}{.}
\end{sphinxVerbatim}

\sphinxAtStartPar
\sphinxstylestrong{See Also:}

\sphinxAtStartPar
\sphinxcode{\sphinxupquote{LayoutGrid()}}

\index{Plus@\spxentry{Plus}}\ignorespaces 

\subsection{Plus()}
\label{\detokenize{reference/graphics:plus}}\label{\detokenize{reference/graphics:index-13}}
\sphinxAtStartPar
\sphinxstyleemphasis{Creates a plus sign as a useable polygon which can be added to a window directly.}

\sphinxAtStartPar
\sphinxstylestrong{Description:}

\sphinxAtStartPar
Creates a polygon in the shape of a  plus sign. Arguments include position in window.
\begin{itemize}
\item {} 
\sphinxAtStartPar
\sphinxcode{\sphinxupquote{\textless{}x\textgreater{}}} and \sphinxcode{\sphinxupquote{\textless{}y\textgreater{}}} is the position of the center

\item {} 
\sphinxAtStartPar
\sphinxcode{\sphinxupquote{\textless{}size\textgreater{}}} or the size of the plus sign in pixels

\item {} 
\sphinxAtStartPar
\sphinxcode{\sphinxupquote{\textless{}width\textgreater{}}} thickness of the plus

\item {} 
\sphinxAtStartPar
\sphinxcode{\sphinxupquote{\textless{}color\textgreater{}}} is a color object (not just the name)

\sphinxAtStartPar
Like other drawn objects, the plus must then be added to the window to appear.

\end{itemize}

\sphinxAtStartPar
\sphinxstylestrong{Usage:}

\begin{sphinxVerbatim}[commandchars=\\\{\}]
\PYG{k}{define}\PYG{+w}{ }\PYG{n+nf}{Plus}\PYG{p}{(}\PYG{p}{.}\PYG{p}{.}\PYG{p}{.}\PYG{p}{)}
\end{sphinxVerbatim}

\sphinxAtStartPar
\sphinxstylestrong{Example:}

\begin{sphinxVerbatim}[commandchars=\\\{\}]
\PYG{n+nv}{win}\PYG{+w}{ }\PYG{o}{\PYGZlt{}\PYGZhy{}}\PYG{+w}{ }\PYG{n+nf}{MakeWindow}\PYG{p}{(}\PYG{p}{)}
\PYG{n+nv}{p1}\PYG{+w}{ }\PYG{o}{\PYGZlt{}\PYGZhy{}}\PYG{+w}{ }\PYG{n+nf}{Plus}\PYG{p}{(}\PYG{l+m+mi}{100}\PYG{p}{,}\PYG{l+m+mi}{100}\PYG{p}{,}\PYG{l+m+mi}{80}\PYG{p}{,}\PYG{l+m+mi}{15}\PYG{p}{,}\PYG{n+nf}{MakeColor}\PYG{p}{(}\PYG{l+s+s2}{\PYGZdq{}red\PYGZdq{}}\PYG{p}{)}\PYG{p}{)}
\PYG{n+nf}{AddObject}\PYG{p}{(}\PYG{n+nv}{p1}\PYG{p}{,}\PYG{n+nv}{win}\PYG{p}{)}
\PYG{n+nf}{Draw}\PYG{p}{(}\PYG{p}{)}
\end{sphinxVerbatim}

\sphinxAtStartPar
\sphinxstylestrong{See Also:}

\sphinxAtStartPar
\sphinxcode{\sphinxupquote{BlockE()}}, \sphinxcode{\sphinxupquote{Polygon()}}, \sphinxcode{\sphinxupquote{MakeStarPoints()}},
\sphinxcode{\sphinxupquote{MakeNGonPoints()}}

\index{ReflectPoints@\spxentry{ReflectPoints}}\ignorespaces 

\subsection{ReflectPoints()}
\label{\detokenize{reference/graphics:reflectpoints}}\label{\detokenize{reference/graphics:index-14}}
\sphinxAtStartPar
\sphinxstyleemphasis{Reflects points on vertical axis}

\sphinxAtStartPar
\sphinxstylestrong{Description:}

\sphinxAtStartPar
Takes a set of points (defined in a joined list  {[}{[}x1,x2,x3,…{]},{[}y1,y2,y3,…{]}{]} and reflects them around the vertical axis x=0, returning a similar {[}{[}x{]},{[}y{]}{]} list.  Identical to \sphinxcode{\sphinxupquote{ZoomPoints(pts,\sphinxhyphen{}1,1)}}

\sphinxAtStartPar
\sphinxstylestrong{Usage:}

\begin{sphinxVerbatim}[commandchars=\\\{\}]
\PYG{k}{define}\PYG{+w}{ }\PYG{n+nf}{ReflectPoints}\PYG{p}{(}\PYG{p}{.}\PYG{p}{.}\PYG{p}{.}\PYG{p}{)}
\end{sphinxVerbatim}

\sphinxAtStartPar
\sphinxstylestrong{Example:}

\begin{sphinxVerbatim}[commandchars=\\\{\}]
\PYG{n+nv}{points}\PYG{+w}{ }\PYG{o}{\PYGZlt{}\PYGZhy{}}\PYG{+w}{ }\PYG{p}{[}\PYG{p}{[}\PYG{l+m+mi}{1}\PYG{p}{,}\PYG{l+m+mi}{2}\PYG{p}{,}\PYG{l+m+mi}{3}\PYG{p}{,}\PYG{l+m+mi}{4}\PYG{p}{]}\PYG{p}{,}\PYG{p}{[}\PYG{l+m+mi}{20}\PYG{p}{,}\PYG{l+m+mi}{21}\PYG{p}{,}\PYG{l+m+mi}{22}\PYG{p}{,}\PYG{l+m+mi}{23}\PYG{p}{]}\PYG{p}{]}
\PYG{n+nv}{newpoints}\PYG{+w}{ }\PYG{o}{\PYGZlt{}\PYGZhy{}}\PYG{+w}{ }\PYG{n+nf}{ReflectPoints}\PYG{p}{(}\PYG{n+nv}{points}\PYG{p}{)}
\end{sphinxVerbatim}

\sphinxAtStartPar
\sphinxstylestrong{See Also:}

\sphinxAtStartPar
\sphinxcode{\sphinxupquote{ZoomPoints()}}, \sphinxcode{\sphinxupquote{RotatePoints()}}

\index{ResetCanvas@\spxentry{ResetCanvas}}\ignorespaces 

\subsection{ResetCanvas()}
\label{\detokenize{reference/graphics:resetcanvas}}\label{\detokenize{reference/graphics:index-15}}
\sphinxAtStartPar
\sphinxstyleemphasis{Resets a canvas to its background, erasing anything drawn on the canvas}

\sphinxAtStartPar
\sphinxstylestrong{Description:}

\sphinxAtStartPar
Resets a canvas, so that anything drawn onto it is   erased and returned to its background color.  Implemented by   resetting the background color to itself:

\begin{sphinxVerbatim}[commandchars=\\\{\}]
 canvas.color \PYGZlt{}\PYGZhy{} canvas.

The function does not return the canvas,   but has the side effect of resetting it.
\end{sphinxVerbatim}

\sphinxAtStartPar
\sphinxstylestrong{Usage:}

\begin{sphinxVerbatim}[commandchars=\\\{\}]
\PYG{k}{define}\PYG{+w}{ }\PYG{n+nf}{ResetCanvas}\PYG{p}{(}\PYG{p}{.}\PYG{p}{.}\PYG{p}{.}\PYG{p}{)}
\end{sphinxVerbatim}

\sphinxAtStartPar
\sphinxstylestrong{Example:}

\begin{sphinxVerbatim}[commandchars=\\\{\}]
\PYG{c+c1}{\PYGZsh{}create a canvas, add pixel noise, then reset and repeat.}
\PYG{k}{define}\PYG{+w}{ }\PYG{n+nf}{Start}\PYG{p}{(}\PYG{n+nv}{p}\PYG{p}{)}
\PYG{p}{\PYGZob{}}
\PYG{+w}{  }\PYG{n+nv+vg}{gWin}\PYG{+w}{ }\PYG{o}{\PYGZlt{}\PYGZhy{}}\PYG{+w}{ }\PYG{n+nf}{MakeWindow}\PYG{p}{(}\PYG{p}{)}
\PYG{+w}{  }\PYG{n+nv}{canvas}\PYG{+w}{ }\PYG{o}{\PYGZlt{}\PYGZhy{}}\PYG{+w}{ }\PYG{n+nf}{MakeCanvas}\PYG{p}{(}\PYG{l+m+mi}{100}\PYG{p}{,}\PYG{l+m+mi}{100}\PYG{p}{,}\PYG{n+nf}{MakeColor}\PYG{p}{(}\PYG{l+s+s2}{\PYGZdq{}black\PYGZdq{}}\PYG{p}{)}\PYG{p}{)}
\PYG{+w}{  }\PYG{n+nf}{AddObject}\PYG{p}{(}\PYG{n+nv}{canvas}\PYG{p}{,}\PYG{n+nv+vg}{gWin}\PYG{p}{)}\PYG{p}{;}\PYG{+w}{ }\PYG{n+nf}{Move}\PYG{p}{(}\PYG{n+nv}{canvas}\PYG{p}{,}\PYG{l+m+mi}{300}\PYG{p}{,}\PYG{l+m+mi}{300}\PYG{p}{)}
\PYG{+w}{  }\PYG{n+nf}{Draw}\PYG{p}{(}\PYG{p}{)}
\PYG{+w}{  }\PYG{n+nv}{white}\PYG{+w}{ }\PYG{o}{\PYGZlt{}\PYGZhy{}}\PYG{+w}{ }\PYG{n+nf}{MakeColor}\PYG{p}{(}\PYG{l+s+s2}{\PYGZdq{}white\PYGZdq{}}\PYG{p}{)}
\PYG{+w}{  }\PYG{c+c1}{\PYGZsh{}\PYGZsh{}add pixel noise}
\PYG{+w}{  }\PYG{n+nv}{j}\PYG{+w}{ }\PYG{o}{\PYGZlt{}\PYGZhy{}}\PYG{+w}{ }\PYG{l+m+mi}{1}
\PYG{+w}{  }\PYG{k}{while}\PYG{p}{(}\PYG{n+nv}{j}\PYG{+w}{ }\PYG{o}{\PYGZlt{}}\PYG{+w}{ }\PYG{l+m+mi}{5}\PYG{p}{)}
\PYG{+w}{   }\PYG{p}{\PYGZob{}}
\PYG{+w}{  }\PYG{n+nv}{i}\PYG{+w}{ }\PYG{o}{\PYGZlt{}\PYGZhy{}}\PYG{+w}{ }\PYG{l+m+mi}{1}
\PYG{+w}{  }\PYG{k}{while}\PYG{p}{(}\PYG{n+nv}{i}\PYG{+w}{ }\PYG{o}{\PYGZlt{}}\PYG{+w}{ }\PYG{l+m+mi}{200}\PYG{p}{)}
\PYG{+w}{   }\PYG{p}{\PYGZob{}}
\PYG{+w}{     }\PYG{n+nf}{SetPixel}\PYG{p}{(}\PYG{n+nv}{canvas}\PYG{p}{,}\PYG{n+nf}{Round}\PYG{p}{(}\PYG{n+nf}{Random}\PYG{p}{(}\PYG{p}{)}\PYG{o}{*}\PYG{l+m+mi}{100}\PYG{p}{)}\PYG{p}{,}
\PYG{+w}{              }\PYG{n+nf}{Round}\PYG{p}{(}\PYG{n+nf}{Random}\PYG{p}{(}\PYG{p}{)}\PYG{o}{*}\PYG{l+m+mi}{100}\PYG{p}{)}\PYG{p}{,}\PYG{n+nv}{white}\PYG{p}{)}
\PYG{+w}{     }\PYG{n+nv}{i}\PYG{+w}{ }\PYG{o}{\PYGZlt{}\PYGZhy{}}\PYG{+w}{ }\PYG{n+nv}{i}\PYG{+w}{ }\PYG{o}{+}\PYG{l+m+mi}{1}
\PYG{+w}{   }\PYG{p}{\PYGZcb{}}
\PYG{+w}{  }\PYG{n+nf}{Draw}\PYG{p}{(}\PYG{p}{)}
\PYG{+w}{  }\PYG{n+nf}{WaitForAnyKeyPress}\PYG{p}{(}\PYG{p}{)}
\PYG{+w}{  }\PYG{n+nf}{ResetCanvas}\PYG{p}{(}\PYG{n+nv}{canvas}\PYG{p}{)}
\PYG{+w}{  }\PYG{n+nf}{Draw}\PYG{p}{(}\PYG{p}{)}
\PYG{+w}{   }\PYG{n+nv}{j}\PYG{+w}{ }\PYG{o}{\PYGZlt{}\PYGZhy{}}\PYG{+w}{ }\PYG{n+nv}{j}\PYG{+w}{ }\PYG{o}{+}\PYG{+w}{ }\PYG{l+m+mi}{1}
\PYG{+w}{  }\PYG{p}{\PYGZcb{}}
\PYG{+w}{  }\PYG{n+nf}{WaitForAnyKeyPress}\PYG{p}{(}\PYG{p}{)}

\PYG{p}{\PYGZcb{}}
\end{sphinxVerbatim}

\sphinxAtStartPar
\sphinxstylestrong{See Also:}

\sphinxAtStartPar
\sphinxcode{\sphinxupquote{SetPixel()}}, \sphinxcode{\sphinxupquote{MakeCanvas()}}, \sphinxcode{\sphinxupquote{Draw()}}

\index{RGBtoHSV@\spxentry{RGBtoHSV}}\ignorespaces 

\subsection{RGBtoHSV()}
\label{\detokenize{reference/graphics:rgbtohsv}}\label{\detokenize{reference/graphics:index-16}}
\sphinxAtStartPar
\sphinxstyleemphasis{Converts a color to HSV triple}

\sphinxAtStartPar
\sphinxstylestrong{Description:}

\sphinxAtStartPar
Converts a color object to HSV values.  May be useful for computing color\sphinxhyphen{}space distances an so on.  No HSVtoRGB is currently implemented.

\sphinxAtStartPar
\sphinxstylestrong{Usage:}

\begin{sphinxVerbatim}[commandchars=\\\{\}]
\PYG{k}{define}\PYG{+w}{ }\PYG{n+nf}{RGBtoHSV}\PYG{p}{(}\PYG{p}{.}\PYG{p}{.}\PYG{p}{.}\PYG{p}{)}
\end{sphinxVerbatim}

\sphinxAtStartPar
\sphinxstylestrong{Example:}

\begin{sphinxVerbatim}[commandchars=\\\{\}]
\PYG{n+nv}{x}\PYG{+w}{ }\PYG{o}{\PYGZlt{}\PYGZhy{}}\PYG{+w}{ }\PYG{n+nf}{RGBtoHSV}\PYG{p}{(}\PYG{n+nf}{MakeColor}\PYG{p}{(}\PYGZdq{}\PYG{n+nv}{red}\PYG{p}{)}\PYG{p}{)}
\end{sphinxVerbatim}

\sphinxAtStartPar
\sphinxstylestrong{See Also:}

\sphinxAtStartPar
\sphinxcode{\sphinxupquote{MakeColor()}}, \sphinxcode{\sphinxupquote{MakeColorRGB()}}

\index{RotatePoints@\spxentry{RotatePoints}}\ignorespaces 

\subsection{RotatePoints()}
\label{\detokenize{reference/graphics:rotatepoints}}\label{\detokenize{reference/graphics:index-17}}
\sphinxAtStartPar
\sphinxstylestrong{Description:}

\sphinxAtStartPar
Takes a set of points (defined in a joined list  \sphinxcode{\sphinxupquote{{[}{[}x1,x2,x3,...{]},}} \sphinxcode{\sphinxupquote{{[}y1,y2,y3,...{]}{]}}} and rotates them \sphinxcode{\sphinxupquote{\textless{}angle\textgreater{}}} degrees around the point \sphinxcode{\sphinxupquote{{[}0,0{]}}},  returning a similar \sphinxcode{\sphinxupquote{{[}{[}x{]},{[}y{]}{]}}} list.

\sphinxAtStartPar
\sphinxstylestrong{Usage:}

\begin{sphinxVerbatim}[commandchars=\\\{\}]
\PYG{k}{define}\PYG{+w}{ }\PYG{n+nf}{RotatePoints}\PYG{p}{(}\PYG{p}{.}\PYG{p}{.}\PYG{p}{.}\PYG{p}{)}
\end{sphinxVerbatim}

\sphinxAtStartPar
\sphinxstylestrong{Example:}

\begin{sphinxVerbatim}[commandchars=\\\{\}]
\PYG{n+nv}{points}\PYG{+w}{ }\PYG{o}{\PYGZlt{}\PYGZhy{}}\PYG{+w}{ }\PYG{p}{[}\PYG{p}{[}\PYG{l+m+mi}{1}\PYG{p}{,}\PYG{l+m+mi}{2}\PYG{p}{,}\PYG{l+m+mi}{3}\PYG{p}{,}\PYG{l+m+mi}{4}\PYG{p}{]}\PYG{p}{,}\PYG{p}{[}\PYG{l+m+mi}{20}\PYG{p}{,}\PYG{l+m+mi}{21}\PYG{p}{,}\PYG{l+m+mi}{22}\PYG{p}{,}\PYG{l+m+mi}{23}\PYG{p}{]}\PYG{p}{]}
\PYG{n+nv}{newpoints}\PYG{+w}{ }\PYG{o}{\PYGZlt{}\PYGZhy{}}\PYG{+w}{ }\PYG{n+nf}{RotatePoints}\PYG{p}{(}\PYG{n+nv}{points}\PYG{p}{,}\PYG{l+m+mi}{10}\PYG{p}{)}
\end{sphinxVerbatim}

\sphinxAtStartPar
\sphinxstylestrong{See Also:}

\sphinxAtStartPar
\sphinxcode{\sphinxupquote{ZoomPoints()}}, \sphinxcode{\sphinxupquote{ReflectPoints()}}

\index{SegmentsIntersect@\spxentry{SegmentsIntersect}}\ignorespaces 

\subsection{SegmentsIntersect()}
\label{\detokenize{reference/graphics:segmentsintersect}}\label{\detokenize{reference/graphics:index-18}}
\sphinxAtStartPar
\sphinxstyleemphasis{Determines whether line segment ab intersects cd.}

\sphinxAtStartPar
\sphinxstylestrong{Description:}

\sphinxAtStartPar
Determines whether two line segments, defined by   four xy point pairs, intersect. Two line segments that share a   corner return 0, although they could be considered to intersect.  This function is defined in pebl\sphinxhyphen{}lib/Graphics.pbl

\sphinxAtStartPar
\sphinxstylestrong{Usage:}

\begin{sphinxVerbatim}[commandchars=\\\{\}]
\PYG{k}{define}\PYG{+w}{ }\PYG{n+nf}{SegmentsIntersect}\PYG{p}{(}\PYG{p}{.}\PYG{p}{.}\PYG{p}{.}\PYG{p}{)}
\end{sphinxVerbatim}

\sphinxAtStartPar
\sphinxstylestrong{Example:}

\begin{sphinxVerbatim}[commandchars=\\\{\}]
\PYG{n+nf}{SegmentsIntersect}\PYG{p}{(}\PYG{l+m+mi}{1}\PYG{p}{,}\PYG{l+m+mi}{0}\PYG{p}{,}\PYG{l+m+mi}{2}\PYG{p}{,}\PYG{l+m+mi}{0}\PYG{p}{,}
\PYG{+w}{                  }\PYG{l+m+mi}{1}\PYG{p}{,}\PYG{l+m+mi}{2}\PYG{p}{,}\PYG{l+m+mi}{2}\PYG{p}{,}\PYG{l+m+mi}{2}\PYG{p}{)}\PYG{+w}{  }\PYG{c+c1}{\PYGZsh{}0}

\PYG{c+c1}{\PYGZsh{}returns 0, though they share (1,0)}
\PYG{n+nf}{SegmentsIntersect}\PYG{p}{(}\PYG{l+m+mi}{1}\PYG{p}{,}\PYG{l+m+mi}{0}\PYG{p}{,}\PYG{l+m+mi}{2}\PYG{p}{,}\PYG{l+m+mi}{0}\PYG{p}{,}
\PYG{+w}{                   }\PYG{l+m+mi}{1}\PYG{p}{,}\PYG{l+m+mi}{0}\PYG{p}{,}\PYG{l+m+mi}{2}\PYG{p}{,}\PYG{l+m+mi}{2}\PYG{p}{)}
\PYG{n+nf}{SegmentsIntersect}\PYG{p}{(}\PYG{l+m+mi}{1}\PYG{p}{,}\PYG{l+m+mi}{1}\PYG{p}{,}\PYG{l+m+mi}{3}\PYG{p}{,}\PYG{l+m+mi}{1}\PYG{p}{,}
\PYG{+w}{                  }\PYG{l+m+mi}{2}\PYG{p}{,}\PYG{l+m+mi}{2}\PYG{p}{,}\PYG{l+m+mi}{2}\PYG{p}{,}\PYG{l+m+mi}{0}\PYG{p}{)}\PYG{+w}{  }\PYG{c+c1}{\PYGZsh{}1}
\end{sphinxVerbatim}

\sphinxAtStartPar
\sphinxstylestrong{See Also:}

\sphinxAtStartPar
\sphinxcode{\sphinxupquote{GetAngle3}}, \sphinxcode{\sphinxupquote{ToRight}}

\index{ToRight@\spxentry{ToRight}}\ignorespaces 

\subsection{ToRight()}
\label{\detokenize{reference/graphics:toright}}\label{\detokenize{reference/graphics:index-19}}
\sphinxAtStartPar
\sphinxstyleemphasis{Determines whether p3 is te the right of line p1p2}

\sphinxAtStartPar
\sphinxstylestrong{Description:}

\sphinxAtStartPar
Determines whether a point p3 is ‘to the right’   of a line segment defined by p1  to p2.  Works essentially by   computing the determinant.

\sphinxAtStartPar
\sphinxstylestrong{Usage:}

\begin{sphinxVerbatim}[commandchars=\\\{\}]
\PYG{k}{define}\PYG{+w}{ }\PYG{n+nf}{ToRight}\PYG{p}{(}\PYG{p}{.}\PYG{p}{.}\PYG{p}{.}\PYG{p}{)}
\end{sphinxVerbatim}

\sphinxAtStartPar
\sphinxstylestrong{Example:}

\begin{sphinxVerbatim}[commandchars=\\\{\}]
\PYG{n+nv}{a}\PYG{+w}{ }\PYG{o}{\PYGZlt{}\PYGZhy{}}\PYG{+w}{ }\PYG{p}{[}\PYG{l+m+mi}{100}\PYG{p}{,}\PYG{l+m+mi}{0}\PYG{p}{]}
\PYG{n+nv}{b}\PYG{+w}{ }\PYG{o}{\PYGZlt{}\PYGZhy{}}\PYG{+w}{ }\PYG{p}{[}\PYG{l+m+mi}{100}\PYG{p}{,}\PYG{l+m+mi}{100}\PYG{p}{]}
\PYG{n+nv}{c}\PYG{+w}{ }\PYG{o}{\PYGZlt{}\PYGZhy{}}\PYG{+w}{ }\PYG{p}{[}\PYG{l+m+mi}{150}\PYG{p}{,}\PYG{l+m+mi}{50}\PYG{p}{]}
\PYG{n+nf}{ToRight}\PYG{p}{(}\PYG{n+nv}{a}\PYG{p}{,}\PYG{n+nv}{b}\PYG{p}{,}\PYG{n+nv}{c}\PYG{p}{)}\PYG{+w}{ }\PYG{c+c1}{\PYGZsh{} returns 1; true}
\PYG{n+nf}{ToRight}\PYG{p}{(}\PYG{n+nv}{b}\PYG{p}{,}\PYG{n+nv}{a}\PYG{p}{,}\PYG{n+nv}{c}\PYG{p}{)}\PYG{+w}{ }\PYG{c+c1}{\PYGZsh{} returns 0; false}
\end{sphinxVerbatim}

\sphinxAtStartPar
\sphinxstylestrong{See Also:}

\sphinxAtStartPar
\sphinxcode{\sphinxupquote{GetAngle()}}, \sphinxcode{\sphinxupquote{GetAngle3()}}, \sphinxcode{\sphinxupquote{SegmentsIntersect()}}

\index{ZoomPoints@\spxentry{ZoomPoints}}\ignorespaces 

\subsection{ZoomPoints()}
\label{\detokenize{reference/graphics:zoompoints}}\label{\detokenize{reference/graphics:index-20}}
\sphinxAtStartPar
\sphinxstyleemphasis{Zooms a set of points in 2 directions}

\sphinxAtStartPar
\sphinxstylestrong{Description:}

\sphinxAtStartPar
Takes a set of points (defined in a joined list  {[}{[}x1,x2,x3,…{]},{[}y1,y2,y3,…{]}{]} and adjusts them in the x and y direction independently, returning a similar {[}{[}x{]},{[}y{]}{]} list.  Note: The original points should be centered at zero, because the get adjusted relative to zero, not relative to their center.

\sphinxAtStartPar
\sphinxstylestrong{Usage:}

\begin{sphinxVerbatim}[commandchars=\\\{\}]
\PYG{k}{define}\PYG{+w}{ }\PYG{n+nf}{ZoomPoints}\PYG{p}{(}\PYG{p}{.}\PYG{p}{.}\PYG{p}{.}\PYG{p}{)}
\end{sphinxVerbatim}

\sphinxAtStartPar
\sphinxstylestrong{Example:}

\begin{sphinxVerbatim}[commandchars=\\\{\}]
\PYG{n+nv}{points}\PYG{+w}{ }\PYG{o}{\PYGZlt{}\PYGZhy{}}\PYG{+w}{ }\PYG{p}{[}\PYG{p}{[}\PYG{l+m+mi}{1}\PYG{p}{,}\PYG{l+m+mi}{2}\PYG{p}{,}\PYG{l+m+mi}{3}\PYG{p}{,}\PYG{l+m+mi}{4}\PYG{p}{]}\PYG{p}{,}\PYG{p}{[}\PYG{l+m+mi}{20}\PYG{p}{,}\PYG{l+m+mi}{21}\PYG{p}{,}\PYG{l+m+mi}{22}\PYG{p}{,}\PYG{l+m+mi}{23}\PYG{p}{]}\PYG{p}{]}
\PYG{n+nv}{newpoints}\PYG{+w}{ }\PYG{o}{\PYGZlt{}\PYGZhy{}}\PYG{+w}{ }\PYG{n+nf}{ZoomPoints}\PYG{p}{(}\PYG{n+nv}{points}\PYG{p}{,}\PYG{l+m+mi}{2}\PYG{p}{,}\PYG{l+m+mf}{.5}\PYG{p}{)}
\PYG{c+c1}{\PYGZsh{}\PYGZsh{}Produces [[2,4,6,8],[10,11.5,11,11.5]]}
\end{sphinxVerbatim}

\sphinxAtStartPar
\sphinxstylestrong{See Also:}

\sphinxAtStartPar
\sphinxcode{\sphinxupquote{RotatePoints()}}, \sphinxcode{\sphinxupquote{ReflectPoints()}}


\subsection{Functions Pending Documentation}
\label{\detokenize{reference/graphics:functions-pending-documentation}}
\index{GetMinDist@\spxentry{GetMinDist}}\ignorespaces 

\subsection{GetMinDist()}
\label{\detokenize{reference/graphics:getmindist}}\label{\detokenize{reference/graphics:index-21}}
\sphinxAtStartPar
\sphinxstyleemphasis{Finds the minimum distance between any two points in a list}

\sphinxAtStartPar
\sphinxstylestrong{Description:}

\sphinxAtStartPar
Computes and returns the minimum distance between any pair of points in a list of {[}x,y{]} coordinate pairs. This function examines all possible pairs of points and returns the smallest distance found. Useful for validating point distributions, checking spacing constraints, or analyzing geometric layouts.

\sphinxAtStartPar
\sphinxstylestrong{Usage:}

\begin{sphinxVerbatim}[commandchars=\\\{\}]
\PYG{k}{define}\PYG{+w}{ }\PYG{n+nf}{GetMinDist}\PYG{p}{(}\PYG{n+nv}{pts}\PYG{p}{)}
\end{sphinxVerbatim}

\sphinxAtStartPar
\sphinxstylestrong{Example:}

\begin{sphinxVerbatim}[commandchars=\\\{\}]
\PYG{n+nv}{points}\PYG{+w}{ }\PYG{o}{\PYGZlt{}\PYGZhy{}}\PYG{+w}{ }\PYG{p}{[}\PYG{p}{[}\PYG{l+m+mi}{0}\PYG{p}{,}\PYG{+w}{ }\PYG{l+m+mi}{0}\PYG{p}{]}\PYG{p}{,}\PYG{+w}{ }\PYG{p}{[}\PYG{l+m+mi}{10}\PYG{p}{,}\PYG{+w}{ }\PYG{l+m+mi}{0}\PYG{p}{]}\PYG{p}{,}\PYG{+w}{ }\PYG{p}{[}\PYG{l+m+mi}{5}\PYG{p}{,}\PYG{+w}{ }\PYG{l+m+mi}{5}\PYG{p}{]}\PYG{p}{,}\PYG{+w}{ }\PYG{p}{[}\PYG{l+m+mi}{100}\PYG{p}{,}\PYG{+w}{ }\PYG{l+m+mi}{100}\PYG{p}{]}\PYG{p}{]}
\PYG{n+nv}{minDist}\PYG{+w}{ }\PYG{o}{\PYGZlt{}\PYGZhy{}}\PYG{+w}{ }\PYG{n+nf}{GetMinDist}\PYG{p}{(}\PYG{n+nv}{points}\PYG{p}{)}
\PYG{n+nf}{Print}\PYG{p}{(}\PYG{n+nv}{minDist}\PYG{p}{)}
\PYG{c+c1}{\PYGZsh{} Result: 7.07107  (distance between [5,5] and [10,0])}

\PYG{c+c1}{\PYGZsh{} Check if points are spaced far enough apart}
\PYG{n+nv}{layout}\PYG{+w}{ }\PYG{o}{\PYGZlt{}\PYGZhy{}}\PYG{+w}{ }\PYG{n+nf}{NonOverlapLayout}\PYG{p}{(}\PYG{l+m+mi}{100}\PYG{p}{,}\PYG{+w}{ }\PYG{l+m+mi}{400}\PYG{p}{,}\PYG{+w}{ }\PYG{l+m+mi}{100}\PYG{p}{,}\PYG{+w}{ }\PYG{l+m+mi}{400}\PYG{p}{,}\PYG{+w}{ }\PYG{l+m+mi}{50}\PYG{p}{,}\PYG{+w}{ }\PYG{l+m+mi}{20}\PYG{p}{)}
\PYG{n+nv}{spacing}\PYG{+w}{ }\PYG{o}{\PYGZlt{}\PYGZhy{}}\PYG{+w}{ }\PYG{n+nf}{GetMinDist}\PYG{p}{(}\PYG{n+nv}{layout}\PYG{p}{)}
\PYG{k}{if}\PYG{p}{(}\PYG{n+nv}{spacing}\PYG{+w}{ }\PYG{o}{\PYGZlt{}}\PYG{+w}{ }\PYG{l+m+mi}{50}\PYG{p}{)}
\PYG{p}{\PYGZob{}}
\PYG{+w}{   }\PYG{n+nf}{Print}\PYG{p}{(}\PYG{l+s+s2}{\PYGZdq{}Warning: some points are too close together\PYGZdq{}}\PYG{p}{)}
\PYG{p}{\PYGZcb{}}
\end{sphinxVerbatim}

\sphinxAtStartPar
\sphinxstylestrong{See Also:}

\sphinxAtStartPar
\sphinxcode{\sphinxupquote{NonOverlapLayout()}}, \sphinxcode{\sphinxupquote{Dist()}}

\index{HideObject@\spxentry{HideObject}}\ignorespaces 

\subsection{HideObject()}
\label{\detokenize{reference/graphics:hideobject}}\label{\detokenize{reference/graphics:index-22}}
\sphinxAtStartPar
\sphinxstyleemphasis{Hides a custom graphical object}

\sphinxAtStartPar
\sphinxstylestrong{Description:}

\sphinxAtStartPar
Hides a custom graphical object by hiding its canvas. This function is designed for use with complex custom objects that have a \sphinxcode{\sphinxupquote{.canv}} property containing their graphical representation. The object remains in memory but is not displayed until shown again with \sphinxcode{\sphinxupquote{ShowObject()}}. This is useful for temporarily removing custom stimuli from view without destroying them.

\sphinxAtStartPar
\sphinxstylestrong{Usage:}

\begin{sphinxVerbatim}[commandchars=\\\{\}]
\PYG{k}{define}\PYG{+w}{ }\PYG{n+nf}{HideObject}\PYG{p}{(}\PYG{n+nv}{obj}\PYG{p}{)}
\end{sphinxVerbatim}

\sphinxAtStartPar
\sphinxstylestrong{Example:}

\begin{sphinxVerbatim}[commandchars=\\\{\}]
\PYG{c+c1}{\PYGZsh{} Create a custom object with a canvas}
\PYG{n+nv}{myObject}\PYG{+w}{ }\PYG{o}{\PYGZlt{}\PYGZhy{}}\PYG{+w}{ }\PYG{n+nf}{MakeCustomShape}\PYG{p}{(}\PYG{p}{)}\PYG{+w}{  }\PYG{c+c1}{\PYGZsh{} hypothetical function}

\PYG{c+c1}{\PYGZsh{} Display the object}
\PYG{n+nf}{ShowObject}\PYG{p}{(}\PYG{n+nv}{myObject}\PYG{p}{)}
\PYG{n+nf}{Draw}\PYG{p}{(}\PYG{p}{)}
\PYG{n+nf}{Wait}\PYG{p}{(}\PYG{l+m+mi}{1000}\PYG{p}{)}

\PYG{c+c1}{\PYGZsh{} Hide the object temporarily}
\PYG{n+nf}{HideObject}\PYG{p}{(}\PYG{n+nv}{myObject}\PYG{p}{)}
\PYG{n+nf}{Draw}\PYG{p}{(}\PYG{p}{)}
\PYG{n+nf}{Wait}\PYG{p}{(}\PYG{l+m+mi}{1000}\PYG{p}{)}

\PYG{c+c1}{\PYGZsh{} Show it again}
\PYG{n+nf}{ShowObject}\PYG{p}{(}\PYG{n+nv}{myObject}\PYG{p}{)}
\PYG{n+nf}{Draw}\PYG{p}{(}\PYG{p}{)}
\end{sphinxVerbatim}

\sphinxAtStartPar
\sphinxstylestrong{See Also:}

\sphinxAtStartPar
\sphinxcode{\sphinxupquote{ShowObject()}}, \sphinxcode{\sphinxupquote{Hide()}}, \sphinxcode{\sphinxupquote{Show()}}

\index{LandoltRing@\spxentry{LandoltRing}}\ignorespaces 

\subsection{LandoltRing()}
\label{\detokenize{reference/graphics:landoltring}}\label{\detokenize{reference/graphics:index-23}}
\sphinxAtStartPar
\sphinxstyleemphasis{Creates a Landolt C ring stimulus for visual acuity testing}

\sphinxAtStartPar
\sphinxstylestrong{Description:}

\sphinxAtStartPar
Creates a Landolt C (also known as Landolt ring), a classic visual acuity stimulus used in optometry and vision research. The Landolt C is a ring with a gap at a specified angular position. Subjects identify the location of the gap to assess visual acuity. The function returns a canvas object containing the rendered stimulus.

\sphinxAtStartPar
The Landolt C was introduced by Edmund Landolt in 1888 (Landolt, E. Methode optométrique simple. \sphinxstyleemphasis{Bull Mem Soc Fran Ophtalmol.} 1888;6:213\textendash{}4) and remains a standard stimulus in visual psychophysics and clinical vision testing.

\sphinxAtStartPar
Parameters:
\begin{itemize}
\item {} 
\sphinxAtStartPar
\sphinxcode{\sphinxupquote{outer}}: Outer diameter of the ring in pixels

\item {} 
\sphinxAtStartPar
\sphinxcode{\sphinxupquote{inner}}: Inner diameter (size of the central hole) in pixels

\item {} 
\sphinxAtStartPar
\sphinxcode{\sphinxupquote{angle}}: Angular position of the gap in degrees (0=right, 90=down, 180=left, 270=up)

\item {} 
\sphinxAtStartPar
\sphinxcode{\sphinxupquote{lgap}}: Width of the gap in pixels

\item {} 
\sphinxAtStartPar
\sphinxcode{\sphinxupquote{color}}: Color object for the ring

\item {} 
\sphinxAtStartPar
\sphinxcode{\sphinxupquote{bgcolor}}: Color object for the background and gap

\end{itemize}

\sphinxAtStartPar
\sphinxstylestrong{Usage:}

\begin{sphinxVerbatim}[commandchars=\\\{\}]
\PYG{k}{define}\PYG{+w}{ }\PYG{n+nf}{LandoltRing}\PYG{p}{(}\PYG{n+nv}{outer}\PYG{p}{,}\PYG{+w}{ }\PYG{n+nv}{inner}\PYG{p}{,}\PYG{+w}{ }\PYG{n+nv}{angle}\PYG{p}{,}\PYG{+w}{ }\PYG{n+nv}{lgap}\PYG{p}{,}\PYG{+w}{ }\PYG{n+nv}{color}\PYG{p}{,}\PYG{+w}{ }\PYG{n+nv}{bgcolor}\PYG{p}{)}
\end{sphinxVerbatim}

\sphinxAtStartPar
\sphinxstylestrong{Example:}

\begin{sphinxVerbatim}[commandchars=\\\{\}]
\PYG{n+nv+vg}{gWin}\PYG{+w}{ }\PYG{o}{\PYGZlt{}\PYGZhy{}}\PYG{+w}{ }\PYG{n+nf}{MakeWindow}\PYG{p}{(}\PYG{p}{)}
\PYG{n+nv}{black}\PYG{+w}{ }\PYG{o}{\PYGZlt{}\PYGZhy{}}\PYG{+w}{ }\PYG{n+nf}{MakeColor}\PYG{p}{(}\PYG{l+s+s2}{\PYGZdq{}black\PYGZdq{}}\PYG{p}{)}
\PYG{n+nv}{white}\PYG{+w}{ }\PYG{o}{\PYGZlt{}\PYGZhy{}}\PYG{+w}{ }\PYG{n+nf}{MakeColor}\PYG{p}{(}\PYG{l+s+s2}{\PYGZdq{}white\PYGZdq{}}\PYG{p}{)}

\PYG{c+c1}{\PYGZsh{} Create a Landolt C with gap on the right (0 degrees)}
\PYG{n+nv}{ring1}\PYG{+w}{ }\PYG{o}{\PYGZlt{}\PYGZhy{}}\PYG{+w}{ }\PYG{n+nf}{LandoltRing}\PYG{p}{(}\PYG{l+m+mi}{100}\PYG{p}{,}\PYG{+w}{ }\PYG{l+m+mi}{60}\PYG{p}{,}\PYG{+w}{ }\PYG{l+m+mi}{0}\PYG{p}{,}\PYG{+w}{ }\PYG{l+m+mi}{20}\PYG{p}{,}\PYG{+w}{ }\PYG{n+nv}{black}\PYG{p}{,}\PYG{+w}{ }\PYG{n+nv}{white}\PYG{p}{)}
\PYG{n+nf}{AddObject}\PYG{p}{(}\PYG{n+nv}{ring1}\PYG{p}{,}\PYG{+w}{ }\PYG{n+nv+vg}{gWin}\PYG{p}{)}
\PYG{n+nf}{Move}\PYG{p}{(}\PYG{n+nv}{ring1}\PYG{p}{,}\PYG{+w}{ }\PYG{l+m+mi}{200}\PYG{p}{,}\PYG{+w}{ }\PYG{l+m+mi}{200}\PYG{p}{)}

\PYG{c+c1}{\PYGZsh{} Create a smaller ring with gap at bottom (90 degrees)}
\PYG{n+nv}{ring2}\PYG{+w}{ }\PYG{o}{\PYGZlt{}\PYGZhy{}}\PYG{+w}{ }\PYG{n+nf}{LandoltRing}\PYG{p}{(}\PYG{l+m+mi}{60}\PYG{p}{,}\PYG{+w}{ }\PYG{l+m+mi}{40}\PYG{p}{,}\PYG{+w}{ }\PYG{l+m+mi}{90}\PYG{p}{,}\PYG{+w}{ }\PYG{l+m+mi}{12}\PYG{p}{,}\PYG{+w}{ }\PYG{n+nv}{black}\PYG{p}{,}\PYG{+w}{ }\PYG{n+nv}{white}\PYG{p}{)}
\PYG{n+nf}{AddObject}\PYG{p}{(}\PYG{n+nv}{ring2}\PYG{p}{,}\PYG{+w}{ }\PYG{n+nv+vg}{gWin}\PYG{p}{)}
\PYG{n+nf}{Move}\PYG{p}{(}\PYG{n+nv}{ring2}\PYG{p}{,}\PYG{+w}{ }\PYG{l+m+mi}{400}\PYG{p}{,}\PYG{+w}{ }\PYG{l+m+mi}{200}\PYG{p}{)}

\PYG{c+c1}{\PYGZsh{} Create rings at different orientations for acuity test}
\PYG{n+nv}{angles}\PYG{+w}{ }\PYG{o}{\PYGZlt{}\PYGZhy{}}\PYG{+w}{ }\PYG{p}{[}\PYG{l+m+mi}{0}\PYG{p}{,}\PYG{+w}{ }\PYG{l+m+mi}{45}\PYG{p}{,}\PYG{+w}{ }\PYG{l+m+mi}{90}\PYG{p}{,}\PYG{+w}{ }\PYG{l+m+mi}{135}\PYG{p}{,}\PYG{+w}{ }\PYG{l+m+mi}{180}\PYG{p}{,}\PYG{+w}{ }\PYG{l+m+mi}{225}\PYG{p}{,}\PYG{+w}{ }\PYG{l+m+mi}{270}\PYG{p}{,}\PYG{+w}{ }\PYG{l+m+mi}{315}\PYG{p}{]}
\PYG{k}{loop}\PYG{p}{(}\PYG{n+nv}{i}\PYG{p}{,}\PYG{+w}{ }\PYG{n+nf}{Sequence}\PYG{p}{(}\PYG{l+m+mi}{1}\PYG{p}{,}\PYG{+w}{ }\PYG{n+nf}{Length}\PYG{p}{(}\PYG{n+nv}{angles}\PYG{p}{)}\PYG{p}{,}\PYG{+w}{ }\PYG{l+m+mi}{1}\PYG{p}{)}\PYG{p}{)}
\PYG{p}{\PYGZob{}}
\PYG{+w}{   }\PYG{n+nv}{ring}\PYG{+w}{ }\PYG{o}{\PYGZlt{}\PYGZhy{}}\PYG{+w}{ }\PYG{n+nf}{LandoltRing}\PYG{p}{(}\PYG{l+m+mi}{80}\PYG{p}{,}\PYG{+w}{ }\PYG{l+m+mi}{50}\PYG{p}{,}\PYG{+w}{ }\PYG{n+nf}{Nth}\PYG{p}{(}\PYG{n+nv}{angles}\PYG{p}{,}\PYG{+w}{ }\PYG{n+nv}{i}\PYG{p}{)}\PYG{p}{,}\PYG{+w}{ }\PYG{l+m+mi}{15}\PYG{p}{,}\PYG{+w}{ }\PYG{n+nv}{black}\PYG{p}{,}\PYG{+w}{ }\PYG{n+nv}{white}\PYG{p}{)}
\PYG{+w}{   }\PYG{n+nf}{AddObject}\PYG{p}{(}\PYG{n+nv}{ring}\PYG{p}{,}\PYG{+w}{ }\PYG{n+nv+vg}{gWin}\PYG{p}{)}
\PYG{+w}{   }\PYG{n+nf}{Move}\PYG{p}{(}\PYG{n+nv}{ring}\PYG{p}{,}\PYG{+w}{ }\PYG{l+m+mi}{100}\PYG{+w}{ }\PYG{o}{+}\PYG{+w}{ }\PYG{n+nv}{i}\PYG{o}{*}\PYG{l+m+mi}{80}\PYG{p}{,}\PYG{+w}{ }\PYG{l+m+mi}{300}\PYG{p}{)}
\PYG{p}{\PYGZcb{}}

\PYG{n+nf}{Draw}\PYG{p}{(}\PYG{p}{)}
\PYG{n+nf}{WaitForAnyKeyPress}\PYG{p}{(}\PYG{p}{)}
\end{sphinxVerbatim}

\sphinxAtStartPar
\sphinxstylestrong{See Also:}

\sphinxAtStartPar
\sphinxcode{\sphinxupquote{Circle()}}, \sphinxcode{\sphinxupquote{BlockE()}}, \sphinxcode{\sphinxupquote{MakeCanvas()}}

\index{MakeTable@\spxentry{MakeTable}}\ignorespaces 

\subsection{MakeTable()}
\label{\detokenize{reference/graphics:maketable}}\label{\detokenize{reference/graphics:index-24}}
\sphinxAtStartPar
\sphinxstyleemphasis{Creates a formatted table display with headers and data}

\sphinxAtStartPar
\sphinxstylestrong{Description:}

\sphinxAtStartPar
Creates a visual table object containing data arranged in rows and columns with headers. Returns a custom table object drawn on a canvas that can be added to a window and positioned like other graphical objects. The table includes horizontal rules separating the header from data and bordering the table. Headers can be multi\sphinxhyphen{}row (nested lists). Useful for displaying results, feedback, or structured information to participants.

\sphinxAtStartPar
Parameters:
\begin{itemize}
\item {} 
\sphinxAtStartPar
\sphinxcode{\sphinxupquote{data}}: Nested list of data values (rows × columns)

\item {} 
\sphinxAtStartPar
\sphinxcode{\sphinxupquote{header}}: List of column headers (can be nested for multi\sphinxhyphen{}row headers)

\item {} 
\sphinxAtStartPar
\sphinxcode{\sphinxupquote{width}}: Total width of the table in pixels

\item {} 
\sphinxAtStartPar
\sphinxcode{\sphinxupquote{height}}: Total height of the table in pixels

\item {} 
\sphinxAtStartPar
\sphinxcode{\sphinxupquote{fontsize}}: Base font size for table text

\item {} 
\sphinxAtStartPar
\sphinxcode{\sphinxupquote{fgcol}}: Foreground color object for text and lines

\item {} 
\sphinxAtStartPar
\sphinxcode{\sphinxupquote{bgcolor}}: Background color object for the table

\item {} 
\sphinxAtStartPar
\sphinxcode{\sphinxupquote{layout}}: Layout mode (default: 1)

\end{itemize}

\sphinxAtStartPar
\sphinxstylestrong{Usage:}

\begin{sphinxVerbatim}[commandchars=\\\{\}]
\PYG{k}{define}\PYG{+w}{ }\PYG{n+nf}{MakeTable}\PYG{p}{(}\PYG{n+nv}{data}\PYG{p}{,}\PYG{+w}{ }\PYG{n+nv}{header}\PYG{p}{,}\PYG{+w}{ }\PYG{n+nv}{width}\PYG{p}{,}\PYG{+w}{ }\PYG{n+nv}{height}\PYG{p}{,}\PYG{+w}{ }\PYG{n+nv}{fontsize}\PYG{p}{,}\PYG{+w}{ }\PYG{n+nv}{fgcol}\PYG{p}{,}\PYG{+w}{ }\PYG{n+nv}{bgcol}\PYG{p}{,}\PYG{+w}{ }\PYG{n+nv}{layout}\PYG{o}{:}\PYG{+w}{ }\PYG{l+m+mi}{1}\PYG{p}{)}
\end{sphinxVerbatim}

\sphinxAtStartPar
\sphinxstylestrong{Example:}

\begin{sphinxVerbatim}[commandchars=\\\{\}]
\PYG{n+nv+vg}{gWin}\PYG{+w}{ }\PYG{o}{\PYGZlt{}\PYGZhy{}}\PYG{+w}{ }\PYG{n+nf}{MakeWindow}\PYG{p}{(}\PYG{p}{)}
\PYG{n+nv}{black}\PYG{+w}{ }\PYG{o}{\PYGZlt{}\PYGZhy{}}\PYG{+w}{ }\PYG{n+nf}{MakeColor}\PYG{p}{(}\PYG{l+s+s2}{\PYGZdq{}black\PYGZdq{}}\PYG{p}{)}
\PYG{n+nv}{white}\PYG{+w}{ }\PYG{o}{\PYGZlt{}\PYGZhy{}}\PYG{+w}{ }\PYG{n+nf}{MakeColor}\PYG{p}{(}\PYG{l+s+s2}{\PYGZdq{}white\PYGZdq{}}\PYG{p}{)}

\PYG{c+c1}{\PYGZsh{} Create a simple results table}
\PYG{n+nv}{headers}\PYG{+w}{ }\PYG{o}{\PYGZlt{}\PYGZhy{}}\PYG{+w}{ }\PYG{p}{[}\PYG{l+s+s2}{\PYGZdq{}Trial\PYGZdq{}}\PYG{p}{,}\PYG{+w}{ }\PYG{l+s+s2}{\PYGZdq{}RT (ms)\PYGZdq{}}\PYG{p}{,}\PYG{+w}{ }\PYG{l+s+s2}{\PYGZdq{}Accuracy\PYGZdq{}}\PYG{p}{]}
\PYG{n+nv}{data}\PYG{+w}{ }\PYG{o}{\PYGZlt{}\PYGZhy{}}\PYG{+w}{ }\PYG{p}{[}\PYG{p}{[}\PYG{l+m+mi}{1}\PYG{p}{,}\PYG{+w}{ }\PYG{l+m+mi}{523}\PYG{p}{,}\PYG{+w}{ }\PYG{l+s+s2}{\PYGZdq{}Correct\PYGZdq{}}\PYG{p}{]}\PYG{p}{,}
\PYG{+w}{         }\PYG{p}{[}\PYG{l+m+mi}{2}\PYG{p}{,}\PYG{+w}{ }\PYG{l+m+mi}{456}\PYG{p}{,}\PYG{+w}{ }\PYG{l+s+s2}{\PYGZdq{}Correct\PYGZdq{}}\PYG{p}{]}\PYG{p}{,}
\PYG{+w}{         }\PYG{p}{[}\PYG{l+m+mi}{3}\PYG{p}{,}\PYG{+w}{ }\PYG{l+m+mi}{678}\PYG{p}{,}\PYG{+w}{ }\PYG{l+s+s2}{\PYGZdq{}Error\PYGZdq{}}\PYG{p}{]}\PYG{p}{,}
\PYG{+w}{         }\PYG{p}{[}\PYG{l+m+mi}{4}\PYG{p}{,}\PYG{+w}{ }\PYG{l+m+mi}{501}\PYG{p}{,}\PYG{+w}{ }\PYG{l+s+s2}{\PYGZdq{}Correct\PYGZdq{}}\PYG{p}{]}\PYG{p}{]}

\PYG{n+nv}{table}\PYG{+w}{ }\PYG{o}{\PYGZlt{}\PYGZhy{}}\PYG{+w}{ }\PYG{n+nf}{MakeTable}\PYG{p}{(}\PYG{n+nv}{data}\PYG{p}{,}\PYG{+w}{ }\PYG{n+nv}{headers}\PYG{p}{,}\PYG{+w}{ }\PYG{l+m+mi}{400}\PYG{p}{,}\PYG{+w}{ }\PYG{l+m+mi}{300}\PYG{p}{,}\PYG{+w}{ }\PYG{l+m+mi}{16}\PYG{p}{,}\PYG{+w}{ }\PYG{n+nv}{black}\PYG{p}{,}\PYG{+w}{ }\PYG{n+nv}{white}\PYG{p}{)}
\PYG{n+nf}{AddObject}\PYG{p}{(}\PYG{n+nv}{table.canv}\PYG{p}{,}\PYG{+w}{ }\PYG{n+nv+vg}{gWin}\PYG{p}{)}
\PYG{n+nf}{Move}\PYG{p}{(}\PYG{n+nv}{table.canv}\PYG{p}{,}\PYG{+w}{ }\PYG{l+m+mi}{400}\PYG{p}{,}\PYG{+w}{ }\PYG{l+m+mi}{300}\PYG{p}{)}
\PYG{n+nf}{Draw}\PYG{p}{(}\PYG{p}{)}
\PYG{n+nf}{WaitForAnyKeyPress}\PYG{p}{(}\PYG{p}{)}
\end{sphinxVerbatim}

\sphinxAtStartPar
\sphinxstylestrong{See Also:}

\sphinxAtStartPar
\sphinxcode{\sphinxupquote{MakeCanvas()}}, \sphinxcode{\sphinxupquote{MakeLabel()}}, \sphinxcode{\sphinxupquote{EasyLabel()}}

\index{ShowObject@\spxentry{ShowObject}}\ignorespaces 

\subsection{ShowObject()}
\label{\detokenize{reference/graphics:showobject}}\label{\detokenize{reference/graphics:index-25}}
\sphinxAtStartPar
\sphinxstyleemphasis{Shows a custom graphical object}

\sphinxAtStartPar
\sphinxstylestrong{Description:}

\sphinxAtStartPar
Shows a custom graphical object by displaying its canvas. This function is designed for use with complex custom objects that have a \sphinxcode{\sphinxupquote{.canv}} property containing their graphical representation. Use this to display objects that were previously hidden with \sphinxcode{\sphinxupquote{HideObject()}} or to initially display custom objects after creation. The object must have been created with a canvas property for this function to work properly.

\sphinxAtStartPar
\sphinxstylestrong{Usage:}

\begin{sphinxVerbatim}[commandchars=\\\{\}]
\PYG{k}{define}\PYG{+w}{ }\PYG{n+nf}{ShowObject}\PYG{p}{(}\PYG{n+nv}{obj}\PYG{p}{)}
\end{sphinxVerbatim}

\sphinxAtStartPar
\sphinxstylestrong{Example:}

\begin{sphinxVerbatim}[commandchars=\\\{\}]
\PYG{c+c1}{\PYGZsh{} Create a custom object with a canvas}
\PYG{n+nv}{myObject}\PYG{+w}{ }\PYG{o}{\PYGZlt{}\PYGZhy{}}\PYG{+w}{ }\PYG{n+nf}{MakeCustomShape}\PYG{p}{(}\PYG{p}{)}\PYG{+w}{  }\PYG{c+c1}{\PYGZsh{} hypothetical function}

\PYG{c+c1}{\PYGZsh{} Initially hide the object}
\PYG{n+nf}{HideObject}\PYG{p}{(}\PYG{n+nv}{myObject}\PYG{p}{)}
\PYG{n+nf}{Draw}\PYG{p}{(}\PYG{p}{)}

\PYG{c+c1}{\PYGZsh{} Wait for user input}
\PYG{n+nf}{WaitForAnyKeyPress}\PYG{p}{(}\PYG{p}{)}

\PYG{c+c1}{\PYGZsh{} Now show the object}
\PYG{n+nf}{ShowObject}\PYG{p}{(}\PYG{n+nv}{myObject}\PYG{p}{)}
\PYG{n+nf}{Draw}\PYG{p}{(}\PYG{p}{)}

\PYG{c+c1}{\PYGZsh{} Toggle visibility}
\PYG{k}{if}\PYG{p}{(}\PYG{n+nv}{someCondition}\PYG{p}{)}
\PYG{p}{\PYGZob{}}
\PYG{+w}{   }\PYG{n+nf}{ShowObject}\PYG{p}{(}\PYG{n+nv}{myObject}\PYG{p}{)}
\PYG{p}{\PYGZcb{}}\PYG{+w}{ }\PYG{k}{else}\PYG{+w}{ }\PYG{p}{\PYGZob{}}
\PYG{+w}{   }\PYG{n+nf}{HideObject}\PYG{p}{(}\PYG{n+nv}{myObject}\PYG{p}{)}
\PYG{p}{\PYGZcb{}}
\end{sphinxVerbatim}

\sphinxAtStartPar
\sphinxstylestrong{See Also:}

\sphinxAtStartPar
\sphinxcode{\sphinxupquote{HideObject()}}, \sphinxcode{\sphinxupquote{Show()}}, \sphinxcode{\sphinxupquote{Hide()}}


\subsection{Functions Under Investigation}
\label{\detokenize{reference/graphics:functions-under-investigation}}
\index{ThickLine2@\spxentry{ThickLine2}}\ignorespaces 

\subsection{ThickLine2()}
\label{\detokenize{reference/graphics:thickline2}}\label{\detokenize{reference/graphics:index-26}}
\begin{sphinxadmonition}{warning}{Warning:}
\sphinxAtStartPar
\sphinxstylestrong{Under investigation.} This function’s status is being reviewed.
\end{sphinxadmonition}

\sphinxAtStartPar
\sphinxstylestrong{Usage:}

\begin{sphinxVerbatim}[commandchars=\\\{\}]
\PYG{k}{define}\PYG{+w}{ }\PYG{n+nf}{ThickLine2}\PYG{p}{(}\PYG{n+nv}{x1}\PYG{p}{,}\PYG{n+nv}{y1}\PYG{p}{,}\PYG{n+nv}{x2}\PYG{p}{,}\PYG{n+nv}{y2}\PYG{p}{,}\PYG{n+nv}{size}\PYG{p}{,}\PYG{n+nv}{color}\PYG{p}{)}
\end{sphinxVerbatim}

\sphinxstepscope


\section{HTML Library \sphinxhyphen{} HTML Generation}
\label{\detokenize{reference/html:html-library-html-generation}}\label{\detokenize{reference/html::doc}}
\sphinxAtStartPar
This library contains functions for generating HTML markup and web content.

\begin{sphinxShadowBox}
\sphinxstyletopictitle{Function Index}
\begin{itemize}
\item {} 
\sphinxAtStartPar
\phantomsection\label{\detokenize{reference/html:id1}}{\hyperref[\detokenize{reference/html:b}]{\sphinxcrossref{B()}}}

\item {} 
\sphinxAtStartPar
\phantomsection\label{\detokenize{reference/html:id2}}{\hyperref[\detokenize{reference/html:br}]{\sphinxcrossref{BR()}}}

\item {} 
\sphinxAtStartPar
\phantomsection\label{\detokenize{reference/html:id3}}{\hyperref[\detokenize{reference/html:ct}]{\sphinxcrossref{CT()}}}

\item {} 
\sphinxAtStartPar
\phantomsection\label{\detokenize{reference/html:id4}}{\hyperref[\detokenize{reference/html:entag}]{\sphinxcrossref{Entag()}}}

\item {} 
\sphinxAtStartPar
\phantomsection\label{\detokenize{reference/html:id5}}{\hyperref[\detokenize{reference/html:h}]{\sphinxcrossref{H()}}}

\item {} 
\sphinxAtStartPar
\phantomsection\label{\detokenize{reference/html:id6}}{\hyperref[\detokenize{reference/html:hl}]{\sphinxcrossref{HL()}}}

\item {} 
\sphinxAtStartPar
\phantomsection\label{\detokenize{reference/html:id7}}{\hyperref[\detokenize{reference/html:img}]{\sphinxcrossref{Img()}}}

\item {} 
\sphinxAtStartPar
\phantomsection\label{\detokenize{reference/html:id8}}{\hyperref[\detokenize{reference/html:makedivpage}]{\sphinxcrossref{MakeDivPage()}}}

\item {} 
\sphinxAtStartPar
\phantomsection\label{\detokenize{reference/html:id9}}{\hyperref[\detokenize{reference/html:ot}]{\sphinxcrossref{OT()}}}

\item {} 
\sphinxAtStartPar
\phantomsection\label{\detokenize{reference/html:id10}}{\hyperref[\detokenize{reference/html:p}]{\sphinxcrossref{P()}}}

\item {} 
\sphinxAtStartPar
\phantomsection\label{\detokenize{reference/html:id11}}{\hyperref[\detokenize{reference/html:page}]{\sphinxcrossref{Page()}}}

\item {} 
\sphinxAtStartPar
\phantomsection\label{\detokenize{reference/html:id12}}{\hyperref[\detokenize{reference/html:table}]{\sphinxcrossref{Table()}}}

\end{itemize}
\end{sphinxShadowBox}

\index{B@\spxentry{B}}\ignorespaces 

\subsection{B()}
\label{\detokenize{reference/html:b}}\label{\detokenize{reference/html:index-0}}
\sphinxAtStartPar
\sphinxstylestrong{Description:}

\sphinxAtStartPar
Implements the HTML \sphinxcode{\sphinxupquote{\textless{}b\textgreater{}}} tag. Wraps the provided text in bold tags to display it in bold font weight when rendered in HTML.

\sphinxAtStartPar
\sphinxstylestrong{Usage:}

\begin{sphinxVerbatim}[commandchars=\\\{\}]
\PYG{k}{define}\PYG{+w}{ }\PYG{n+nf}{B}\PYG{p}{(}\PYG{n+nv}{text}\PYG{p}{)}
\end{sphinxVerbatim}

\sphinxAtStartPar
\sphinxstylestrong{Example:}

\begin{sphinxVerbatim}[commandchars=\\\{\}]
\PYG{n+nv}{report}\PYG{+w}{ }\PYG{o}{\PYGZlt{}\PYGZhy{}}\PYG{+w}{ }\PYG{n+nf}{B}\PYG{p}{(}\PYG{l+s+s2}{\PYGZdq{}Important Result\PYGZdq{}}\PYG{p}{)}
\PYG{n+nf}{FilePrint}\PYG{p}{(}\PYG{n+nv}{file}\PYG{p}{,}\PYG{+w}{ }\PYG{n+nv}{report}\PYG{p}{)}
\PYG{c+c1}{\PYGZsh{}\PYGZsh{} Produces: \PYGZlt{}b\PYGZgt{}Important Result\PYGZlt{}/b\PYGZgt{}}
\end{sphinxVerbatim}

\sphinxAtStartPar
\sphinxstylestrong{See Also:}

\sphinxAtStartPar
\sphinxcode{\sphinxupquote{P()}}, \sphinxcode{\sphinxupquote{H()}}, \sphinxcode{\sphinxupquote{Entag()}}

\index{BR@\spxentry{BR}}\ignorespaces 

\subsection{BR()}
\label{\detokenize{reference/html:br}}\label{\detokenize{reference/html:index-1}}
\sphinxAtStartPar
\sphinxstylestrong{Description:}

\sphinxAtStartPar
Implements the HTML \sphinxcode{\sphinxupquote{\textless{}br\textgreater{}}} tag. Returns a line break tag to create a new line in HTML output without starting a new paragraph.

\sphinxAtStartPar
\sphinxstylestrong{Usage:}

\begin{sphinxVerbatim}[commandchars=\\\{\}]
\PYG{k}{define}\PYG{+w}{ }\PYG{n+nf}{BR}\PYG{p}{(}\PYG{p}{)}
\end{sphinxVerbatim}

\sphinxAtStartPar
\sphinxstylestrong{Example:}

\begin{sphinxVerbatim}[commandchars=\\\{\}]
\PYG{n+nv}{text}\PYG{+w}{ }\PYG{o}{\PYGZlt{}\PYGZhy{}}\PYG{+w}{ }\PYG{l+s+s2}{\PYGZdq{}Line 1\PYGZdq{}}\PYG{+w}{ }\PYG{o}{+}\PYG{+w}{ }\PYG{n+nf}{BR}\PYG{p}{(}\PYG{p}{)}\PYG{+w}{ }\PYG{o}{+}\PYG{+w}{ }\PYG{l+s+s2}{\PYGZdq{}Line 2\PYGZdq{}}
\PYG{n+nf}{FilePrint}\PYG{p}{(}\PYG{n+nv}{file}\PYG{p}{,}\PYG{+w}{ }\PYG{n+nv}{text}\PYG{p}{)}
\PYG{c+c1}{\PYGZsh{}\PYGZsh{} Produces: Line 1\PYGZlt{}br\PYGZgt{}Line 2}
\end{sphinxVerbatim}

\sphinxAtStartPar
\sphinxstylestrong{See Also:}

\sphinxAtStartPar
\sphinxcode{\sphinxupquote{P()}}, \sphinxcode{\sphinxupquote{HL()}}

\index{CT@\spxentry{CT}}\ignorespaces 

\subsection{CT()}
\label{\detokenize{reference/html:ct}}\label{\detokenize{reference/html:index-2}}
\sphinxAtStartPar
\sphinxstylestrong{Description:}

\sphinxAtStartPar
Closes an HTML tag by generating the closing tag syntax. This is a helper function used by other HTML functions to create properly formatted closing tags. Takes a tag name and returns \sphinxcode{\sphinxupquote{\textless{}/tagname\textgreater{}}}.

\sphinxAtStartPar
\sphinxstylestrong{Usage:}

\begin{sphinxVerbatim}[commandchars=\\\{\}]
\PYG{k}{define}\PYG{+w}{ }\PYG{n+nf}{CT}\PYG{p}{(}\PYG{n+nv}{tag}\PYG{p}{)}
\end{sphinxVerbatim}

\sphinxAtStartPar
\sphinxstylestrong{Example:}

\begin{sphinxVerbatim}[commandchars=\\\{\}]
\PYG{n+nv}{closing}\PYG{+w}{ }\PYG{o}{\PYGZlt{}\PYGZhy{}}\PYG{+w}{ }\PYG{n+nf}{CT}\PYG{p}{(}\PYG{l+s+s2}{\PYGZdq{}div\PYGZdq{}}\PYG{p}{)}
\PYG{c+c1}{\PYGZsh{}\PYGZsh{} Produces: \PYGZlt{}/div\PYGZgt{}}
\end{sphinxVerbatim}

\sphinxAtStartPar
\sphinxstylestrong{See Also:}

\sphinxAtStartPar
\sphinxcode{\sphinxupquote{OT()}}, \sphinxcode{\sphinxupquote{Entag()}}

\index{Entag@\spxentry{Entag}}\ignorespaces 

\subsection{Entag()}
\label{\detokenize{reference/html:entag}}\label{\detokenize{reference/html:index-3}}
\sphinxAtStartPar
\sphinxstylestrong{Description:}

\sphinxAtStartPar
Generic function to wrap body content in any HTML tag. Takes a tag name and body text, then returns the body wrapped in opening and closing tags. Useful for creating custom HTML elements not covered by specific functions.

\sphinxAtStartPar
\sphinxstylestrong{Usage:}

\begin{sphinxVerbatim}[commandchars=\\\{\}]
\PYG{k}{define}\PYG{+w}{ }\PYG{n+nf}{Entag}\PYG{p}{(}\PYG{n+nv}{tag}\PYG{p}{,}\PYG{+w}{ }\PYG{n+nv}{body}\PYG{p}{)}
\end{sphinxVerbatim}

\sphinxAtStartPar
\sphinxstylestrong{Example:}

\begin{sphinxVerbatim}[commandchars=\\\{\}]
\PYG{n+nv}{emphasized}\PYG{+w}{ }\PYG{o}{\PYGZlt{}\PYGZhy{}}\PYG{+w}{ }\PYG{n+nf}{Entag}\PYG{p}{(}\PYG{l+s+s2}{\PYGZdq{}em\PYGZdq{}}\PYG{p}{,}\PYG{+w}{ }\PYG{l+s+s2}{\PYGZdq{}This is important\PYGZdq{}}\PYG{p}{)}
\PYG{c+c1}{\PYGZsh{}\PYGZsh{} Produces: \PYGZlt{}em\PYGZgt{}This is important\PYGZlt{}/em\PYGZgt{}}

\PYG{n+nv}{div}\PYG{+w}{ }\PYG{o}{\PYGZlt{}\PYGZhy{}}\PYG{+w}{ }\PYG{n+nf}{Entag}\PYG{p}{(}\PYG{l+s+s2}{\PYGZdq{}div\PYGZdq{}}\PYG{p}{,}\PYG{+w}{ }\PYG{l+s+s2}{\PYGZdq{}Content in a div\PYGZdq{}}\PYG{p}{)}
\PYG{c+c1}{\PYGZsh{}\PYGZsh{} Produces: \PYGZlt{}div\PYGZgt{}Content in a div\PYGZlt{}/div\PYGZgt{}}
\end{sphinxVerbatim}

\sphinxAtStartPar
\sphinxstylestrong{See Also:}

\sphinxAtStartPar
\sphinxcode{\sphinxupquote{OT()}}, \sphinxcode{\sphinxupquote{CT()}}, \sphinxcode{\sphinxupquote{P()}}, \sphinxcode{\sphinxupquote{B()}}

\index{H@\spxentry{H}}\ignorespaces 

\subsection{H()}
\label{\detokenize{reference/html:h}}\label{\detokenize{reference/html:index-4}}
\sphinxAtStartPar
\sphinxstylestrong{Description:}

\sphinxAtStartPar
Implements HTML header tags \sphinxcode{\sphinxupquote{\textless{}h1\textgreater{}}} through \sphinxcode{\sphinxupquote{\textless{}h6\textgreater{}}}. Takes text and a level (1\sphinxhyphen{}6) and wraps the text in the appropriate header tag. Level 1 is the largest heading, level 6 is the smallest.

\sphinxAtStartPar
\sphinxstylestrong{Usage:}

\begin{sphinxVerbatim}[commandchars=\\\{\}]
\PYG{k}{define}\PYG{+w}{ }\PYG{n+nf}{H}\PYG{p}{(}\PYG{n+nv}{text}\PYG{p}{,}\PYG{+w}{ }\PYG{n+nv}{level}\PYG{p}{)}
\end{sphinxVerbatim}

\sphinxAtStartPar
\sphinxstylestrong{Example:}

\begin{sphinxVerbatim}[commandchars=\\\{\}]
\PYG{n+nv}{title}\PYG{+w}{ }\PYG{o}{\PYGZlt{}\PYGZhy{}}\PYG{+w}{ }\PYG{n+nf}{H}\PYG{p}{(}\PYG{l+s+s2}{\PYGZdq{}Test Results\PYGZdq{}}\PYG{p}{,}\PYG{+w}{ }\PYG{l+m+mi}{1}\PYG{p}{)}
\PYG{c+c1}{\PYGZsh{}\PYGZsh{} Produces: \PYGZlt{}h1\PYGZgt{}Test Results\PYGZlt{}/h1\PYGZgt{}}

\PYG{n+nv}{subhead}\PYG{+w}{ }\PYG{o}{\PYGZlt{}\PYGZhy{}}\PYG{+w}{ }\PYG{n+nf}{H}\PYG{p}{(}\PYG{l+s+s2}{\PYGZdq{}Section A\PYGZdq{}}\PYG{p}{,}\PYG{+w}{ }\PYG{l+m+mi}{2}\PYG{p}{)}
\PYG{c+c1}{\PYGZsh{}\PYGZsh{} Produces: \PYGZlt{}h2\PYGZgt{}Section A\PYGZlt{}/h2\PYGZgt{}}
\end{sphinxVerbatim}

\sphinxAtStartPar
\sphinxstylestrong{See Also:}

\sphinxAtStartPar
\sphinxcode{\sphinxupquote{P()}}, \sphinxcode{\sphinxupquote{B()}}

\index{HL@\spxentry{HL}}\ignorespaces 

\subsection{HL()}
\label{\detokenize{reference/html:hl}}\label{\detokenize{reference/html:index-5}}
\sphinxAtStartPar
\sphinxstylestrong{Description:}

\sphinxAtStartPar
Creates a horizontal line element in HTML. Returns \sphinxcode{\sphinxupquote{\textless{}hl\textgreater{}}} tag. Note: This appears to be a non\sphinxhyphen{}standard tag; standard HTML uses \sphinxcode{\sphinxupquote{\textless{}hr\textgreater{}}} for horizontal rules.

\sphinxAtStartPar
\sphinxstylestrong{Usage:}

\begin{sphinxVerbatim}[commandchars=\\\{\}]
\PYG{k}{define}\PYG{+w}{ }\PYG{n+nf}{HL}\PYG{p}{(}\PYG{p}{)}
\end{sphinxVerbatim}

\sphinxAtStartPar
\sphinxstylestrong{Example:}

\begin{sphinxVerbatim}[commandchars=\\\{\}]
\PYG{n+nv}{separator}\PYG{+w}{ }\PYG{o}{\PYGZlt{}\PYGZhy{}}\PYG{+w}{ }\PYG{n+nf}{HL}\PYG{p}{(}\PYG{p}{)}
\PYG{n+nf}{FilePrint}\PYG{p}{(}\PYG{n+nv}{file}\PYG{p}{,}\PYG{+w}{ }\PYG{n+nv}{separator}\PYG{p}{)}
\end{sphinxVerbatim}

\sphinxAtStartPar
\sphinxstylestrong{See Also:}

\sphinxAtStartPar
\sphinxcode{\sphinxupquote{BR()}}, \sphinxcode{\sphinxupquote{P()}}

\index{Img@\spxentry{Img}}\ignorespaces 

\subsection{Img()}
\label{\detokenize{reference/html:img}}\label{\detokenize{reference/html:index-6}}
\sphinxAtStartPar
\sphinxstylestrong{Description:}

\sphinxAtStartPar
Implements the HTML \sphinxcode{\sphinxupquote{\textless{}img\textgreater{}}} tag. Creates an image element with specified filename and width. The filename should be a path to the image file, and width is specified in pixels.

\sphinxAtStartPar
\sphinxstylestrong{Usage:}

\begin{sphinxVerbatim}[commandchars=\\\{\}]
\PYG{k}{define}\PYG{+w}{ }\PYG{n+nf}{Img}\PYG{p}{(}\PYG{n+nv}{filename}\PYG{p}{,}\PYG{+w}{ }\PYG{n+nv}{width}\PYG{p}{)}
\end{sphinxVerbatim}

\sphinxAtStartPar
\sphinxstylestrong{Example:}

\begin{sphinxVerbatim}[commandchars=\\\{\}]
\PYG{n+nv}{image}\PYG{+w}{ }\PYG{o}{\PYGZlt{}\PYGZhy{}}\PYG{+w}{ }\PYG{n+nf}{Img}\PYG{p}{(}\PYG{l+s+s2}{\PYGZdq{}results\PYGZus{}chart.png\PYGZdq{}}\PYG{p}{,}\PYG{+w}{ }\PYG{l+m+mi}{600}\PYG{p}{)}
\PYG{c+c1}{\PYGZsh{}\PYGZsh{} Produces: \PYGZlt{}img src=\PYGZsq{}results\PYGZus{}chart.png\PYGZsq{} width=600/\PYGZgt{}}
\end{sphinxVerbatim}

\sphinxAtStartPar
\sphinxstylestrong{See Also:}

\sphinxAtStartPar
\sphinxcode{\sphinxupquote{Page()}}, \sphinxcode{\sphinxupquote{Table()}}

\index{MakeDivPage@\spxentry{MakeDivPage}}\ignorespaces 

\subsection{MakeDivPage()}
\label{\detokenize{reference/html:makedivpage}}\label{\detokenize{reference/html:index-7}}
\sphinxAtStartPar
\sphinxstylestrong{Description:}

\sphinxAtStartPar
Creates a page\sphinxhyphen{}formatted div container with automatic page numbering. Uses CSS classes ‘page’ and ‘subpage’ for styling (defined in \sphinxcode{\sphinxupquote{Page()}} stylesheet). Automatically increments a global page counter (gPage) for multi\sphinxhyphen{}page reports. Useful for creating printable reports with consistent page formatting.

\sphinxAtStartPar
\sphinxstylestrong{Usage:}

\begin{sphinxVerbatim}[commandchars=\\\{\}]
\PYG{k}{define}\PYG{+w}{ }\PYG{n+nf}{MakeDivPage}\PYG{p}{(}\PYG{n+nv}{text}\PYG{p}{)}
\end{sphinxVerbatim}

\sphinxAtStartPar
\sphinxstylestrong{Example:}

\begin{sphinxVerbatim}[commandchars=\\\{\}]
\PYG{n+nv}{page1}\PYG{+w}{ }\PYG{o}{\PYGZlt{}\PYGZhy{}}\PYG{+w}{ }\PYG{n+nf}{MakeDivPage}\PYG{p}{(}\PYG{n+nf}{H}\PYG{p}{(}\PYG{l+s+s2}{\PYGZdq{}Report\PYGZdq{}}\PYG{p}{,}\PYG{+w}{ }\PYG{l+m+mi}{1}\PYG{p}{)}\PYG{+w}{ }\PYG{o}{+}\PYG{+w}{ }\PYG{n+nf}{P}\PYG{p}{(}\PYG{l+s+s2}{\PYGZdq{}Page content here\PYGZdq{}}\PYG{p}{)}\PYG{p}{)}
\PYG{n+nv}{page2}\PYG{+w}{ }\PYG{o}{\PYGZlt{}\PYGZhy{}}\PYG{+w}{ }\PYG{n+nf}{MakeDivPage}\PYG{p}{(}\PYG{n+nf}{H}\PYG{p}{(}\PYG{l+s+s2}{\PYGZdq{}Continued\PYGZdq{}}\PYG{p}{,}\PYG{+w}{ }\PYG{l+m+mi}{2}\PYG{p}{)}\PYG{+w}{ }\PYG{o}{+}\PYG{+w}{ }\PYG{n+nf}{P}\PYG{p}{(}\PYG{l+s+s2}{\PYGZdq{}More content\PYGZdq{}}\PYG{p}{)}\PYG{p}{)}
\PYG{n+nv}{report}\PYG{+w}{ }\PYG{o}{\PYGZlt{}\PYGZhy{}}\PYG{+w}{ }\PYG{n+nf}{Page}\PYG{p}{(}\PYG{n+nv}{page1}\PYG{+w}{ }\PYG{o}{+}\PYG{+w}{ }\PYG{n+nv}{page2}\PYG{p}{)}
\PYG{n+nf}{FilePrint}\PYG{p}{(}\PYG{n+nv}{file}\PYG{p}{,}\PYG{+w}{ }\PYG{n+nv}{report}\PYG{p}{)}
\end{sphinxVerbatim}

\sphinxAtStartPar
\sphinxstylestrong{See Also:}

\sphinxAtStartPar
\sphinxcode{\sphinxupquote{Page()}}, \sphinxcode{\sphinxupquote{H()}}, \sphinxcode{\sphinxupquote{P()}}

\index{OT@\spxentry{OT}}\ignorespaces 

\subsection{OT()}
\label{\detokenize{reference/html:ot}}\label{\detokenize{reference/html:index-8}}
\sphinxAtStartPar
\sphinxstylestrong{Description:}

\sphinxAtStartPar
Opens an HTML tag by generating the opening tag syntax. This is a helper function used by other HTML functions to create properly formatted opening tags. Takes a tag name and returns \sphinxcode{\sphinxupquote{\textless{}tagname\textgreater{}}}.

\sphinxAtStartPar
\sphinxstylestrong{Usage:}

\begin{sphinxVerbatim}[commandchars=\\\{\}]
\PYG{k}{define}\PYG{+w}{ }\PYG{n+nf}{OT}\PYG{p}{(}\PYG{n+nv}{tag}\PYG{p}{)}
\end{sphinxVerbatim}

\sphinxAtStartPar
\sphinxstylestrong{Example:}

\begin{sphinxVerbatim}[commandchars=\\\{\}]
\PYG{n+nv}{opening}\PYG{+w}{ }\PYG{o}{\PYGZlt{}\PYGZhy{}}\PYG{+w}{ }\PYG{n+nf}{OT}\PYG{p}{(}\PYG{l+s+s2}{\PYGZdq{}div\PYGZdq{}}\PYG{p}{)}
\PYG{c+c1}{\PYGZsh{}\PYGZsh{} Produces: \PYGZlt{}div\PYGZgt{}}
\end{sphinxVerbatim}

\sphinxAtStartPar
\sphinxstylestrong{See Also:}

\sphinxAtStartPar
\sphinxcode{\sphinxupquote{CT()}}, \sphinxcode{\sphinxupquote{Entag()}}

\index{P@\spxentry{P}}\ignorespaces 

\subsection{P()}
\label{\detokenize{reference/html:p}}\label{\detokenize{reference/html:index-9}}
\sphinxAtStartPar
\sphinxstylestrong{Description:}

\sphinxAtStartPar
Implements the HTML \sphinxcode{\sphinxupquote{\textless{}p\textgreater{}}} tag. Wraps the provided text in paragraph tags to create a standard HTML paragraph element.

\sphinxAtStartPar
\sphinxstylestrong{Usage:}

\begin{sphinxVerbatim}[commandchars=\\\{\}]
\PYG{k}{define}\PYG{+w}{ }\PYG{n+nf}{P}\PYG{p}{(}\PYG{n+nv}{text}\PYG{p}{)}
\end{sphinxVerbatim}

\sphinxAtStartPar
\sphinxstylestrong{Example:}

\begin{sphinxVerbatim}[commandchars=\\\{\}]
\PYG{n+nv}{paragraph}\PYG{+w}{ }\PYG{o}{\PYGZlt{}\PYGZhy{}}\PYG{+w}{ }\PYG{n+nf}{P}\PYG{p}{(}\PYG{l+s+s2}{\PYGZdq{}This is the first paragraph of the report.\PYGZdq{}}\PYG{p}{)}
\PYG{n+nf}{FilePrint}\PYG{p}{(}\PYG{n+nv}{file}\PYG{p}{,}\PYG{+w}{ }\PYG{n+nv}{paragraph}\PYG{p}{)}
\PYG{c+c1}{\PYGZsh{}\PYGZsh{} Produces: \PYGZlt{}p\PYGZgt{}This is the first paragraph of the report.\PYGZlt{}/p\PYGZgt{}}
\end{sphinxVerbatim}

\sphinxAtStartPar
\sphinxstylestrong{See Also:}

\sphinxAtStartPar
\sphinxcode{\sphinxupquote{H()}}, \sphinxcode{\sphinxupquote{B()}}, \sphinxcode{\sphinxupquote{BR()}}

\index{Page@\spxentry{Page}}\ignorespaces 

\subsection{Page()}
\label{\detokenize{reference/html:page}}\label{\detokenize{reference/html:index-10}}
\sphinxAtStartPar
\sphinxstylestrong{Description:}

\sphinxAtStartPar
Creates a complete HTML document with CSS styling suitable for printable reports. Wraps content in full HTML structure including head, style, and body tags. Provides default CSS for letter\sphinxhyphen{}size pages with print\sphinxhyphen{}friendly styling, or accepts custom CSS. The default stylesheet includes responsive table styling and page formatting optimized for 8.5x11 inch paper.

\sphinxAtStartPar
\sphinxstylestrong{Usage:}

\begin{sphinxVerbatim}[commandchars=\\\{\}]
\PYG{k}{define}\PYG{+w}{ }\PYG{n+nf}{Page}\PYG{p}{(}\PYG{n+nv}{text}\PYG{p}{,}\PYG{+w}{ }\PYG{n+nv}{style}\PYG{o}{:}\PYG{l+m+mi}{0}\PYG{p}{)}
\end{sphinxVerbatim}

\sphinxAtStartPar
\sphinxstylestrong{Example:}

\begin{sphinxVerbatim}[commandchars=\\\{\}]
\PYG{n+nv}{content}\PYG{+w}{ }\PYG{o}{\PYGZlt{}\PYGZhy{}}\PYG{+w}{ }\PYG{n+nf}{H}\PYG{p}{(}\PYG{l+s+s2}{\PYGZdq{}Test Report\PYGZdq{}}\PYG{p}{,}\PYG{+w}{ }\PYG{l+m+mi}{1}\PYG{p}{)}\PYG{+w}{ }\PYG{o}{+}\PYG{+w}{ }\PYG{n+nf}{P}\PYG{p}{(}\PYG{l+s+s2}{\PYGZdq{}Results below:\PYGZdq{}}\PYG{p}{)}\PYG{+w}{ }\PYG{o}{+}\PYG{+w}{ }\PYG{n+nf}{Table}\PYG{p}{(}\PYG{n+nv}{data}\PYG{p}{,}\PYG{+w}{ }\PYG{p}{[}\PYG{l+s+s2}{\PYGZdq{}Name\PYGZdq{}}\PYG{p}{,}\PYG{+w}{ }\PYG{l+s+s2}{\PYGZdq{}Score\PYGZdq{}}\PYG{p}{]}\PYG{p}{)}
\PYG{n+nv}{html}\PYG{+w}{ }\PYG{o}{\PYGZlt{}\PYGZhy{}}\PYG{+w}{ }\PYG{n+nf}{Page}\PYG{p}{(}\PYG{n+nv}{content}\PYG{p}{)}
\PYG{n+nf}{FilePrint}\PYG{p}{(}\PYG{n+nv}{file}\PYG{p}{,}\PYG{+w}{ }\PYG{n+nv}{html}\PYG{p}{)}

\PYG{c+c1}{\PYGZsh{}\PYGZsh{} With custom CSS:}
\PYG{n+nv}{customCSS}\PYG{+w}{ }\PYG{o}{\PYGZlt{}\PYGZhy{}}\PYG{+w}{ }\PYG{l+s+s2}{\PYGZdq{}body \PYGZob{} font\PYGZhy{}family: Arial; \PYGZcb{}\PYGZdq{}}
\PYG{n+nv}{html}\PYG{+w}{ }\PYG{o}{\PYGZlt{}\PYGZhy{}}\PYG{+w}{ }\PYG{n+nf}{Page}\PYG{p}{(}\PYG{n+nv}{content}\PYG{p}{,}\PYG{+w}{ }\PYG{n+nv}{customCSS}\PYG{p}{)}
\end{sphinxVerbatim}

\sphinxAtStartPar
\sphinxstylestrong{See Also:}

\sphinxAtStartPar
\sphinxcode{\sphinxupquote{MakeDivPage()}}, \sphinxcode{\sphinxupquote{Table()}}, \sphinxcode{\sphinxupquote{H()}}, \sphinxcode{\sphinxupquote{P()}}

\index{Table@\spxentry{Table}}\ignorespaces 

\subsection{Table()}
\label{\detokenize{reference/html:table}}\label{\detokenize{reference/html:index-11}}
\sphinxAtStartPar
\sphinxstylestrong{Description:}

\sphinxAtStartPar
Implements HTML \sphinxcode{\sphinxupquote{\textless{}table\textgreater{}}} markup. Converts a nested list (list of rows, where each row is a list of cells) into an HTML table. Optionally accepts a header list to create table column headers using \sphinxcode{\sphinxupquote{\textless{}thead\textgreater{}}} and \sphinxcode{\sphinxupquote{\textless{}th\textgreater{}}} tags. Data rows are automatically wrapped in \sphinxcode{\sphinxupquote{\textless{}tr\textgreater{}}} and \sphinxcode{\sphinxupquote{\textless{}td\textgreater{}}} tags. Works with the CSS styling provided by \sphinxcode{\sphinxupquote{Page()}} for formatted, printable tables.

\sphinxAtStartPar
\sphinxstylestrong{Usage:}

\begin{sphinxVerbatim}[commandchars=\\\{\}]
\PYG{k}{define}\PYG{+w}{ }\PYG{n+nf}{Table}\PYG{p}{(}\PYG{n+nv}{tab}\PYG{p}{,}\PYG{+w}{ }\PYG{n+nv}{header}\PYG{o}{:}\PYG{l+s+s2}{\PYGZdq{}\PYGZdq{}}\PYG{p}{)}
\end{sphinxVerbatim}

\sphinxAtStartPar
\sphinxstylestrong{Example:}

\begin{sphinxVerbatim}[commandchars=\\\{\}]
\PYG{c+c1}{\PYGZsh{}\PYGZsh{} Simple table without headers:}
\PYG{n+nv}{data}\PYG{+w}{ }\PYG{o}{\PYGZlt{}\PYGZhy{}}\PYG{+w}{ }\PYG{p}{[}\PYG{p}{[}\PYG{l+s+s2}{\PYGZdq{}John\PYGZdq{}}\PYG{p}{,}\PYG{+w}{ }\PYG{l+m+mi}{85}\PYG{p}{]}\PYG{p}{,}\PYG{+w}{ }\PYG{p}{[}\PYG{l+s+s2}{\PYGZdq{}Mary\PYGZdq{}}\PYG{p}{,}\PYG{+w}{ }\PYG{l+m+mi}{92}\PYG{p}{]}\PYG{p}{,}\PYG{+w}{ }\PYG{p}{[}\PYG{l+s+s2}{\PYGZdq{}Bob\PYGZdq{}}\PYG{p}{,}\PYG{+w}{ }\PYG{l+m+mi}{78}\PYG{p}{]}\PYG{p}{]}
\PYG{n+nv}{table}\PYG{+w}{ }\PYG{o}{\PYGZlt{}\PYGZhy{}}\PYG{+w}{ }\PYG{n+nf}{Table}\PYG{p}{(}\PYG{n+nv}{data}\PYG{p}{)}

\PYG{c+c1}{\PYGZsh{}\PYGZsh{} Table with headers:}
\PYG{n+nv}{data}\PYG{+w}{ }\PYG{o}{\PYGZlt{}\PYGZhy{}}\PYG{+w}{ }\PYG{p}{[}\PYG{p}{[}\PYG{l+s+s2}{\PYGZdq{}John\PYGZdq{}}\PYG{p}{,}\PYG{+w}{ }\PYG{l+m+mi}{85}\PYG{p}{]}\PYG{p}{,}\PYG{+w}{ }\PYG{p}{[}\PYG{l+s+s2}{\PYGZdq{}Mary\PYGZdq{}}\PYG{p}{,}\PYG{+w}{ }\PYG{l+m+mi}{92}\PYG{p}{]}\PYG{p}{,}\PYG{+w}{ }\PYG{p}{[}\PYG{l+s+s2}{\PYGZdq{}Bob\PYGZdq{}}\PYG{p}{,}\PYG{+w}{ }\PYG{l+m+mi}{78}\PYG{p}{]}\PYG{p}{]}
\PYG{n+nv}{headers}\PYG{+w}{ }\PYG{o}{\PYGZlt{}\PYGZhy{}}\PYG{+w}{ }\PYG{p}{[}\PYG{l+s+s2}{\PYGZdq{}Name\PYGZdq{}}\PYG{p}{,}\PYG{+w}{ }\PYG{l+s+s2}{\PYGZdq{}Score\PYGZdq{}}\PYG{p}{]}
\PYG{n+nv}{table}\PYG{+w}{ }\PYG{o}{\PYGZlt{}\PYGZhy{}}\PYG{+w}{ }\PYG{n+nf}{Table}\PYG{p}{(}\PYG{n+nv}{data}\PYG{p}{,}\PYG{+w}{ }\PYG{n+nv}{headers}\PYG{p}{)}

\PYG{c+c1}{\PYGZsh{}\PYGZsh{} In a full report:}
\PYG{n+nv}{report}\PYG{+w}{ }\PYG{o}{\PYGZlt{}\PYGZhy{}}\PYG{+w}{ }\PYG{n+nf}{Page}\PYG{p}{(}\PYG{n+nf}{H}\PYG{p}{(}\PYG{l+s+s2}{\PYGZdq{}Results\PYGZdq{}}\PYG{p}{,}\PYG{+w}{ }\PYG{l+m+mi}{1}\PYG{p}{)}\PYG{+w}{ }\PYG{o}{+}\PYG{+w}{ }\PYG{n+nf}{Table}\PYG{p}{(}\PYG{n+nv}{data}\PYG{p}{,}\PYG{+w}{ }\PYG{n+nv}{headers}\PYG{p}{)}\PYG{p}{)}
\PYG{n+nf}{FilePrint}\PYG{p}{(}\PYG{n+nv}{file}\PYG{p}{,}\PYG{+w}{ }\PYG{n+nv}{report}\PYG{p}{)}
\end{sphinxVerbatim}

\sphinxAtStartPar
\sphinxstylestrong{See Also:}

\sphinxAtStartPar
\sphinxcode{\sphinxupquote{Page()}}, \sphinxcode{\sphinxupquote{MakeDivPage()}}, \sphinxcode{\sphinxupquote{Entag()}}

\sphinxstepscope


\section{Math Library \sphinxhyphen{} Extended Mathematical}
\label{\detokenize{reference/math:math-library-extended-mathematical}}\label{\detokenize{reference/math::doc}}
\sphinxAtStartPar
This library contains extended mathematical functions beyond the core PEBLMath namespace.

\begin{sphinxShadowBox}
\sphinxstyletopictitle{Function Index}
\begin{itemize}
\item {} 
\sphinxAtStartPar
\phantomsection\label{\detokenize{reference/math:id1}}{\hyperref[\detokenize{reference/math:bound}]{\sphinxcrossref{Bound()}}}

\item {} 
\sphinxAtStartPar
\phantomsection\label{\detokenize{reference/math:id2}}{\hyperref[\detokenize{reference/math:cumnorminv}]{\sphinxcrossref{CumNormInv()}}}

\item {} 
\sphinxAtStartPar
\phantomsection\label{\detokenize{reference/math:id3}}{\hyperref[\detokenize{reference/math:cumsum}]{\sphinxcrossref{CumSum()}}}

\item {} 
\sphinxAtStartPar
\phantomsection\label{\detokenize{reference/math:id4}}{\hyperref[\detokenize{reference/math:dist}]{\sphinxcrossref{Dist()}}}

\item {} 
\sphinxAtStartPar
\phantomsection\label{\detokenize{reference/math:id5}}{\hyperref[\detokenize{reference/math:filter}]{\sphinxcrossref{Filter()}}}

\item {} 
\sphinxAtStartPar
\phantomsection\label{\detokenize{reference/math:id6}}{\hyperref[\detokenize{reference/math:match}]{\sphinxcrossref{Match()}}}

\item {} 
\sphinxAtStartPar
\phantomsection\label{\detokenize{reference/math:id7}}{\hyperref[\detokenize{reference/math:normaldensity}]{\sphinxcrossref{NormalDensity()}}}

\item {} 
\sphinxAtStartPar
\phantomsection\label{\detokenize{reference/math:id8}}{\hyperref[\detokenize{reference/math:order}]{\sphinxcrossref{Order()}}}

\item {} 
\sphinxAtStartPar
\phantomsection\label{\detokenize{reference/math:id9}}{\hyperref[\detokenize{reference/math:rank}]{\sphinxcrossref{Rank()}}}

\item {} 
\sphinxAtStartPar
\phantomsection\label{\detokenize{reference/math:id10}}{\hyperref[\detokenize{reference/math:sdtbeta}]{\sphinxcrossref{SDTBeta()}}}

\item {} 
\sphinxAtStartPar
\phantomsection\label{\detokenize{reference/math:id11}}{\hyperref[\detokenize{reference/math:sdtdprime}]{\sphinxcrossref{SDTDPrime()}}}

\item {} 
\sphinxAtStartPar
\phantomsection\label{\detokenize{reference/math:id12}}{\hyperref[\detokenize{reference/math:sum}]{\sphinxcrossref{Sum()}}}

\item {} 
\sphinxAtStartPar
\phantomsection\label{\detokenize{reference/math:id13}}{\hyperref[\detokenize{reference/math:summarystats}]{\sphinxcrossref{SummaryStats()}}}

\item {} 
\sphinxAtStartPar
\phantomsection\label{\detokenize{reference/math:id14}}{\hyperref[\detokenize{reference/math:vecsum}]{\sphinxcrossref{VecSum()}}}

\item {} 
\sphinxAtStartPar
\phantomsection\label{\detokenize{reference/math:id15}}{\hyperref[\detokenize{reference/math:vectimes}]{\sphinxcrossref{VecTimes()}}}

\item {} 
\sphinxAtStartPar
\phantomsection\label{\detokenize{reference/math:id16}}{\hyperref[\detokenize{reference/math:max}]{\sphinxcrossref{Max()}}}

\item {} 
\sphinxAtStartPar
\phantomsection\label{\detokenize{reference/math:id17}}{\hyperref[\detokenize{reference/math:median}]{\sphinxcrossref{Median()}}}

\item {} 
\sphinxAtStartPar
\phantomsection\label{\detokenize{reference/math:id18}}{\hyperref[\detokenize{reference/math:min}]{\sphinxcrossref{Min()}}}

\item {} 
\sphinxAtStartPar
\phantomsection\label{\detokenize{reference/math:id19}}{\hyperref[\detokenize{reference/math:stddev}]{\sphinxcrossref{StdDev()}}}

\end{itemize}
\end{sphinxShadowBox}

\index{Bound@\spxentry{Bound}}\ignorespaces 

\subsection{Bound()}
\label{\detokenize{reference/math:bound}}\label{\detokenize{reference/math:index-0}}
\sphinxAtStartPar
\sphinxstyleemphasis{Returns val, bounded by min and max.}

\sphinxAtStartPar
\sphinxstylestrong{Description:}

\sphinxAtStartPar
This makes sure
number is between min and max;
if min\textgreater{}max, it will return max, soyou need to check if that
isn’t the right behavior.

\sphinxAtStartPar
\sphinxstylestrong{Usage:}

\begin{sphinxVerbatim}[commandchars=\\\{\}]
\PYG{k}{define}\PYG{+w}{ }\PYG{n+nf}{Bound}\PYG{p}{(}\PYG{n+nv}{number}\PYG{p}{,}\PYG{n+nv}{min}\PYG{p}{,}\PYG{n+nv}{max}\PYG{p}{)}
\end{sphinxVerbatim}

\index{CumNormInv@\spxentry{CumNormInv}}\ignorespaces 

\subsection{CumNormInv()}
\label{\detokenize{reference/math:cumnorminv}}\label{\detokenize{reference/math:index-1}}
\sphinxAtStartPar
\sphinxstyleemphasis{Returns accurate numerical approximation of cumulative normal inverse.}

\sphinxAtStartPar
\sphinxstylestrong{Description:}

\sphinxAtStartPar
This function takes a probability and returns the    corresponding z\sphinxhyphen{}score for the cumulative standard normal distribution.   It uses an accurate numerical approximation from: \sphinxcode{\sphinxupquote{http://home.online.no/\textasciitilde{}pjacklam/notes/invnorm}}

\sphinxAtStartPar
\sphinxstylestrong{Usage:}

\begin{sphinxVerbatim}[commandchars=\\\{\}]
\PYG{k}{define}\PYG{+w}{ }\PYG{n+nf}{CumNormInv}\PYG{p}{(}\PYG{p}{.}\PYG{p}{.}\PYG{p}{.}\PYG{p}{)}
\end{sphinxVerbatim}

\sphinxAtStartPar
\sphinxstylestrong{Example:}

\begin{sphinxVerbatim}[commandchars=\\\{\}]
\PYG{n+nf}{Print}\PYG{p}{(}\PYG{n+nf}{CumNormInv}\PYG{p}{(}\PYG{l+m+mi}{0}\PYG{p}{)}\PYG{p}{)}\PYG{+w}{    }\PYG{c+c1}{\PYGZsh{}= NA}
\PYG{+w}{ }\PYG{n+nf}{Print}\PYG{p}{(}\PYG{n+nf}{CumNormInv}\PYG{p}{(}\PYG{l+m+mf}{.01}\PYG{p}{)}\PYG{p}{)}\PYG{+w}{ }\PYG{c+c1}{\PYGZsh{}= \PYGZhy{}2.32634}
\PYG{+w}{ }\PYG{n+nf}{Print}\PYG{p}{(}\PYG{n+nf}{CumNormInv}\PYG{p}{(}\PYG{l+m+mf}{.5}\PYG{p}{)}\PYG{p}{)}\PYG{+w}{  }\PYG{c+c1}{\PYGZsh{}= 0}
\PYG{+w}{ }\PYG{n+nf}{Print}\PYG{p}{(}\PYG{n+nf}{CumNormInv}\PYG{p}{(}\PYG{l+m+mf}{.9}\PYG{p}{)}\PYG{p}{)}\PYG{+w}{  }\PYG{c+c1}{\PYGZsh{}= 1.28}
\PYG{+w}{ }\PYG{n+nf}{Print}\PYG{p}{(}\PYG{n+nf}{CumNormInv}\PYG{p}{(}\PYG{l+m+mi}{1}\PYG{p}{)}\PYG{p}{)}\PYG{+w}{   }\PYG{c+c1}{\PYGZsh{}= NA}
\end{sphinxVerbatim}

\sphinxAtStartPar
\sphinxstylestrong{See Also:}

\sphinxAtStartPar
\sphinxcode{\sphinxupquote{NormalDensity()}}, \sphinxcode{\sphinxupquote{RandomNormal()}}

\index{CumSum@\spxentry{CumSum}}\ignorespaces 

\subsection{CumSum()}
\label{\detokenize{reference/math:cumsum}}\label{\detokenize{reference/math:index-2}}
\sphinxAtStartPar
\sphinxstyleemphasis{Returns the cumulative sums of a set of numbers}

\sphinxAtStartPar
\sphinxstylestrong{Description:}

\sphinxAtStartPar
Returns the cumulative sum  of \sphinxcode{\sphinxupquote{\textless{}list\textgreater{}}}.

\sphinxAtStartPar
\sphinxstylestrong{Usage:}

\begin{sphinxVerbatim}[commandchars=\\\{\}]
\PYG{k}{define}\PYG{+w}{ }\PYG{n+nf}{CumSum}\PYG{p}{(}\PYG{p}{.}\PYG{p}{.}\PYG{p}{.}\PYG{p}{)}
\end{sphinxVerbatim}

\sphinxAtStartPar
\sphinxstylestrong{Example:}

\begin{sphinxVerbatim}[commandchars=\\\{\}]
\PYG{n+nv}{sum}\PYG{+w}{ }\PYG{o}{\PYGZlt{}\PYGZhy{}}\PYG{+w}{ }\PYG{n+nf}{CumSum}\PYG{p}{(}\PYG{p}{[}\PYG{l+m+mi}{1}\PYG{p}{,}\PYG{l+m+mi}{2}\PYG{p}{,}\PYG{l+m+mi}{3}\PYG{p}{,}\PYG{l+m+mi}{3}\PYG{p}{,}\PYG{l+m+mi}{4}\PYG{p}{,}\PYG{l+m+mi}{7}\PYG{p}{]}\PYG{p}{)}
\PYG{c+c1}{\PYGZsh{} == [1,3,6,9,13,20]}
\end{sphinxVerbatim}

\sphinxAtStartPar
\sphinxstylestrong{See Also:}

\sphinxAtStartPar
\sphinxcode{\sphinxupquote{Min()}}, \sphinxcode{\sphinxupquote{Max()}}, \sphinxcode{\sphinxupquote{Mean()}}, \sphinxcode{\sphinxupquote{Median()}}, \sphinxcode{\sphinxupquote{Quantile()}}, \sphinxcode{\sphinxupquote{StDev()}}

\index{Dist@\spxentry{Dist}}\ignorespaces 

\subsection{Dist()}
\label{\detokenize{reference/math:dist}}\label{\detokenize{reference/math:index-3}}
\sphinxAtStartPar
\sphinxstyleemphasis{Returns distance between two points.}

\sphinxAtStartPar
\sphinxstylestrong{Description:}

\sphinxAtStartPar
Returns Euclidean distance between two points.   Each point should be {[}x,y{]}, and any additional items in the list are   ignored.

\sphinxAtStartPar
\sphinxstylestrong{Usage:}

\begin{sphinxVerbatim}[commandchars=\\\{\}]
\PYG{k}{define}\PYG{+w}{ }\PYG{n+nf}{Dist}\PYG{p}{(}\PYG{p}{.}\PYG{p}{.}\PYG{p}{.}\PYG{p}{)}
\end{sphinxVerbatim}

\sphinxAtStartPar
\sphinxstylestrong{Example:}

\begin{sphinxVerbatim}[commandchars=\\\{\}]
\PYG{n+nv}{p1}\PYG{+w}{ }\PYG{o}{\PYGZlt{}\PYGZhy{}}\PYG{+w}{ }\PYG{p}{[}\PYG{l+m+mi}{0}\PYG{p}{,}\PYG{l+m+mi}{0}\PYG{p}{]}
\PYG{n+nv}{p2}\PYG{+w}{ }\PYG{o}{\PYGZlt{}\PYGZhy{}}\PYG{+w}{ }\PYG{p}{[}\PYG{l+m+mi}{3}\PYG{p}{,}\PYG{l+m+mi}{4}\PYG{p}{]}
\PYG{n+nv}{d}\PYG{+w}{ }\PYG{o}{\PYGZlt{}\PYGZhy{}}\PYG{+w}{ }\PYG{n+nf}{Dist}\PYG{p}{(}\PYG{n+nv}{p1}\PYG{p}{,}\PYG{n+nv}{p2}\PYG{p}{)}\PYG{+w}{  }\PYG{c+c1}{\PYGZsh{}d is 5}
\end{sphinxVerbatim}

\index{Filter@\spxentry{Filter}}\ignorespaces 

\subsection{Filter()}
\label{\detokenize{reference/math:filter}}\label{\detokenize{reference/math:index-4}}
\sphinxAtStartPar
\sphinxstyleemphasis{Filters a list based on a 0/1 list produced by Match.}

\sphinxAtStartPar
\sphinxstylestrong{Description:}

\sphinxAtStartPar
Returns a subset of \sphinxcode{\sphinxupquote{\textless{}list\textgreater{}}}, depending on whether the \sphinxcode{\sphinxupquote{\textless{}filter\textgreater{}}} list is zero or nonzero.  Both arguments must be lists of the same length.

\sphinxAtStartPar
\sphinxstylestrong{Usage:}

\begin{sphinxVerbatim}[commandchars=\\\{\}]
\PYG{k}{define}\PYG{+w}{ }\PYG{n+nf}{Filter}\PYG{p}{(}\PYG{p}{.}\PYG{p}{.}\PYG{p}{.}\PYG{p}{)}
\end{sphinxVerbatim}

\sphinxAtStartPar
\sphinxstylestrong{Example:}

\begin{sphinxVerbatim}[commandchars=\\\{\}]
\PYG{n+nv}{x}\PYG{+w}{ }\PYG{o}{\PYGZlt{}\PYGZhy{}}\PYG{+w}{ }\PYG{p}{[}\PYG{l+m+mi}{1}\PYG{p}{,}\PYG{l+m+mi}{2}\PYG{p}{,}\PYG{l+m+mi}{3}\PYG{p}{,}\PYG{l+m+mi}{3}\PYG{p}{,}\PYG{l+m+mi}{2}\PYG{p}{,}\PYG{l+m+mi}{2}\PYG{p}{,}\PYG{l+m+mi}{1}\PYG{p}{]}
\PYG{n+nf}{Print}\PYG{p}{(}\PYG{n+nf}{Filter}\PYG{p}{(}\PYG{n+nv}{x}\PYG{p}{,}\PYG{p}{[}\PYG{l+m+mi}{1}\PYG{p}{,}\PYG{l+m+mi}{1}\PYG{p}{,}\PYG{l+m+mi}{1}\PYG{p}{,}\PYG{l+m+mi}{0}\PYG{p}{,}\PYG{l+m+mi}{0}\PYG{p}{,}\PYG{l+m+mi}{0}\PYG{p}{,}\PYG{l+m+mi}{0}\PYG{p}{]}\PYG{p}{)}\PYG{p}{)}\PYG{+w}{ }\PYG{c+c1}{\PYGZsh{}\PYGZsh{}==[1,2,3]}
\PYG{n+nf}{Print}\PYG{p}{(}\PYG{n+nf}{Filter}\PYG{p}{(}\PYG{n+nv}{x}\PYG{p}{,}\PYG{n+nf}{Match}\PYG{p}{(}\PYG{n+nv}{x}\PYG{p}{,}\PYG{l+m+mi}{1}\PYG{p}{)}\PYG{p}{)}\PYG{p}{)}\PYG{+w}{      }\PYG{c+c1}{\PYGZsh{}\PYGZsh{}== [1,1]}
\end{sphinxVerbatim}

\sphinxAtStartPar
\sphinxstylestrong{See Also:}

\sphinxAtStartPar
\sphinxcode{\sphinxupquote{Match()}}, \sphinxcode{\sphinxupquote{Subset()}}, \sphinxcode{\sphinxupquote{Lookup()}}

\index{Match@\spxentry{Match}}\ignorespaces 

\subsection{Match()}
\label{\detokenize{reference/math:match}}\label{\detokenize{reference/math:index-5}}
\sphinxAtStartPar
\sphinxstyleemphasis{Returns a list of 0/1s, indicating which elements of list match item.}

\sphinxAtStartPar
\sphinxstylestrong{Description:}

\sphinxAtStartPar
Returns a list of 0/1, indicating which elements of  \sphinxcode{\sphinxupquote{\textless{}list\textgreater{}}} match \sphinxcode{\sphinxupquote{\textless{}target\textgreater{}}}

\sphinxAtStartPar
\sphinxstylestrong{Usage:}

\begin{sphinxVerbatim}[commandchars=\\\{\}]
\PYG{k}{define}\PYG{+w}{ }\PYG{n+nf}{Match}\PYG{p}{(}\PYG{p}{.}\PYG{p}{.}\PYG{p}{.}\PYG{p}{)}
\end{sphinxVerbatim}

\sphinxAtStartPar
\sphinxstylestrong{Example:}

\begin{sphinxVerbatim}[commandchars=\\\{\}]
\PYG{n+nv}{x}\PYG{+w}{ }\PYG{o}{\PYGZlt{}\PYGZhy{}}\PYG{+w}{ }\PYG{p}{[}\PYG{l+m+mi}{1}\PYG{p}{,}\PYG{l+m+mi}{2}\PYG{p}{,}\PYG{l+m+mi}{3}\PYG{p}{,}\PYG{l+m+mi}{3}\PYG{p}{,}\PYG{l+m+mi}{2}\PYG{p}{,}\PYG{l+m+mi}{2}\PYG{p}{,}\PYG{l+m+mi}{1}\PYG{p}{]}
\PYG{n+nf}{Print}\PYG{p}{(}\PYG{n+nf}{Match}\PYG{p}{(}\PYG{n+nv}{x}\PYG{p}{,}\PYG{l+m+mi}{1}\PYG{p}{)}\PYG{p}{)}\PYG{+w}{  }\PYG{c+c1}{\PYGZsh{}\PYGZsh{}== [1,0,0,0,0,0,1]}
\PYG{n+nf}{Print}\PYG{p}{(}\PYG{n+nf}{Match}\PYG{p}{(}\PYG{n+nv}{x}\PYG{p}{,}\PYG{l+m+mi}{2}\PYG{p}{)}\PYG{p}{)}\PYG{+w}{  }\PYG{c+c1}{\PYGZsh{}\PYGZsh{}== [0,1,0,0,1,1,0]}
\PYG{n+nf}{Print}\PYG{p}{(}\PYG{+w}{ }\PYG{n+nf}{Match}\PYG{p}{(}\PYG{n+nv}{x}\PYG{p}{,}\PYG{l+m+mi}{3}\PYG{p}{)}\PYG{+w}{  }\PYG{c+c1}{\PYGZsh{}\PYGZsh{}== [0,0,1,1,0,0,0]}
\end{sphinxVerbatim}

\sphinxAtStartPar
\sphinxstylestrong{See Also:}

\sphinxAtStartPar
\sphinxcode{\sphinxupquote{Filter()}}, \sphinxcode{\sphinxupquote{Subset()}}, \sphinxcode{\sphinxupquote{Lookup()}}

\index{NormalDensity@\spxentry{NormalDensity}}\ignorespaces 

\subsection{NormalDensity()}
\label{\detokenize{reference/math:normaldensity}}\label{\detokenize{reference/math:index-6}}
\sphinxAtStartPar
\sphinxstyleemphasis{Returns density of standard normal distribution.}

\sphinxAtStartPar
\sphinxstylestrong{Description:}

\sphinxAtStartPar
Computes density of normal standard distribution

\sphinxAtStartPar
\sphinxstylestrong{Usage:}

\begin{sphinxVerbatim}[commandchars=\\\{\}]
\PYG{k}{define}\PYG{+w}{ }\PYG{n+nf}{NormalDensity}\PYG{p}{(}\PYG{p}{.}\PYG{p}{.}\PYG{p}{.}\PYG{p}{)}
\end{sphinxVerbatim}

\sphinxAtStartPar
\sphinxstylestrong{Example:}

\begin{sphinxVerbatim}[commandchars=\\\{\}]
\PYG{n+nf}{Print}\PYG{p}{(}\PYG{n+nf}{NormalDensity}\PYG{p}{(}\PYG{o}{\PYGZhy{}}\PYG{l+m+mi}{100}\PYG{p}{)}\PYG{p}{)}\PYG{+w}{     }\PYG{c+c1}{\PYGZsh{} 1.8391e\PYGZhy{}2171}
\PYG{n+nf}{Print}\PYG{p}{(}\PYG{n+nf}{NormalDensity}\PYG{p}{(}\PYG{o}{\PYGZhy{}}\PYG{l+m+mf}{2.32635}\PYG{p}{)}\PYG{p}{)}\PYG{+w}{ }\PYG{c+c1}{\PYGZsh{}5.97}
\PYG{n+nf}{Print}\PYG{p}{(}\PYG{n+nf}{NormalDensity}\PYG{p}{(}\PYG{l+m+mi}{0}\PYG{p}{)}\PYG{p}{)}\PYG{+w}{        }\PYG{c+c1}{\PYGZsh{}0.398942}
\PYG{n+nf}{Print}\PYG{p}{(}\PYG{n+nf}{NormalDensity}\PYG{p}{(}\PYG{l+m+mf}{1.28155}\PYG{p}{)}\PYG{p}{)}\PYG{+w}{  }\PYG{c+c1}{\PYGZsh{}.90687}
\PYG{n+nf}{Print}\PYG{p}{(}\PYG{n+nf}{NormalDensity}\PYG{p}{(}\PYG{l+m+mi}{1000}\PYG{p}{)}\PYG{p}{)}\PYG{+w}{     }\PYG{c+c1}{\PYGZsh{}inf}
\end{sphinxVerbatim}

\sphinxAtStartPar
\sphinxstylestrong{See Also:}

\sphinxAtStartPar
\sphinxcode{\sphinxupquote{RandomNormal()}}, \sphinxcode{\sphinxupquote{CumNormInv()}}

\index{Order@\spxentry{Order}}\ignorespaces 

\subsection{Order()}
\label{\detokenize{reference/math:order}}\label{\detokenize{reference/math:index-7}}
\sphinxAtStartPar
\sphinxstylestrong{Description:}

\sphinxAtStartPar
Returns a list of indices describing the order of values by position, from min to max.

\sphinxAtStartPar
\sphinxstylestrong{Usage:}

\begin{sphinxVerbatim}[commandchars=\\\{\}]
\PYG{k}{define}\PYG{+w}{ }\PYG{n+nf}{Order}\PYG{p}{(}\PYG{p}{.}\PYG{p}{.}\PYG{p}{.}\PYG{p}{)}
\end{sphinxVerbatim}

\sphinxAtStartPar
\sphinxstylestrong{Example:}

\begin{sphinxVerbatim}[commandchars=\\\{\}]
\PYG{+w}{ }\PYG{n+nv}{n}\PYG{+w}{ }\PYG{o}{\PYGZlt{}\PYGZhy{}}\PYG{+w}{ }\PYG{p}{[}\PYG{l+m+mi}{33}\PYG{p}{,}\PYG{l+m+mi}{12}\PYG{p}{,}\PYG{l+m+mi}{1}\PYG{p}{,}\PYG{l+m+mi}{5}\PYG{p}{,}\PYG{l+m+mi}{9}\PYG{p}{]}
\PYG{+w}{ }\PYG{n+nv}{o}\PYG{+w}{ }\PYG{o}{\PYGZlt{}\PYGZhy{}}\PYG{+w}{ }\PYG{n+nf}{Order}\PYG{p}{(}\PYG{n+nv}{n}\PYG{p}{)}
\PYG{n+nf}{Print}\PYG{p}{(}\PYG{n+nv}{o}\PYG{p}{)}\PYG{+w}{ }\PYG{c+c1}{\PYGZsh{}should print [3,4,5,2,1]}
\end{sphinxVerbatim}

\sphinxAtStartPar
\sphinxstylestrong{See Also:}

\sphinxAtStartPar
\sphinxcode{\sphinxupquote{Rank()}}

\index{Rank@\spxentry{Rank}}\ignorespaces 

\subsection{Rank()}
\label{\detokenize{reference/math:rank}}\label{\detokenize{reference/math:index-8}}
\sphinxAtStartPar
\sphinxstylestrong{Description:}

\sphinxAtStartPar
Returns a list of numbers describing the rank of   each position, from min to max.  The same as calling \sphinxcode{\sphinxupquote{Order(Order(x))}}.

\sphinxAtStartPar
\sphinxstylestrong{Usage:}

\begin{sphinxVerbatim}[commandchars=\\\{\}]
\PYG{k}{define}\PYG{+w}{ }\PYG{n+nf}{Rank}\PYG{p}{(}\PYG{p}{.}\PYG{p}{.}\PYG{p}{.}\PYG{p}{)}
\end{sphinxVerbatim}

\sphinxAtStartPar
\sphinxstylestrong{Example:}

\begin{sphinxVerbatim}[commandchars=\\\{\}]
\PYG{+w}{ }\PYG{n+nv}{n}\PYG{+w}{ }\PYG{o}{\PYGZlt{}\PYGZhy{}}\PYG{+w}{ }\PYG{p}{[}\PYG{l+m+mi}{33}\PYG{p}{,}\PYG{l+m+mi}{12}\PYG{p}{,}\PYG{l+m+mi}{1}\PYG{p}{,}\PYG{l+m+mi}{5}\PYG{p}{,}\PYG{l+m+mi}{9}\PYG{p}{]}
\PYG{+w}{ }\PYG{n+nv}{o}\PYG{+w}{ }\PYG{o}{\PYGZlt{}\PYGZhy{}}\PYG{+w}{ }\PYG{n+nf}{Rank}\PYG{p}{(}\PYG{n+nv}{n}\PYG{p}{)}
\PYG{n+nf}{Print}\PYG{p}{(}\PYG{n+nv}{o}\PYG{p}{)}\PYG{+w}{ }\PYG{c+c1}{\PYGZsh{}should print [5,4,1,2,3]}
\end{sphinxVerbatim}

\sphinxAtStartPar
\sphinxstylestrong{See Also:}

\sphinxAtStartPar
\sphinxcode{\sphinxupquote{Order()}}

\index{SDTBeta@\spxentry{SDTBeta}}\ignorespaces 

\subsection{SDTBeta()}
\label{\detokenize{reference/math:sdtbeta}}\label{\detokenize{reference/math:index-9}}
\sphinxAtStartPar
\sphinxstyleemphasis{Computes SDT beta.}

\sphinxAtStartPar
\sphinxstylestrong{Description:}

\sphinxAtStartPar
\sphinxcode{\sphinxupquote{SDTBeta}} computes beta, as defined by signal detection theory.  This is a measure of decision bias based on hit rate and false alarm rate.

\sphinxAtStartPar
\sphinxstylestrong{Usage:}

\begin{sphinxVerbatim}[commandchars=\\\{\}]
\PYG{k}{define}\PYG{+w}{ }\PYG{n+nf}{SDTBeta}\PYG{p}{(}\PYG{p}{.}\PYG{p}{.}\PYG{p}{.}\PYG{p}{)}
\end{sphinxVerbatim}

\sphinxAtStartPar
\sphinxstylestrong{Example:}

\begin{sphinxVerbatim}[commandchars=\\\{\}]
\PYG{n+nf}{Print}\PYG{p}{(}\PYG{n+nf}{SDTBeta}\PYG{p}{(}\PYG{l+m+mf}{.1}\PYG{p}{,}\PYG{l+m+mf}{.9}\PYG{p}{)}\PYG{p}{)}
\PYG{n+nf}{Print}\PYG{p}{(}\PYG{n+nf}{SDTBeta}\PYG{p}{(}\PYG{l+m+mf}{.1}\PYG{p}{,}\PYG{l+m+mf}{.5}\PYG{p}{)}\PYG{p}{)}
\PYG{n+nf}{Print}\PYG{p}{(}\PYG{n+nf}{SDTBeta}\PYG{p}{(}\PYG{l+m+mf}{.5}\PYG{p}{,}\PYG{l+m+mf}{.5}\PYG{p}{)}\PYG{p}{)}
\PYG{n+nf}{Print}\PYG{p}{(}\PYG{n+nf}{SDTBeta}\PYG{p}{(}\PYG{l+m+mf}{.8}\PYG{p}{,}\PYG{l+m+mf}{.9}\PYG{p}{)}\PYG{p}{)}
\PYG{n+nf}{Print}\PYG{p}{(}\PYG{n+nf}{SDTbeta}\PYG{p}{(}\PYG{l+m+mf}{.9}\PYG{p}{,}\PYG{l+m+mf}{.95}\PYG{p}{)}\PYG{p}{)}
\end{sphinxVerbatim}

\sphinxAtStartPar
\sphinxstylestrong{See Also:}

\sphinxAtStartPar
\sphinxcode{\sphinxupquote{SDTDPrime()}}

\index{SDTDPrime@\spxentry{SDTDPrime}}\ignorespaces 

\subsection{SDTDPrime()}
\label{\detokenize{reference/math:sdtdprime}}\label{\detokenize{reference/math:index-10}}
\sphinxAtStartPar
\sphinxstyleemphasis{Computes SDT dprime.}

\sphinxAtStartPar
\sphinxstylestrong{Description:}

\sphinxAtStartPar
\sphinxcode{\sphinxupquote{SDTDPrime}} computes d\sphinxhyphen{}prime, as defined by signal detection theory.  This is a measure of sensitivy based jointly on hit rate and false alarm rate.

\sphinxAtStartPar
\sphinxstylestrong{Usage:}

\begin{sphinxVerbatim}[commandchars=\\\{\}]
\PYG{k}{define}\PYG{+w}{ }\PYG{n+nf}{SDTDPrime}\PYG{p}{(}\PYG{p}{.}\PYG{p}{.}\PYG{p}{.}\PYG{p}{)}
\end{sphinxVerbatim}

\sphinxAtStartPar
\sphinxstylestrong{Example:}

\begin{sphinxVerbatim}[commandchars=\\\{\}]
\PYG{n+nf}{Print}\PYG{p}{(}\PYG{n+nf}{SDTDPrime}\PYG{p}{(}\PYG{l+m+mf}{.1}\PYG{p}{,}\PYG{l+m+mf}{.9}\PYG{p}{)}\PYG{p}{)}\PYG{+w}{  }\PYG{c+c1}{\PYGZsh{}2.56431}
\PYG{n+nf}{Print}\PYG{p}{(}\PYG{n+nf}{SDTDPrime}\PYG{p}{(}\PYG{l+m+mf}{.1}\PYG{p}{,}\PYG{l+m+mf}{.5}\PYG{p}{)}\PYG{p}{)}\PYG{+w}{  }\PYG{c+c1}{\PYGZsh{}1.28155}
\PYG{n+nf}{Print}\PYG{p}{(}\PYG{n+nf}{SDTDPrime}\PYG{p}{(}\PYG{l+m+mf}{.5}\PYG{p}{,}\PYG{l+m+mf}{.5}\PYG{p}{)}\PYG{p}{)}\PYG{+w}{  }\PYG{c+c1}{\PYGZsh{}0}
\PYG{n+nf}{Print}\PYG{p}{(}\PYG{n+nf}{SDTDPrime}\PYG{p}{(}\PYG{l+m+mf}{.8}\PYG{p}{,}\PYG{l+m+mf}{.9}\PYG{p}{)}\PYG{p}{)}\PYG{+w}{  }\PYG{c+c1}{\PYGZsh{}.43993}
\PYG{n+nf}{Print}\PYG{p}{(}\PYG{n+nf}{SDTDPrime}\PYG{p}{(}\PYG{l+m+mf}{.9}\PYG{p}{,}\PYG{l+m+mf}{.95}\PYG{p}{)}\PYG{p}{)}\PYG{+w}{ }\PYG{c+c1}{\PYGZsh{}.363302}
\end{sphinxVerbatim}

\sphinxAtStartPar
\sphinxstylestrong{See Also:}

\sphinxAtStartPar
\sphinxcode{\sphinxupquote{SDTBeta()}},

\index{Sum@\spxentry{Sum}}\ignorespaces 

\subsection{Sum()}
\label{\detokenize{reference/math:sum}}\label{\detokenize{reference/math:index-11}}
\sphinxAtStartPar
\sphinxstylestrong{Description:}

\sphinxAtStartPar
Returns the sum  of \sphinxcode{\sphinxupquote{\textless{}list\textgreater{}}}.

\sphinxAtStartPar
\sphinxstylestrong{Usage:}

\begin{sphinxVerbatim}[commandchars=\\\{\}]
\PYG{k}{define}\PYG{+w}{ }\PYG{n+nf}{Sum}\PYG{p}{(}\PYG{p}{.}\PYG{p}{.}\PYG{p}{.}\PYG{p}{)}
\end{sphinxVerbatim}

\sphinxAtStartPar
\sphinxstylestrong{Example:}

\begin{sphinxVerbatim}[commandchars=\\\{\}]
\PYG{n+nv}{sum}\PYG{+w}{ }\PYG{o}{\PYGZlt{}\PYGZhy{}}\PYG{+w}{ }\PYG{n+nf}{Sum}\PYG{p}{(}\PYG{p}{[}\PYG{l+m+mi}{3}\PYG{p}{,}\PYG{l+m+mi}{5}\PYG{p}{,}\PYG{l+m+mi}{99}\PYG{p}{,}\PYG{l+m+mi}{12}\PYG{p}{,}\PYG{l+m+mf}{1.3}\PYG{p}{,}\PYG{l+m+mi}{15}\PYG{p}{]}\PYG{p}{)}\PYG{+w}{      }\PYG{c+c1}{\PYGZsh{} == 135.3}
\end{sphinxVerbatim}

\sphinxAtStartPar
\sphinxstylestrong{See Also:}

\sphinxAtStartPar
\sphinxcode{\sphinxupquote{Min()}}, \sphinxcode{\sphinxupquote{Max()}}, \sphinxcode{\sphinxupquote{Mean()}}, \sphinxcode{\sphinxupquote{Median()}}, \sphinxcode{\sphinxupquote{Quantile()}}, \sphinxcode{\sphinxupquote{StDev()}}

\index{SummaryStats@\spxentry{SummaryStats}}\ignorespaces 

\subsection{SummaryStats()}
\label{\detokenize{reference/math:summarystats}}\label{\detokenize{reference/math:index-12}}
\sphinxAtStartPar
\sphinxstylestrong{Description:}

\sphinxAtStartPar
Computes summary statistics for a data list,   aggregated by labels in a condition list. For each condition (distinct label in the \sphinxcode{\sphinxupquote{\textless{}cond\textgreater{}}} list), it will  return a list with the following entries: \sphinxcode{\sphinxupquote{\textless{}cond\textgreater{}}} \sphinxcode{\sphinxupquote{\textless{}N\textgreater{}}} \sphinxcode{\sphinxupquote{\textless{}median\textgreater{}}} \sphinxcode{\sphinxupquote{\textless{}mean\textgreater{}}} \sphinxcode{\sphinxupquote{\textless{}sd\textgreater{}}}

\sphinxAtStartPar
\sphinxstylestrong{Usage:}

\begin{sphinxVerbatim}[commandchars=\\\{\}]
\PYG{k}{define}\PYG{+w}{ }\PYG{n+nf}{SummaryStats}\PYG{p}{(}\PYG{p}{.}\PYG{p}{.}\PYG{p}{.}\PYG{p}{)}
\end{sphinxVerbatim}

\sphinxAtStartPar
\sphinxstylestrong{Example:}

\begin{sphinxVerbatim}[commandchars=\\\{\}]
\PYG{+w}{  }\PYG{n+nv}{dat}\PYG{+w}{ }\PYG{o}{\PYGZlt{}\PYGZhy{}}\PYG{+w}{ }\PYG{p}{[}\PYG{l+m+mf}{1.1}\PYG{p}{,}\PYG{l+m+mf}{1.2}\PYG{p}{,}\PYG{l+m+mf}{1.3}\PYG{p}{,}\PYG{l+m+mf}{2.1}\PYG{p}{,}\PYG{l+m+mf}{2.2}\PYG{p}{,}\PYG{l+m+mf}{2.3}\PYG{p}{]}
\PYG{+w}{  }\PYG{n+nv}{cond}\PYG{+w}{ }\PYG{o}{\PYGZlt{}\PYGZhy{}}\PYG{+w}{ }\PYG{p}{[}\PYG{l+m+mi}{1}\PYG{p}{,}\PYG{l+m+mi}{1}\PYG{p}{,}\PYG{l+m+mi}{1}\PYG{p}{,}\PYG{l+m+mi}{2}\PYG{p}{,}\PYG{l+m+mi}{2}\PYG{p}{,}\PYG{l+m+mi}{2}\PYG{p}{]}
\PYG{+w}{  }\PYG{n+nf}{Print}\PYG{p}{(}\PYG{n+nf}{SummaryStats}\PYG{p}{(}\PYG{n+nv}{dat}\PYG{p}{,}\PYG{n+nv}{cond}\PYG{p}{)}\PYG{p}{)}

R\PYG{n+nv}{esult}\PYG{o}{:}

\PYG{p}{[}\PYG{p}{[}\PYG{l+m+mi}{1}\PYG{p}{,}\PYG{+w}{ }\PYG{l+m+mi}{3}\PYG{p}{,}\PYG{+w}{ }\PYG{l+m+mf}{1.1}\PYG{p}{,}\PYG{+w}{ }\PYG{l+m+mf}{1.2}\PYG{p}{,}\PYG{+w}{ }\PYG{l+m+mf}{0.0816497}\PYG{p}{]}
\PYG{p}{,}\PYG{+w}{ }\PYG{p}{[}\PYG{l+m+mi}{2}\PYG{p}{,}\PYG{+w}{ }\PYG{l+m+mi}{3}\PYG{p}{,}\PYG{+w}{ }\PYG{l+m+mf}{2.1}\PYG{p}{,}\PYG{+w}{ }\PYG{l+m+mf}{2.2}\PYG{p}{,}\PYG{+w}{ }\PYG{l+m+mf}{0.0816497}\PYG{p}{]}
\PYG{p}{]}
\end{sphinxVerbatim}

\sphinxAtStartPar
\sphinxstylestrong{See Also:}

\sphinxAtStartPar
\sphinxcode{\sphinxupquote{StDev()}}, \sphinxcode{\sphinxupquote{Min()}}, \sphinxcode{\sphinxupquote{Max()}}, \sphinxcode{\sphinxupquote{Mean()}}, \sphinxcode{\sphinxupquote{Median()}}, \sphinxcode{\sphinxupquote{Quantile()}}, \sphinxcode{\sphinxupquote{Sum()}}

\index{VecSum@\spxentry{VecSum}}\ignorespaces 

\subsection{VecSum()}
\label{\detokenize{reference/math:vecsum}}\label{\detokenize{reference/math:index-13}}
\sphinxAtStartPar
\sphinxstyleemphasis{Returns the pairwise sums of two lists of numbers}

\sphinxAtStartPar
\sphinxstylestrong{Description:}

\sphinxAtStartPar
Returns the pairwise sums of \sphinxcode{\sphinxupquote{\textless{}list1\textgreater{}}} and \sphinxcode{\sphinxupquote{\textless{}list2\textgreater{}}}.

\sphinxAtStartPar
\sphinxstylestrong{Usage:}

\begin{sphinxVerbatim}[commandchars=\\\{\}]
\PYG{k}{define}\PYG{+w}{ }\PYG{n+nf}{VecSum}\PYG{p}{(}\PYG{p}{.}\PYG{p}{.}\PYG{p}{.}\PYG{p}{)}
\end{sphinxVerbatim}

\sphinxAtStartPar
\sphinxstylestrong{Example:}

\begin{sphinxVerbatim}[commandchars=\\\{\}]
\PYG{n+nv}{sum}\PYG{+w}{ }\PYG{o}{\PYGZlt{}\PYGZhy{}}\PYG{+w}{ }\PYG{n+nf}{VecSum}\PYG{p}{(}\PYG{p}{[}\PYG{l+m+mi}{1}\PYG{p}{,}\PYG{l+m+mi}{1}\PYG{p}{,}\PYG{l+m+mi}{1}\PYG{p}{,}\PYG{l+m+mi}{1}\PYG{p}{,}\PYG{l+m+mi}{2}\PYG{p}{]}\PYG{p}{,}\PYG{p}{[}\PYG{l+m+mi}{2}\PYG{p}{,}\PYG{l+m+mi}{3}\PYG{p}{,}\PYG{l+m+mi}{4}\PYG{p}{,}\PYG{l+m+mi}{3}\PYG{p}{,}\PYG{l+m+mi}{2}\PYG{p}{]}\PYG{p}{)}
\PYG{c+c1}{\PYGZsh{}\PYGZsh{} == [3,4,5,4,4]}
\end{sphinxVerbatim}

\sphinxAtStartPar
\sphinxstylestrong{See Also:}

\sphinxAtStartPar
\sphinxcode{\sphinxupquote{VecTimes()}}, \sphinxcode{\sphinxupquote{CumSum()}}, \sphinxcode{\sphinxupquote{Median()}}, \sphinxcode{\sphinxupquote{Quantile()}}

\index{VecTimes@\spxentry{VecTimes}}\ignorespaces 

\subsection{VecTimes()}
\label{\detokenize{reference/math:vectimes}}\label{\detokenize{reference/math:index-14}}
\sphinxAtStartPar
\sphinxstyleemphasis{Returns the pairwise products of two lists of numbers}

\sphinxAtStartPar
\sphinxstylestrong{Description:}

\sphinxAtStartPar
Returns the pairwise sums of \sphinxcode{\sphinxupquote{\textless{}list1\textgreater{}}} and \sphinxcode{\sphinxupquote{\textless{}list2\textgreater{}}}.

\sphinxAtStartPar
\sphinxstylestrong{Usage:}

\begin{sphinxVerbatim}[commandchars=\\\{\}]
\PYG{k}{define}\PYG{+w}{ }\PYG{n+nf}{VecTimes}\PYG{p}{(}\PYG{p}{.}\PYG{p}{.}\PYG{p}{.}\PYG{p}{)}
\end{sphinxVerbatim}

\sphinxAtStartPar
\sphinxstylestrong{Example:}

\begin{sphinxVerbatim}[commandchars=\\\{\}]
\PYG{n+nv}{prod}\PYG{+w}{ }\PYG{o}{\PYGZlt{}\PYGZhy{}}\PYG{+w}{ }\PYG{n+nf}{VecTimes}\PYG{p}{(}\PYG{p}{[}\PYG{l+m+mi}{1}\PYG{p}{,}\PYG{l+m+mi}{1}\PYG{p}{,}\PYG{l+m+mi}{2}\PYG{p}{,}\PYG{l+m+mi}{2}\PYG{p}{,}\PYG{l+m+mi}{3}\PYG{p}{]}\PYG{p}{,}\PYG{p}{[}\PYG{l+m+mi}{2}\PYG{p}{,}\PYG{l+m+mi}{3}\PYG{p}{,}\PYG{l+m+mi}{4}\PYG{p}{,}\PYG{l+m+mi}{3}\PYG{p}{,}\PYG{l+m+mi}{2}\PYG{p}{]}\PYG{p}{)}
\PYG{c+c1}{\PYGZsh{}\PYGZsh{} == [2,3,8,6,6]}
\end{sphinxVerbatim}

\sphinxAtStartPar
\sphinxstylestrong{See Also:}

\sphinxAtStartPar
\sphinxcode{\sphinxupquote{VecSum()}}, \sphinxcode{\sphinxupquote{Mean()}}, \sphinxcode{\sphinxupquote{CumSum()}}

\index{Max@\spxentry{Max}}\ignorespaces 

\subsection{Max()}
\label{\detokenize{reference/math:max}}\label{\detokenize{reference/math:index-15}}
\sphinxAtStartPar
\sphinxstyleemphasis{Returns the largest value in a list}

\sphinxAtStartPar
\sphinxstylestrong{Description:}

\sphinxAtStartPar
Max returns the largest value in a list. This is a PEBL function that wraps the compiled Max function, adding error checking to ensure the argument is a list.

\sphinxAtStartPar
\sphinxstylestrong{Usage:}

\begin{sphinxVerbatim}[commandchars=\\\{\}]
\PYG{k}{define}\PYG{+w}{ }\PYG{n+nf}{Max}\PYG{p}{(}\PYG{n+nv}{list}\PYG{p}{)}
\end{sphinxVerbatim}

\sphinxAtStartPar
\sphinxstylestrong{Example:}

\begin{sphinxVerbatim}[commandchars=\\\{\}]
\PYG{n+nv}{numbers}\PYG{+w}{ }\PYG{o}{\PYGZlt{}\PYGZhy{}}\PYG{+w}{ }\PYG{p}{[}\PYG{l+m+mi}{3}\PYG{p}{,}\PYG{+w}{ }\PYG{l+m+mi}{7}\PYG{p}{,}\PYG{+w}{ }\PYG{l+m+mi}{2}\PYG{p}{,}\PYG{+w}{ }\PYG{l+m+mi}{9}\PYG{p}{,}\PYG{+w}{ }\PYG{l+m+mi}{4}\PYG{p}{]}
\PYG{n+nv}{max\PYGZus{}value}\PYG{+w}{ }\PYG{o}{\PYGZlt{}\PYGZhy{}}\PYG{+w}{ }\PYG{n+nf}{Max}\PYG{p}{(}\PYG{n+nv}{numbers}\PYG{p}{)}\PYG{+w}{  }\PYG{c+c1}{\PYGZsh{} Returns 9}
\end{sphinxVerbatim}

\sphinxAtStartPar
\sphinxstylestrong{See Also:}

\sphinxAtStartPar
\sphinxcode{\sphinxupquote{Min()}}, \sphinxcode{\sphinxupquote{Mean()}}, \sphinxcode{\sphinxupquote{Median()}}, \sphinxcode{\sphinxupquote{StdDev()}}

\index{Median@\spxentry{Median}}\ignorespaces 

\subsection{Median()}
\label{\detokenize{reference/math:median}}\label{\detokenize{reference/math:index-16}}
\sphinxAtStartPar
\sphinxstyleemphasis{Returns the median value of a list}

\sphinxAtStartPar
\sphinxstylestrong{Description:}

\sphinxAtStartPar
Median returns the median value of a list. If the list has an even number of elements, it returns the average of the two middle values. This is a PEBL function that provides error checking and handles edge cases.

\sphinxAtStartPar
\sphinxstylestrong{Usage:}

\begin{sphinxVerbatim}[commandchars=\\\{\}]
\PYG{k}{define}\PYG{+w}{ }\PYG{n+nf}{Median}\PYG{p}{(}\PYG{n+nv}{list}\PYG{p}{)}
\end{sphinxVerbatim}

\sphinxAtStartPar
\sphinxstylestrong{Example:}

\begin{sphinxVerbatim}[commandchars=\\\{\}]
\PYG{n+nv}{numbers1}\PYG{+w}{ }\PYG{o}{\PYGZlt{}\PYGZhy{}}\PYG{+w}{ }\PYG{p}{[}\PYG{l+m+mi}{3}\PYG{p}{,}\PYG{+w}{ }\PYG{l+m+mi}{7}\PYG{p}{,}\PYG{+w}{ }\PYG{l+m+mi}{2}\PYG{p}{,}\PYG{+w}{ }\PYG{l+m+mi}{9}\PYG{p}{,}\PYG{+w}{ }\PYG{l+m+mi}{4}\PYG{p}{]}
\PYG{n+nv}{med1}\PYG{+w}{ }\PYG{o}{\PYGZlt{}\PYGZhy{}}\PYG{+w}{ }\PYG{n+nf}{Median}\PYG{p}{(}\PYG{n+nv}{numbers1}\PYG{p}{)}\PYG{+w}{  }\PYG{c+c1}{\PYGZsh{} Returns 4}

\PYG{n+nv}{numbers2}\PYG{+w}{ }\PYG{o}{\PYGZlt{}\PYGZhy{}}\PYG{+w}{ }\PYG{p}{[}\PYG{l+m+mi}{1}\PYG{p}{,}\PYG{+w}{ }\PYG{l+m+mi}{2}\PYG{p}{,}\PYG{+w}{ }\PYG{l+m+mi}{3}\PYG{p}{,}\PYG{+w}{ }\PYG{l+m+mi}{4}\PYG{p}{]}
\PYG{n+nv}{med2}\PYG{+w}{ }\PYG{o}{\PYGZlt{}\PYGZhy{}}\PYG{+w}{ }\PYG{n+nf}{Median}\PYG{p}{(}\PYG{n+nv}{numbers2}\PYG{p}{)}\PYG{+w}{  }\PYG{c+c1}{\PYGZsh{} Returns 2.5}
\end{sphinxVerbatim}

\sphinxAtStartPar
\sphinxstylestrong{See Also:}

\sphinxAtStartPar
\sphinxcode{\sphinxupquote{Mean()}}, \sphinxcode{\sphinxupquote{Min()}}, \sphinxcode{\sphinxupquote{Max()}}, \sphinxcode{\sphinxupquote{Quantile()}}, \sphinxcode{\sphinxupquote{StdDev()}}

\index{Min@\spxentry{Min}}\ignorespaces 

\subsection{Min()}
\label{\detokenize{reference/math:min}}\label{\detokenize{reference/math:index-17}}
\sphinxAtStartPar
\sphinxstyleemphasis{Returns the smallest value in a list}

\sphinxAtStartPar
\sphinxstylestrong{Description:}

\sphinxAtStartPar
Min returns the smallest value in a list. This is a PEBL function that wraps the compiled Min function, adding error checking to ensure the argument is a list.

\sphinxAtStartPar
\sphinxstylestrong{Usage:}

\begin{sphinxVerbatim}[commandchars=\\\{\}]
\PYG{k}{define}\PYG{+w}{ }\PYG{n+nf}{Min}\PYG{p}{(}\PYG{n+nv}{list}\PYG{p}{)}
\end{sphinxVerbatim}

\sphinxAtStartPar
\sphinxstylestrong{Example:}

\begin{sphinxVerbatim}[commandchars=\\\{\}]
\PYG{n+nv}{numbers}\PYG{+w}{ }\PYG{o}{\PYGZlt{}\PYGZhy{}}\PYG{+w}{ }\PYG{p}{[}\PYG{l+m+mi}{3}\PYG{p}{,}\PYG{+w}{ }\PYG{l+m+mi}{7}\PYG{p}{,}\PYG{+w}{ }\PYG{l+m+mi}{2}\PYG{p}{,}\PYG{+w}{ }\PYG{l+m+mi}{9}\PYG{p}{,}\PYG{+w}{ }\PYG{l+m+mi}{4}\PYG{p}{]}
\PYG{n+nv}{min\PYGZus{}value}\PYG{+w}{ }\PYG{o}{\PYGZlt{}\PYGZhy{}}\PYG{+w}{ }\PYG{n+nf}{Min}\PYG{p}{(}\PYG{n+nv}{numbers}\PYG{p}{)}\PYG{+w}{  }\PYG{c+c1}{\PYGZsh{} Returns 2}
\end{sphinxVerbatim}

\sphinxAtStartPar
\sphinxstylestrong{See Also:}

\sphinxAtStartPar
\sphinxcode{\sphinxupquote{Max()}}, \sphinxcode{\sphinxupquote{Mean()}}, \sphinxcode{\sphinxupquote{Median()}}, \sphinxcode{\sphinxupquote{StdDev()}}

\index{StdDev@\spxentry{StdDev}}\ignorespaces 

\subsection{StdDev()}
\label{\detokenize{reference/math:stddev}}\label{\detokenize{reference/math:index-18}}
\sphinxAtStartPar
\sphinxstyleemphasis{Returns the standard deviation of a list}

\sphinxAtStartPar
\sphinxstylestrong{Description:}

\sphinxAtStartPar
StdDev computes the standard deviation of a list of numbers. It uses the formula: sqrt(n * sum(x\textasciicircum{}2) \sphinxhyphen{} (sum(x))\textasciicircum{}2) / n. Returns 0 for empty lists. This is a PEBL function implemented in Math.pbl.

\sphinxAtStartPar
\sphinxstylestrong{Usage:}

\begin{sphinxVerbatim}[commandchars=\\\{\}]
\PYG{k}{define}\PYG{+w}{ }\PYG{n+nf}{StdDev}\PYG{p}{(}\PYG{n+nv}{list}\PYG{p}{)}
\end{sphinxVerbatim}

\sphinxAtStartPar
\sphinxstylestrong{Example:}

\begin{sphinxVerbatim}[commandchars=\\\{\}]
\PYG{n+nv}{data}\PYG{+w}{ }\PYG{o}{\PYGZlt{}\PYGZhy{}}\PYG{+w}{ }\PYG{p}{[}\PYG{l+m+mi}{2}\PYG{p}{,}\PYG{+w}{ }\PYG{l+m+mi}{4}\PYG{p}{,}\PYG{+w}{ }\PYG{l+m+mi}{4}\PYG{p}{,}\PYG{+w}{ }\PYG{l+m+mi}{4}\PYG{p}{,}\PYG{+w}{ }\PYG{l+m+mi}{5}\PYG{p}{,}\PYG{+w}{ }\PYG{l+m+mi}{5}\PYG{p}{,}\PYG{+w}{ }\PYG{l+m+mi}{7}\PYG{p}{,}\PYG{+w}{ }\PYG{l+m+mi}{9}\PYG{p}{]}
\PYG{n+nv}{sd}\PYG{+w}{ }\PYG{o}{\PYGZlt{}\PYGZhy{}}\PYG{+w}{ }\PYG{n+nf}{StdDev}\PYG{p}{(}\PYG{n+nv}{data}\PYG{p}{)}\PYG{+w}{  }\PYG{c+c1}{\PYGZsh{} Calculates standard deviation}
\end{sphinxVerbatim}

\sphinxAtStartPar
\sphinxstylestrong{See Also:}

\sphinxAtStartPar
\sphinxcode{\sphinxupquote{Mean()}}, \sphinxcode{\sphinxupquote{Median()}}, \sphinxcode{\sphinxupquote{Min()}}, \sphinxcode{\sphinxupquote{Max()}}, \sphinxcode{\sphinxupquote{Sum()}}, \sphinxcode{\sphinxupquote{SummaryStats()}}

\sphinxstepscope


\section{UI Library \sphinxhyphen{} User Interface}
\label{\detokenize{reference/ui:ui-library-user-interface}}\label{\detokenize{reference/ui::doc}}
\sphinxAtStartPar
This library contains functions for creating user interface elements like buttons, textboxes, and checkboxes.

\begin{sphinxShadowBox}
\sphinxstyletopictitle{Function Index}
\begin{itemize}
\item {} 
\sphinxAtStartPar
\phantomsection\label{\detokenize{reference/ui:id1}}{\hyperref[\detokenize{reference/ui:clearscrollboxthumbcapture}]{\sphinxcrossref{ClearScrollboxThumbCapture()}}}

\item {} 
\sphinxAtStartPar
\phantomsection\label{\detokenize{reference/ui:id2}}{\hyperref[\detokenize{reference/ui:clickonmenu}]{\sphinxcrossref{ClickOnMenu()}}}

\item {} 
\sphinxAtStartPar
\phantomsection\label{\detokenize{reference/ui:id3}}{\hyperref[\detokenize{reference/ui:clickonscrollbox}]{\sphinxcrossref{ClickOnScrollbox()}}}

\item {} 
\sphinxAtStartPar
\phantomsection\label{\detokenize{reference/ui:id4}}{\hyperref[\detokenize{reference/ui:drawpulldown}]{\sphinxcrossref{DrawPulldown()}}}

\item {} 
\sphinxAtStartPar
\phantomsection\label{\detokenize{reference/ui:id5}}{\hyperref[\detokenize{reference/ui:drawscrollingtextbox}]{\sphinxcrossref{DrawScrollingTextBox()}}}

\item {} 
\sphinxAtStartPar
\phantomsection\label{\detokenize{reference/ui:id6}}{\hyperref[\detokenize{reference/ui:editscrollboxvalue}]{\sphinxcrossref{EditScrollboxValue()}}}

\item {} 
\sphinxAtStartPar
\phantomsection\label{\detokenize{reference/ui:id7}}{\hyperref[\detokenize{reference/ui:getfulllinebreaks}]{\sphinxcrossref{GetFullLineBreaks()}}}

\item {} 
\sphinxAtStartPar
\phantomsection\label{\detokenize{reference/ui:id8}}{\hyperref[\detokenize{reference/ui:insidemenu}]{\sphinxcrossref{InsideMenu()}}}

\item {} 
\sphinxAtStartPar
\phantomsection\label{\detokenize{reference/ui:id9}}{\hyperref[\detokenize{reference/ui:insidetb}]{\sphinxcrossref{InsideTB()}}}

\item {} 
\sphinxAtStartPar
\phantomsection\label{\detokenize{reference/ui:id10}}{\hyperref[\detokenize{reference/ui:makebutton}]{\sphinxcrossref{MakeButton()}}}

\item {} 
\sphinxAtStartPar
\phantomsection\label{\detokenize{reference/ui:id11}}{\hyperref[\detokenize{reference/ui:makemenu}]{\sphinxcrossref{MakeMenu()}}}

\item {} 
\sphinxAtStartPar
\phantomsection\label{\detokenize{reference/ui:id12}}{\hyperref[\detokenize{reference/ui:makemenuitem}]{\sphinxcrossref{MakeMenuItem()}}}

\item {} 
\sphinxAtStartPar
\phantomsection\label{\detokenize{reference/ui:id13}}{\hyperref[\detokenize{reference/ui:makescrollbox}]{\sphinxcrossref{MakeScrollBox()}}}

\item {} 
\sphinxAtStartPar
\phantomsection\label{\detokenize{reference/ui:id14}}{\hyperref[\detokenize{reference/ui:makescrollingtextbox}]{\sphinxcrossref{MakeScrollingTextBox()}}}

\item {} 
\sphinxAtStartPar
\phantomsection\label{\detokenize{reference/ui:id15}}{\hyperref[\detokenize{reference/ui:maketextlist}]{\sphinxcrossref{MakeTextList()}}}

\item {} 
\sphinxAtStartPar
\phantomsection\label{\detokenize{reference/ui:id16}}{\hyperref[\detokenize{reference/ui:opensubmenus}]{\sphinxcrossref{OpenSubMenus()}}}

\item {} 
\sphinxAtStartPar
\phantomsection\label{\detokenize{reference/ui:id17}}{\hyperref[\detokenize{reference/ui:popupentrybox}]{\sphinxcrossref{PopUpEntryBox()}}}

\item {} 
\sphinxAtStartPar
\phantomsection\label{\detokenize{reference/ui:id18}}{\hyperref[\detokenize{reference/ui:pushbutton}]{\sphinxcrossref{PushButton()}}}

\item {} 
\sphinxAtStartPar
\phantomsection\label{\detokenize{reference/ui:id19}}{\hyperref[\detokenize{reference/ui:setscrollingtext}]{\sphinxcrossref{SetScrollingText()}}}

\item {} 
\sphinxAtStartPar
\phantomsection\label{\detokenize{reference/ui:id20}}{\hyperref[\detokenize{reference/ui:settextboxcursorfromclick}]{\sphinxcrossref{SetTextBoxCursorFromClick()}}}

\item {} 
\sphinxAtStartPar
\phantomsection\label{\detokenize{reference/ui:id21}}{\hyperref[\detokenize{reference/ui:sortdir}]{\sphinxcrossref{SortDir()}}}

\item {} 
\sphinxAtStartPar
\phantomsection\label{\detokenize{reference/ui:id22}}{\hyperref[\detokenize{reference/ui:updatepulldown}]{\sphinxcrossref{UpdatePulldown()}}}

\item {} 
\sphinxAtStartPar
\phantomsection\label{\detokenize{reference/ui:id23}}{\hyperref[\detokenize{reference/ui:updatescrollbox}]{\sphinxcrossref{UpdateScrollbox()}}}

\end{itemize}
\end{sphinxShadowBox}

\index{ClearScrollboxThumbCapture@\spxentry{ClearScrollboxThumbCapture}}\ignorespaces 

\subsection{ClearScrollboxThumbCapture()}
\label{\detokenize{reference/ui:clearscrollboxthumbcapture}}\label{\detokenize{reference/ui:index-0}}
\sphinxAtStartPar
\sphinxstylestrong{Description:}

\sphinxAtStartPar
this supposedly gets called on a mouse release event?
it should also get called when you leave the window…

\sphinxAtStartPar
\sphinxstylestrong{Usage:}

\begin{sphinxVerbatim}[commandchars=\\\{\}]
\PYG{k}{define}\PYG{+w}{ }\PYG{n+nf}{ClearScrollboxThumbCapture}\PYG{p}{(}\PYG{n+nv}{obj}\PYG{p}{,}\PYG{n+nv}{p}\PYG{p}{,}\PYG{n+nv}{event}\PYG{o}{:}\PYG{l+m+mi}{0}\PYG{p}{)}\PYG{c+c1}{\PYGZsh{}+}
\end{sphinxVerbatim}

\index{ClickOnMenu@\spxentry{ClickOnMenu}}\ignorespaces 

\subsection{ClickOnMenu()}
\label{\detokenize{reference/ui:clickonmenu}}\label{\detokenize{reference/ui:index-1}}
\sphinxAtStartPar
\sphinxstyleemphasis{Handles menu click, calling the .clickon function of menu.}

\sphinxAtStartPar
\sphinxstylestrong{Description:}

\sphinxAtStartPar
Handles clicking on a menu item. It will call the .clickon property of that item, and then hide the menu.

\sphinxAtStartPar
\sphinxstylestrong{Usage:}

\begin{sphinxVerbatim}[commandchars=\\\{\}]
\PYG{k}{define}\PYG{+w}{ }\PYG{n+nf}{ClickOnMenu}\PYG{p}{(}\PYG{p}{.}\PYG{p}{.}\PYG{p}{.}\PYG{p}{)}
\end{sphinxVerbatim}

\sphinxAtStartPar
\sphinxstylestrong{Example:}

\begin{sphinxVerbatim}[commandchars=\\\{\}]
T\PYG{n+nv}{his}\PYG{+w}{ }\PYG{n+nv}{creates}\PYG{+w}{ }\PYG{n+nv}{a}\PYG{+w}{ }\PYG{n+nv}{menu}\PYG{+w}{ }\PYG{k}{and}\PYG{+w}{ }\PYG{n+nv}{awaits}\PYG{+w}{ }\PYG{n+nv}{clicking}\PYG{+w}{ }\PYG{n+nv}{on}\PYG{p}{.}\PYG{+w}{  }M\PYG{n+nv}{ore}\PYG{+w}{ }\PYG{n+nv}{complete}\PYG{+w}{ }\PYG{n+nv}{examples}\PYG{+w}{ }\PYG{n+nv}{are}\PYG{+w}{ }\PYG{n+nv}{available}\PYG{+w}{ }\PYG{n+nv}{in}\PYG{+w}{ }\PYG{n+nv}{ui.pbl}\PYG{p}{.}\PYG{+w}{  }I\PYG{n+nv}{t}\PYG{+w}{ }\PYG{n+nv}{requires}\PYG{+w}{ }\PYG{n+nv}{that}\PYG{+w}{ }M\PYG{n+nv}{yMessage}\PYG{+w}{ }\PYG{n+nv}{is}\PYG{+w}{ }\PYG{n+nv}{created}\PYG{+w}{ }\PYG{n+nv}{somewhere}


\PYG{+w}{   }\PYG{n+nv}{menu1}\PYG{+w}{ }\PYG{o}{\PYGZlt{}\PYGZhy{}}\PYG{+w}{ }\PYG{n+nf}{MakeMenuItem}\PYG{p}{(}\PYG{l+s+s2}{\PYGZdq{}File\PYGZdq{}}\PYG{p}{,}\PYG{l+m+mi}{0}\PYG{p}{,}\PYG{l+m+mi}{0}\PYG{p}{,}\PYG{n+nv+vg}{gWin}\PYG{p}{,}\PYG{l+m+mi}{14}\PYG{p}{,}\PYG{l+m+mi}{10}\PYG{p}{,}\PYG{l+s+s2}{\PYGZdq{}MYMESSAGE\PYGZdq{}}\PYG{p}{)}


\PYG{+w}{   }\PYG{n+nv}{menu2}\PYG{o}{\PYGZlt{}\PYGZhy{}}\PYG{+w}{ }\PYG{n+nf}{MakeMenu}\PYG{p}{(}\PYG{l+s+s2}{\PYGZdq{}Edit\PYGZdq{}}\PYG{p}{,}\PYG{l+m+mi}{70}\PYG{p}{,}\PYG{l+m+mi}{0}\PYG{p}{,}\PYG{n+nv+vg}{gWin}\PYG{p}{,}\PYG{l+m+mi}{14}\PYG{p}{,}\PYG{l+m+mi}{10}\PYG{p}{,}\PYG{+w}{ }\PYG{l+s+s2}{\PYGZdq{}MYMESSAGE\PYGZdq{}}\PYG{p}{)}

\PYG{+w}{   }\PYG{n+nv}{menus}\PYG{+w}{ }\PYG{o}{\PYGZlt{}\PYGZhy{}}\PYG{+w}{ }\PYG{p}{[}\PYG{n+nv}{menu1}\PYG{p}{,}\PYG{n+nv}{menu2}\PYG{p}{]}
\PYG{+w}{   }\PYG{n+nv}{opt}\PYG{+w}{ }\PYG{o}{\PYGZlt{}\PYGZhy{}}\PYG{+w}{ }\PYG{n+nf}{WaitForClickOntarget}\PYG{p}{(}\PYG{n+nv}{menu}\PYG{p}{,}\PYG{p}{[}\PYG{l+m+mi}{1}\PYG{p}{,}\PYG{l+m+mi}{2}\PYG{p}{]}\PYG{p}{)}
\PYG{+w}{   }\PYG{n+nf}{ClickOnMenu}\PYG{p}{(}\PYG{n+nf}{Nth}\PYG{p}{(}\PYG{n+nv}{menus}\PYG{p}{,}\PYG{n+nv}{opt}\PYG{p}{)}\PYG{p}{,}\PYG{n+nv+vg}{gClick}\PYG{p}{)}
\end{sphinxVerbatim}

\sphinxAtStartPar
\sphinxstylestrong{See Also:}

\sphinxAtStartPar
\sphinxcode{\sphinxupquote{MakeMenu()}}, \sphinxcode{\sphinxupquote{OpenSubMenus()}}, \sphinxcode{\sphinxupquote{MakeMenuItem()}}

\index{ClickOnScrollbox@\spxentry{ClickOnScrollbox}}\ignorespaces 

\subsection{ClickOnScrollbox()}
\label{\detokenize{reference/ui:clickonscrollbox}}\label{\detokenize{reference/ui:index-2}}
\sphinxAtStartPar
\sphinxstyleemphasis{Handles click on scrollbox.}

\sphinxAtStartPar
\sphinxstylestrong{Description:}

\sphinxAtStartPar
Handles a click event on the a  \sphinxcode{\sphinxupquote{ScrollBox}}. This should be called after one checks (e.g., via InsideTB) whether the scrollbox was actually clicked on.  It will handle scrolling, moving via the thumb, up/down arrows, and reselection. It is also used to interact with \sphinxcode{\sphinxupquote{ScrollingTextBox}}  objects. This function name is bound to the .clickon property of scrollboxes, so it can be called using CallFunction (see example below).

\sphinxAtStartPar
\sphinxstylestrong{Usage:}

\begin{sphinxVerbatim}[commandchars=\\\{\}]
\PYG{k}{define}\PYG{+w}{ }\PYG{n+nf}{ClickOnScrollbox}\PYG{p}{(}\PYG{p}{.}\PYG{p}{.}\PYG{p}{.}\PYG{p}{)}
\end{sphinxVerbatim}

\sphinxAtStartPar
\sphinxstylestrong{Example:}

\begin{sphinxVerbatim}[commandchars=\\\{\}]
S\PYG{n+nv}{ee}\PYG{+w}{ }\PYG{n+nv}{ui.pbl}\PYG{+w}{ }\PYG{n+nv}{in}\PYG{+w}{ }\PYG{n+nv}{the}\PYG{+w}{ }\PYG{n+nv}{demo}\PYG{+w}{ }\PYG{n+nv}{directory}\PYG{+w}{ }\PYG{n+nv}{for}\PYG{+w}{ }\PYG{n+nv}{examples}\PYG{+w}{ }\PYG{n+nv}{of}\PYG{+w}{ }\PYG{n+nv}{the}\PYG{+w}{ }\PYG{n+nv}{use}\PYG{+w}{ }\PYG{n+nv}{of}\PYG{+w}{ }\PYG{n+nv}{a}\PYG{+w}{ }\PYG{n+nv}{scrolling}\PYG{+w}{ }\PYG{n+nv}{text}\PYG{+w}{ }\PYG{n+nv}{box}\PYG{p}{.}\PYG{+w}{  }A\PYG{+w}{ }\PYG{n+nv}{brief}\PYG{+w}{ }\PYG{n+nv}{example}\PYG{+w}{ }\PYG{n+nv}{follows}\PYG{o}{:}


\PYG{+w}{   }\PYG{n+nv}{sb}\PYG{+w}{ }\PYG{o}{\PYGZlt{}\PYGZhy{}}\PYG{+w}{ }\PYG{n+nf}{MakeScrollBox}\PYG{p}{(}\PYG{n+nf}{Sequence}\PYG{p}{(}\PYG{l+m+mi}{1}\PYG{p}{,}\PYG{l+m+mi}{50}\PYG{p}{,}\PYG{l+m+mi}{1}\PYG{p}{)}\PYG{p}{,}\PYG{l+s+s2}{\PYGZdq{}The numbers\PYGZdq{}}\PYG{p}{,}\PYG{l+m+mi}{40}\PYG{p}{,}\PYG{l+m+mi}{40}\PYG{p}{,}\PYG{n+nv+vg}{gWin}\PYG{p}{,}\PYG{l+m+mi}{12}\PYG{p}{,}\PYG{l+m+mi}{150}\PYG{p}{,}\PYG{l+m+mi}{500}\PYG{p}{,}\PYG{l+m+mi}{3}\PYG{p}{)}
\PYG{+w}{   }\PYG{n+nf}{Draw}\PYG{p}{(}\PYG{p}{)}

\PYG{+w}{   }\PYG{n+nv}{resp}\PYG{+w}{ }\PYG{o}{\PYGZlt{}\PYGZhy{}}\PYG{+w}{ }\PYG{n+nf}{WaitForClickOntarget}\PYG{p}{(}\PYG{p}{[}\PYG{n+nv}{sb}\PYG{p}{]}\PYG{p}{,}\PYG{p}{[}\PYG{l+m+mi}{1}\PYG{p}{]}\PYG{p}{)}
\PYG{+w}{    }\PYG{n+nf}{ClickOnScrollbox}\PYG{p}{(}\PYG{n+nv}{sb}\PYG{p}{,}\PYG{n+nv+vg}{gClick}\PYG{p}{)}

\PYG{+w}{    }\PYG{c+c1}{\PYGZsh{}Alternately:   CallFunction(sb.clickon,[sb,gClick])}


\PYG{+w}{   }\PYG{c+c1}{\PYGZsh{}\PYGZsh{}change the selected items}
\PYG{+w}{   }\PYG{n+nv}{sb.list}\PYG{+w}{ }\PYG{o}{\PYGZlt{}\PYGZhy{}}\PYG{+w}{ }\PYG{n+nf}{Sequence}\PYG{p}{(}\PYG{n+nv}{sb.selected}\PYG{p}{,}\PYG{n+nv}{sb.selected}\PYG{o}{+}\PYG{l+m+mi}{50}\PYG{p}{,}\PYG{l+m+mi}{1}\PYG{p}{)}
\PYG{+w}{   }\PYG{n+nf}{UpdateScrollbox}\PYG{p}{(}\PYG{n+nv}{sb}\PYG{p}{)}
\PYG{+w}{   }\PYG{n+nf}{DrawScrollbox}\PYG{p}{(}\PYG{n+nv}{sb}\PYG{p}{)}
\PYG{+w}{   }\PYG{n+nf}{Draw}\PYG{p}{(}\PYG{p}{)}
\end{sphinxVerbatim}

\sphinxAtStartPar
\sphinxstylestrong{See Also:}

\sphinxAtStartPar
\sphinxcode{\sphinxupquote{MakeScrollingTextBox}}
\sphinxcode{\sphinxupquote{MakeScrollBox}}
\sphinxcode{\sphinxupquote{UpdateScrollBox}}
\sphinxcode{\sphinxupquote{DrawScrollBox}}

\index{DrawPulldown@\spxentry{DrawPulldown}}\ignorespaces 

\subsection{DrawPulldown()}
\label{\detokenize{reference/ui:drawpulldown}}\label{\detokenize{reference/ui:index-3}}
\sphinxAtStartPar
\sphinxstyleemphasis{Redraws a pulldonw if state changes.}

\sphinxAtStartPar
\sphinxstylestrong{Description:}

\sphinxAtStartPar
This handles layout/drawing of a pulldown box. This does not actually call Draw() on the window, and so an additional draw command is needed before the output is displayed.  The main use case for this function is if you need to manually change the selected object (by changing .selected). This will redraw the pulldown with the new selection.

\sphinxAtStartPar
\sphinxstylestrong{Usage:}

\begin{sphinxVerbatim}[commandchars=\\\{\}]
\PYG{k}{define}\PYG{+w}{ }\PYG{n+nf}{DrawPulldown}\PYG{p}{(}\PYG{p}{.}\PYG{p}{.}\PYG{p}{.}\PYG{p}{)}
\end{sphinxVerbatim}

\sphinxAtStartPar
\sphinxstylestrong{Example:}

\begin{sphinxVerbatim}[commandchars=\\\{\}]
\PYG{n+nv}{options}\PYG{+w}{  }\PYG{o}{\PYGZlt{}\PYGZhy{}}\PYG{+w}{ }\PYG{n+nf}{MakePulldownButton}\PYG{p}{(}\PYG{p}{[}\PYG{l+s+s2}{\PYGZdq{}A\PYGZdq{}}\PYG{p}{,}B\PYG{l+s+s2}{\PYGZdq{},\PYGZdq{}}C\PYGZdq{}\PYG{p}{]}\PYG{p}{,}\PYG{l+m+mi}{400}\PYG{p}{,}\PYG{l+m+mi}{250}\PYG{p}{,}\PYG{n+nv+vg}{gWin}\PYG{p}{,}\PYG{l+m+mi}{14}\PYG{p}{,}\PYG{l+m+mi}{100}\PYG{p}{,}\PYG{l+m+mi}{1}\PYG{p}{)}
\PYG{n+nf}{Draw}\PYG{p}{(}\PYG{p}{)}
\PYG{n+nf}{WaitForAnyKeyPress}\PYG{p}{(}\PYG{p}{)}
\PYG{n+nv}{options.selected}\PYG{+w}{ }\PYG{o}{\PYGZlt{}\PYGZhy{}}\PYG{+w}{ }\PYG{l+m+mi}{2}
\PYG{n+nf}{DrawPulldown}\PYG{p}{(}\PYG{n+nv}{options}\PYG{p}{)}
\PYG{n+nf}{Draw}\PYG{p}{(}\PYG{p}{)}
\PYG{n+nf}{WaitForAnyKeyPress}\PYG{p}{(}\PYG{p}{)}
\end{sphinxVerbatim}

\sphinxAtStartPar
\sphinxstylestrong{See Also:}

\sphinxAtStartPar
\sphinxcode{\sphinxupquote{MakePullDown()}}, \sphinxcode{\sphinxupquote{Pulldown()}}, \sphinxcode{\sphinxupquote{UpdatePulldown()}}

\index{DrawScrollingTextBox@\spxentry{DrawScrollingTextBox}}\ignorespaces 

\subsection{DrawScrollingTextBox()}
\label{\detokenize{reference/ui:drawscrollingtextbox}}\label{\detokenize{reference/ui:index-4}}
\sphinxAtStartPar
\sphinxstylestrong{Description:}

\sphinxAtStartPar
this draws the current state of the scrollbox.
It should be called directly whenever things like the scrollbar,
offset, selected item are changed, but not when the list changes.
the only material side effect it can have is changing selected, which will update
to ensure it stays within bounds.

\sphinxAtStartPar
\sphinxstylestrong{Usage:}

\begin{sphinxVerbatim}[commandchars=\\\{\}]
\PYG{k}{define}\PYG{+w}{ }\PYG{n+nf}{DrawScrollingTextBox}\PYG{p}{(}\PYG{n+nv}{obj}\PYG{p}{)}
\end{sphinxVerbatim}

\index{EditScrollboxValue@\spxentry{EditScrollboxValue}}\ignorespaces 

\subsection{EditScrollboxValue()}
\label{\detokenize{reference/ui:editscrollboxvalue}}\label{\detokenize{reference/ui:index-5}}
\sphinxAtStartPar
\sphinxstylestrong{Description:}

\sphinxAtStartPar
make this separate so you can override for more custom edits.
see launcher experiment chain, where the chain is just a set of labels
that link to the ‘real’ chain.

\sphinxAtStartPar
\sphinxstylestrong{Usage:}

\begin{sphinxVerbatim}[commandchars=\\\{\}]
\PYG{k}{define}\PYG{+w}{ }\PYG{n+nf}{EditScrollboxValue}\PYG{p}{(}\PYG{n+nv}{win}\PYG{p}{,}\PYG{n+nv}{click}\PYG{p}{,}\PYG{n+nv}{default}\PYG{p}{,}\PYG{n+nv}{selected}\PYG{p}{)}
\end{sphinxVerbatim}

\index{GetFullLineBreaks@\spxentry{GetFullLineBreaks}}\ignorespaces 

\subsection{GetFullLineBreaks()}
\label{\detokenize{reference/ui:getfulllinebreaks}}\label{\detokenize{reference/ui:index-6}}
\sphinxAtStartPar
\sphinxstylestrong{Description:}

\sphinxAtStartPar
this attempts to get the full set of linebreaks from the
text attached to tb

\sphinxAtStartPar
\sphinxstylestrong{Usage:}

\begin{sphinxVerbatim}[commandchars=\\\{\}]
\PYG{k}{define}\PYG{+w}{ }\PYG{n+nf}{GetFullLineBreaks}\PYG{p}{(}\PYG{n+nv}{tb}\PYG{p}{,}\PYG{n+nv}{text}\PYG{p}{)}
\end{sphinxVerbatim}

\index{InsideMenu@\spxentry{InsideMenu}}\ignorespaces 

\subsection{InsideMenu()}
\label{\detokenize{reference/ui:insidemenu}}\label{\detokenize{reference/ui:index-7}}
\sphinxAtStartPar
\sphinxstylestrong{Description:}

\sphinxAtStartPar
This is offset from upper left corner

\sphinxAtStartPar
\sphinxstylestrong{Usage:}

\begin{sphinxVerbatim}[commandchars=\\\{\}]
\PYG{k}{define}\PYG{+w}{ }\PYG{n+nf}{InsideMenu}\PYG{p}{(}\PYG{n+nv}{xy}\PYG{p}{,}\PYG{n+nv}{object}\PYG{p}{)}\PYG{c+c1}{\PYGZsh{}+}
\end{sphinxVerbatim}

\index{InsideTB@\spxentry{InsideTB}}\ignorespaces 

\subsection{InsideTB()}
\label{\detokenize{reference/ui:insidetb}}\label{\detokenize{reference/ui:index-8}}
\sphinxAtStartPar
\sphinxstyleemphasis{Determine inside for a textbox\sphinxhyphen{}style object (location is upper left)}

\sphinxAtStartPar
\sphinxstylestrong{Description:}

\sphinxAtStartPar
Determines whether an \sphinxcode{\sphinxupquote{{[}x,y{]}}} point is inside an object having .x, .y, .width, and .height properties, with .x and .y representing the upper left corner of the object.  This is bound to the .inside property of many custom ui objects.  The \sphinxcode{\sphinxupquote{Inside}} function will use the function bound to the .inside property for any custom object having that property, and so this function’s use is mainly hidden from users.

\sphinxAtStartPar
\sphinxstylestrong{Usage:}

\begin{sphinxVerbatim}[commandchars=\\\{\}]
\PYG{k}{define}\PYG{+w}{ }\PYG{n+nf}{InsideTB}\PYG{p}{(}\PYG{p}{.}\PYG{p}{.}\PYG{p}{.}\PYG{p}{)}
\end{sphinxVerbatim}

\sphinxAtStartPar
\sphinxstylestrong{Example:}

\begin{sphinxVerbatim}[commandchars=\\\{\}]
\PYG{n+nv}{pulldown}\PYG{+w}{ }\PYG{o}{\PYGZlt{}\PYGZhy{}}\PYG{+w}{ }\PYG{n+nf}{MakePulldown}\PYG{p}{(}\PYG{p}{[}\PYG{l+s+s2}{\PYGZdq{}one\PYGZdq{}}\PYG{p}{,}\PYG{l+s+s2}{\PYGZdq{}two\PYGZdq{}}\PYG{p}{,}\PYG{l+s+s2}{\PYGZdq{}three\PYGZdq{}}\PYG{p}{,}\PYG{l+s+s2}{\PYGZdq{}four\PYGZdq{}}\PYG{p}{]}\PYG{p}{,}\PYG{l+m+mi}{400}\PYG{o}{\PYGZhy{}}\PYG{l+m+mi}{75}\PYG{p}{,}\PYG{l+m+mi}{300}\PYG{p}{,}\PYG{n+nv+vg}{gWin}\PYG{p}{,}\PYG{l+m+mi}{12}\PYG{p}{,}\PYG{l+m+mi}{150}\PYG{p}{,}\PYG{l+m+mi}{1}\PYG{p}{)}

\PYG{k}{if}\PYG{p}{(}\PYG{n+nf}{InsideTB}\PYG{p}{(}\PYG{p}{[}\PYG{l+m+mi}{300}\PYG{p}{,}\PYG{l+m+mi}{300}\PYG{p}{]}\PYG{p}{,}\PYG{n+nv}{pulldown}\PYG{p}{)}\PYG{p}{)}
\PYG{+w}{ }\PYG{p}{\PYGZob{}}
\PYG{+w}{   }\PYG{n+nf}{Print}\PYG{p}{(}\PYG{l+s+s2}{\PYGZdq{}INSIDE\PYGZdq{}}\PYG{p}{)}
\PYG{+w}{  }\PYG{p}{\PYGZcb{}}
\end{sphinxVerbatim}

\sphinxAtStartPar
\sphinxstylestrong{See Also:}

\sphinxAtStartPar
\sphinxcode{\sphinxupquote{Inside()}}, \sphinxcode{\sphinxupquote{MoveObject()}}, \sphinxcode{\sphinxupquote{ClickOn()}}, \sphinxcode{\sphinxupquote{DrawObject()}}

\index{MakeButton@\spxentry{MakeButton}}\ignorespaces 

\subsection{MakeButton()}
\label{\detokenize{reference/ui:makebutton}}\label{\detokenize{reference/ui:index-9}}
\sphinxAtStartPar
\sphinxstyleemphasis{Makes a button for clicking on.}

\sphinxAtStartPar
\sphinxstylestrong{Description:}

\sphinxAtStartPar
Creates a button on a window that can be clicked and launches actions. The button is always 20 pixels high (using images in media images), with a rounded grey background.  The label text will be shrunk to fit the width, although this should be avoided as it can look strange. A button is a custom object made from images and text. It has a property ‘clickon’ that is bound to ‘PushButton’  A button will look like this: \sphinxstylestrong{Usage:}

\begin{sphinxVerbatim}[commandchars=\\\{\}]
\PYG{k}{define}\PYG{+w}{ }\PYG{n+nf}{MakeButton}\PYG{p}{(}\PYG{p}{.}\PYG{p}{.}\PYG{p}{.}\PYG{p}{)}
\end{sphinxVerbatim}

\sphinxAtStartPar
\sphinxstylestrong{Example:}

\begin{sphinxVerbatim}[commandchars=\\\{\}]
T\PYG{n+nv}{he}\PYG{+w}{ }\PYG{n+nv}{following}\PYG{+w}{ }\PYG{n+nv}{creates}\PYG{+w}{ }\PYG{n+nv}{a}\PYG{+w}{ }\PYG{n+nv}{button}\PYG{p}{,}\PYG{+w}{ }\PYG{n+nv}{waits}\PYG{+w}{ }\PYG{n+nv}{for}\PYG{+w}{ }\PYG{n+nv}{you}\PYG{+w}{ }\PYG{n+nv}{to}\PYG{+w}{ }\PYG{n+nv}{click}\PYG{+w}{ }\PYG{n+nv}{on}\PYG{+w}{ }\PYG{n+nv}{it}\PYG{p}{,}\PYG{+w}{ }\PYG{k}{and}\PYG{+w}{ }\PYG{n+nv}{animates}\PYG{+w}{ }\PYG{n+nv}{a}\PYG{+w}{ }\PYG{n+nv}{button}\PYG{+w}{ }\PYG{n+nv}{press}


\PYG{+w}{ }\PYG{n+nv}{done}\PYG{+w}{ }\PYG{o}{\PYGZlt{}\PYGZhy{}}\PYG{+w}{ }\PYG{n+nf}{MakeButton}\PYG{p}{(}\PYG{l+s+s2}{\PYGZdq{}QUIT\PYGZdq{}}\PYG{p}{,}\PYG{l+m+mi}{400}\PYG{p}{,}\PYG{l+m+mi}{250}\PYG{p}{,}\PYG{n+nv+vg}{gWin}\PYG{p}{,}\PYG{l+m+mi}{150}\PYG{p}{)}
\PYG{+w}{ }\PYG{n+nv}{resp}\PYG{+w}{ }\PYG{o}{\PYGZlt{}\PYGZhy{}}\PYG{+w}{ }\PYG{n+nf}{WaitForClickOntarget}\PYG{p}{(}\PYG{p}{[}\PYG{n+nv}{done}\PYG{p}{]}\PYG{p}{,}\PYG{p}{[}\PYG{l+m+mi}{1}\PYG{p}{]}\PYG{p}{)}
\PYG{+w}{ }\PYG{n+nf}{CallFunction}\PYG{p}{(}\PYG{n+nv}{done.clickon}\PYG{p}{,}\PYG{p}{[}\PYG{n+nv}{done}\PYG{p}{,}\PYG{n+nv+vg}{gClick}\PYG{p}{]}\PYG{p}{)}
\end{sphinxVerbatim}

\sphinxAtStartPar
\sphinxstylestrong{See Also:}

\sphinxAtStartPar
\sphinxcode{\sphinxupquote{PushButton()}}, \sphinxcode{\sphinxupquote{MakeCheckBox()}}

\index{MakeMenu@\spxentry{MakeMenu}}\ignorespaces 

\subsection{MakeMenu()}
\label{\detokenize{reference/ui:makemenu}}\label{\detokenize{reference/ui:index-10}}
\sphinxAtStartPar
\sphinxstyleemphasis{Creates menu with suboptions.}

\sphinxAtStartPar
\sphinxstylestrong{Description:}

\sphinxAtStartPar
Creates a menu containing multiple menu items, that automatically call functions specified by the command.

\sphinxAtStartPar
\sphinxstylestrong{Usage:}

\begin{sphinxVerbatim}[commandchars=\\\{\}]
\PYG{k}{define}\PYG{+w}{ }\PYG{n+nf}{MakeMenu}\PYG{p}{(}\PYG{p}{.}\PYG{p}{.}\PYG{p}{.}\PYG{p}{)}
\end{sphinxVerbatim}

\sphinxAtStartPar
\sphinxstylestrong{Example:}

\begin{sphinxVerbatim}[commandchars=\\\{\}]
T\PYG{n+nv}{his}\PYG{+w}{ }\PYG{n+nv}{creates}\PYG{+w}{ }\PYG{n+nv}{a}\PYG{+w}{ }\PYG{n+nv}{menu}\PYG{+w}{ }\PYG{k}{and}\PYG{+w}{ }\PYG{n+nv}{awaits}\PYG{+w}{ }\PYG{n+nv}{clicking}\PYG{+w}{ }\PYG{n+nv}{on}\PYG{p}{.}\PYG{+w}{  }M\PYG{n+nv}{ore}\PYG{+w}{ }\PYG{n+nv}{complete}\PYG{+w}{ }\PYG{n+nv}{examples}\PYG{+w}{ }\PYG{n+nv}{are}\PYG{+w}{ }\PYG{n+nv}{available}\PYG{+w}{ }\PYG{n+nv}{in}\PYG{+w}{ }\PYG{n+nv}{ui.pbl}\PYG{p}{.}\PYG{+w}{  }I\PYG{n+nv}{t}\PYG{+w}{ }\PYG{n+nv}{requires}\PYG{+w}{ }\PYG{n+nv}{that}\PYG{+w}{ }M\PYG{n+nv}{yMessage}\PYG{+w}{ }\PYG{n+nv}{is}\PYG{+w}{ }\PYG{n+nv}{created}\PYG{+w}{ }\PYG{n+nv}{somewhere}


\PYG{+w}{   }\PYG{n+nv}{menu1}\PYG{+w}{ }\PYG{o}{\PYGZlt{}\PYGZhy{}}\PYG{+w}{ }\PYG{n+nf}{MakeMenu}\PYG{p}{(}\PYG{l+s+s2}{\PYGZdq{}File\PYGZdq{}}\PYG{p}{,}\PYG{l+m+mi}{0}\PYG{p}{,}\PYG{l+m+mi}{0}\PYG{p}{,}\PYG{n+nv+vg}{gWin}\PYG{p}{,}\PYG{l+m+mi}{14}\PYG{p}{,}\PYG{l+m+mi}{10}\PYG{p}{,}
\PYG{+w}{              }\PYG{p}{[}\PYG{l+s+s2}{\PYGZdq{}Open\PYGZdq{}}\PYG{p}{,}\PYG{l+s+s2}{\PYGZdq{}Save\PYGZdq{}}\PYG{p}{,}\PYG{l+s+s2}{\PYGZdq{}Save as\PYGZdq{}}\PYG{p}{,}\PYG{l+s+s2}{\PYGZdq{}Quit\PYGZdq{}}\PYG{p}{]}\PYG{p}{,}
\PYG{+w}{              }\PYG{p}{[}\PYG{l+s+s2}{\PYGZdq{}MYMESSAGE\PYGZdq{}}\PYG{p}{,}\PYG{l+s+s2}{\PYGZdq{}MYMESSAGE\PYGZdq{}}\PYG{p}{,}\PYG{l+s+s2}{\PYGZdq{}MYMESSAGE\PYGZdq{}}\PYG{p}{,}\PYG{l+s+s2}{\PYGZdq{}MYMESSAGE\PYGZdq{}}\PYG{p}{]}\PYG{p}{)}


\PYG{+w}{   }\PYG{n+nv}{menu2}\PYG{o}{\PYGZlt{}\PYGZhy{}}\PYG{+w}{ }\PYG{n+nf}{MakeMenu}\PYG{p}{(}\PYG{l+s+s2}{\PYGZdq{}Edit\PYGZdq{}}\PYG{p}{,}\PYG{l+m+mi}{70}\PYG{p}{,}\PYG{l+m+mi}{0}\PYG{p}{,}\PYG{n+nv+vg}{gWin}\PYG{p}{,}\PYG{l+m+mi}{14}\PYG{p}{,}\PYG{l+m+mi}{10}\PYG{p}{,}
\PYG{+w}{              }\PYG{p}{[}\PYG{l+s+s2}{\PYGZdq{}Cut\PYGZdq{}}\PYG{p}{,}\PYG{l+s+s2}{\PYGZdq{}Copy\PYGZdq{}}\PYG{p}{,}\PYG{l+s+s2}{\PYGZdq{}Paste\PYGZdq{}}\PYG{p}{,}\PYG{l+s+s2}{\PYGZdq{}Select\PYGZdq{}}\PYG{p}{]}\PYG{p}{,}
\PYG{+w}{              }\PYG{p}{[}\PYG{l+s+s2}{\PYGZdq{}MYMESSAGE\PYGZdq{}}\PYG{p}{,}\PYG{l+s+s2}{\PYGZdq{}MYMESSAGE\PYGZdq{}}\PYG{p}{,}\PYG{l+s+s2}{\PYGZdq{}MYMESSAGE\PYGZdq{}}\PYG{p}{,}\PYG{l+s+s2}{\PYGZdq{}MYMESSAGE\PYGZdq{}}\PYG{p}{]}\PYG{p}{)}

\PYG{+w}{   }\PYG{n+nv}{menu}\PYG{+w}{ }\PYG{o}{\PYGZlt{}\PYGZhy{}}\PYG{+w}{ }\PYG{p}{[}\PYG{n+nv}{menu1}\PYG{p}{,}\PYG{n+nv}{menu2}\PYG{p}{]}
\PYG{+w}{   }\PYG{n+nv}{opt}\PYG{+w}{ }\PYG{o}{\PYGZlt{}\PYGZhy{}}\PYG{+w}{ }\PYG{n+nf}{WaitForClickOntarget}\PYG{p}{(}\PYG{n+nv}{menu}\PYG{p}{,}\PYG{p}{[}\PYG{l+m+mi}{1}\PYG{p}{,}\PYG{l+m+mi}{2}\PYG{p}{]}\PYG{p}{)}
\PYG{+w}{   }\PYG{n+nf}{ClickOnMenu}\PYG{p}{(}\PYG{n+nf}{Nth}\PYG{p}{(}\PYG{n+nv}{menu}\PYG{p}{,}\PYG{n+nv}{opt}\PYG{p}{)}\PYG{p}{,}\PYG{n+nv+vg}{gClick}\PYG{p}{)}
\end{sphinxVerbatim}

\sphinxAtStartPar
\sphinxstylestrong{See Also:}

\sphinxAtStartPar
\sphinxcode{\sphinxupquote{MakeMenuItem()}}, \sphinxcode{\sphinxupquote{OpenSubMenus()}}, \sphinxcode{\sphinxupquote{ClickOnMenu()}}

\index{MakeMenuItem@\spxentry{MakeMenuItem}}\ignorespaces 

\subsection{MakeMenuItem()}
\label{\detokenize{reference/ui:makemenuitem}}\label{\detokenize{reference/ui:index-11}}
\sphinxAtStartPar
\sphinxstyleemphasis{Creates menu sub\sphinxhyphen{}item.}

\sphinxAtStartPar
\sphinxstylestrong{Description:}

\sphinxAtStartPar
Creates a single menu containing a label, whose .clickon property is bound to some other function.

\sphinxAtStartPar
\sphinxstylestrong{Usage:}

\begin{sphinxVerbatim}[commandchars=\\\{\}]
\PYG{k}{define}\PYG{+w}{ }\PYG{n+nf}{MakeMenuItem}\PYG{p}{(}\PYG{p}{.}\PYG{p}{.}\PYG{p}{.}\PYG{p}{)}
\end{sphinxVerbatim}

\sphinxAtStartPar
\sphinxstylestrong{Example:}

\begin{sphinxVerbatim}[commandchars=\\\{\}]
T\PYG{n+nv}{his}\PYG{+w}{ }\PYG{n+nv}{creates}\PYG{+w}{ }\PYG{n+nv}{a}\PYG{+w}{ }\PYG{n+nv}{menu}\PYG{+w}{ }\PYG{k}{and}\PYG{+w}{ }\PYG{n+nv}{awaits}\PYG{+w}{ }\PYG{n+nv}{clicking}\PYG{+w}{ }\PYG{n+nv}{on}\PYG{p}{.}\PYG{+w}{  }M\PYG{n+nv}{ore}\PYG{+w}{ }\PYG{n+nv}{complete}\PYG{+w}{ }\PYG{n+nv}{examples}\PYG{+w}{ }\PYG{n+nv}{are}\PYG{+w}{ }\PYG{n+nv}{available}\PYG{+w}{ }\PYG{n+nv}{in}\PYG{+w}{ }\PYG{n+nv}{ui.pbl}\PYG{p}{.}\PYG{+w}{  }I\PYG{n+nv}{t}\PYG{+w}{ }\PYG{n+nv}{requires}\PYG{+w}{ }\PYG{n+nv}{that}\PYG{+w}{ }M\PYG{n+nv}{yMessage}\PYG{+w}{ }\PYG{n+nv}{is}\PYG{+w}{ }\PYG{n+nv}{created}\PYG{+w}{ }\PYG{n+nv}{somewhere}


\PYG{+w}{   }\PYG{n+nv}{menu1}\PYG{+w}{ }\PYG{o}{\PYGZlt{}\PYGZhy{}}\PYG{+w}{ }\PYG{n+nf}{MakeMenuItem}\PYG{p}{(}\PYG{l+s+s2}{\PYGZdq{}File\PYGZdq{}}\PYG{p}{,}\PYG{l+m+mi}{0}\PYG{p}{,}\PYG{l+m+mi}{0}\PYG{p}{,}\PYG{n+nv+vg}{gWin}\PYG{p}{,}\PYG{l+m+mi}{14}\PYG{p}{,}\PYG{l+m+mi}{10}\PYG{p}{,}\PYG{l+s+s2}{\PYGZdq{}MYMESSAGE\PYGZdq{}}\PYG{p}{)}


\PYG{+w}{   }\PYG{n+nv}{menu2}\PYG{o}{\PYGZlt{}\PYGZhy{}}\PYG{+w}{ }\PYG{n+nf}{MakeMenu}\PYG{p}{(}\PYG{l+s+s2}{\PYGZdq{}Edit\PYGZdq{}}\PYG{p}{,}\PYG{l+m+mi}{70}\PYG{p}{,}\PYG{l+m+mi}{0}\PYG{p}{,}\PYG{n+nv+vg}{gWin}\PYG{p}{,}\PYG{l+m+mi}{14}\PYG{p}{,}\PYG{l+m+mi}{10}\PYG{p}{,}\PYG{+w}{ }\PYG{l+s+s2}{\PYGZdq{}MYMESSAGE\PYGZdq{}}\PYG{p}{)}

\PYG{+w}{   }\PYG{n+nv}{menus}\PYG{+w}{ }\PYG{o}{\PYGZlt{}\PYGZhy{}}\PYG{+w}{ }\PYG{p}{[}\PYG{n+nv}{menu1}\PYG{p}{,}\PYG{n+nv}{menu2}\PYG{p}{]}
\PYG{+w}{   }\PYG{n+nv}{opt}\PYG{+w}{ }\PYG{o}{\PYGZlt{}\PYGZhy{}}\PYG{+w}{ }\PYG{n+nf}{WaitForClickOntarget}\PYG{p}{(}\PYG{n+nv}{menu}\PYG{p}{,}\PYG{p}{[}\PYG{l+m+mi}{1}\PYG{p}{,}\PYG{l+m+mi}{2}\PYG{p}{]}\PYG{p}{)}
\PYG{+w}{   }\PYG{n+nf}{ClickOnMenu}\PYG{p}{(}\PYG{n+nf}{Nth}\PYG{p}{(}\PYG{n+nv}{menus}\PYG{p}{,}\PYG{n+nv}{opt}\PYG{p}{)}\PYG{p}{,}\PYG{n+nv+vg}{gClick}\PYG{p}{)}
\end{sphinxVerbatim}

\sphinxAtStartPar
\sphinxstylestrong{See Also:}

\sphinxAtStartPar
\sphinxcode{\sphinxupquote{MakeMenu()}}, \sphinxcode{\sphinxupquote{OpenSubMenus()}}, \sphinxcode{\sphinxupquote{ClickOnMenu}}

\index{MakeScrollBox@\spxentry{MakeScrollBox}}\ignorespaces 

\subsection{MakeScrollBox()}
\label{\detokenize{reference/ui:makescrollbox}}\label{\detokenize{reference/ui:index-12}}
\sphinxAtStartPar
\sphinxstyleemphasis{Make a scrolling selection box.}

\sphinxAtStartPar
\sphinxstylestrong{Description:}

\sphinxAtStartPar
Creates a graphical object that displays and allows selection of a list of items, and scrolls if the text gets too big.   It has a property ‘clickon’ that is bound to ‘ClickOnScrollBox’  A Scrolling textbox looks like this: \sphinxstylestrong{Usage:}

\begin{sphinxVerbatim}[commandchars=\\\{\}]
\PYG{k}{define}\PYG{+w}{ }\PYG{n+nf}{MakeScrollBox}\PYG{p}{(}\PYG{p}{.}\PYG{p}{.}\PYG{p}{.}\PYG{p}{)}
\end{sphinxVerbatim}

\sphinxAtStartPar
\sphinxstylestrong{Example:}

\begin{sphinxVerbatim}[commandchars=\\\{\}]
S\PYG{n+nv}{ee}\PYG{+w}{ }\PYG{n+nv}{ui.pbl}\PYG{+w}{ }\PYG{n+nv}{in}\PYG{+w}{ }\PYG{n+nv}{the}\PYG{+w}{ }\PYG{n+nv}{demo}\PYG{+w}{ }\PYG{n+nv}{directory}\PYG{+w}{ }\PYG{n+nv}{for}\PYG{+w}{ }\PYG{n+nv}{examples}\PYG{+w}{ }\PYG{n+nv}{of}\PYG{+w}{ }\PYG{n+nv}{the}\PYG{+w}{ }\PYG{n+nv}{use}\PYG{+w}{ }\PYG{n+nv}{of}\PYG{+w}{ }\PYG{n+nv}{a}\PYG{+w}{ }\PYG{n+nv}{scrolling}\PYG{+w}{ }\PYG{n+nv}{text}\PYG{+w}{ }\PYG{n+nv}{box}



\PYG{+w}{  }\PYG{n+nv}{sb}\PYG{+w}{ }\PYG{o}{\PYGZlt{}\PYGZhy{}}\PYG{+w}{ }\PYG{n+nf}{MakeScrollBox}\PYG{p}{(}\PYG{n+nf}{Sequence}\PYG{p}{(}\PYG{l+m+mi}{1}\PYG{p}{,}\PYG{l+m+mi}{50}\PYG{p}{,}\PYG{l+m+mi}{1}\PYG{p}{)}\PYG{p}{,}\PYG{l+s+s2}{\PYGZdq{}The numbers\PYGZdq{}}\PYG{p}{,}\PYG{l+m+mi}{40}\PYG{p}{,}\PYG{l+m+mi}{40}\PYG{p}{,}\PYG{n+nv+vg}{gWin}\PYG{p}{,}\PYG{l+m+mi}{12}\PYG{p}{,}\PYG{l+m+mi}{150}\PYG{p}{,}\PYG{l+m+mi}{500}\PYG{p}{,}\PYG{l+m+mi}{3}\PYG{p}{)}

\PYG{+w}{   }\PYG{n+nf}{Draw}\PYG{p}{(}\PYG{p}{)}
\PYG{+w}{   }\PYG{n+nv}{resp}\PYG{+w}{ }\PYG{o}{\PYGZlt{}\PYGZhy{}}\PYG{+w}{ }\PYG{n+nf}{WaitForClickOntarget}\PYG{p}{(}\PYG{p}{[}\PYG{n+nv}{sb}\PYG{p}{]}\PYG{p}{,}\PYG{p}{[}\PYG{l+m+mi}{1}\PYG{p}{]}\PYG{p}{)}
\PYG{+w}{   }\PYG{n+nf}{CallFunction}\PYG{p}{(}\PYG{n+nv}{sb.clickon}\PYG{p}{,}\PYG{p}{[}\PYG{n+nv}{sb}\PYG{p}{,}\PYG{n+nv+vg}{gClick}\PYG{p}{]}\PYG{p}{)}
\PYG{+w}{   }\PYG{c+c1}{\PYGZsh{}Alternately: ClickOnScrollbox(sb,gClick)}
\end{sphinxVerbatim}

\sphinxAtStartPar
\sphinxstylestrong{See Also:}

\sphinxAtStartPar
\sphinxcode{\sphinxupquote{SetScrollingText}}
\sphinxcode{\sphinxupquote{MakeScrollingTextBox}}
\sphinxcode{\sphinxupquote{UpdateScrollBox}}
\sphinxcode{\sphinxupquote{DrawScrollBox}}
\sphinxcode{\sphinxupquote{ClickOnScrollBox}}

\index{MakeScrollingTextBox@\spxentry{MakeScrollingTextBox}}\ignorespaces 

\subsection{MakeScrollingTextBox()}
\label{\detokenize{reference/ui:makescrollingtextbox}}\label{\detokenize{reference/ui:index-13}}
\sphinxAtStartPar
\sphinxstyleemphasis{Make a box for text that can be scrolled if too long.}

\sphinxAtStartPar
\sphinxstylestrong{Description:}

\sphinxAtStartPar
Creates a graphical object that displays a block of text, and scrolls if the text gets too big. It uses a \sphinxcode{\sphinxupquote{Scrollbox}} as its base, but handles parsing the text into lines and hides the selection box.  Thus, no ‘selection’ is displayed (although it actually exists), and a .text property is added to hold the text being displayed.  It has a property ‘clickon’ that is bound to ‘ClickOnScrollBox’  A Scrolling textbox looks like this: \sphinxstylestrong{Usage:}

\begin{sphinxVerbatim}[commandchars=\\\{\}]
\PYG{k}{define}\PYG{+w}{ }\PYG{n+nf}{MakeScrollingTextBox}\PYG{p}{(}\PYG{p}{.}\PYG{p}{.}\PYG{p}{.}\PYG{p}{)}
\end{sphinxVerbatim}

\sphinxAtStartPar
\sphinxstylestrong{Example:}

\begin{sphinxVerbatim}[commandchars=\\\{\}]
S\PYG{n+nv}{ee}\PYG{+w}{ }\PYG{n+nv}{ui.pbl}\PYG{+w}{ }\PYG{n+nv}{in}\PYG{+w}{ }\PYG{n+nv}{the}\PYG{+w}{ }\PYG{n+nv}{demo}\PYG{+w}{ }\PYG{n+nv}{directory}\PYG{+w}{ }\PYG{n+nv}{for}\PYG{+w}{ }\PYG{n+nv}{examples}\PYG{+w}{ }\PYG{n+nv}{of}\PYG{+w}{ }\PYG{n+nv}{the}\PYG{+w}{ }\PYG{n+nv}{use}\PYG{+w}{ }\PYG{n+nv}{of}\PYG{+w}{ }\PYG{n+nv}{a}\PYG{+w}{ }\PYG{n+nv}{scrolling}\PYG{+w}{ }\PYG{n+nv}{text}\PYG{+w}{ }\PYG{n+nv}{box}



\PYG{+w}{  }\PYG{n+nv}{textscroll}\PYG{+w}{ }\PYG{o}{\PYGZlt{}\PYGZhy{}}\PYG{+w}{ }\PYG{n+nf}{MakeScrollingTextBox}\PYG{p}{(}\PYG{l+s+s2}{\PYGZdq{}\PYGZdq{}}\PYG{p}{,}\PYG{l+m+mi}{200}\PYG{p}{,}\PYG{l+m+mi}{50}\PYG{p}{,}\PYG{n+nv+vg}{gWin}\PYG{p}{,}\PYG{l+m+mi}{12}\PYG{p}{,}
\PYG{+w}{                                        }\PYG{l+m+mi}{300}\PYG{p}{,}\PYG{l+m+mi}{150}\PYG{p}{,}\PYG{l+m+mi}{0}\PYG{p}{)}

\PYG{+w}{  }\PYG{n+nf}{SetScrollingText}\PYG{p}{(}\PYG{n+nv}{textscroll}\PYG{p}{,}\PYG{n+nf}{FileReadText}\PYG{p}{(}\PYG{l+s+s2}{\PYGZdq{}Uppercase.txt\PYGZdq{}}\PYG{p}{)}\PYG{p}{)}
\PYG{+w}{   }\PYG{n+nf}{Draw}\PYG{p}{(}\PYG{p}{)}
\PYG{+w}{  }\PYG{n+nv}{resp}\PYG{+w}{ }\PYG{o}{\PYGZlt{}\PYGZhy{}}\PYG{+w}{ }\PYG{n+nf}{WaitForClickOntarget}\PYG{p}{(}\PYG{p}{[}\PYG{n+nv}{textscroll}\PYG{p}{]}\PYG{p}{,}\PYG{p}{[}\PYG{l+m+mi}{1}\PYG{p}{]}\PYG{p}{)}
\PYG{+w}{   }\PYG{n+nf}{CallFunction}\PYG{p}{(}\PYG{n+nv}{textscroll.clickon}\PYG{p}{,}\PYG{p}{[}\PYG{n+nv}{textscroll}\PYG{p}{,}\PYG{n+nv+vg}{gClick}\PYG{p}{]}\PYG{p}{)}
\end{sphinxVerbatim}

\sphinxAtStartPar
\sphinxstylestrong{See Also:}

\sphinxAtStartPar
\sphinxcode{\sphinxupquote{SetScrollingText}}
\sphinxcode{\sphinxupquote{MakeScrollBox}}
\sphinxcode{\sphinxupquote{UpdateScrollBox}}
\sphinxcode{\sphinxupquote{DrawScrollBox}}
\sphinxcode{\sphinxupquote{ClickOnScrollBox}}

\index{MakeTextList@\spxentry{MakeTextList}}\ignorespaces 

\subsection{MakeTextList()}
\label{\detokenize{reference/ui:maketextlist}}\label{\detokenize{reference/ui:index-14}}
\sphinxAtStartPar
\sphinxstyleemphasis{Creates a text body from a list.}

\sphinxAtStartPar
\sphinxstylestrong{Description:}

\sphinxAtStartPar
This takes a list and creates a block of text with carriage returns, ensuring each item of the list is on its own line; it also requires an offset, skipping the first lines of the list.  It is mostly a helper function used by \sphinxcode{\sphinxupquote{Scrollbox}} objects to help format.  It will make text out of the entire list, so  you should be sure to cut off the end for efficiency if you only want to display some of the lines.

\sphinxAtStartPar
\sphinxstylestrong{Usage:}

\begin{sphinxVerbatim}[commandchars=\\\{\}]
\PYG{k}{define}\PYG{+w}{ }\PYG{n+nf}{MakeTextList}\PYG{p}{(}\PYG{p}{.}\PYG{p}{.}\PYG{p}{.}\PYG{p}{)}
\end{sphinxVerbatim}

\sphinxAtStartPar
\sphinxstylestrong{Example:}

\begin{sphinxVerbatim}[commandchars=\\\{\}]
\PYG{n+nv}{letters}\PYG{+w}{ }\PYG{o}{\PYGZlt{}\PYGZhy{}}\PYG{+w}{ }\PYG{n+nf}{FileReadList}\PYG{p}{(}\PYG{l+s+s2}{\PYGZdq{}Uppercase.txt\PYGZdq{}}\PYG{p}{)}
\PYG{n+nv}{out}\PYG{+w}{ }\PYG{o}{\PYGZlt{}\PYGZhy{}}\PYG{+w}{ }\PYG{n+nf}{MakeTextList}\PYG{p}{(}\PYG{n+nv}{letters}\PYG{p}{,}\PYG{l+m+mi}{20}\PYG{p}{,}\PYG{l+s+s2}{\PYGZdq{}\PYGZhy{}\PYGZhy{}\PYGZdq{}}\PYG{p}{)}

T\PYG{n+nv}{he}\PYG{+w}{ }\PYG{n+nv}{above}\PYG{+w}{ }\PYG{n+nv}{code}\PYG{+w}{ }\PYG{n+nv}{will}\PYG{+w}{ }\PYG{n+nv}{create}\PYG{+w}{ }\PYG{n+nv}{the}\PYG{+w}{ }\PYG{n+nv}{following}\PYG{o}{:}

\PYG{o}{\PYGZhy{}}\PYG{o}{\PYGZhy{}}\PYG{n+nv}{u}
\PYG{o}{\PYGZhy{}}\PYG{o}{\PYGZhy{}}\PYG{n+nv}{v}
\PYG{o}{\PYGZhy{}}\PYG{o}{\PYGZhy{}}\PYG{n+nv}{w}
\PYG{o}{\PYGZhy{}}\PYG{o}{\PYGZhy{}}\PYG{n+nv}{x}
\PYG{o}{\PYGZhy{}}\PYG{o}{\PYGZhy{}}\PYG{n+nv}{y}
\PYG{o}{\PYGZhy{}}\PYG{o}{\PYGZhy{}}\PYG{n+nv}{z}
\end{sphinxVerbatim}

\sphinxAtStartPar
\sphinxstylestrong{See Also:}

\sphinxAtStartPar
\sphinxcode{\sphinxupquote{ListToString}}

\index{OpenSubMenus@\spxentry{OpenSubMenus}}\ignorespaces 

\subsection{OpenSubMenus()}
\label{\detokenize{reference/ui:opensubmenus}}\label{\detokenize{reference/ui:index-15}}
\sphinxAtStartPar
\sphinxstyleemphasis{Opens the sub\sphinxhyphen{}menus of a menu.}

\sphinxAtStartPar
\sphinxstylestrong{Description:}

\sphinxAtStartPar
Used by ClickOnMenu to open, display a submenu and get a click.

\sphinxAtStartPar
\sphinxstylestrong{Usage:}

\begin{sphinxVerbatim}[commandchars=\\\{\}]
\PYG{k}{define}\PYG{+w}{ }\PYG{n+nf}{OpenSubMenus}\PYG{p}{(}\PYG{p}{.}\PYG{p}{.}\PYG{p}{.}\PYG{p}{)}
\end{sphinxVerbatim}

\sphinxAtStartPar
\sphinxstylestrong{Example:}

\begin{sphinxVerbatim}[commandchars=\\\{\}]
T\PYG{n+nv}{his}\PYG{+w}{ }\PYG{n+nv}{creates}\PYG{+w}{ }\PYG{n+nv}{a}\PYG{+w}{ }\PYG{n+nv}{menu}\PYG{+w}{ }\PYG{k}{and}\PYG{+w}{ }\PYG{n+nv}{awaits}\PYG{+w}{ }\PYG{n+nv}{clicking}\PYG{+w}{ }\PYG{n+nv}{on}\PYG{p}{.}\PYG{+w}{  }M\PYG{n+nv}{ore}\PYG{+w}{ }\PYG{n+nv}{complete}\PYG{+w}{ }\PYG{n+nv}{examples}\PYG{+w}{ }\PYG{n+nv}{are}\PYG{+w}{ }\PYG{n+nv}{available}\PYG{+w}{ }\PYG{n+nv}{in}\PYG{+w}{ }\PYG{n+nv}{ui.pbl}\PYG{p}{.}\PYG{+w}{  }I\PYG{n+nv}{t}\PYG{+w}{ }\PYG{n+nv}{requires}\PYG{+w}{ }\PYG{n+nv}{that}\PYG{+w}{ }M\PYG{n+nv}{yMessage}\PYG{+w}{ }\PYG{n+nv}{is}\PYG{+w}{ }\PYG{n+nv}{created}\PYG{+w}{ }\PYG{n+nv}{somewhere}


\PYG{+w}{   }\PYG{n+nv}{menu1}\PYG{+w}{ }\PYG{o}{\PYGZlt{}\PYGZhy{}}\PYG{+w}{ }\PYG{n+nf}{MakeMenuItem}\PYG{p}{(}\PYG{l+s+s2}{\PYGZdq{}File\PYGZdq{}}\PYG{p}{,}\PYG{l+m+mi}{0}\PYG{p}{,}\PYG{l+m+mi}{0}\PYG{p}{,}\PYG{n+nv+vg}{gWin}\PYG{p}{,}\PYG{l+m+mi}{14}\PYG{p}{,}\PYG{l+m+mi}{10}\PYG{p}{,}\PYG{l+s+s2}{\PYGZdq{}MYMESSAGE\PYGZdq{}}\PYG{p}{)}


\PYG{+w}{   }\PYG{n+nv}{menu2}\PYG{o}{\PYGZlt{}\PYGZhy{}}\PYG{+w}{ }\PYG{n+nf}{MakeMenu}\PYG{p}{(}\PYG{l+s+s2}{\PYGZdq{}Edit\PYGZdq{}}\PYG{p}{,}\PYG{l+m+mi}{70}\PYG{p}{,}\PYG{l+m+mi}{0}\PYG{p}{,}\PYG{n+nv+vg}{gWin}\PYG{p}{,}\PYG{l+m+mi}{14}\PYG{p}{,}\PYG{l+m+mi}{10}\PYG{p}{,}\PYG{+w}{ }\PYG{l+s+s2}{\PYGZdq{}MYMESSAGE\PYGZdq{}}\PYG{p}{)}

\PYG{+w}{   }\PYG{n+nv}{menus}\PYG{+w}{ }\PYG{o}{\PYGZlt{}\PYGZhy{}}\PYG{+w}{ }\PYG{p}{[}\PYG{n+nv}{menu1}\PYG{p}{,}\PYG{n+nv}{menu2}\PYG{p}{]}
\PYG{+w}{   }\PYG{n+nv}{opt}\PYG{+w}{ }\PYG{o}{\PYGZlt{}\PYGZhy{}}\PYG{+w}{ }\PYG{n+nf}{WaitForClickOntarget}\PYG{p}{(}\PYG{n+nv}{menu}\PYG{p}{,}\PYG{p}{[}\PYG{l+m+mi}{1}\PYG{p}{,}\PYG{l+m+mi}{2}\PYG{p}{]}\PYG{p}{)}
\PYG{+w}{   }\PYG{n+nf}{ClickOnMenu}\PYG{p}{(}\PYG{n+nf}{Nth}\PYG{p}{(}\PYG{n+nv}{menus}\PYG{p}{,}\PYG{n+nv}{opt}\PYG{p}{)}\PYG{p}{,}\PYG{n+nv+vg}{gClick}\PYG{p}{)}
\end{sphinxVerbatim}

\sphinxAtStartPar
\sphinxstylestrong{See Also:}

\sphinxAtStartPar
\sphinxcode{\sphinxupquote{MakeMenu()}}, \sphinxcode{\sphinxupquote{OpenSubMenus()}}, \sphinxcode{\sphinxupquote{MakeMenuItem()}}

\index{PopUpEntryBox@\spxentry{PopUpEntryBox}}\ignorespaces 

\subsection{PopUpEntryBox()}
\label{\detokenize{reference/ui:popupentrybox}}\label{\detokenize{reference/ui:index-16}}
\sphinxAtStartPar
\sphinxstylestrong{Description:}

\sphinxAtStartPar
Creates a small text\sphinxhyphen{}entry box at a specified location..

\sphinxAtStartPar
\sphinxstylestrong{Usage:}

\begin{sphinxVerbatim}[commandchars=\\\{\}]
\PYG{k}{define}\PYG{+w}{ }\PYG{n+nf}{PopUpEntryBox}\PYG{p}{(}\PYG{p}{.}\PYG{p}{.}\PYG{p}{.}\PYG{p}{)}
\end{sphinxVerbatim}

\sphinxAtStartPar
\sphinxstylestrong{Example:}

\begin{sphinxVerbatim}[commandchars=\\\{\}]
\PYG{n+nv}{subnum}\PYG{+w}{ }\PYG{o}{\PYGZlt{}\PYGZhy{}}\PYG{+w}{ }\PYG{n+nf}{PopUpEntryBox}\PYG{p}{(}\PYG{l+s+s2}{\PYGZdq{}Enter particpant code\PYGZdq{}}\PYG{p}{,}\PYG{n+nv+vg}{gWin}\PYG{p}{,}\PYG{p}{[}\PYG{l+m+mi}{100}\PYG{p}{,}\PYG{l+m+mi}{100}\PYG{p}{]}\PYG{p}{)}
\end{sphinxVerbatim}

\sphinxAtStartPar
\sphinxstylestrong{See Also:}

\sphinxAtStartPar
\sphinxcode{\sphinxupquote{MessageBox}} \sphinxcode{\sphinxupquote{GetEasyInput}}, \sphinxcode{\sphinxupquote{PopUpMessageBox}}

\index{PushButton@\spxentry{PushButton}}\ignorespaces 

\subsection{PushButton()}
\label{\detokenize{reference/ui:pushbutton}}\label{\detokenize{reference/ui:index-17}}
\sphinxAtStartPar
\sphinxstyleemphasis{Pushes a button and releases.}

\sphinxAtStartPar
\sphinxstylestrong{Description:}

\sphinxAtStartPar
Animates a button\sphinxhyphen{}pushing. It takes a button created using the MakeButton function and will animate a downclick when the mouse is down, and release when the mouse is unclicked.  To conform with general object handlers, it requires specifying a mouse click position, which could be {[}0,0{]}, or gclick. This function is bound to the property ‘clickon’ of any button, allowing you to handle mouse clicks universally for many different objects.

\sphinxAtStartPar
\sphinxstylestrong{Usage:}

\begin{sphinxVerbatim}[commandchars=\\\{\}]
\PYG{k}{define}\PYG{+w}{ }\PYG{n+nf}{PushButton}\PYG{p}{(}\PYG{p}{.}\PYG{p}{.}\PYG{p}{.}\PYG{p}{)}
\end{sphinxVerbatim}

\sphinxAtStartPar
\sphinxstylestrong{Example:}

\begin{sphinxVerbatim}[commandchars=\\\{\}]
T\PYG{n+nv}{he}\PYG{+w}{ }\PYG{n+nv}{following}\PYG{+w}{ }\PYG{n+nv}{creates}\PYG{+w}{ }\PYG{n+nv}{a}\PYG{+w}{ }\PYG{n+nv}{button}\PYG{p}{,}\PYG{+w}{ }\PYG{n+nv}{waits}\PYG{+w}{ }\PYG{n+nv}{for}\PYG{+w}{ }\PYG{n+nv}{you}\PYG{+w}{ }\PYG{n+nv}{to}\PYG{+w}{ }\PYG{n+nv}{click}\PYG{+w}{ }\PYG{n+nv}{on}\PYG{+w}{ }\PYG{n+nv}{it}\PYG{p}{,}\PYG{+w}{ }\PYG{k}{and}\PYG{+w}{ }\PYG{n+nv}{animates}\PYG{+w}{ }\PYG{n+nv}{a}\PYG{+w}{ }\PYG{n+nv}{button}\PYG{+w}{ }\PYG{n+nv}{press}


\PYG{+w}{ }\PYG{n+nv}{done}\PYG{+w}{ }\PYG{o}{\PYGZlt{}\PYGZhy{}}\PYG{+w}{ }\PYG{n+nf}{MakeButton}\PYG{p}{(}\PYG{l+s+s2}{\PYGZdq{}QUIT\PYGZdq{}}\PYG{p}{,}\PYG{l+m+mi}{400}\PYG{p}{,}\PYG{l+m+mi}{250}\PYG{p}{,}\PYG{n+nv+vg}{gWin}\PYG{p}{,}\PYG{l+m+mi}{150}\PYG{p}{)}
\PYG{+w}{ }\PYG{n+nv}{resp}\PYG{+w}{ }\PYG{o}{\PYGZlt{}\PYGZhy{}}\PYG{+w}{ }\PYG{n+nf}{WaitForClickOntarget}\PYG{p}{(}\PYG{p}{[}\PYG{n+nv}{done}\PYG{p}{]}\PYG{p}{,}\PYG{p}{[}\PYG{l+m+mi}{1}\PYG{p}{]}\PYG{p}{)}
\PYG{+w}{ }\PYG{n+nf}{PushButton}\PYG{p}{(}\PYG{n+nv}{done}\PYG{p}{,}\PYG{p}{[}\PYG{l+m+mi}{0}\PYG{p}{,}\PYG{l+m+mi}{0}\PYG{p}{]}\PYG{p}{)}


T\PYG{n+nv}{o}\PYG{+w}{ }\PYG{n+nv}{handle}\PYG{+w}{ }\PYG{n+nv}{multiple}\PYG{+w}{ }\PYG{n+nv}{buttons}\PYG{p}{,}\PYG{+w}{ }\PYG{n+nv}{you}\PYG{+w}{ }\PYG{n+nv}{can}\PYG{+w}{ }\PYG{n+nv}{do}\PYG{o}{:}


\PYG{+w}{ }\PYG{n+nv}{done}\PYG{+w}{ }\PYG{o}{\PYGZlt{}\PYGZhy{}}\PYG{+w}{ }\PYG{n+nf}{MakeButton}\PYG{p}{(}\PYG{l+s+s2}{\PYGZdq{}QUIT\PYGZdq{}}\PYG{p}{,}\PYG{l+m+mi}{400}\PYG{p}{,}\PYG{l+m+mi}{250}\PYG{p}{,}\PYG{n+nv+vg}{gWin}\PYG{p}{,}\PYG{l+m+mi}{150}\PYG{p}{)}
\PYG{+w}{ }\PYG{n+nv}{ok}\PYG{+w}{ }\PYG{o}{\PYGZlt{}\PYGZhy{}}\PYG{+w}{   }\PYG{n+nf}{MakeButton}\PYG{p}{(}\PYG{l+s+s2}{\PYGZdq{}OK\PYGZdq{}}\PYG{p}{,}\PYG{l+m+mi}{400}\PYG{p}{,}\PYG{l+m+mi}{250}\PYG{p}{,}\PYG{n+nv+vg}{gWin}\PYG{p}{,}\PYG{l+m+mi}{150}\PYG{p}{)}

\PYG{+w}{ }\PYG{n+nv}{resp}\PYG{+w}{ }\PYG{o}{\PYGZlt{}\PYGZhy{}}\PYG{+w}{ }\PYG{l+m+mi}{2}
\PYG{+w}{ }\PYG{k}{while}\PYG{+w}{ }\PYG{p}{(}\PYG{n+nv}{resp}\PYG{+w}{ }\PYG{o}{!=}\PYG{+w}{ }\PYG{l+m+mi}{1}\PYG{p}{)}
\PYG{+w}{ }\PYG{p}{\PYGZob{}}
\PYG{+w}{  }\PYG{n+nf}{Draw}\PYG{p}{(}\PYG{p}{)}
\PYG{+w}{  }\PYG{n+nv}{resp}\PYG{+w}{ }\PYG{o}{\PYGZlt{}\PYGZhy{}}\PYG{+w}{ }\PYG{n+nf}{WaitForClickOntarget}\PYG{p}{(}\PYG{p}{[}\PYG{n+nv}{done}\PYG{p}{,}\PYG{n+nv}{ok}\PYG{p}{]}\PYG{p}{,}\PYG{p}{[}\PYG{l+m+mi}{1}\PYG{p}{,}\PYG{l+m+mi}{2}\PYG{p}{]}\PYG{p}{)}
\PYG{+w}{  }\PYG{n+nv}{obj}\PYG{+w}{ }\PYG{o}{\PYGZlt{}\PYGZhy{}}\PYG{+w}{ }\PYG{n+nf}{Nth}\PYG{p}{(}\PYG{p}{[}\PYG{n+nv}{done}\PYG{p}{,}\PYG{n+nv}{ok}\PYG{p}{]}\PYG{p}{,}\PYG{n+nv}{resp}\PYG{p}{)}
\PYG{+w}{  }\PYG{n+nf}{CallFunction}\PYG{p}{(}\PYG{n+nv}{obj.clickon}\PYG{p}{,}\PYG{p}{[}\PYG{n+nv}{obj}\PYG{p}{,}\PYG{n+nv+vg}{gClick}\PYG{p}{]}\PYG{p}{)}
\PYG{+w}{ }\PYG{p}{\PYGZcb{}}
\end{sphinxVerbatim}

\sphinxAtStartPar
\sphinxstylestrong{See Also:}

\sphinxAtStartPar
\sphinxcode{\sphinxupquote{MakeCheckBox()}}

\index{SetScrollingText@\spxentry{SetScrollingText}}\ignorespaces 

\subsection{SetScrollingText()}
\label{\detokenize{reference/ui:setscrollingtext}}\label{\detokenize{reference/ui:index-18}}
\sphinxAtStartPar
\sphinxstyleemphasis{Changes text of a scrolling textbox.}

\sphinxAtStartPar
\sphinxstylestrong{Description:}

\sphinxAtStartPar
This updates the text in a \sphinxcode{\sphinxupquote{ScrollingTextBox}}. Because text must be parsed to be put into the box, you cannot just update the .text property, but instead should use this function.

\sphinxAtStartPar
\sphinxstylestrong{Usage:}

\begin{sphinxVerbatim}[commandchars=\\\{\}]
\PYG{k}{define}\PYG{+w}{ }\PYG{n+nf}{SetScrollingText}\PYG{p}{(}\PYG{p}{.}\PYG{p}{.}\PYG{p}{.}\PYG{p}{)}
\end{sphinxVerbatim}

\sphinxAtStartPar
\sphinxstylestrong{Example:}

\begin{sphinxVerbatim}[commandchars=\\\{\}]
S\PYG{n+nv}{ee}\PYG{+w}{ }\PYG{n+nv}{ui.pbl}\PYG{+w}{ }\PYG{n+nv}{in}\PYG{+w}{ }\PYG{n+nv}{the}\PYG{+w}{ }\PYG{n+nv}{demo}\PYG{+w}{ }\PYG{n+nv}{directory}\PYG{+w}{ }\PYG{n+nv}{for}\PYG{+w}{ }\PYG{n+nv}{examples}\PYG{+w}{ }\PYG{n+nv}{of}\PYG{+w}{ }\PYG{n+nv}{the}\PYG{+w}{ }\PYG{n+nv}{use}\PYG{+w}{ }\PYG{n+nv}{of}\PYG{+w}{ }\PYG{n+nv}{a}\PYG{+w}{ }\PYG{n+nv}{scrolling}\PYG{+w}{ }\PYG{n+nv}{text}\PYG{+w}{ }\PYG{n+nv}{box}\PYG{p}{.}\PYG{+w}{  }A\PYG{+w}{ }\PYG{n+nv}{brief}\PYG{+w}{ }\PYG{n+nv}{example}\PYG{+w}{ }\PYG{n+nv}{follows}\PYG{o}{:}


\PYG{+w}{  }\PYG{n+nv}{textscroll}\PYG{+w}{ }\PYG{o}{\PYGZlt{}\PYGZhy{}}\PYG{+w}{ }\PYG{n+nf}{MakeScrollingTextBox}\PYG{p}{(}\PYG{l+s+s2}{\PYGZdq{}\PYGZdq{}}\PYG{p}{,}\PYG{l+m+mi}{200}\PYG{p}{,}\PYG{l+m+mi}{50}\PYG{p}{,}\PYG{n+nv+vg}{gWin}\PYG{p}{,}\PYG{l+m+mi}{12}\PYG{p}{,}
\PYG{+w}{                                        }\PYG{l+m+mi}{300}\PYG{p}{,}\PYG{l+m+mi}{150}\PYG{p}{,}\PYG{l+m+mi}{0}\PYG{p}{)}

\PYG{+w}{  }\PYG{n+nf}{SetScrollingText}\PYG{p}{(}\PYG{n+nv}{textscroll}\PYG{p}{,}\PYG{n+nf}{FileReadText}\PYG{p}{(}\PYG{l+s+s2}{\PYGZdq{}Uppercase.txt\PYGZdq{}}\PYG{p}{)}\PYG{p}{)}
\PYG{+w}{   }\PYG{n+nf}{Draw}\PYG{p}{(}\PYG{p}{)}
\PYG{+w}{  }\PYG{n+nv}{resp}\PYG{+w}{ }\PYG{o}{\PYGZlt{}\PYGZhy{}}\PYG{+w}{ }\PYG{n+nf}{WaitForClickOntarget}\PYG{p}{(}\PYG{p}{[}\PYG{n+nv}{textscroll}\PYG{p}{]}\PYG{p}{,}\PYG{p}{[}\PYG{l+m+mi}{1}\PYG{p}{]}\PYG{p}{)}
\PYG{+w}{   }\PYG{n+nf}{CallFunction}\PYG{p}{(}\PYG{n+nv}{textscroll.clickon}\PYG{p}{,}\PYG{p}{[}\PYG{n+nv}{textscroll}\PYG{p}{,}\PYG{n+nv+vg}{gClick}\PYG{p}{]}\PYG{p}{)}
\end{sphinxVerbatim}

\sphinxAtStartPar
\sphinxstylestrong{See Also:}

\sphinxAtStartPar
\sphinxcode{\sphinxupquote{MakeScrollingTextBox}}
\sphinxcode{\sphinxupquote{MakeScrollBox}}
\sphinxcode{\sphinxupquote{UpdateScrollBox}}
\sphinxcode{\sphinxupquote{DrawScrollBox}}
\sphinxcode{\sphinxupquote{ClickOnScrollBox}}

\index{SetTextBoxCursorFromClick@\spxentry{SetTextBoxCursorFromClick}}\ignorespaces 

\subsection{SetTextBoxCursorFromClick()}
\label{\detokenize{reference/ui:settextboxcursorfromclick}}\label{\detokenize{reference/ui:index-19}}
\sphinxAtStartPar
\sphinxstylestrong{Description:}

\sphinxAtStartPar
this is used directly by a compiled function GetInput0
to reset the cursor position in any getinput thing.

\sphinxAtStartPar
\sphinxstylestrong{Usage:}

\begin{sphinxVerbatim}[commandchars=\\\{\}]
\PYG{k}{define}\PYG{+w}{ }\PYG{n+nf}{SetTextBoxCursorFromClick}\PYG{p}{(}\PYG{n+nv}{box}\PYG{p}{,}\PYG{+w}{ }\PYG{n+nv}{exit}\PYG{p}{,}\PYG{+w}{ }\PYG{n+nv}{click}\PYG{p}{)}
\end{sphinxVerbatim}

\index{SortDir@\spxentry{SortDir}}\ignorespaces 

\subsection{SortDir()}
\label{\detokenize{reference/ui:sortdir}}\label{\detokenize{reference/ui:index-20}}
\sphinxAtStartPar
\sphinxstylestrong{Description:}

\sphinxAtStartPar
This sorts the directory by subdirs
then alphabetically.

\sphinxAtStartPar
\sphinxstylestrong{Usage:}

\begin{sphinxVerbatim}[commandchars=\\\{\}]
\PYG{k}{define}\PYG{+w}{ }\PYG{n+nf}{SortDir}\PYG{p}{(}\PYG{n+nv}{inlist}\PYG{p}{,}\PYG{n+nv}{path}\PYG{p}{)}
\end{sphinxVerbatim}

\index{UpdatePulldown@\spxentry{UpdatePulldown}}\ignorespaces 

\subsection{UpdatePulldown()}
\label{\detokenize{reference/ui:updatepulldown}}\label{\detokenize{reference/ui:index-21}}
\sphinxAtStartPar
\sphinxstyleemphasis{Updates the list of a pulldown.}

\sphinxAtStartPar
\sphinxstylestrong{Description:}

\sphinxAtStartPar
This changes the list being used in a Pulldown object.  It tries to maintain the same selected option (matching the text of the previous selection), but if not found will select index 1. It calls \sphinxcode{\sphinxupquote{DrawPullDown}} when complete, but a \sphinxcode{\sphinxupquote{Draw()}} command must be issued before the pulldown changes will appear.

\sphinxAtStartPar
\sphinxstylestrong{Usage:}

\begin{sphinxVerbatim}[commandchars=\\\{\}]
\PYG{k}{define}\PYG{+w}{ }\PYG{n+nf}{UpdatePulldown}\PYG{p}{(}\PYG{p}{.}\PYG{p}{.}\PYG{p}{.}\PYG{p}{)}
\end{sphinxVerbatim}

\sphinxAtStartPar
\sphinxstylestrong{Example:}

\begin{sphinxVerbatim}[commandchars=\\\{\}]
\PYG{n+nv}{options}\PYG{+w}{  }\PYG{o}{\PYGZlt{}\PYGZhy{}}\PYG{+w}{ }\PYG{n+nf}{MakePulldownButton}\PYG{p}{(}\PYG{p}{[}\PYG{l+s+s2}{\PYGZdq{}A\PYGZdq{}}\PYG{p}{,}B\PYG{l+s+s2}{\PYGZdq{},\PYGZdq{}}C\PYG{l+s+s2}{\PYGZdq{}],400,250,gWin,14,100,3)}
\PYG{l+s+s2}{Draw()}
\PYG{l+s+s2}{WaitForAnyKeyPress()}

\PYG{l+s+s2}{\PYGZsh{}\PYGZsh{}This should add a fourth option but C should still be selected.}
\PYG{l+s+s2}{UpdatePullDown(options,[\PYGZdq{}}A\PYG{l+s+s2}{\PYGZdq{},\PYGZdq{}}B\PYG{l+s+s2}{\PYGZdq{},\PYGZdq{}}C\PYG{l+s+s2}{\PYGZdq{},\PYGZdq{}}D\PYGZdq{}\PYG{p}{]}\PYG{p}{)}
\PYG{n+nf}{Draw}\PYG{p}{(}\PYG{p}{)}
\PYG{n+nf}{WaitForAnyKeyPress}\PYG{p}{(}\PYG{p}{)}
\end{sphinxVerbatim}

\sphinxAtStartPar
\sphinxstylestrong{See Also:}

\sphinxAtStartPar
\sphinxcode{\sphinxupquote{MakePullDown()}}, \sphinxcode{\sphinxupquote{Pulldown()}}, \sphinxcode{\sphinxupquote{DrawPulldown()}}

\index{UpdateScrollbox@\spxentry{UpdateScrollbox}}\ignorespaces 

\subsection{UpdateScrollbox()}
\label{\detokenize{reference/ui:updatescrollbox}}\label{\detokenize{reference/ui:index-22}}
\sphinxAtStartPar
\sphinxstyleemphasis{Recalculates scrollbox layout.}

\sphinxAtStartPar
\sphinxstylestrong{Description:}

\sphinxAtStartPar
This updates the layout of a \sphinxcode{\sphinxupquote{ScrollBox}}. It should be used if you manually change the .list or .listoffset properties.  It won’t actually redraw the scrollbox (which is done by DrawScrollbox).

\sphinxAtStartPar
\sphinxstylestrong{Usage:}

\begin{sphinxVerbatim}[commandchars=\\\{\}]
\PYG{k}{define}\PYG{+w}{ }\PYG{n+nf}{UpdateScrollbox}\PYG{p}{(}\PYG{p}{.}\PYG{p}{.}\PYG{p}{.}\PYG{p}{)}
\end{sphinxVerbatim}

\sphinxAtStartPar
\sphinxstylestrong{Example:}

\begin{sphinxVerbatim}[commandchars=\\\{\}]
S\PYG{n+nv}{ee}\PYG{+w}{ }\PYG{n+nv}{ui.pbl}\PYG{+w}{ }\PYG{n+nv}{in}\PYG{+w}{ }\PYG{n+nv}{the}\PYG{+w}{ }\PYG{n+nv}{demo}\PYG{+w}{ }\PYG{n+nv}{directory}\PYG{+w}{ }\PYG{n+nv}{for}\PYG{+w}{ }\PYG{n+nv}{examples}\PYG{+w}{ }\PYG{n+nv}{of}\PYG{+w}{ }\PYG{n+nv}{the}\PYG{+w}{ }\PYG{n+nv}{use}\PYG{+w}{ }\PYG{n+nv}{of}\PYG{+w}{ }\PYG{n+nv}{a}\PYG{+w}{ }\PYG{n+nv}{scrolling}\PYG{+w}{ }\PYG{n+nv}{text}\PYG{+w}{ }\PYG{n+nv}{box}\PYG{p}{.}\PYG{+w}{  }A\PYG{+w}{ }\PYG{n+nv}{brief}\PYG{+w}{ }\PYG{n+nv}{example}\PYG{+w}{ }\PYG{n+nv}{follows}\PYG{o}{:}


\PYG{+w}{   }\PYG{n+nv}{sb}\PYG{+w}{ }\PYG{o}{\PYGZlt{}\PYGZhy{}}\PYG{+w}{ }\PYG{n+nf}{MakeScrollBox}\PYG{p}{(}\PYG{n+nf}{Sequence}\PYG{p}{(}\PYG{l+m+mi}{1}\PYG{p}{,}\PYG{l+m+mi}{50}\PYG{p}{,}\PYG{l+m+mi}{1}\PYG{p}{)}\PYG{p}{,}\PYG{l+s+s2}{\PYGZdq{}The numbers\PYGZdq{}}\PYG{p}{,}\PYG{l+m+mi}{40}\PYG{p}{,}\PYG{l+m+mi}{40}\PYG{p}{,}\PYG{n+nv+vg}{gWin}\PYG{p}{,}\PYG{l+m+mi}{12}\PYG{p}{,}\PYG{l+m+mi}{150}\PYG{p}{,}\PYG{l+m+mi}{500}\PYG{p}{,}\PYG{l+m+mi}{3}\PYG{p}{)}
\PYG{+w}{   }\PYG{n+nf}{Draw}\PYG{p}{(}\PYG{p}{)}

\PYG{+w}{   }\PYG{n+nv}{resp}\PYG{+w}{ }\PYG{o}{\PYGZlt{}\PYGZhy{}}\PYG{+w}{ }\PYG{n+nf}{WaitForClickOntarget}\PYG{p}{(}\PYG{p}{[}\PYG{n+nv}{sb}\PYG{p}{]}\PYG{p}{,}\PYG{p}{[}\PYG{l+m+mi}{1}\PYG{p}{]}\PYG{p}{)}
\PYG{+w}{   }\PYG{n+nf}{CallFunction}\PYG{p}{(}\PYG{n+nv}{sb.clickon}\PYG{p}{,}\PYG{p}{[}\PYG{n+nv}{sb}\PYG{p}{,}\PYG{n+nv+vg}{gClick}\PYG{p}{]}\PYG{p}{)}
\PYG{+w}{   }\PYG{c+c1}{\PYGZsh{}Alternately: ClickOnScrollbox(sb,gClick)}

\PYG{+w}{   }\PYG{c+c1}{\PYGZsh{}\PYGZsh{}change the selected items}
\PYG{+w}{   }\PYG{n+nv}{sb.list}\PYG{+w}{ }\PYG{o}{\PYGZlt{}\PYGZhy{}}\PYG{+w}{ }\PYG{n+nf}{Sequence}\PYG{p}{(}\PYG{n+nv}{sb.selected}\PYG{p}{,}\PYG{n+nv}{sb.selected}\PYG{o}{+}\PYG{l+m+mi}{50}\PYG{p}{,}\PYG{l+m+mi}{1}\PYG{p}{)}
\PYG{+w}{   }\PYG{n+nf}{UpdateScrollbox}\PYG{p}{(}\PYG{n+nv}{sb}\PYG{p}{)}
\PYG{+w}{   }\PYG{n+nf}{DrawScrollbox}\PYG{p}{(}\PYG{n+nv}{sb}\PYG{p}{)}
\PYG{+w}{   }\PYG{n+nf}{Draw}\PYG{p}{(}\PYG{p}{)}
\end{sphinxVerbatim}

\sphinxAtStartPar
\sphinxstylestrong{See Also:}

\sphinxAtStartPar
\sphinxcode{\sphinxupquote{MakeScrollingTextBox}}
\sphinxcode{\sphinxupquote{MakeScrollBox}}
\sphinxcode{\sphinxupquote{DrawScrollBox}}
\sphinxcode{\sphinxupquote{ClickOnScrollBox}}

\sphinxstepscope


\section{Utility Library \sphinxhyphen{} Helpers and Utilities}
\label{\detokenize{reference/utility:utility-library-helpers-and-utilities}}\label{\detokenize{reference/utility::doc}}
\sphinxAtStartPar
This library contains utility functions for file operations, data management, and common helper tasks.

\begin{sphinxShadowBox}
\sphinxstyletopictitle{Function Index}
\begin{itemize}
\item {} 
\sphinxAtStartPar
\phantomsection\label{\detokenize{reference/utility:id3}}{\hyperref[\detokenize{reference/utility:calibratescreen}]{\sphinxcrossref{CalibrateScreen()}}}

\item {} 
\sphinxAtStartPar
\phantomsection\label{\detokenize{reference/utility:id4}}{\hyperref[\detokenize{reference/utility:concatenatelist}]{\sphinxcrossref{ConcatenateList()}}}

\item {} 
\sphinxAtStartPar
\phantomsection\label{\detokenize{reference/utility:id5}}{\hyperref[\detokenize{reference/utility:convertipstring}]{\sphinxcrossref{ConvertIPString()}}}

\item {} 
\sphinxAtStartPar
\phantomsection\label{\detokenize{reference/utility:id6}}{\hyperref[\detokenize{reference/utility:cr}]{\sphinxcrossref{CR()}}}

\item {} 
\sphinxAtStartPar
\phantomsection\label{\detokenize{reference/utility:id7}}{\hyperref[\detokenize{reference/utility:dirlisttotext}]{\sphinxcrossref{DirlistToText()}}}

\item {} 
\sphinxAtStartPar
\phantomsection\label{\detokenize{reference/utility:id8}}{\hyperref[\detokenize{reference/utility:drawobject}]{\sphinxcrossref{DrawObject()}}}

\item {} 
\sphinxAtStartPar
\phantomsection\label{\detokenize{reference/utility:id9}}{\hyperref[\detokenize{reference/utility:easylabel}]{\sphinxcrossref{EasyLabel()}}}

\item {} 
\sphinxAtStartPar
\phantomsection\label{\detokenize{reference/utility:id10}}{\hyperref[\detokenize{reference/utility:easytextbox}]{\sphinxcrossref{EasyTextBox()}}}

\item {} 
\sphinxAtStartPar
\phantomsection\label{\detokenize{reference/utility:id11}}{\hyperref[\detokenize{reference/utility:endswith}]{\sphinxcrossref{EndsWith()}}}

\item {} 
\sphinxAtStartPar
\phantomsection\label{\detokenize{reference/utility:id12}}{\hyperref[\detokenize{reference/utility:enquote}]{\sphinxcrossref{Enquote()}}}

\item {} 
\sphinxAtStartPar
\phantomsection\label{\detokenize{reference/utility:id13}}{\hyperref[\detokenize{reference/utility:fetchtext}]{\sphinxcrossref{FetchText()}}}

\item {} 
\sphinxAtStartPar
\phantomsection\label{\detokenize{reference/utility:id14}}{\hyperref[\detokenize{reference/utility:fileprintlist}]{\sphinxcrossref{FilePrintList()}}}

\item {} 
\sphinxAtStartPar
\phantomsection\label{\detokenize{reference/utility:id15}}{\hyperref[\detokenize{reference/utility:formattext}]{\sphinxcrossref{FormatText()}}}

\item {} 
\sphinxAtStartPar
\phantomsection\label{\detokenize{reference/utility:id16}}{\hyperref[\detokenize{reference/utility:geteasychoice}]{\sphinxcrossref{GetEasyChoice()}}}

\item {} 
\sphinxAtStartPar
\phantomsection\label{\detokenize{reference/utility:id17}}{\hyperref[\detokenize{reference/utility:geteasyinput}]{\sphinxcrossref{GetEasyInput()}}}

\item {} 
\sphinxAtStartPar
\phantomsection\label{\detokenize{reference/utility:id18}}{\hyperref[\detokenize{reference/utility:geteasymultichoice}]{\sphinxcrossref{GetEasyMultiChoice()}}}

\item {} 
\sphinxAtStartPar
\phantomsection\label{\detokenize{reference/utility:id19}}{\hyperref[\detokenize{reference/utility:getnewdatafile}]{\sphinxcrossref{GetNewDataFile()}}}

\item {} 
\sphinxAtStartPar
\phantomsection\label{\detokenize{reference/utility:id20}}{\hyperref[\detokenize{reference/utility:getnimhdemographics}]{\sphinxcrossref{GetNIMHDemographics()}}}

\item {} 
\sphinxAtStartPar
\phantomsection\label{\detokenize{reference/utility:id21}}{\hyperref[\detokenize{reference/utility:getsubnum}]{\sphinxcrossref{GetSubNum()}}}

\item {} 
\sphinxAtStartPar
\phantomsection\label{\detokenize{reference/utility:id22}}{\hyperref[\detokenize{reference/utility:initializeupload}]{\sphinxcrossref{InitializeUpload()}}}

\item {} 
\sphinxAtStartPar
\phantomsection\label{\detokenize{reference/utility:id23}}{\hyperref[\detokenize{reference/utility:inside}]{\sphinxcrossref{Inside()}}}

\item {} 
\sphinxAtStartPar
\phantomsection\label{\detokenize{reference/utility:id24}}{\hyperref[\detokenize{reference/utility:isurl}]{\sphinxcrossref{IsURL()}}}

\item {} 
\sphinxAtStartPar
\phantomsection\label{\detokenize{reference/utility:id25}}{\hyperref[\detokenize{reference/utility:jsontext}]{\sphinxcrossref{JSONText()}}}

\item {} 
\sphinxAtStartPar
\phantomsection\label{\detokenize{reference/utility:id26}}{\hyperref[\detokenize{reference/utility:likerttrial}]{\sphinxcrossref{LikertTrial()}}}

\item {} 
\sphinxAtStartPar
\phantomsection\label{\detokenize{reference/utility:id27}}{\hyperref[\detokenize{reference/utility:listtohumantext}]{\sphinxcrossref{ListToHumanText()}}}

\item {} 
\sphinxAtStartPar
\phantomsection\label{\detokenize{reference/utility:id28}}{\hyperref[\detokenize{reference/utility:lookup}]{\sphinxcrossref{Lookup()}}}

\item {} 
\sphinxAtStartPar
\phantomsection\label{\detokenize{reference/utility:id29}}{\hyperref[\detokenize{reference/utility:makeparameterobject}]{\sphinxcrossref{MakeParameterObject()}}}

\item {} 
\sphinxAtStartPar
\phantomsection\label{\detokenize{reference/utility:id30}}{\hyperref[\detokenize{reference/utility:messagebox}]{\sphinxcrossref{MessageBox()}}}

\item {} 
\sphinxAtStartPar
\phantomsection\label{\detokenize{reference/utility:id31}}{\hyperref[\detokenize{reference/utility:movecenter}]{\sphinxcrossref{MoveCenter()}}}

\item {} 
\sphinxAtStartPar
\phantomsection\label{\detokenize{reference/utility:id32}}{\hyperref[\detokenize{reference/utility:movecorner}]{\sphinxcrossref{MoveCorner()}}}

\item {} 
\sphinxAtStartPar
\phantomsection\label{\detokenize{reference/utility:id33}}{\hyperref[\detokenize{reference/utility:moveobject}]{\sphinxcrossref{MoveObject()}}}

\item {} 
\sphinxAtStartPar
\phantomsection\label{\detokenize{reference/utility:id34}}{\hyperref[\detokenize{reference/utility:printlist}]{\sphinxcrossref{PrintList()}}}

\item {} 
\sphinxAtStartPar
\phantomsection\label{\detokenize{reference/utility:id35}}{\hyperref[\detokenize{reference/utility:readcsv}]{\sphinxcrossref{ReadCSV()}}}

\item {} 
\sphinxAtStartPar
\phantomsection\label{\detokenize{reference/utility:id36}}{\hyperref[\detokenize{reference/utility:readjsonparameters}]{\sphinxcrossref{ReadJSONParameters()}}}

\item {} 
\sphinxAtStartPar
\phantomsection\label{\detokenize{reference/utility:id37}}{\hyperref[\detokenize{reference/utility:removeobjects}]{\sphinxcrossref{RemoveObjects()}}}

\item {} 
\sphinxAtStartPar
\phantomsection\label{\detokenize{reference/utility:id38}}{\hyperref[\detokenize{reference/utility:replacechar}]{\sphinxcrossref{ReplaceChar()}}}

\item {} 
\sphinxAtStartPar
\phantomsection\label{\detokenize{reference/utility:id39}}{\hyperref[\detokenize{reference/utility:splitstringslow}]{\sphinxcrossref{SplitStringSlow()}}}

\item {} 
\sphinxAtStartPar
\phantomsection\label{\detokenize{reference/utility:id40}}{\hyperref[\detokenize{reference/utility:stripquotes}]{\sphinxcrossref{StripQuotes()}}}

\item {} 
\sphinxAtStartPar
\phantomsection\label{\detokenize{reference/utility:id41}}{\hyperref[\detokenize{reference/utility:stripspace}]{\sphinxcrossref{StripSpace()}}}

\item {} 
\sphinxAtStartPar
\phantomsection\label{\detokenize{reference/utility:id42}}{\hyperref[\detokenize{reference/utility:tab}]{\sphinxcrossref{Tab()}}}

\item {} 
\sphinxAtStartPar
\phantomsection\label{\detokenize{reference/utility:id43}}{\hyperref[\detokenize{reference/utility:waitforbuttonclickontarget}]{\sphinxcrossref{WaitForButtonClickOnTarget()}}}

\item {} 
\sphinxAtStartPar
\phantomsection\label{\detokenize{reference/utility:id44}}{\hyperref[\detokenize{reference/utility:waitforclickontarget}]{\sphinxcrossref{WaitForClickOnTarget()}}}

\item {} 
\sphinxAtStartPar
\phantomsection\label{\detokenize{reference/utility:id45}}{\hyperref[\detokenize{reference/utility:waitforclickontargetwithtimeout}]{\sphinxcrossref{WaitForClickOnTargetWithTimeout()}}}

\item {} 
\sphinxAtStartPar
\phantomsection\label{\detokenize{reference/utility:id46}}{\hyperref[\detokenize{reference/utility:waitfordownclick}]{\sphinxcrossref{WaitForDownClick()}}}

\item {} 
\sphinxAtStartPar
\phantomsection\label{\detokenize{reference/utility:id47}}{\hyperref[\detokenize{reference/utility:yesnotrial}]{\sphinxcrossref{YesNoTrial()}}}

\item {} 
\sphinxAtStartPar
\phantomsection\label{\detokenize{reference/utility:id48}}{\hyperref[\detokenize{reference/utility:zeropad}]{\sphinxcrossref{ZeroPad()}}}

\item {} 
\sphinxAtStartPar
\phantomsection\label{\detokenize{reference/utility:id49}}{\hyperref[\detokenize{reference/utility:functions-pending-documentation}]{\sphinxcrossref{Functions Pending Documentation}}}

\item {} 
\sphinxAtStartPar
\phantomsection\label{\detokenize{reference/utility:id50}}{\hyperref[\detokenize{reference/utility:appenddirlist}]{\sphinxcrossref{AppendDirList()}}}

\item {} 
\sphinxAtStartPar
\phantomsection\label{\detokenize{reference/utility:id51}}{\hyperref[\detokenize{reference/utility:createparameters}]{\sphinxcrossref{CreateParameters()}}}

\item {} 
\sphinxAtStartPar
\phantomsection\label{\detokenize{reference/utility:id52}}{\hyperref[\detokenize{reference/utility:dirtotext}]{\sphinxcrossref{DirToText()}}}

\item {} 
\sphinxAtStartPar
\phantomsection\label{\detokenize{reference/utility:id53}}{\hyperref[\detokenize{reference/utility:getdirectory}]{\sphinxcrossref{GetDirectory()}}}

\item {} 
\sphinxAtStartPar
\phantomsection\label{\detokenize{reference/utility:id54}}{\hyperref[\detokenize{reference/utility:getnewsubnum}]{\sphinxcrossref{GetNewSubNum()}}}

\item {} 
\sphinxAtStartPar
\phantomsection\label{\detokenize{reference/utility:id55}}{\hyperref[\detokenize{reference/utility:readtranslation}]{\sphinxcrossref{ReadTranslation()}}}

\item {} 
\sphinxAtStartPar
\phantomsection\label{\detokenize{reference/utility:id56}}{\hyperref[\detokenize{reference/utility:readtranslationjson}]{\sphinxcrossref{ReadTranslationJSON()}}}

\item {} 
\sphinxAtStartPar
\phantomsection\label{\detokenize{reference/utility:id57}}{\hyperref[\detokenize{reference/utility:substitutestrings}]{\sphinxcrossref{SubstituteStrings()}}}

\item {} 
\sphinxAtStartPar
\phantomsection\label{\detokenize{reference/utility:id58}}{\hyperref[\detokenize{reference/utility:syncdatafile}]{\sphinxcrossref{SyncDataFile()}}}

\item {} 
\sphinxAtStartPar
\phantomsection\label{\detokenize{reference/utility:id59}}{\hyperref[\detokenize{reference/utility:uploadfile}]{\sphinxcrossref{UploadFile()}}}

\item {} 
\sphinxAtStartPar
\phantomsection\label{\detokenize{reference/utility:id60}}{\hyperref[\detokenize{reference/utility:clickon}]{\sphinxcrossref{ClickOn()}}}

\item {} 
\sphinxAtStartPar
\phantomsection\label{\detokenize{reference/utility:id61}}{\hyperref[\detokenize{reference/utility:countdown}]{\sphinxcrossref{Countdown()}}}

\item {} 
\sphinxAtStartPar
\phantomsection\label{\detokenize{reference/utility:id62}}{\hyperref[\detokenize{reference/utility:getinput}]{\sphinxcrossref{GetInput()}}}

\end{itemize}
\end{sphinxShadowBox}

\index{CalibrateScreen@\spxentry{CalibrateScreen}}\ignorespaces 

\subsection{CalibrateScreen()}
\label{\detokenize{reference/utility:calibratescreen}}\label{\detokenize{reference/utility:index-0}}
\sphinxAtStartPar
\sphinxstylestrong{Description:}

\sphinxAtStartPar
Main calibration function
Returns custom object with calibration data

\sphinxAtStartPar
\sphinxstylestrong{Usage:}

\begin{sphinxVerbatim}[commandchars=\\\{\}]
\PYG{k}{define}\PYG{+w}{ }\PYG{n+nf}{CalibrateScreen}\PYG{p}{(}\PYG{n+nv}{win}\PYG{p}{)}
\end{sphinxVerbatim}

\index{ConcatenateList@\spxentry{ConcatenateList}}\ignorespaces 

\subsection{ConcatenateList()}
\label{\detokenize{reference/utility:concatenatelist}}\label{\detokenize{reference/utility:index-1}}
\sphinxAtStartPar
\sphinxstyleemphasis{Combines list}

\sphinxAtStartPar
\sphinxstylestrong{Description:}

\sphinxAtStartPar
Combines a list together to form a single string. Like ListToString but defaults to a separator of “ “ (space).

\sphinxAtStartPar
\sphinxstylestrong{Usage:}

\begin{sphinxVerbatim}[commandchars=\\\{\}]
\PYG{k}{define}\PYG{+w}{ }\PYG{n+nf}{ConcatenateList}\PYG{p}{(}\PYG{p}{.}\PYG{p}{.}\PYG{p}{.}\PYG{p}{)}
\end{sphinxVerbatim}

\sphinxAtStartPar
\sphinxstylestrong{Example:}

\begin{sphinxVerbatim}[commandchars=\\\{\}]
\PYG{n+nf}{ConcatenateList}\PYG{p}{(}\PYG{p}{[}\PYG{l+m+mi}{1}\PYG{p}{,}\PYG{l+m+mi}{2}\PYG{p}{,}\PYG{l+m+mi}{3}\PYG{p}{,}\PYG{l+m+mi}{444}\PYG{p}{]}\PYG{p}{)}\PYG{+w}{                    }\PYG{c+c1}{\PYGZsh{} == \PYGZdq{}1 2 3 444\PYGZdq{}}
\PYG{n+nf}{ConcatenateList}\PYG{p}{(}\PYG{p}{[}\PYG{l+s+s2}{\PYGZdq{}a\PYGZdq{}}\PYG{p}{,}\PYG{l+s+s2}{\PYGZdq{}b\PYGZdq{}}\PYG{p}{,}\PYG{l+s+s2}{\PYGZdq{}c\PYGZdq{}}\PYG{p}{,}\PYG{l+s+s2}{\PYGZdq{}d\PYGZdq{}}\PYG{p}{,}\PYG{l+s+s2}{\PYGZdq{}e\PYGZdq{}}\PYG{p}{]}\PYG{p}{,}\PYG{l+s+s2}{\PYGZdq{},\PYGZdq{}}\PYG{p}{)}\PYG{+w}{      }\PYG{c+c1}{\PYGZsh{} == \PYGZdq{}a,b,c,d,e\PYGZdq{}}
\end{sphinxVerbatim}

\sphinxAtStartPar
\sphinxstylestrong{See Also:}
\begin{description}
\sphinxlineitem{\sphinxcode{\sphinxupquote{SubString()}}, \sphinxcode{\sphinxupquote{StringLength()}}, \sphinxcode{\sphinxupquote{FoldList()}},}
\sphinxAtStartPar
\sphinxcode{\sphinxupquote{ModList()}}

\end{description}

\index{ConvertIPString@\spxentry{ConvertIPString}}\ignorespaces 

\subsection{ConvertIPString()}
\label{\detokenize{reference/utility:convertipstring}}\label{\detokenize{reference/utility:index-2}}
\sphinxAtStartPar
\sphinxstyleemphasis{Converts an ip\sphinxhyphen{}number\sphinxhyphen{}as\sphinxhyphen{}string to usable address}

\sphinxAtStartPar
\sphinxstylestrong{Description:}

\sphinxAtStartPar
Converts an IP address specified as a string into   an integer that can be used by ConnectToIP.

\sphinxAtStartPar
\sphinxstylestrong{Usage:}

\begin{sphinxVerbatim}[commandchars=\\\{\}]
\PYG{k}{define}\PYG{+w}{ }\PYG{n+nf}{ConvertIPString}\PYG{p}{(}\PYG{p}{.}\PYG{p}{.}\PYG{p}{.}\PYG{p}{)}
\end{sphinxVerbatim}

\sphinxAtStartPar
\sphinxstylestrong{Example:}

\begin{sphinxVerbatim}[commandchars=\\\{\}]
S\PYG{n+nv}{ee}\PYG{+w}{ }\PYG{n+nv}{nim.pbl}\PYG{+w}{ }\PYG{n+nv}{for}\PYG{+w}{ }\PYG{n+nv}{example}\PYG{+w}{ }\PYG{n+nv}{of}\PYG{+w}{ }\PYG{n+nv}{two}\PYG{o}{\PYGZhy{}}\PYG{n+nv}{way}\PYG{+w}{ }\PYG{n+nv}{network}\PYG{+w}{ }\PYG{n+nv}{connection}\PYG{p}{.}

\PYG{+w}{  }\PYG{n+nv}{ip}\PYG{+w}{ }\PYG{o}{\PYGZlt{}\PYGZhy{}}\PYG{+w}{ }\PYG{n+nf}{ConvertIPString}\PYG{p}{(}\PYG{l+s+s2}{\PYGZdq{}192.168.0.1\PYGZdq{}}\PYG{p}{)}
\PYG{+w}{  }\PYG{n+nv}{net}\PYG{+w}{ }\PYG{o}{\PYGZlt{}\PYGZhy{}}\PYG{+w}{ }\PYG{n+nf}{ConnectToHost}\PYG{p}{(}\PYG{n+nv}{ip}\PYG{p}{,}\PYG{l+m+mi}{1234}\PYG{p}{)}
\PYG{+w}{  }\PYG{n+nv}{dat}\PYG{+w}{ }\PYG{o}{\PYGZlt{}\PYGZhy{}}\PYG{+w}{ }\PYG{n+nf}{GetData}\PYG{p}{(}\PYG{n+nv}{net}\PYG{p}{,}\PYG{l+m+mi}{20}\PYG{p}{)}
\PYG{+w}{  }\PYG{n+nf}{Print}\PYG{p}{(}\PYG{n+nv}{dat}\PYG{p}{)}
\PYG{+w}{  }\PYG{n+nf}{CloseNetworkConnection}\PYG{p}{(}\PYG{n+nv}{net}\PYG{p}{)}
\end{sphinxVerbatim}

\sphinxAtStartPar
\sphinxstylestrong{See Also:}
\begin{description}
\sphinxlineitem{\sphinxcode{\sphinxupquote{ConnectToHost()}}, \sphinxcode{\sphinxupquote{ConnectToIP()}}, \sphinxcode{\sphinxupquote{GetData()}}, \sphinxcode{\sphinxupquote{WaitForNetworkConnection()}},}
\sphinxAtStartPar
\sphinxcode{\sphinxupquote{SendData()}}, \sphinxcode{\sphinxupquote{CloseNetworkConnection()}}

\end{description}

\index{CR@\spxentry{CR}}\ignorespaces 

\subsection{CR()}
\label{\detokenize{reference/utility:cr}}\label{\detokenize{reference/utility:index-3}}
\sphinxAtStartPar
\sphinxstylestrong{Description:}

\sphinxAtStartPar
Produces \sphinxcode{\sphinxupquote{\textless{}number\textgreater{}}} linefeeds which can be added to a   string and printed or saved to a file.  CR is an abbreviation for {\color{red}\bfseries{}\textasciigrave{}\textasciigrave{}}Carriage Return’’.

\sphinxAtStartPar
\sphinxstylestrong{Usage:}

\begin{sphinxVerbatim}[commandchars=\\\{\}]
\PYG{k}{define}\PYG{+w}{ }\PYG{n+nf}{CR}\PYG{p}{(}\PYG{p}{.}\PYG{p}{.}\PYG{p}{.}\PYG{p}{)}
\end{sphinxVerbatim}

\sphinxAtStartPar
\sphinxstylestrong{Example:}

\begin{sphinxVerbatim}[commandchars=\\\{\}]
\PYG{n+nf}{Print}\PYG{p}{(}\PYG{l+s+s2}{\PYGZdq{}Number: \PYGZdq{}}\PYG{+w}{  }\PYG{n+nf}{Tab}\PYG{p}{(}\PYG{l+m+mi}{1}\PYG{p}{)}\PYG{+w}{ }\PYG{o}{+}\PYG{+w}{ }\PYG{n+nv}{number}\PYG{+w}{  }\PYG{o}{+}\PYG{+w}{ }\PYG{n+nf}{CR}\PYG{p}{(}\PYG{l+m+mi}{2}\PYG{p}{)}\PYG{p}{)}
\PYG{n+nf}{Print}\PYG{p}{(}\PYG{l+s+s2}{\PYGZdq{}We needed space before this line.\PYGZdq{}}\PYG{p}{)}
\end{sphinxVerbatim}

\sphinxAtStartPar
\sphinxstylestrong{See Also:}

\sphinxAtStartPar
\sphinxcode{\sphinxupquote{Format()}}, \sphinxcode{\sphinxupquote{Tab()}}

\index{DirlistToText@\spxentry{DirlistToText}}\ignorespaces 

\subsection{DirlistToText()}
\label{\detokenize{reference/utility:dirlisttotext}}\label{\detokenize{reference/utility:index-4}}
\sphinxAtStartPar
\sphinxstylestrong{Description:}

\sphinxAtStartPar
appends a set of nested directories into a path.

\sphinxAtStartPar
\sphinxstylestrong{Usage:}

\begin{sphinxVerbatim}[commandchars=\\\{\}]
\PYG{k}{define}\PYG{+w}{ }\PYG{n+nf}{DirlistToText}\PYG{p}{(}\PYG{n+nv}{list}\PYG{p}{)}
\end{sphinxVerbatim}

\index{DrawObject@\spxentry{DrawObject}}\ignorespaces 

\subsection{DrawObject()}
\label{\detokenize{reference/utility:drawobject}}\label{\detokenize{reference/utility:index-5}}
\sphinxAtStartPar
\sphinxstyleemphasis{Calls the .draw property of an object}

\sphinxAtStartPar
\sphinxstylestrong{Description:}

\sphinxAtStartPar
Calls the function named by the  .draw property of a custom object.  Useful for handling drawing of a bunch of different objects. This is essentially the same as CallFunction(obj.draw, {[}obj{]}), but falls back to a normal Draw() command so it handles built\sphinxhyphen{}in objects as well.

\sphinxAtStartPar
\sphinxstylestrong{Usage:}

\begin{sphinxVerbatim}[commandchars=\\\{\}]
\PYG{k}{define}\PYG{+w}{ }\PYG{n+nf}{DrawObject}\PYG{p}{(}\PYG{p}{.}\PYG{p}{.}\PYG{p}{.}\PYG{p}{)}
\end{sphinxVerbatim}

\sphinxAtStartPar
\sphinxstylestrong{Example:}

\begin{sphinxVerbatim}[commandchars=\\\{\}]
\PYG{c+c1}{\PYGZsh{}\PYGZsh{}This overrides buttons placement at the center:}
\PYG{n+nv}{done}\PYG{+w}{ }\PYG{o}{\PYGZlt{}\PYGZhy{}}\PYG{+w}{ }\PYG{n+nf}{MakeButton}\PYG{p}{(}\PYG{l+s+s2}{\PYGZdq{}QUIT\PYGZdq{}}\PYG{p}{,}\PYG{l+m+mi}{400}\PYG{p}{,}\PYG{l+m+mi}{250}\PYG{p}{,}\PYG{n+nv+vg}{gWin}\PYG{p}{,}\PYG{l+m+mi}{150}\PYG{p}{)}
\PYG{n+nf}{WaitForClickOnTarget}\PYG{p}{(}\PYG{p}{[}\PYG{n+nv}{done}\PYG{p}{]}\PYG{p}{,}\PYG{p}{[}\PYG{l+m+mi}{1}\PYG{p}{]}\PYG{p}{)}
\PYG{n+nf}{Clickon}\PYG{p}{(}\PYG{n+nv}{done}\PYG{p}{,}\PYG{n+nv+vg}{gClick}\PYG{p}{)}
\PYG{n+nf}{DrawObject}\PYG{p}{(}\PYG{n+nv}{done}\PYG{p}{)}
\end{sphinxVerbatim}

\sphinxAtStartPar
\sphinxstylestrong{See Also:}

\sphinxAtStartPar
\sphinxcode{\sphinxupquote{Inside()}}, \sphinxcode{\sphinxupquote{ClickOnCheckbox()}}, \sphinxcode{\sphinxupquote{MoveObject()}}, \sphinxcode{\sphinxupquote{Draw()}}

\index{EasyLabel@\spxentry{EasyLabel}}\ignorespaces 

\subsection{EasyLabel()}
\label{\detokenize{reference/utility:easylabel}}\label{\detokenize{reference/utility:index-6}}
\sphinxAtStartPar
\sphinxstylestrong{Description:}

\sphinxAtStartPar
Creates and adds to the window location a label   at specified location. Uses standard vera font with grey background.    (May in the future get background color from window).   Easy\sphinxhyphen{}to\sphinxhyphen{}use replacement for the \sphinxcode{\sphinxupquote{MakeFont}},  \textasciitilde{}\textasciigrave{}\textasciigrave{}MakeLabel\textasciigrave{}\textasciigrave{},  \textasciitilde{} \sphinxcode{\sphinxupquote{AddObject}}, \textasciitilde{} \sphinxcode{\sphinxupquote{Move}}, steps you typically have to go through.    The optional argument fontsize defaults to 16\sphinxhyphen{}point.  The optional argument fg specifies a color name (e.g., \sphinxcode{\sphinxupquote{"red"}}) to use, and style specifies the font style, where 0,1,2,3 = normal, italic, bold, bolditalic.

\sphinxAtStartPar
\sphinxstylestrong{Usage:}

\begin{sphinxVerbatim}[commandchars=\\\{\}]
\PYG{k}{define}\PYG{+w}{ }\PYG{n+nf}{EasyLabel}\PYG{p}{(}\PYG{p}{.}\PYG{p}{.}\PYG{p}{.}\PYG{p}{)}
\end{sphinxVerbatim}

\sphinxAtStartPar
\sphinxstylestrong{Example:}

\begin{sphinxVerbatim}[commandchars=\\\{\}]
\PYG{n+nv}{win}\PYG{+w}{ }\PYG{o}{\PYGZlt{}\PYGZhy{}}\PYG{+w}{ }\PYG{n+nf}{MakeWindow}\PYG{p}{(}\PYG{p}{)}
\PYG{n+nv}{lab}\PYG{+w}{ }\PYG{o}{\PYGZlt{}\PYGZhy{}}\PYG{+w}{ }\PYG{n+nf}{EasyLabel}\PYG{p}{(}\PYG{l+s+s2}{\PYGZdq{}What?\PYGZdq{}}\PYG{p}{,}\PYG{l+m+mi}{200}\PYG{p}{,}\PYG{l+m+mi}{100}\PYG{p}{,}\PYG{n+nv}{win}\PYG{p}{)}
\PYG{n+nf}{Draw}\PYG{p}{(}\PYG{p}{)}
\PYG{n+nv}{lab}\PYG{+w}{ }\PYG{o}{\PYGZlt{}\PYGZhy{}}\PYG{+w}{ }\PYG{n+nf}{EasyLabel}\PYG{p}{(}\PYG{l+s+s2}{\PYGZdq{}What?\PYGZdq{}}\PYG{p}{,}\PYG{l+m+mi}{200}\PYG{p}{,}\PYG{l+m+mi}{100}\PYG{p}{,}\PYG{n+nv}{win}\PYG{p}{,}\PYG{l+m+mi}{12}\PYG{p}{)}
\end{sphinxVerbatim}

\sphinxAtStartPar
\sphinxstylestrong{See Also:}

\sphinxAtStartPar
\sphinxcode{\sphinxupquote{EasyTextBox()}}, \sphinxcode{\sphinxupquote{MakeLabel()}}

\index{EasyTextBox@\spxentry{EasyTextBox}}\ignorespaces 

\subsection{EasyTextBox()}
\label{\detokenize{reference/utility:easytextbox}}\label{\detokenize{reference/utility:index-7}}
\sphinxAtStartPar
\sphinxstylestrong{Description:}

\sphinxAtStartPar
Creates and adds to the window location a textbox   at specified location. Uses standard vera font with white background.   Easy\sphinxhyphen{}to\sphinxhyphen{}use replacement for the MakeFont,MakeTextBox,   AddObject, Move, steps.    The optional arguments fgcolor and bgcolor should specify color names (like white and black).  By default, the textbox is created with a foreground of \sphinxcode{\sphinxupquote{"black"}} and a background of \sphinxcode{\sphinxupquote{"white"}}.

\sphinxAtStartPar
\sphinxstylestrong{Usage:}

\begin{sphinxVerbatim}[commandchars=\\\{\}]
\PYG{k}{define}\PYG{+w}{ }\PYG{n+nf}{EasyTextBox}\PYG{p}{(}\PYG{p}{.}\PYG{p}{.}\PYG{p}{.}\PYG{p}{)}
\end{sphinxVerbatim}

\sphinxAtStartPar
\sphinxstylestrong{Example:}

\begin{sphinxVerbatim}[commandchars=\\\{\}]
\PYG{n+nv}{win}\PYG{+w}{ }\PYG{o}{\PYGZlt{}\PYGZhy{}}\PYG{+w}{ }\PYG{n+nf}{MakeWindow}\PYG{p}{(}\PYG{p}{)}
\PYG{n+nv}{entry}\PYG{+w}{ }\PYG{o}{\PYGZlt{}\PYGZhy{}}\PYG{+w}{ }\PYG{n+nf}{EasyTextBox}\PYG{p}{(}\PYG{l+s+s2}{\PYGZdq{}1 2 3 4 5\PYGZdq{}}\PYG{p}{,}\PYG{l+m+mi}{200}\PYG{p}{,}\PYG{l+m+mi}{100}\PYG{p}{,}
\PYG{+w}{                      }\PYG{n+nv}{win}\PYG{p}{,}\PYG{l+m+mi}{12}\PYG{p}{,}\PYG{l+m+mi}{200}\PYG{p}{,}\PYG{l+m+mi}{50}\PYG{p}{)}
\PYG{n+nf}{Draw}\PYG{p}{(}\PYG{p}{)}
\PYG{n+nv}{entry}\PYG{+w}{ }\PYG{o}{\PYGZlt{}\PYGZhy{}}\PYG{+w}{ }\PYG{n+nf}{EasyTextBox}\PYG{p}{(}\PYG{l+s+s2}{\PYGZdq{}1 2 3 4 5\PYGZdq{}}\PYG{p}{,}\PYG{l+m+mi}{200}\PYG{p}{,}\PYG{l+m+mi}{100}\PYG{p}{,}
\PYG{+w}{                      }\PYG{n+nv}{win}\PYG{p}{,}\PYG{l+m+mi}{12}\PYG{p}{,}\PYG{l+m+mi}{200}\PYG{p}{,}\PYG{l+m+mi}{50}\PYG{p}{,}\PYG{l+s+s2}{\PYGZdq{}red\PYGZdq{}}\PYG{p}{,}\PYG{l+s+s2}{\PYGZdq{}blue\PYGZdq{}}\PYG{p}{)}
\PYG{n+nf}{Draw}\PYG{p}{(}\PYG{p}{)}
\end{sphinxVerbatim}

\sphinxAtStartPar
\sphinxstylestrong{See Also:}

\sphinxAtStartPar
\sphinxcode{\sphinxupquote{EasyLabel()}}, \sphinxcode{\sphinxupquote{MakeTextBox()}}

\index{EndsWith@\spxentry{EndsWith}}\ignorespaces 

\subsection{EndsWith()}
\label{\detokenize{reference/utility:endswith}}\label{\detokenize{reference/utility:index-8}}
\sphinxAtStartPar
\sphinxstylestrong{Description:}


\subsubsection{Parameter file handling \sphinxhyphen{} supports both legacy CSV and modern JSON formats}
\label{\detokenize{reference/utility:parameter-file-handling-supports-both-legacy-csv-and-modern-json-formats}}
\sphinxAtStartPar
Helper function: Check if a string ends with a given suffix
Used to detect .json file extensions for parameter files

\sphinxAtStartPar
\sphinxstylestrong{Usage:}

\begin{sphinxVerbatim}[commandchars=\\\{\}]
\PYG{k}{define}\PYG{+w}{ }\PYG{n+nf}{EndsWith}\PYG{p}{(}\PYG{n+nv}{string}\PYG{p}{,}\PYG{+w}{ }\PYG{n+nv}{suffix}\PYG{p}{)}
\end{sphinxVerbatim}

\index{Enquote@\spxentry{Enquote}}\ignorespaces 

\subsection{Enquote()}
\label{\detokenize{reference/utility:enquote}}\label{\detokenize{reference/utility:index-9}}
\sphinxAtStartPar
\sphinxstyleemphasis{Returns string surrounded by quote marks.}

\sphinxAtStartPar
\sphinxstylestrong{Description:}

\sphinxAtStartPar
Surrounds the argument with quotes.

\sphinxAtStartPar
\sphinxstylestrong{Usage:}

\begin{sphinxVerbatim}[commandchars=\\\{\}]
\PYG{k}{define}\PYG{+w}{ }\PYG{n+nf}{Enquote}\PYG{p}{(}\PYG{p}{.}\PYG{p}{.}\PYG{p}{.}\PYG{p}{)}
\end{sphinxVerbatim}

\sphinxAtStartPar
\sphinxstylestrong{Example:}

\begin{sphinxVerbatim}[commandchars=\\\{\}]
\PYG{c+c1}{\PYGZsh{}\PYGZsh{}use to add quoted text to instructions.}
\PYG{n+nv}{instructions}\PYG{+w}{ }\PYG{o}{\PYGZlt{}\PYGZhy{}}\PYG{+w}{ }\PYG{l+s+s2}{\PYGZdq{}Respond whenever you see an \PYGZdq{}}\PYG{o}{+}
\PYG{+w}{                 }\PYG{n+nf}{Enquote}\PYG{p}{(}\PYG{l+s+s2}{\PYGZdq{}X\PYGZdq{}}\PYG{p}{)}

\PYG{+w}{ }\PYG{c+c1}{\PYGZsh{}\PYGZsh{}Use it for saving data that may have spaces:}
\PYG{+w}{ }\PYG{n+nv}{resp}\PYG{+w}{ }\PYG{o}{\PYGZlt{}\PYGZhy{}}\PYG{+w}{  }\PYG{n+nf}{GetInput}\PYG{p}{(}\PYG{n+nv}{tb}\PYG{p}{,}\PYG{+w}{ }\PYG{l+s+s2}{\PYGZdq{}\PYGZlt{}enter\PYGZgt{}\PYGZdq{}}\PYG{p}{)}
\PYG{+w}{ }\PYG{n+nf}{FilePrint}\PYG{p}{(}\PYG{n+nv}{fileout}\PYG{p}{,}\PYG{+w}{ }\PYG{n+nf}{Enquote}\PYG{p}{(}\PYG{n+nv}{resp}\PYG{p}{)}\PYG{p}{)}
\end{sphinxVerbatim}

\sphinxAtStartPar
\sphinxstylestrong{See Also:}

\sphinxAtStartPar
gQuote

\index{FetchText@\spxentry{FetchText}}\ignorespaces 

\subsection{FetchText()}
\label{\detokenize{reference/utility:fetchtext}}\label{\detokenize{reference/utility:index-10}}
\sphinxAtStartPar
\sphinxstylestrong{Description:}

\sphinxAtStartPar
Helper function: Fetch text from URL or local file
Handles both \sphinxurl{http://} URLs and local file paths

\sphinxAtStartPar
\sphinxstylestrong{Usage:}

\begin{sphinxVerbatim}[commandchars=\\\{\}]
\PYG{k}{define}\PYG{+w}{ }\PYG{n+nf}{FetchText}\PYG{p}{(}\PYG{n+nv}{source}\PYG{p}{)}
\end{sphinxVerbatim}

\index{FilePrintList@\spxentry{FilePrintList}}\ignorespaces 

\subsection{FilePrintList()}
\label{\detokenize{reference/utility:fileprintlist}}\label{\detokenize{reference/utility:index-11}}
\sphinxAtStartPar
\sphinxstylestrong{Description:}

\sphinxAtStartPar
Prints a list to a file, without the ‘,’s or {[}{]}   characters. Puts a carriage return at the end.  Returns a string   that was printed.  If a list contains other lists, the printing will   wrap multiple lines and the internal lists will be printed as   normal.  To avoid this, try FilePrintList(file,Flatten(list)).

\sphinxAtStartPar
\sphinxstylestrong{Usage:}

\begin{sphinxVerbatim}[commandchars=\\\{\}]
\PYG{k}{define}\PYG{+w}{ }\PYG{n+nf}{FilePrintList}\PYG{p}{(}\PYG{p}{.}\PYG{p}{.}\PYG{p}{.}\PYG{p}{)}
\end{sphinxVerbatim}

\sphinxAtStartPar
\sphinxstylestrong{Example:}

\begin{sphinxVerbatim}[commandchars=\\\{\}]
\PYG{n+nf}{FilePrintList}\PYG{p}{(}\PYG{n+nv}{fstream}\PYG{p}{,}\PYG{+w}{ }\PYG{p}{[}\PYG{l+m+mi}{1}\PYG{p}{,}\PYG{l+m+mi}{2}\PYG{p}{,}\PYG{l+m+mi}{3}\PYG{p}{,}\PYG{l+m+mi}{4}\PYG{p}{,}\PYG{l+m+mi}{5}\PYG{p}{,}\PYG{l+m+mi}{5}\PYG{p}{,}\PYG{l+m+mi}{5}\PYG{p}{]}\PYG{p}{)}
\PYG{c+c1}{\PYGZsh{}\PYGZsh{}}
\PYG{c+c1}{\PYGZsh{}\PYGZsh{}  Produces:}
\PYG{c+c1}{\PYGZsh{}\PYGZsh{}1 2 3 4 5 5 5}
\PYG{n+nf}{FilePrintList}\PYG{p}{(}\PYG{n+nv}{fstream}\PYG{p}{,}\PYG{p}{[}\PYG{p}{[}\PYG{l+m+mi}{1}\PYG{p}{,}\PYG{l+m+mi}{2}\PYG{p}{]}\PYG{p}{,}\PYG{p}{[}\PYG{l+m+mi}{3}\PYG{p}{,}\PYG{l+m+mi}{4}\PYG{p}{]}\PYG{p}{,}\PYG{p}{[}\PYG{l+m+mi}{5}\PYG{p}{,}\PYG{l+m+mi}{6}\PYG{p}{]}\PYG{p}{]}\PYG{p}{)}
\PYG{c+c1}{\PYGZsh{}Produces:}
\PYG{c+c1}{\PYGZsh{} [1,2]}
\PYG{c+c1}{\PYGZsh{},[3,4]}
\PYG{c+c1}{\PYGZsh{},[5,6]}

\PYG{n+nf}{FilePrintList}\PYG{p}{(}\PYG{n+nv}{fstream}\PYG{p}{,}\PYG{n+nf}{Flatten}\PYG{p}{(}\PYG{p}{[}\PYG{p}{[}\PYG{l+m+mi}{1}\PYG{p}{,}\PYG{l+m+mi}{2}\PYG{p}{]}\PYG{p}{,}\PYG{p}{[}\PYG{l+m+mi}{3}\PYG{p}{,}\PYG{l+m+mi}{4}\PYG{p}{]}\PYG{p}{,}\PYG{p}{[}\PYG{l+m+mi}{5}\PYG{p}{,}\PYG{l+m+mi}{6}\PYG{p}{]}\PYG{p}{]}\PYG{p}{)}\PYG{p}{)}
\PYG{c+c1}{\PYGZsh{}Produces:}
\PYG{c+c1}{\PYGZsh{} 1 2 3 4 5 6}
\end{sphinxVerbatim}

\sphinxAtStartPar
\sphinxstylestrong{See Also:}

\sphinxAtStartPar
\sphinxcode{\sphinxupquote{Print()}}, \sphinxcode{\sphinxupquote{Print\_()}}, \sphinxcode{\sphinxupquote{FilePrint()}}, \sphinxcode{\sphinxupquote{FilePrint\_()}}, \sphinxcode{\sphinxupquote{PrintList()}},

\index{FormatText@\spxentry{FormatText}}\ignorespaces 

\subsection{FormatText()}
\label{\detokenize{reference/utility:formattext}}\label{\detokenize{reference/utility:index-12}}
\sphinxAtStartPar
\sphinxstylestrong{Description:}

\sphinxAtStartPar
this works at replacing carriage returns (n) etc. from text

\sphinxAtStartPar
\sphinxstylestrong{Usage:}

\begin{sphinxVerbatim}[commandchars=\\\{\}]
\PYG{k}{define}\PYG{+w}{ }\PYG{n+nf}{FormatText}\PYG{p}{(}\PYG{n+nv}{text}\PYG{p}{)}
\end{sphinxVerbatim}

\index{GetEasyChoice@\spxentry{GetEasyChoice}}\ignorespaces 

\subsection{GetEasyChoice()}
\label{\detokenize{reference/utility:geteasychoice}}\label{\detokenize{reference/utility:index-13}}
\sphinxAtStartPar
\sphinxstyleemphasis{Simple multiple choice}

\sphinxAtStartPar
\sphinxstylestrong{Description:}

\sphinxAtStartPar
Hides what is on the screen and presents a textbox with   specified message, and a series of options to select from. Returns element from corresponding position of the \sphinxcode{\sphinxupquote{\textless{}output\textgreater{}}} list.

\sphinxAtStartPar
\sphinxstylestrong{Usage:}

\begin{sphinxVerbatim}[commandchars=\\\{\}]
\PYG{k}{define}\PYG{+w}{ }\PYG{n+nf}{GetEasyChoice}\PYG{p}{(}\PYG{p}{.}\PYG{p}{.}\PYG{p}{.}\PYG{p}{)}
\end{sphinxVerbatim}

\sphinxAtStartPar
\sphinxstylestrong{Example:}

\begin{sphinxVerbatim}[commandchars=\\\{\}]
T\PYG{n+nv}{he}\PYG{+w}{ }\PYG{n+nv}{code}\PYG{+w}{ }\PYG{n+nv}{snippet}\PYG{+w}{ }\PYG{n+nv}{below}\PYG{+w}{ }\PYG{n+nv}{produces}\PYG{+w}{ }\PYG{n+nv}{the}\PYG{+w}{ }\PYG{n+nv}{following}\PYG{+w}{ }\PYG{n+nv}{screen}\PYG{o}{:}\PYG{+w}{ }\PYG{n+nv+vg}{gWin}\PYG{+w}{ }\PYG{o}{\PYGZlt{}\PYGZhy{}}\PYG{+w}{ }\PYG{n+nf}{MakeWindow}\PYG{p}{(}\PYG{l+s+s2}{\PYGZdq{}white\PYGZdq{}}\PYG{p}{)}
\PYG{+w}{ }\PYG{n+nv}{inp}\PYG{+w}{ }\PYG{o}{\PYGZlt{}\PYGZhy{}}\PYG{+w}{  }\PYG{n+nf}{GetEasyChoice}\PYG{p}{(}\PYG{l+s+s2}{\PYGZdq{}What Year are you in school\PYGZdq{}}\PYG{p}{,}
\PYG{+w}{                        }\PYG{p}{[}\PYG{l+s+s2}{\PYGZdq{}First\PYGZhy{}year\PYGZdq{}}\PYG{p}{,}\PYG{l+s+s2}{\PYGZdq{}Sophomore\PYGZdq{}}\PYG{p}{,}
\PYG{+w}{                        }\PYG{l+s+s2}{\PYGZdq{}Junior\PYGZdq{}}\PYG{p}{,}\PYG{l+s+s2}{\PYGZdq{}Senior\PYGZdq{}}\PYG{p}{,}\PYG{l+s+s2}{\PYGZdq{}Other\PYGZdq{}}\PYG{p}{]}\PYG{p}{,}
\PYG{+w}{                        }\PYG{p}{[}\PYG{l+m+mi}{1}\PYG{p}{,}\PYG{l+m+mi}{2}\PYG{p}{,}\PYG{l+m+mi}{3}\PYG{p}{,}\PYG{l+m+mi}{4}\PYG{p}{,}\PYG{l+m+mi}{5}\PYG{p}{]}\PYG{p}{,}\PYG{+w}{  }\PYG{n+nv+vg}{gWin}\PYG{p}{)}
\end{sphinxVerbatim}

\sphinxAtStartPar
\sphinxstylestrong{See Also:}

\sphinxAtStartPar
\sphinxcode{\sphinxupquote{MessageBox()}}, \sphinxcode{\sphinxupquote{GetEasyChoice()}}, \sphinxcode{\sphinxupquote{EasyTextBox()}}

\index{GetEasyInput@\spxentry{GetEasyInput}}\ignorespaces 

\subsection{GetEasyInput()}
\label{\detokenize{reference/utility:geteasyinput}}\label{\detokenize{reference/utility:index-14}}
\sphinxAtStartPar
\sphinxstyleemphasis{Gets typed input based on a prompt.}

\sphinxAtStartPar
\sphinxstylestrong{Description:}

\sphinxAtStartPar
Hides what is on the screen and presents a textbox with   specified message, and a second text box to enter input.  Continues   when ‘enter’ it hit at the end of text entry.

\sphinxAtStartPar
\sphinxstylestrong{Usage:}

\begin{sphinxVerbatim}[commandchars=\\\{\}]
\PYG{k}{define}\PYG{+w}{ }\PYG{n+nf}{GetEasyInput}\PYG{p}{(}\PYG{p}{.}\PYG{p}{.}\PYG{p}{.}\PYG{p}{)}
\end{sphinxVerbatim}

\sphinxAtStartPar
\sphinxstylestrong{Example:}

\begin{sphinxVerbatim}[commandchars=\\\{\}]
\PYG{n+nv+vg}{gWin}\PYG{+w}{ }\PYG{o}{\PYGZlt{}\PYGZhy{}}\PYG{+w}{ }\PYG{n+nf}{MakeWindow}\PYG{p}{(}\PYG{p}{)}
\PYG{n+nv}{inp}\PYG{+w}{ }\PYG{o}{\PYGZlt{}\PYGZhy{}}\PYG{+w}{  }\PYG{n+nf}{GetEasyInput}\PYG{p}{(}\PYG{l+s+s2}{\PYGZdq{}Enter Participant ID Code\PYGZdq{}}\PYG{p}{,}\PYG{n+nv+vg}{gWin}\PYG{p}{)}
\end{sphinxVerbatim}

\sphinxAtStartPar
\sphinxstylestrong{See Also:}

\sphinxAtStartPar
\sphinxcode{\sphinxupquote{MessageBox()}}, \sphinxcode{\sphinxupquote{GetEasyChoice()}}, \sphinxcode{\sphinxupquote{EasyTextBox()}}

\index{GetEasyMultiChoice@\spxentry{GetEasyMultiChoice}}\ignorespaces 

\subsection{GetEasyMultiChoice()}
\label{\detokenize{reference/utility:geteasymultichoice}}\label{\detokenize{reference/utility:index-15}}
\sphinxAtStartPar
\sphinxstyleemphasis{Simple select\sphinxhyphen{}multiple choice}

\sphinxAtStartPar
\sphinxstylestrong{Description:}

\sphinxAtStartPar
The minchoices and maxchoices gives the range of the number of choices permitted.

\sphinxAtStartPar
\sphinxstylestrong{Usage:}

\begin{sphinxVerbatim}[commandchars=\\\{\}]
\PYG{k}{define}\PYG{+w}{ }\PYG{n+nf}{GetEasyMultiChoice}\PYG{p}{(}\PYG{n+nv}{text}\PYG{p}{,}\PYG{n+nv}{choices}\PYG{p}{,}\PYG{n+nv}{output}\PYG{p}{,}\PYG{n+nv}{win}\PYG{p}{,}\PYG{n+nv}{minchoices}\PYG{o}{:}\PYG{l+m+mi}{1}\PYG{p}{,}\PYG{+w}{ }\PYG{n+nv}{maxchoices}\PYG{o}{:}\PYG{l+m+mi}{1}\PYG{p}{)}
\end{sphinxVerbatim}

\index{GetNewDataFile@\spxentry{GetNewDataFile}}\ignorespaces 

\subsection{GetNewDataFile()}
\label{\detokenize{reference/utility:getnewdatafile}}\label{\detokenize{reference/utility:index-16}}
\sphinxAtStartPar
\sphinxstyleemphasis{Opens a data file in subnum directory}

\sphinxAtStartPar
\sphinxstylestrong{Description:}

\sphinxAtStartPar
Creates a data file for output, asking for either append or renumbering the subject code if the specified file is already in use.

\sphinxAtStartPar
\sphinxstylestrong{Usage:}

\begin{sphinxVerbatim}[commandchars=\\\{\}]
\PYG{k}{define}\PYG{+w}{ }\PYG{n+nf}{GetNewDataFile}\PYG{p}{(}\PYG{p}{.}\PYG{p}{.}\PYG{p}{.}\PYG{p}{)}
\end{sphinxVerbatim}

\sphinxAtStartPar
\sphinxstylestrong{Example:}

\begin{sphinxVerbatim}[commandchars=\\\{\}]
\PYG{+w}{  }\PYG{n+nv}{file1}\PYG{+w}{ }\PYG{o}{\PYGZlt{}\PYGZhy{}}\PYG{+w}{ }\PYG{n+nf}{GetNewDataFile}\PYG{p}{(}\PYG{l+s+s2}{\PYGZdq{}1\PYGZdq{}}\PYG{p}{,}\PYG{n+nv+vg}{gWin}\PYG{p}{,}\PYG{l+s+s2}{\PYGZdq{}memorytest\PYGZdq{}}\PYG{p}{,}\PYG{l+s+s2}{\PYGZdq{}csv\PYGZdq{}}\PYG{p}{,}
\PYG{+w}{                  }\PYG{l+s+s2}{\PYGZdq{}sub,trial,word,answer,rt,corr\PYGZdq{}}\PYG{p}{)}
\PYG{c+c1}{\PYGZsh{}\PYGZsh{}above creates a file data\PYGZbs{}1\PYGZbs{}memorytest\PYGZhy{}1.csv}

\PYG{+w}{ }\PYG{n+nv}{file2}\PYG{+w}{ }\PYG{o}{\PYGZlt{}\PYGZhy{}}\PYG{+w}{ }\PYG{n+nf}{GetNewDataFile}\PYG{p}{(}\PYG{l+s+s2}{\PYGZdq{}1\PYGZdq{}}\PYG{p}{,}\PYG{n+nv+vg}{gWin}\PYG{p}{,}\PYG{l+s+s2}{\PYGZdq{}memorytest\PYGZdq{}}\PYG{p}{,}\PYG{l+s+s2}{\PYGZdq{}csv\PYGZdq{}}\PYG{p}{,}
\PYG{+w}{                  }\PYG{l+s+s2}{\PYGZdq{}sub,trial,word,answer,rt,corr\PYGZdq{}}\PYG{p}{)}
\PYG{c+c1}{\PYGZsh{} above will prompt you for new subject code}

\PYG{+w}{ }\PYG{n+nv}{file3}\PYG{+w}{ }\PYG{o}{\PYGZlt{}\PYGZhy{}}\PYG{+w}{ }\PYG{n+nf}{GetNewDataFile}\PYG{p}{(}\PYG{l+s+s2}{\PYGZdq{}1\PYGZdq{}}\PYG{p}{,}\PYG{n+nv+vg}{gWin}\PYG{p}{,}\PYG{l+s+s2}{\PYGZdq{}memorytest\PYGZhy{}report\PYGZdq{}}\PYG{p}{,}\PYG{l+s+s2}{\PYGZdq{}txt\PYGZdq{}}\PYG{p}{,}
\PYG{+w}{                  }\PYG{l+s+s2}{\PYGZdq{}\PYGZdq{}}\PYG{p}{)}
\PYG{c+c1}{\PYGZsh{}\PYGZsh{}No header is needed on a text\PYGZhy{}based report file.}
\end{sphinxVerbatim}

\sphinxAtStartPar
\sphinxstylestrong{See Also:}

\sphinxAtStartPar
\sphinxcode{\sphinxupquote{FileOpenWrite()}}, \sphinxcode{\sphinxupquote{FileOpenAppend()}}, \sphinxcode{\sphinxupquote{FileOpenOverwrite()}}

\index{GetNIMHDemographics@\spxentry{GetNIMHDemographics}}\ignorespaces 

\subsection{GetNIMHDemographics()}
\label{\detokenize{reference/utility:getnimhdemographics}}\label{\detokenize{reference/utility:index-17}}
\sphinxAtStartPar
\sphinxstyleemphasis{Asks NIMH\sphinxhyphen{}related questions}

\sphinxAtStartPar
\sphinxstylestrong{Description:}

\sphinxAtStartPar
Gets demographic information that are normally required for NIMH\sphinxhyphen{}related research.  Currently are gender (M/F/prefer not to say), ethnicity (Hispanic or not), and race (A.I./Alaskan, Asian/A.A., Hawaiian, black/A.A., white/Caucasian, other).               It then prints their responses in a single line in the demographics file, along with any special code you supply and a time/date stamp. This code might include a subject number, experiment number, or something else, but many informed consent forms assure the subject that this information cannot be tied back to them or their data, so be careful about what you record. The file output will look something like:

\begin{sphinxVerbatim}[commandchars=\\\{\}]
\PYGZhy{}\PYGZhy{}\PYGZhy{}\PYGZhy{}  31,Thu May 12 17:00:35 2011,F,hisp,asian,3331 32,Thu May 12 22:49:10 2011,M,nothisp,amind,3332 \PYGZhy{}\PYGZhy{}\PYGZhy{}\PYGZhy{}

     The first column is the user\PYGZhy{}specified code (in this    case, indicating the experiment number).  The middle columns    indicate date/time, and the last three columns indicate         gender (M, F, other), Hispanic (Y/N), and race.
\end{sphinxVerbatim}

\sphinxAtStartPar
\sphinxstylestrong{Usage:}

\begin{sphinxVerbatim}[commandchars=\\\{\}]
\PYG{k}{define}\PYG{+w}{ }\PYG{n+nf}{GetNIMHDemographics}\PYG{p}{(}\PYG{p}{.}\PYG{p}{.}\PYG{p}{.}\PYG{p}{)}
\end{sphinxVerbatim}

\sphinxAtStartPar
\sphinxstylestrong{Example:}

\begin{sphinxVerbatim}[commandchars=\\\{\}]
\PYG{n+nf}{GetNIMHDemographics}\PYG{p}{(}\PYG{l+s+s2}{\PYGZdq{}x0413\PYGZdq{}}\PYG{p}{,}\PYG{+w}{ }\PYG{n+nv+vg}{gwindow}\PYG{p}{,}
\PYG{+w}{                    }\PYG{l+s+s2}{\PYGZdq{}x0413\PYGZhy{}demographics.txt\PYGZdq{}}\PYG{p}{)}
\end{sphinxVerbatim}

\index{GetSubNum@\spxentry{GetSubNum}}\ignorespaces 

\subsection{GetSubNum()}
\label{\detokenize{reference/utility:getsubnum}}\label{\detokenize{reference/utility:index-18}}
\sphinxAtStartPar
\sphinxstyleemphasis{Asks user to enter subject number}

\sphinxAtStartPar
\sphinxstylestrong{Description:}

\sphinxAtStartPar
Creates dialog to ask user to input a subject code

\sphinxAtStartPar
\sphinxstylestrong{Usage:}

\begin{sphinxVerbatim}[commandchars=\\\{\}]
\PYG{k}{define}\PYG{+w}{ }\PYG{n+nf}{GetSubNum}\PYG{p}{(}\PYG{p}{.}\PYG{p}{.}\PYG{p}{.}\PYG{p}{)}
\end{sphinxVerbatim}

\sphinxAtStartPar
\sphinxstylestrong{Example:}

\begin{sphinxVerbatim}[commandchars=\\\{\}]
\PYG{c+c1}{\PYGZsh{}\PYGZsh{} Put this at the beginning of an experiment,}
\PYG{c+c1}{\PYGZsh{}\PYGZsh{} after a window gWin has been defined.}
\PYG{c+c1}{\PYGZsh{}\PYGZsh{}}
\PYG{+w}{ }\PYG{k}{if}\PYG{p}{(}\PYG{n+nv+vg}{gSubNum}\PYG{+w}{ }\PYG{o}{==}\PYG{+w}{ }\PYG{l+m+mi}{0}\PYG{p}{)}
\PYG{+w}{  }\PYG{p}{\PYGZob{}}
\PYG{+w}{    }\PYG{n+nv+vg}{gSubNum}\PYG{+w}{ }\PYG{o}{\PYGZlt{}\PYGZhy{}}\PYG{+w}{ }\PYG{n+nf}{GetSubNum}\PYG{p}{(}\PYG{n+nv+vg}{gWin}\PYG{p}{)}
\PYG{+w}{  }\PYG{p}{\PYGZcb{}}

N\PYG{n+nv}{ote}\PYG{o}{:}\PYG{+w}{ }\PYG{n+nv+vg}{gSubNum}\PYG{+w}{ }\PYG{n+nv}{can}\PYG{+w}{ }\PYG{n+nv}{also}\PYG{+w}{ }\PYG{n+nv}{be}\PYG{+w}{ }\PYG{n+nv}{set}\PYG{+w}{ }\PYG{n+nv}{from}\PYG{+w}{ }\PYG{n+nv}{the}\PYG{+w}{ }\PYG{n+nv}{command}\PYG{+w}{ }\PYG{n+nv}{line}\PYG{p}{.}
\end{sphinxVerbatim}

\index{InitializeUpload@\spxentry{InitializeUpload}}\ignorespaces 

\subsection{InitializeUpload()}
\label{\detokenize{reference/utility:initializeupload}}\label{\detokenize{reference/utility:index-19}}
\sphinxAtStartPar
\sphinxstylestrong{Description:}


\subsubsection{Token\sphinxhyphen{}based multi\sphinxhyphen{}test hosting support}
\label{\detokenize{reference/utility:token-based-multi-test-hosting-support}}
\sphinxAtStartPar
Initialize token\sphinxhyphen{}based upload configuration for uploading on emscipten branch.

\sphinxAtStartPar
This is only needed on emscripten to set up the virtual file system to store data
in an easy\sphinxhyphen{}to\sphinxhyphen{}retrieve way.  On native platforms, the Upload

\sphinxAtStartPar
Call this at the start of battery tests that will be hosted online
Reads upload.json (written by JavaScript launcher) and sets up:
\sphinxhyphen{} gToken: Study identifier
\sphinxhyphen{} gTestName: Test name (e.g., “stroop”, “flanker”)
\sphinxhyphen{} gUploadURL: Server endpoint for data upload (not needed here)
\sphinxhyphen{} gParticipant: Participant ID
\sphinxhyphen{} gDataDirectory: Centralized data path: /data/\{token\}/\{test\}/\{participant\}/
\sphinxhyphen{} gUploadSettings: Full configuration object
\sphinxhyphen{} gUseUpload: Flag indicating token mode is active

\sphinxAtStartPar
\sphinxstylestrong{Usage:}

\begin{sphinxVerbatim}[commandchars=\\\{\}]
\PYG{k}{define}\PYG{+w}{ }\PYG{n+nf}{InitializeUpload}\PYG{p}{(}\PYG{n+nv}{file}\PYG{o}{:}\PYG{l+s+s2}{\PYGZdq{}\PYGZdq{}}\PYG{p}{)}
\end{sphinxVerbatim}

\index{Inside@\spxentry{Inside}}\ignorespaces 

\subsection{Inside()}
\label{\detokenize{reference/utility:inside}}\label{\detokenize{reference/utility:index-20}}
\sphinxAtStartPar
\sphinxstyleemphasis{Determines whether a point is inside a graphical object}

\sphinxAtStartPar
\sphinxstylestrong{Description:}

\sphinxAtStartPar
Determines whether an \sphinxcode{\sphinxupquote{{[}x,y{]}}} point is inside another   object.  Will operate correctly for rectangles, squares, circles,   textboxes, images, and labels. For custom objects having a function name bound to their .inside property, it will use that function to test for insideness. \sphinxcode{\sphinxupquote{{[}xylist{]}}} can be a list containing   {[}x,y{]}, and if it is longer the other points will be ignored (such as   the list returned by \sphinxcode{\sphinxupquote{WaitForMouseButton()}}.  Returns 1 if inside, 0   if not inside.

\sphinxAtStartPar
\sphinxstylestrong{Usage:}

\begin{sphinxVerbatim}[commandchars=\\\{\}]
\PYG{k}{define}\PYG{+w}{ }\PYG{n+nf}{Inside}\PYG{p}{(}\PYG{p}{.}\PYG{p}{.}\PYG{p}{.}\PYG{p}{)}
\end{sphinxVerbatim}

\sphinxAtStartPar
\sphinxstylestrong{Example:}

\begin{sphinxVerbatim}[commandchars=\\\{\}]
\PYG{n+nv}{button}\PYG{+w}{ }\PYG{o}{\PYGZlt{}\PYGZhy{}}\PYG{+w}{ }\PYG{n+nf}{EasyLabel}\PYG{p}{(}\PYG{l+s+s2}{\PYGZdq{}Click me to continue\PYGZdq{}}\PYG{p}{,}\PYG{+w}{ }\PYG{l+m+mi}{100}\PYG{p}{,}\PYG{l+m+mi}{100}\PYG{p}{,}\PYG{n+nv+vg}{gWin}\PYG{p}{,}\PYG{l+m+mi}{12}\PYG{p}{)}

\PYG{n+nv}{continue}\PYG{+w}{ }\PYG{o}{\PYGZlt{}\PYGZhy{}}\PYG{+w}{ }\PYG{l+m+mi}{1}
\PYG{k}{while}\PYG{p}{(}\PYG{n+nv}{continue}\PYG{p}{)}
\PYG{p}{\PYGZob{}}
\PYG{+w}{   }\PYG{n+nv}{xy}\PYG{+w}{ }\PYG{o}{\PYGZlt{}\PYGZhy{}}\PYG{+w}{ }\PYG{n+nf}{WaitForMouseButton}\PYG{p}{(}\PYG{p}{)}
\PYG{+w}{   }\PYG{n+nv}{continue}\PYG{+w}{ }\PYG{o}{\PYGZlt{}\PYGZhy{}}\PYG{+w}{ }\PYG{n+nf}{Inside}\PYG{p}{(}\PYG{n+nv}{xy}\PYG{p}{,}\PYG{n+nv}{button}\PYG{p}{)}
\PYG{p}{\PYGZcb{}}
\end{sphinxVerbatim}

\sphinxAtStartPar
\sphinxstylestrong{See Also:}

\sphinxAtStartPar
\sphinxcode{\sphinxupquote{WaitForMouseButton()}}, \sphinxcode{\sphinxupquote{GetMouseCursorPosition()}}, \sphinxcode{\sphinxupquote{InsideTB()}}

\index{IsURL@\spxentry{IsURL}}\ignorespaces 

\subsection{IsURL()}
\label{\detokenize{reference/utility:isurl}}\label{\detokenize{reference/utility:index-21}}
\sphinxAtStartPar
\sphinxstylestrong{Description:}

\sphinxAtStartPar
Helper function: Check if a string is a URL
Used to detect \sphinxurl{http://} or \sphinxurl{https://} URLs for remote parameter loading

\sphinxAtStartPar
\sphinxstylestrong{Usage:}

\begin{sphinxVerbatim}[commandchars=\\\{\}]
\PYG{k}{define}\PYG{+w}{ }\PYG{n+nf}{IsURL}\PYG{p}{(}\PYG{n+nv}{string}\PYG{p}{)}
\end{sphinxVerbatim}

\index{JSONText@\spxentry{JSONText}}\ignorespaces 

\subsection{JSONText()}
\label{\detokenize{reference/utility:jsontext}}\label{\detokenize{reference/utility:index-22}}
\sphinxAtStartPar
\sphinxstylestrong{Description:}

\sphinxAtStartPar
this will print the JSON object in a format that can be saved.
It requires an PCustomObject, which is created with ParseJSON() function.

\sphinxAtStartPar
\sphinxstylestrong{Usage:}

\begin{sphinxVerbatim}[commandchars=\\\{\}]
\PYG{k}{define}\PYG{+w}{ }\PYG{n+nf}{JSONText}\PYG{p}{(}\PYG{n+nv}{obj}\PYG{p}{,}\PYG{+w}{ }\PYG{n+nv}{indent}\PYG{o}{:}\PYG{l+m+mi}{0}\PYG{p}{)}
\end{sphinxVerbatim}

\index{LikertTrial@\spxentry{LikertTrial}}\ignorespaces 

\subsection{LikertTrial()}
\label{\detokenize{reference/utility:likerttrial}}\label{\detokenize{reference/utility:index-23}}
\sphinxAtStartPar
\sphinxstylestrong{Description:}

\sphinxAtStartPar
These helper functions require gTextBox, gHeader, and gFooter to work.

\sphinxAtStartPar
\sphinxstylestrong{Usage:}

\begin{sphinxVerbatim}[commandchars=\\\{\}]
\PYG{k}{define}\PYG{+w}{ }\PYG{n+nf}{LikertTrial}\PYG{p}{(}\PYG{n+nv}{text}\PYG{p}{)}
\end{sphinxVerbatim}

\index{ListToHumanText@\spxentry{ListToHumanText}}\ignorespaces 

\subsection{ListToHumanText()}
\label{\detokenize{reference/utility:listtohumantext}}\label{\detokenize{reference/utility:index-24}}
\sphinxAtStartPar
\sphinxstylestrong{Description:}

\sphinxAtStartPar
Converts a list of a text listing of options

\sphinxAtStartPar
\sphinxstylestrong{Usage:}

\begin{sphinxVerbatim}[commandchars=\\\{\}]
\PYG{k}{define}\PYG{+w}{ }\PYG{n+nf}{ListToHumanText}\PYG{p}{(}\PYG{p}{.}\PYG{p}{.}\PYG{p}{.}\PYG{p}{)}
\end{sphinxVerbatim}

\sphinxAtStartPar
\sphinxstylestrong{Example:}

\begin{sphinxVerbatim}[commandchars=\\\{\}]
\PYG{n+nf}{ListToHumanText}\PYG{p}{(}\PYG{p}{[}\PYG{l+m+mi}{1}\PYG{p}{,}\PYG{l+m+mi}{2}\PYG{p}{,}\PYG{l+m+mi}{3}\PYG{p}{,}\PYG{l+m+mi}{444}\PYG{p}{]}\PYG{p}{)}

\PYG{l+s+s2}{\PYGZdq{}1, 2, 3, or 444\PYGZdq{}}

\PYG{n+nf}{ListToHumanText}\PYG{p}{(}\PYG{p}{[}\PYG{l+s+s2}{\PYGZdq{}a\PYGZdq{}}\PYG{p}{,}\PYG{l+s+s2}{\PYGZdq{}b\PYGZdq{}}\PYG{p}{,}\PYG{l+s+s2}{\PYGZdq{}c\PYGZdq{}}\PYG{p}{,}\PYG{l+s+s2}{\PYGZdq{}d\PYGZdq{}}\PYG{p}{,}\PYG{l+s+s2}{\PYGZdq{}e\PYGZdq{}}\PYG{p}{]}\PYG{p}{,}\PYG{l+s+s2}{\PYGZdq{}and\PYGZdq{}}\PYG{p}{)}
\PYG{l+s+s2}{\PYGZdq{}a, b, c, d, and e\PYGZdq{}}
\end{sphinxVerbatim}

\sphinxAtStartPar
\sphinxstylestrong{See Also:}

\sphinxAtStartPar
\sphinxcode{\sphinxupquote{ConcatenateList()}}, \sphinxcode{\sphinxupquote{PrintList()}}, \sphinxcode{\sphinxupquote{ListToString()}}

\index{Lookup@\spxentry{Lookup}}\ignorespaces 

\subsection{Lookup()}
\label{\detokenize{reference/utility:lookup}}\label{\detokenize{reference/utility:index-25}}
\sphinxAtStartPar
\sphinxstylestrong{Description:}

\sphinxAtStartPar
Returns element in \sphinxcode{\sphinxupquote{\textless{}database\textgreater{}}} corresponding to element of \sphinxcode{\sphinxupquote{\textless{}keylist\textgreater{}}} that matches \sphinxcode{\sphinxupquote{\textless{}key\textgreater{}}}.  If no match exists, Match returns an empty list.

\sphinxAtStartPar
\sphinxstylestrong{Usage:}

\begin{sphinxVerbatim}[commandchars=\\\{\}]
\PYG{k}{define}\PYG{+w}{ }\PYG{n+nf}{Lookup}\PYG{p}{(}\PYG{p}{.}\PYG{p}{.}\PYG{p}{.}\PYG{p}{)}
\end{sphinxVerbatim}

\sphinxAtStartPar
\sphinxstylestrong{Example:}

\begin{sphinxVerbatim}[commandchars=\\\{\}]
\PYG{+w}{ }\PYG{n+nv}{keys}\PYG{+w}{     }\PYG{o}{\PYGZlt{}\PYGZhy{}}\PYG{+w}{ }\PYG{p}{[}\PYG{l+m+mi}{1}\PYG{p}{,}\PYG{l+m+mi}{2}\PYG{p}{,}\PYG{l+m+mi}{3}\PYG{p}{,}\PYG{l+m+mi}{4}\PYG{p}{,}\PYG{l+m+mi}{5}\PYG{p}{]}
\PYG{+w}{ }\PYG{n+nv}{database}\PYG{+w}{ }\PYG{o}{\PYGZlt{}\PYGZhy{}}\PYG{+w}{ }\PYG{p}{[}\PYG{l+s+s2}{\PYGZdq{}market\PYGZdq{}}\PYG{p}{,}\PYG{l+s+s2}{\PYGZdq{}home\PYGZdq{}}\PYG{p}{,}\PYG{l+s+s2}{\PYGZdq{}roast beef\PYGZdq{}}\PYG{p}{,}
\PYG{+w}{              }\PYG{l+s+s2}{\PYGZdq{}none\PYGZdq{}}\PYG{p}{,}\PYG{l+s+s2}{\PYGZdq{}wee wee wee\PYGZdq{}}\PYG{p}{]}
\PYG{+w}{ }\PYG{n+nf}{Print}\PYG{p}{(}\PYG{n+nf}{Lookup}\PYG{p}{(}\PYG{l+m+mi}{3}\PYG{p}{,}\PYG{n+nv}{keys}\PYG{p}{,}\PYG{n+nv}{database}\PYG{p}{)}\PYG{p}{)}\PYG{p}{)}

\PYG{c+c1}{\PYGZsh{}\PYGZsh{} Or, do something like this:}

\PYG{n+nv}{data}\PYG{+w}{  }\PYG{o}{\PYGZlt{}\PYGZhy{}}\PYG{+w}{ }\PYG{p}{[}\PYG{p}{[}\PYG{l+s+s2}{\PYGZdq{}punky\PYGZdq{}}\PYG{p}{,}\PYG{l+s+s2}{\PYGZdq{}brewster\PYGZdq{}}\PYG{p}{]}\PYG{p}{,}
\PYG{+w}{          }\PYG{p}{[}\PYG{l+s+s2}{\PYGZdq{}arnold\PYGZdq{}}\PYG{p}{,}\PYG{l+s+s2}{\PYGZdq{}jackson\PYGZdq{}}\PYG{p}{]}\PYG{p}{,}
\PYG{+w}{          }\PYG{p}{[}\PYG{l+s+s2}{\PYGZdq{}richie\PYGZdq{}}\PYG{p}{,}\PYG{l+s+s2}{\PYGZdq{}cunningham\PYGZdq{}}\PYG{p}{]}\PYG{p}{,}
\PYG{+w}{          }\PYG{p}{[}\PYG{l+s+s2}{\PYGZdq{}alex\PYGZdq{}}\PYG{p}{,}\PYG{l+s+s2}{\PYGZdq{}keaton\PYGZdq{}}\PYG{p}{]}\PYG{p}{]}

\PYG{n+nv}{d2}\PYG{+w}{ }\PYG{o}{\PYGZlt{}\PYGZhy{}}\PYG{+w}{ }\PYG{n+nf}{Transpose}\PYG{p}{(}\PYG{n+nv}{data}\PYG{p}{)}
\PYG{n+nv}{key}\PYG{+w}{ }\PYG{o}{\PYGZlt{}\PYGZhy{}}\PYG{+w}{ }\PYG{n+nf}{First}\PYG{p}{(}\PYG{n+nv}{data}\PYG{p}{)}

\PYG{n+nf}{Print}\PYG{p}{(}\PYG{n+nf}{Lookup}\PYG{p}{(}\PYG{l+s+s2}{\PYGZdq{}alex\PYGZdq{}}\PYG{p}{,}\PYG{+w}{ }\PYG{n+nv}{key}\PYG{p}{,}\PYG{+w}{ }\PYG{n+nv}{data}\PYG{p}{)}\PYG{p}{)}
\PYG{c+c1}{\PYGZsh{}\PYGZsh{}Returns [\PYGZdq{}alex\PYGZdq{},\PYGZdq{}keaton\PYGZdq{}]}
\end{sphinxVerbatim}

\sphinxAtStartPar
\sphinxstylestrong{See Also:}

\sphinxAtStartPar
\sphinxcode{\sphinxupquote{Match()}}

\index{MakeParameterObject@\spxentry{MakeParameterObject}}\ignorespaces 

\subsection{MakeParameterObject()}
\label{\detokenize{reference/utility:makeparameterobject}}\label{\detokenize{reference/utility:index-26}}
\sphinxAtStartPar
\sphinxstylestrong{Description:}

\sphinxAtStartPar
This creates an object called ‘parameters’ with
property\sphinxhyphen{}value pairs specified by pairs, and
will load duplicate properties into lists.

\sphinxAtStartPar
\sphinxstylestrong{Usage:}

\begin{sphinxVerbatim}[commandchars=\\\{\}]
\PYG{k}{define}\PYG{+w}{ }\PYG{n+nf}{MakeParameterObject}\PYG{p}{(}\PYG{n+nv}{pairs}\PYG{p}{)}
\end{sphinxVerbatim}

\index{MessageBox@\spxentry{MessageBox}}\ignorespaces 

\subsection{MessageBox()}
\label{\detokenize{reference/utility:messagebox}}\label{\detokenize{reference/utility:index-27}}
\sphinxAtStartPar
\sphinxstyleemphasis{Pops up a message, overtop the entire screen, and waits for a click to continue.}

\sphinxAtStartPar
\sphinxstylestrong{Description:}

\sphinxAtStartPar
Hides what is on the screen and presents a textbox with   specified message, with a button to click at the bottom to continue.   All arguments after window are optional, but permit changing the size of the text box, (left and right separately), removing the background, and allowing keyboard responses to advance.    By default, if acknowledgement is set to \textless{}OK\textgreater{}, the messagebox will continue when the mouse button clicks   an on\sphinxhyphen{}screen button labeled OK. If set to a key (e.g., ‘x’), it will continue when that key is pressed.

\sphinxAtStartPar
\sphinxstylestrong{Usage:}

\begin{sphinxVerbatim}[commandchars=\\\{\}]
\PYG{k}{define}\PYG{+w}{ }\PYG{n+nf}{MessageBox}\PYG{p}{(}\PYG{p}{.}\PYG{p}{.}\PYG{p}{.}\PYG{p}{)}
\end{sphinxVerbatim}

\sphinxAtStartPar
\sphinxstylestrong{Example:}

\begin{sphinxVerbatim}[commandchars=\\\{\}]
\PYG{n+nv+vg}{gWin}\PYG{+w}{ }\PYG{o}{\PYGZlt{}\PYGZhy{}}\PYG{+w}{ }\PYG{n+nf}{MakeWindow}\PYG{p}{(}\PYG{p}{)}
\PYG{n+nf}{MessageBox}\PYG{p}{(}\PYG{l+s+s2}{\PYGZdq{}Click below to begin.\PYGZdq{}}\PYG{p}{,}\PYG{n+nv+vg}{gWin}\PYG{p}{)}

\PYG{+w}{ }\PYG{n+nf}{MessageBox}\PYG{p}{(}\PYG{l+s+s2}{\PYGZdq{}this makes a messagebox filling the left side, permitting}
\PYG{l+s+s2}{ graphics you might have put on the right to be displayed.\PYGZdq{}}\PYG{p}{,}
\PYG{+w}{            }\PYG{n+nv+vg}{gWin}\PYG{p}{,}\PYG{l+m+mi}{40}\PYG{p}{,}\PYG{l+m+mi}{100}\PYG{p}{,}\PYG{n+nv+vg}{gVideoWidth}\PYG{o}{/}\PYG{l+m+mi}{2}\PYG{p}{,}
\PYG{+w}{            }\PYG{l+m+mi}{300}\PYG{p}{,}\PYG{l+m+mi}{0}\PYG{p}{,}\PYG{l+s+s2}{\PYGZdq{}\PYGZlt{}OK\PYGZgt{}\PYGZdq{}}\PYG{p}{)}

\PYG{+w}{ }\PYG{n+nf}{MessageBox}\PYG{p}{(}\PYG{l+s+s2}{\PYGZdq{}This messagebox allows you to continue by hitting the x or z keys\PYGZdq{}}\PYG{p}{,}
\PYG{+w}{    }\PYG{n+nv+vg}{gWin}\PYG{p}{,}\PYG{l+m+mi}{20}\PYG{p}{,}\PYG{l+m+mi}{100}\PYG{p}{,}\PYG{l+m+mi}{100}\PYG{p}{,}\PYG{l+m+mi}{300}\PYG{p}{,}\PYG{l+m+mi}{1}\PYG{p}{,}\PYG{p}{[}\PYG{l+s+s2}{\PYGZdq{}X\PYGZdq{}}\PYG{p}{,}\PYG{l+s+s2}{\PYGZdq{}Z\PYGZdq{}}\PYG{p}{]}\PYG{p}{)}
\end{sphinxVerbatim}

\sphinxAtStartPar
\sphinxstylestrong{See Also:}

\sphinxAtStartPar
\sphinxcode{\sphinxupquote{GetEasyInput()}}, \sphinxcode{\sphinxupquote{EasyTextBox()}}, \sphinxcode{\sphinxupquote{PopUpMessageBox()}}

\index{MoveCenter@\spxentry{MoveCenter}}\ignorespaces 

\subsection{MoveCenter()}
\label{\detokenize{reference/utility:movecenter}}\label{\detokenize{reference/utility:index-28}}
\sphinxAtStartPar
\sphinxstylestrong{Description:}

\sphinxAtStartPar
Moves a TextBox to a specified location                 according to its center, instead of its upper left corner.

\sphinxAtStartPar
\sphinxstylestrong{Usage:}

\begin{sphinxVerbatim}[commandchars=\\\{\}]
\PYG{k}{define}\PYG{+w}{ }\PYG{n+nf}{MoveCenter}\PYG{p}{(}\PYG{p}{.}\PYG{p}{.}\PYG{p}{.}\PYG{p}{)}
\end{sphinxVerbatim}

\sphinxAtStartPar
\sphinxstylestrong{Example:}

\begin{sphinxVerbatim}[commandchars=\\\{\}]
\PYG{n+nf}{MoveCenter}\PYG{p}{(}T\PYG{n+nv}{extBox}\PYG{p}{,}\PYG{+w}{ }\PYG{l+m+mi}{33}\PYG{p}{,}\PYG{+w}{ }\PYG{l+m+mi}{100}\PYG{p}{)}
\end{sphinxVerbatim}

\sphinxAtStartPar
\sphinxstylestrong{See Also:}

\sphinxAtStartPar
\sphinxcode{\sphinxupquote{Move()}}, \sphinxcode{\sphinxupquote{MoveCenter()}}, \sphinxcode{\sphinxupquote{.X}} and \sphinxcode{\sphinxupquote{.Y}} properties

\index{MoveCorner@\spxentry{MoveCorner}}\ignorespaces 

\subsection{MoveCorner()}
\label{\detokenize{reference/utility:movecorner}}\label{\detokenize{reference/utility:index-29}}
\sphinxAtStartPar
\sphinxstyleemphasis{Moves an image or label by its upper corner.}

\sphinxAtStartPar
\sphinxstylestrong{Description:}

\sphinxAtStartPar
Moves a label or image to a specified location          according to its upper left corner, instead of its center.

\sphinxAtStartPar
\sphinxstylestrong{Usage:}

\begin{sphinxVerbatim}[commandchars=\\\{\}]
\PYG{k}{define}\PYG{+w}{ }\PYG{n+nf}{MoveCorner}\PYG{p}{(}\PYG{p}{.}\PYG{p}{.}\PYG{p}{.}\PYG{p}{)}
\end{sphinxVerbatim}

\sphinxAtStartPar
\sphinxstylestrong{Example:}

\begin{sphinxVerbatim}[commandchars=\\\{\}]
\PYG{n+nf}{MoveCorner}\PYG{p}{(}\PYG{n+nv}{label}\PYG{p}{,}\PYG{+w}{ }\PYG{l+m+mi}{33}\PYG{p}{,}\PYG{+w}{ }\PYG{l+m+mi}{100}\PYG{p}{)}
\end{sphinxVerbatim}

\sphinxAtStartPar
\sphinxstylestrong{See Also:}

\sphinxAtStartPar
\sphinxcode{\sphinxupquote{Move()}}, \sphinxcode{\sphinxupquote{MoveCenter()}}, \sphinxcode{\sphinxupquote{.X}} and \sphinxcode{\sphinxupquote{.Y}} properties

\index{MoveObject@\spxentry{MoveObject}}\ignorespaces 

\subsection{MoveObject()}
\label{\detokenize{reference/utility:moveobject}}\label{\detokenize{reference/utility:index-30}}
\sphinxAtStartPar
\sphinxstyleemphasis{Calls the .move property of an object}

\sphinxAtStartPar
\sphinxstylestrong{Description:}

\sphinxAtStartPar
Calls the function named by the  .move property of a custom object.  Useful if a custom object has complex parts that need to be moved; you can bind .move to a custom move function and then call it (and anything else) using MoveObject. \sphinxcode{\sphinxupquote{MoveObject}} will fall back on a normal move, so you can handle movement of many built\sphinxhyphen{}in objects with it

\sphinxAtStartPar
\sphinxstylestrong{Usage:}

\begin{sphinxVerbatim}[commandchars=\\\{\}]
\PYG{k}{define}\PYG{+w}{ }\PYG{n+nf}{MoveObject}\PYG{p}{(}\PYG{p}{.}\PYG{p}{.}\PYG{p}{.}\PYG{p}{)}
\end{sphinxVerbatim}

\sphinxAtStartPar
\sphinxstylestrong{Example:}

\begin{sphinxVerbatim}[commandchars=\\\{\}]
\PYG{c+c1}{\PYGZsh{}\PYGZsh{}This overrides buttons placement at the center:}
\PYG{n+nv}{done}\PYG{+w}{ }\PYG{o}{\PYGZlt{}\PYGZhy{}}\PYG{+w}{ }\PYG{n+nf}{MakeButton}\PYG{p}{(}\PYG{l+s+s2}{\PYGZdq{}QUIT\PYGZdq{}}\PYG{p}{,}\PYG{l+m+mi}{400}\PYG{p}{,}\PYG{l+m+mi}{250}\PYG{p}{,}\PYG{n+nv+vg}{gWin}\PYG{p}{,}\PYG{l+m+mi}{150}\PYG{p}{)}
\PYG{n+nv}{done.move}\PYG{+w}{ }\PYG{o}{\PYGZlt{}\PYGZhy{}}\PYG{+w}{ }\PYG{l+s+s2}{\PYGZdq{}MoveCorner\PYGZdq{}}
\PYG{n+nf}{MoveObject}\PYG{p}{(}\PYG{n+nv}{done}\PYG{p}{,}\PYG{+w}{ }\PYG{l+m+mi}{100}\PYG{p}{,}\PYG{l+m+mi}{100}\PYG{p}{)}
\end{sphinxVerbatim}

\sphinxAtStartPar
\sphinxstylestrong{See Also:}

\sphinxAtStartPar
\sphinxcode{\sphinxupquote{Inside()}}, \sphinxcode{\sphinxupquote{Move()}}, \sphinxcode{\sphinxupquote{ClickOn()}}, \sphinxcode{\sphinxupquote{DrawObject()}}

\index{PrintList@\spxentry{PrintList}}\ignorespaces 

\subsection{PrintList()}
\label{\detokenize{reference/utility:printlist}}\label{\detokenize{reference/utility:index-31}}
\sphinxAtStartPar
\sphinxstylestrong{Description:}

\sphinxAtStartPar
Prints a list, without the ‘,’s or {[}{]}   characters. Puts a carriage return at the end.  Returns a string   that was printed.  If a list contains other lists, the printing will   wrap multiple lines and the internal lists will be printed as   normal.  To avoid this, try PrintList(Flatten(list)).

\sphinxAtStartPar
\sphinxstylestrong{Usage:}

\begin{sphinxVerbatim}[commandchars=\\\{\}]
\PYG{k}{define}\PYG{+w}{ }\PYG{n+nf}{PrintList}\PYG{p}{(}\PYG{p}{.}\PYG{p}{.}\PYG{p}{.}\PYG{p}{)}
\end{sphinxVerbatim}

\sphinxAtStartPar
\sphinxstylestrong{Example:}

\begin{sphinxVerbatim}[commandchars=\\\{\}]
\PYG{n+nf}{PrintList}\PYG{p}{(}\PYG{+w}{ }\PYG{p}{[}\PYG{l+m+mi}{1}\PYG{p}{,}\PYG{l+m+mi}{2}\PYG{p}{,}\PYG{l+m+mi}{3}\PYG{p}{,}\PYG{l+m+mi}{4}\PYG{p}{,}\PYG{l+m+mi}{5}\PYG{p}{,}\PYG{l+m+mi}{5}\PYG{p}{,}\PYG{l+m+mi}{5}\PYG{p}{]}\PYG{p}{)}
\PYG{c+c1}{\PYGZsh{}\PYGZsh{}}
\PYG{c+c1}{\PYGZsh{}\PYGZsh{}  Produces:}
\PYG{c+c1}{\PYGZsh{}\PYGZsh{}1 2 3 4 5 5 5}
\PYG{n+nf}{PrintList}\PYG{p}{(}\PYG{p}{[}\PYG{p}{[}\PYG{l+m+mi}{1}\PYG{p}{,}\PYG{l+m+mi}{2}\PYG{p}{]}\PYG{p}{,}\PYG{p}{[}\PYG{l+m+mi}{3}\PYG{p}{,}\PYG{l+m+mi}{4}\PYG{p}{]}\PYG{p}{,}\PYG{p}{[}\PYG{l+m+mi}{5}\PYG{p}{,}\PYG{l+m+mi}{6}\PYG{p}{]}\PYG{p}{]}\PYG{p}{)}
\PYG{c+c1}{\PYGZsh{}Produces:}
\PYG{c+c1}{\PYGZsh{} [1,2]}
\PYG{c+c1}{\PYGZsh{},[3,4]}
\PYG{c+c1}{\PYGZsh{},[5,6]}

\PYG{n+nf}{PrintList}\PYG{p}{(}\PYG{n+nf}{Flatten}\PYG{p}{(}\PYG{p}{[}\PYG{p}{[}\PYG{l+m+mi}{1}\PYG{p}{,}\PYG{l+m+mi}{2}\PYG{p}{]}\PYG{p}{,}\PYG{p}{[}\PYG{l+m+mi}{3}\PYG{p}{,}\PYG{l+m+mi}{4}\PYG{p}{]}\PYG{p}{,}\PYG{p}{[}\PYG{l+m+mi}{5}\PYG{p}{,}\PYG{l+m+mi}{6}\PYG{p}{]}\PYG{p}{]}\PYG{p}{)}\PYG{p}{)}
\PYG{c+c1}{\PYGZsh{}Produces:}
\PYG{c+c1}{\PYGZsh{} 1 2 3 4 5 6}
\end{sphinxVerbatim}

\sphinxAtStartPar
\sphinxstylestrong{See Also:}

\sphinxAtStartPar
\sphinxcode{\sphinxupquote{Print()}}, \sphinxcode{\sphinxupquote{Print\_()}}, \sphinxcode{\sphinxupquote{FilePrint()}}, \sphinxcode{\sphinxupquote{FilePrint\_()}}, \sphinxcode{\sphinxupquote{FilePrintList()}},

\index{ReadCSV@\spxentry{ReadCSV}}\ignorespaces 

\subsection{ReadCSV()}
\label{\detokenize{reference/utility:readcsv}}\label{\detokenize{reference/utility:index-32}}
\sphinxAtStartPar
\sphinxstyleemphasis{Opens a csv file returning a table with its elements}

\sphinxAtStartPar
\sphinxstylestrong{Description:}

\sphinxAtStartPar
Reads a comma\sphinxhyphen{}separated  value file into a nested   list.  Need not be named with a .csv extension.  It should properly   strip quotes from cells, and not break entries on commas embedded   within quoted text.

\sphinxAtStartPar
\sphinxstylestrong{Usage:}

\begin{sphinxVerbatim}[commandchars=\\\{\}]
\PYG{k}{define}\PYG{+w}{ }\PYG{n+nf}{ReadCSV}\PYG{p}{(}\PYG{p}{.}\PYG{p}{.}\PYG{p}{.}\PYG{p}{)}
\end{sphinxVerbatim}

\sphinxAtStartPar
\sphinxstylestrong{Example:}

\begin{sphinxVerbatim}[commandchars=\\\{\}]
\PYG{n+nv}{table}\PYG{+w}{ }\PYG{o}{\PYGZlt{}\PYGZhy{}}\PYG{+w}{ }\PYG{n+nf}{ReadCSV}\PYG{p}{(}\PYG{l+s+s2}{\PYGZdq{}datafile.csv\PYGZdq{}}\PYG{p}{)}
\end{sphinxVerbatim}

\sphinxAtStartPar
\sphinxstylestrong{See Also:}

\sphinxAtStartPar
\sphinxcode{\sphinxupquote{FileReadTable()}}, \sphinxcode{\sphinxupquote{FileReadList()}}, \sphinxcode{\sphinxupquote{StripQuotes()}}

\index{ReadJSONParameters@\spxentry{ReadJSONParameters}}\ignorespaces 

\subsection{ReadJSONParameters()}
\label{\detokenize{reference/utility:readjsonparameters}}\label{\detokenize{reference/utility:index-33}}
\sphinxAtStartPar
\sphinxstylestrong{Description:}

\sphinxAtStartPar
Read JSON parameter file and convert to parameter object
JSON format: simple key\sphinxhyphen{}value pairs, e.g., \{“dopractice”: 1, “isi”: 1000\}
Returns a custom object with parameters as properties
Supports both local files and URLs (\sphinxurl{http://} or \sphinxurl{https://})
Returns empty parameter object if fetch/parse fails (allows fallback to defaults)

\sphinxAtStartPar
\sphinxstylestrong{Usage:}

\begin{sphinxVerbatim}[commandchars=\\\{\}]
\PYG{k}{define}\PYG{+w}{ }\PYG{n+nf}{ReadJSONParameters}\PYG{p}{(}\PYG{n+nv}{filename}\PYG{p}{)}
\end{sphinxVerbatim}

\index{RemoveObjects@\spxentry{RemoveObjects}}\ignorespaces 

\subsection{RemoveObjects()}
\label{\detokenize{reference/utility:removeobjects}}\label{\detokenize{reference/utility:index-34}}
\sphinxAtStartPar
\sphinxstyleemphasis{Removes a (possibly nested) list of objects from a parent window}

\sphinxAtStartPar
\sphinxstylestrong{Description:}

\sphinxAtStartPar
This is a recursive removeobjects

\sphinxAtStartPar
\sphinxstylestrong{Usage:}

\begin{sphinxVerbatim}[commandchars=\\\{\}]
\PYG{k}{define}\PYG{+w}{ }\PYG{n+nf}{RemoveObjects}\PYG{p}{(}\PYG{n+nv}{list}\PYG{p}{,}\PYG{n+nv}{win}\PYG{p}{)}
\end{sphinxVerbatim}

\index{ReplaceChar@\spxentry{ReplaceChar}}\ignorespaces 

\subsection{ReplaceChar()}
\label{\detokenize{reference/utility:replacechar}}\label{\detokenize{reference/utility:index-35}}
\sphinxAtStartPar
\sphinxstylestrong{Description:}

\sphinxAtStartPar
Substitutes  \sphinxcode{\sphinxupquote{\textless{}char2\textgreater{}}} for \sphinxcode{\sphinxupquote{\textless{}char\textgreater{}}}   in \sphinxcode{\sphinxupquote{\textless{}string\textgreater{}}}. Useful for saving subject entry data in a file;   replacing spaces with some other character.  The second argument can either be a character to match, or a list of characters to match, in which case they all get replaced with the third argument.

\sphinxAtStartPar
\sphinxstylestrong{Usage:}

\begin{sphinxVerbatim}[commandchars=\\\{\}]
\PYG{k}{define}\PYG{+w}{ }\PYG{n+nf}{ReplaceChar}\PYG{p}{(}\PYG{p}{.}\PYG{p}{.}\PYG{p}{.}\PYG{p}{)}
\end{sphinxVerbatim}

\sphinxAtStartPar
\sphinxstylestrong{Example:}

\begin{sphinxVerbatim}[commandchars=\\\{\}]
\PYG{n+nv}{x}\PYG{+w}{ }\PYG{o}{\PYGZlt{}\PYGZhy{}}\PYG{+w}{ }\PYG{p}{[}\PYG{l+s+s2}{\PYGZdq{}Sing a song of sixpence\PYGZdq{}}\PYG{p}{]}
\PYG{n+nv}{rep}\PYG{+w}{ }\PYG{o}{\PYGZlt{}\PYGZhy{}}\PYG{+w}{ }\PYG{n+nf}{ReplaceChar}\PYG{p}{(}\PYG{n+nv}{x}\PYG{p}{,}\PYG{l+s+s2}{\PYGZdq{} \PYGZdq{}}\PYG{p}{,}\PYG{+w}{ }\PYG{l+s+s2}{\PYGZdq{}\PYGZus{}\PYGZdq{}}\PYG{p}{)}
\PYG{n+nf}{Print}\PYG{p}{(}\PYG{n+nv}{rep}\PYG{p}{)}
\PYG{c+c1}{\PYGZsh{} Result:  Sing\PYGZus{}a\PYGZus{}song\PYGZus{}of\PYGZus{}sixpence}

\PYG{n+nv}{x}\PYG{+w}{ }\PYG{o}{\PYGZlt{}\PYGZhy{}}\PYG{+w}{ }\PYG{p}{[}\PYG{l+s+s2}{\PYGZdq{}sing a song of sixpence\PYGZdq{}}\PYG{p}{]}
\PYG{n+nv}{rep}\PYG{+w}{ }\PYG{o}{\PYGZlt{}\PYGZhy{}}\PYG{+w}{ }\PYG{n+nf}{ReplaceChar}\PYG{p}{(}\PYG{n+nv}{x}\PYG{p}{,}\PYG{p}{[}\PYG{l+s+s2}{\PYGZdq{}s\PYGZdq{}}\PYG{p}{,}\PYG{l+s+s2}{\PYGZdq{}x\PYGZdq{}}\PYG{p}{]}\PYG{p}{,}\PYG{+w}{ }\PYG{l+s+s2}{\PYGZdq{}p\PYGZdq{}}\PYG{p}{)}
\PYG{n+nf}{Print}\PYG{p}{(}\PYG{n+nv}{rep}\PYG{p}{)}
\PYG{c+c1}{\PYGZsh{} Result:  ping a pong of pippence}
\end{sphinxVerbatim}

\sphinxAtStartPar
\sphinxstylestrong{See Also:}

\sphinxAtStartPar
for list items: \sphinxcode{\sphinxupquote{Replace()}} , \sphinxcode{\sphinxupquote{SplitString()}},

\index{SplitStringSlow@\spxentry{SplitStringSlow}}\ignorespaces 

\subsection{SplitStringSlow()}
\label{\detokenize{reference/utility:splitstringslow}}\label{\detokenize{reference/utility:index-36}}
\sphinxAtStartPar
\sphinxstylestrong{Description:}

\sphinxAtStartPar
Splits a string into tokens. \sphinxcode{\sphinxupquote{\textless{}split\textgreater{}}} must be a string. If           \sphinxcode{\sphinxupquote{\textless{}split\textgreater{}}} is not found in \sphinxcode{\sphinxupquote{\textless{}string\textgreater{}}}, a list containing the entire                  string is returned; if split is equal to \sphinxcode{\sphinxupquote{""}}, the each letter                in the string is placed into a different item in the list.  The entire text of \sphinxcode{\sphinxupquote{\textless{}split\textgreater{}}} is used to tokenize, but as a consequence this function is relatively slow, and should be avoided if your string is longer than a few hundred characters.

\sphinxAtStartPar
\sphinxstylestrong{Usage:}

\begin{sphinxVerbatim}[commandchars=\\\{\}]
\PYG{k}{define}\PYG{+w}{ }\PYG{n+nf}{SplitStringSlow}\PYG{p}{(}\PYG{p}{.}\PYG{p}{.}\PYG{p}{.}\PYG{p}{)}
\end{sphinxVerbatim}

\sphinxAtStartPar
\sphinxstylestrong{Example:}

\begin{sphinxVerbatim}[commandchars=\\\{\}]
\PYG{n+nf}{SplitStringSlow}\PYG{p}{(}\PYG{l+s+s2}{\PYGZdq{}Everybody Loves a Clown\PYGZdq{}}\PYG{p}{,}\PYG{+w}{ }\PYG{l+s+s2}{\PYGZdq{} \PYGZdq{}}\PYG{p}{)}
\PYG{c+c1}{\PYGZsh{} Produces [\PYGZdq{}Everybody\PYGZdq{}, \PYGZdq{}Loves\PYGZdq{}, \PYGZdq{}a\PYGZdq{}, \PYGZdq{}Clown\PYGZdq{}]}
\PYG{n+nf}{SplitStringSlow}\PYG{p}{(}\PYG{l+s+s2}{\PYGZdq{}she sells seashells\PYGZdq{}}\PYG{p}{,}\PYG{+w}{ }\PYG{l+s+s2}{\PYGZdq{}ll\PYGZdq{}}\PYG{p}{)}
\PYG{c+c1}{\PYGZsh{}produces [\PYGZdq{}she se\PYGZdq{},\PYGZdq{}s seashe\PYGZdq{}, \PYGZdq{}s\PYGZdq{}]}
\end{sphinxVerbatim}

\sphinxAtStartPar
\sphinxstylestrong{See Also:}

\sphinxAtStartPar
\sphinxcode{\sphinxupquote{Splitstring()}}           \sphinxcode{\sphinxupquote{FindInString()}}, \sphinxcode{\sphinxupquote{ReplaceChar()}}

\index{StripQuotes@\spxentry{StripQuotes}}\ignorespaces 

\subsection{StripQuotes()}
\label{\detokenize{reference/utility:stripquotes}}\label{\detokenize{reference/utility:index-37}}
\sphinxAtStartPar
\sphinxstylestrong{Description:}

\sphinxAtStartPar
Strips quotation marks from the outside of a   string.  Useful if you are reading in data that is quoted.

\sphinxAtStartPar
\sphinxstylestrong{Usage:}

\begin{sphinxVerbatim}[commandchars=\\\{\}]
\PYG{k}{define}\PYG{+w}{ }\PYG{n+nf}{StripQuotes}\PYG{p}{(}\PYG{p}{.}\PYG{p}{.}\PYG{p}{.}\PYG{p}{)}
\end{sphinxVerbatim}

\sphinxAtStartPar
\sphinxstylestrong{Example:}

\begin{sphinxVerbatim}[commandchars=\\\{\}]
\PYG{n+nv}{text}\PYG{+w}{ }\PYG{o}{\PYGZlt{}\PYGZhy{}}\PYG{+w}{ }\PYG{n+nv+vg}{gQuote}\PYG{+w}{ }\PYG{o}{+}\PYG{+w}{ }\PYG{l+s+s2}{\PYGZdq{}abcd\PYGZdq{}}\PYG{+w}{ }\PYG{o}{+}\PYG{+w}{ }\PYG{n+nv+vg}{gQuote}
\PYG{n+nf}{Print}\PYG{p}{(}\PYG{n+nf}{StripQuotes}\PYG{p}{(}\PYG{n+nv}{text}\PYG{p}{)}\PYG{p}{)}\PYG{+w}{  }\PYG{c+c1}{\PYGZsh{}\PYGZsh{} abcd}
\PYG{n+nf}{Print}\PYG{p}{(}\PYG{n+nf}{StripQuotes}\PYG{p}{(}\PYG{l+s+s2}{\PYGZdq{}aaa\PYGZdq{}}\PYG{p}{)}\PYG{p}{)}\PYG{+w}{ }\PYG{c+c1}{\PYGZsh{}\PYGZsh{}aaa}
\end{sphinxVerbatim}

\sphinxAtStartPar
\sphinxstylestrong{See Also:}

\sphinxAtStartPar
\sphinxcode{\sphinxupquote{StripSpace()}}

\index{StripSpace@\spxentry{StripSpace}}\ignorespaces 

\subsection{StripSpace()}
\label{\detokenize{reference/utility:stripspace}}\label{\detokenize{reference/utility:index-38}}
\sphinxAtStartPar
\sphinxstylestrong{Description:}

\sphinxAtStartPar
Strips spaces from the start and end of a   string.  Useful for cleaning up input and such.

\sphinxAtStartPar
\sphinxstylestrong{Usage:}

\begin{sphinxVerbatim}[commandchars=\\\{\}]
\PYG{k}{define}\PYG{+w}{ }\PYG{n+nf}{StripSpace}\PYG{p}{(}\PYG{p}{.}\PYG{p}{.}\PYG{p}{.}\PYG{p}{)}
\end{sphinxVerbatim}

\sphinxAtStartPar
\sphinxstylestrong{Example:}

\begin{sphinxVerbatim}[commandchars=\\\{\}]
\PYG{n+nv}{text}\PYG{+w}{ }\PYG{o}{\PYGZlt{}\PYGZhy{}}\PYG{+w}{  }\PYG{l+s+s2}{\PYGZdq{} abcd  \PYGZdq{}}
\PYG{n+nf}{Print}\PYG{p}{(}\PYG{n+nf}{StripSpace}\PYG{p}{(}\PYG{n+nv}{text}\PYG{p}{)}\PYG{p}{)}\PYG{+w}{  }\PYG{c+c1}{\PYGZsh{}\PYGZsh{} \PYGZsq{}abcd\PYGZsq{}}
\PYG{n+nf}{Print}\PYG{p}{(}\PYG{n+nf}{StripSpace}\PYG{p}{(}\PYG{l+s+s2}{\PYGZdq{}aaa\PYGZdq{}}\PYG{p}{)}\PYG{p}{)}\PYG{+w}{ }\PYG{c+c1}{\PYGZsh{}\PYGZsh{} \PYGZsq{}aaa\PYGZsq{}}
\end{sphinxVerbatim}

\sphinxAtStartPar
\sphinxstylestrong{See Also:}

\sphinxAtStartPar
\sphinxcode{\sphinxupquote{StripQuotes()}}

\index{Tab@\spxentry{Tab}}\ignorespaces 

\subsection{Tab()}
\label{\detokenize{reference/utility:tab}}\label{\detokenize{reference/utility:index-39}}
\sphinxAtStartPar
\sphinxstylestrong{Description:}

\sphinxAtStartPar
Produces a tab character which can be added to a   string. If displayed in a text box, it will use a 4\sphinxhyphen{}item tab stop.

\sphinxAtStartPar
\sphinxstylestrong{Usage:}

\begin{sphinxVerbatim}[commandchars=\\\{\}]
\PYG{k}{define}\PYG{+w}{ }\PYG{n+nf}{Tab}\PYG{p}{(}\PYG{p}{.}\PYG{p}{.}\PYG{p}{.}\PYG{p}{)}
\end{sphinxVerbatim}

\sphinxAtStartPar
\sphinxstylestrong{Example:}

\begin{sphinxVerbatim}[commandchars=\\\{\}]
\PYG{n+nf}{Print}\PYG{p}{(}\PYG{l+s+s2}{\PYGZdq{}Number: \PYGZdq{}}\PYG{+w}{  }\PYG{n+nf}{Tab}\PYG{p}{(}\PYG{l+m+mi}{1}\PYG{p}{)}\PYG{+w}{ }\PYG{o}{+}\PYG{+w}{ }\PYG{n+nv}{number}\PYG{+w}{ }\PYG{p}{)}
\PYG{n+nf}{Print}\PYG{p}{(}\PYG{l+s+s2}{\PYGZdq{}Value: \PYGZdq{}}\PYG{+w}{  }\PYG{n+nf}{Tab}\PYG{p}{(}\PYG{l+m+mi}{1}\PYG{p}{)}\PYG{+w}{ }\PYG{o}{+}\PYG{+w}{ }\PYG{n+nv}{value}\PYG{+w}{ }\PYG{p}{)}
\PYG{n+nf}{Print}\PYG{p}{(}\PYG{l+s+s2}{\PYGZdq{}Size: \PYGZdq{}}\PYG{+w}{  }\PYG{n+nf}{Tab}\PYG{p}{(}\PYG{l+m+mi}{1}\PYG{p}{)}\PYG{+w}{ }\PYG{o}{+}\PYG{+w}{ }\PYG{n+nv}{size}\PYG{+w}{ }\PYG{p}{)}
\end{sphinxVerbatim}

\sphinxAtStartPar
\sphinxstylestrong{See Also:}

\sphinxAtStartPar
\sphinxcode{\sphinxupquote{Format()}}, \sphinxcode{\sphinxupquote{CR()}}

\index{WaitForButtonClickOnTarget@\spxentry{WaitForButtonClickOnTarget}}\ignorespaces 

\subsection{WaitForButtonClickOnTarget()}
\label{\detokenize{reference/utility:waitforbuttonclickontarget}}\label{\detokenize{reference/utility:index-40}}
\sphinxAtStartPar
\sphinxstylestrong{Description:}

\sphinxAtStartPar
targetlist is a set of graphical objects,
keylist is a set of keys whose corresponding
value should be returned when a graphical object is clicked upon.
This modifies the built\sphinxhyphen{}in waitforclickontarget so that it will
Return the button that is clicked, along with the target,
and the target object

\sphinxAtStartPar
\sphinxstylestrong{Usage:}

\begin{sphinxVerbatim}[commandchars=\\\{\}]
\PYG{k}{define}\PYG{+w}{ }\PYG{n+nf}{WaitForButtonClickOnTarget}\PYG{p}{(}\PYG{n+nv}{targetlist}\PYG{p}{,}\PYG{n+nv}{keylist}\PYG{p}{)}
\end{sphinxVerbatim}

\index{WaitForClickOnTarget@\spxentry{WaitForClickOnTarget}}\ignorespaces 

\subsection{WaitForClickOnTarget()}
\label{\detokenize{reference/utility:waitforclickontarget}}\label{\detokenize{reference/utility:index-41}}
\sphinxAtStartPar
\sphinxstyleemphasis{Waits until any of a set of target objects are clicked.}

\sphinxAtStartPar
\sphinxstylestrong{Description:}

\sphinxAtStartPar
Allows you to specify a list of graphical objects in \sphinxcode{\sphinxupquote{\textless{}objectlist\textgreater{}}} and awaits a click   on any one of them, returning the corresponding key in \textless{}keylist\textgreater{}.  Also, sets the    global variable gClick which saves the location of the click, if    you need it for something else.

\sphinxAtStartPar
\sphinxstylestrong{Usage:}

\begin{sphinxVerbatim}[commandchars=\\\{\}]
\PYG{k}{define}\PYG{+w}{ }\PYG{n+nf}{WaitForClickOnTarget}\PYG{p}{(}\PYG{p}{.}\PYG{p}{.}\PYG{p}{.}\PYG{p}{)}
\end{sphinxVerbatim}

\sphinxAtStartPar
\sphinxstylestrong{Example:}

\begin{sphinxVerbatim}[commandchars=\\\{\}]
\PYG{n+nv}{resp}\PYG{+w}{ }\PYG{o}{\PYGZlt{}\PYGZhy{}}\PYG{+w}{ }\PYG{n+nf}{Sequence}\PYG{p}{(}\PYG{l+m+mi}{1}\PYG{p}{,}\PYG{l+m+mi}{5}\PYG{p}{,}\PYG{l+m+mi}{1}\PYG{p}{)}
\PYG{n+nv}{objs}\PYG{+w}{ }\PYG{o}{\PYGZlt{}\PYGZhy{}}\PYG{+w}{ }\PYG{p}{[}\PYG{p}{]}
\PYG{k}{loop}\PYG{p}{(}\PYG{n+nv}{i}\PYG{p}{,}\PYG{n+nv}{resp}\PYG{p}{)}
\PYG{p}{\PYGZob{}}
\PYG{+w}{  }\PYG{n+nv}{tmp}\PYG{+w}{ }\PYG{o}{\PYGZlt{}\PYGZhy{}}\PYG{+w}{ }\PYG{n+nf}{EasyLabel}\PYG{p}{(}\PYG{n+nv}{i}\PYG{+w}{ }\PYG{o}{+}\PYG{l+s+s2}{\PYGZdq{}. \PYGZdq{}}\PYG{p}{,}
\PYG{+w}{           }\PYG{l+m+mi}{100}\PYG{o}{+}\PYG{l+m+mi}{50}\PYG{o}{*}\PYG{n+nv}{i}\PYG{p}{,}\PYG{l+m+mi}{100}\PYG{p}{,}\PYG{n+nv+vg}{gWin}\PYG{p}{,}\PYG{l+m+mi}{25}\PYG{p}{)}
\PYG{+w}{  }\PYG{n+nv}{objs}\PYG{+w}{ }\PYG{o}{\PYGZlt{}\PYGZhy{}}\PYG{+w}{ }\PYG{n+nf}{Append}\PYG{p}{(}\PYG{n+nv}{objs}\PYG{p}{,}\PYG{+w}{ }\PYG{n+nv}{tmp}\PYG{p}{)}
\PYG{p}{\PYGZcb{}}
\PYG{n+nf}{Draw}\PYG{p}{(}\PYG{p}{)}
\PYG{n+nv}{click}\PYG{+w}{  }\PYG{o}{\PYGZlt{}\PYGZhy{}}\PYG{+w}{ }\PYG{n+nf}{WaitForClickOnTarget}\PYG{p}{(}\PYG{n+nv}{objs}\PYG{p}{,}\PYG{n+nv}{resp}\PYG{p}{)}
\PYG{n+nf}{Print}\PYG{p}{(}\PYG{l+s+s2}{\PYGZdq{}You clicked on \PYGZdq{}}\PYG{+w}{ }\PYG{o}{+}\PYG{+w}{ }\PYG{n+nv}{click}\PYG{p}{)}
\PYG{n+nf}{Print}\PYG{p}{(}\PYG{l+s+s2}{\PYGZdq{}Click location: [\PYGZdq{}}\PYG{+w}{ }\PYG{o}{+}\PYG{+w}{ }\PYG{n+nf}{First}\PYG{p}{(}\PYG{n+nv+vg}{gClick}\PYG{p}{)}\PYG{+w}{ }\PYG{o}{+}
\PYG{+w}{      }\PYG{l+s+s2}{\PYGZdq{}, \PYGZdq{}}\PYG{+w}{ }\PYG{o}{+}\PYG{+w}{ }\PYG{n+nf}{Second}\PYG{p}{(}\PYG{n+nv+vg}{gClick}\PYG{p}{)}\PYG{+w}{ }\PYG{o}{+}\PYG{+w}{ }\PYG{l+s+s2}{\PYGZdq{}]\PYGZdq{}}\PYG{p}{)}
\end{sphinxVerbatim}

\index{WaitForClickOnTargetWithTimeout@\spxentry{WaitForClickOnTargetWithTimeout}}\ignorespaces 

\subsection{WaitForClickOnTargetWithTimeout()}
\label{\detokenize{reference/utility:waitforclickontargetwithtimeout}}\label{\detokenize{reference/utility:index-42}}
\sphinxAtStartPar
\sphinxstylestrong{Description:}

\sphinxAtStartPar
Allows you to specify a list of graphical objects in \sphinxcode{\sphinxupquote{\textless{}objectlist\textgreater{}}} and awaits a click   on any one of them, returning the corresponding key in \sphinxcode{\sphinxupquote{\textless{}keylist\textgreater{}}}.  Also, sets the    global variable gClick which saves the location of the click, if    you need it for something else.  The function will return after the specified time limit.    If no response is made by timeout, the text \textless{}timeout\textgreater{} will be returned (instead of the corresponding keylist element), and gClick will be set to {[}\sphinxhyphen{}1, \sphinxhyphen{}1{]}.  This function can also be useful to dynamically update some visual object while waiting for a response.  Give timeout some small number (below 50 ms, as low as 1\sphinxhyphen{}5), and loop over this repeatedly until a ‘proper’ response is given, redrawing a timer or other dynamic visual element each time.  By default, this will only activate when a normal (left\sphinxhyphen{}click) is made on button 1.  However, the three optional arguments button1, button2, and button3 permit waiting for any or all left, right, or center buttons.

\sphinxAtStartPar
\sphinxstylestrong{Usage:}

\begin{sphinxVerbatim}[commandchars=\\\{\}]
\PYG{k}{define}\PYG{+w}{ }\PYG{n+nf}{WaitForClickOnTargetWithTimeout}\PYG{p}{(}\PYG{p}{.}\PYG{p}{.}\PYG{p}{.}\PYG{p}{)}
\end{sphinxVerbatim}

\sphinxAtStartPar
\sphinxstylestrong{Example:}

\begin{sphinxVerbatim}[commandchars=\\\{\}]
\PYG{n+nv}{resp}\PYG{+w}{ }\PYG{o}{\PYGZlt{}\PYGZhy{}}\PYG{+w}{ }\PYG{n+nf}{Sequence}\PYG{p}{(}\PYG{l+m+mi}{1}\PYG{p}{,}\PYG{l+m+mi}{5}\PYG{p}{,}\PYG{l+m+mi}{1}\PYG{p}{)}
\PYG{n+nv}{objs}\PYG{+w}{ }\PYG{o}{\PYGZlt{}\PYGZhy{}}\PYG{+w}{ }\PYG{p}{[}\PYG{p}{]}
\PYG{k}{loop}\PYG{p}{(}\PYG{n+nv}{i}\PYG{p}{,}\PYG{n+nv}{resp}\PYG{p}{)}
\PYG{p}{\PYGZob{}}
\PYG{+w}{  }\PYG{n+nv}{tmp}\PYG{+w}{ }\PYG{o}{\PYGZlt{}\PYGZhy{}}\PYG{+w}{ }\PYG{n+nf}{EasyLabel}\PYG{p}{(}\PYG{n+nv}{i}\PYG{+w}{ }\PYG{o}{+}\PYG{l+s+s2}{\PYGZdq{}. \PYGZdq{}}\PYG{p}{,}
\PYG{+w}{           }\PYG{l+m+mi}{100}\PYG{o}{+}\PYG{l+m+mi}{50}\PYG{o}{*}\PYG{n+nv}{i}\PYG{p}{,}\PYG{l+m+mi}{100}\PYG{p}{,}\PYG{n+nv+vg}{gWin}\PYG{p}{,}\PYG{l+m+mi}{25}\PYG{p}{)}
\PYG{+w}{  }\PYG{n+nv}{objs}\PYG{+w}{ }\PYG{o}{\PYGZlt{}\PYGZhy{}}\PYG{+w}{ }\PYG{n+nf}{Append}\PYG{p}{(}\PYG{n+nv}{objs}\PYG{p}{,}\PYG{+w}{ }\PYG{n+nv}{tmp}\PYG{p}{)}
\PYG{p}{\PYGZcb{}}
\PYG{n+nf}{Draw}\PYG{p}{(}\PYG{p}{)}
\PYG{n+nv}{click}\PYG{+w}{  }\PYG{o}{\PYGZlt{}\PYGZhy{}}\PYG{+w}{ }\PYG{n+nf}{WaitForClickOnTargetWithTimeout}\PYG{p}{(}\PYG{n+nv}{objs}\PYG{p}{,}\PYG{n+nv}{resp}\PYG{p}{,}\PYG{l+m+mi}{3000}\PYG{p}{)}
\PYG{n+nf}{Print}\PYG{p}{(}\PYG{l+s+s2}{\PYGZdq{}You clicked on \PYGZdq{}}\PYG{+w}{ }\PYG{o}{+}\PYG{+w}{ }\PYG{n+nv}{click}\PYG{p}{)}
\PYG{n+nf}{Print}\PYG{p}{(}\PYG{l+s+s2}{\PYGZdq{}Click location: [\PYGZdq{}}\PYG{+w}{ }\PYG{o}{+}\PYG{+w}{ }\PYG{n+nf}{First}\PYG{p}{(}\PYG{n+nv+vg}{gClick}\PYG{p}{)}\PYG{+w}{ }\PYG{o}{+}
\PYG{+w}{      }\PYG{l+s+s2}{\PYGZdq{}, \PYGZdq{}}\PYG{+w}{ }\PYG{o}{+}\PYG{+w}{ }\PYG{n+nf}{Second}\PYG{p}{(}\PYG{n+nv+vg}{gClick}\PYG{p}{)}\PYG{+w}{ }\PYG{o}{+}\PYG{+w}{ }\PYG{l+s+s2}{\PYGZdq{}]\PYGZdq{}}\PYG{p}{)}

\PYG{c+c1}{\PYGZsh{}\PYGZsh{}wait for a center\PYGZhy{}click.}
\PYG{n+nv}{click}\PYG{+w}{  }\PYG{o}{\PYGZlt{}\PYGZhy{}}\PYG{+w}{ }\PYG{n+nf}{WaitForClickOnTargetWithTimeout}\PYG{p}{(}\PYG{n+nv}{objs}\PYG{p}{,}\PYG{n+nv}{resp}\PYG{p}{,}\PYG{l+m+mi}{3000}\PYG{p}{,}\PYG{l+m+mi}{0}\PYG{p}{,}\PYG{l+m+mi}{0}\PYG{p}{,}\PYG{l+m+mi}{1}\PYG{p}{)}
\end{sphinxVerbatim}

\sphinxAtStartPar
\sphinxstylestrong{See Also:}

\sphinxAtStartPar
\sphinxcode{\sphinxupquote{WaitForDownClick()}}, \sphinxcode{\sphinxupquote{WaitForMouseButton()}}

\index{WaitForDownClick@\spxentry{WaitForDownClick}}\ignorespaces 

\subsection{WaitForDownClick()}
\label{\detokenize{reference/utility:waitfordownclick}}\label{\detokenize{reference/utility:index-43}}
\sphinxAtStartPar
\sphinxstyleemphasis{Waits for mouse button to be clicked}

\sphinxAtStartPar
\sphinxstylestrong{Description:}

\sphinxAtStartPar
Will wait until the mouse button is clicked down.  Returns   the same 4\sphinxhyphen{}tuple as \sphinxcode{\sphinxupquote{WaitForMouseButton:}}

\begin{sphinxVerbatim}[commandchars=\\\{\}]
 [xpos,    ypos,     button id [1\PYGZhy{}3],     \PYGZdq{}\PYGZlt{}pressed\PYGZgt{}\PYGZdq{} or \PYGZdq{}\PYGZlt{}released\PYGZgt{}\PYGZdq{}]

but the last element will always be ``\PYGZlt{}pressed\PYGZgt{}``.  Useful as a \PYGZsq{}click mouse to continue\PYGZsq{} probe.
\end{sphinxVerbatim}

\sphinxAtStartPar
\sphinxstylestrong{Usage:}

\begin{sphinxVerbatim}[commandchars=\\\{\}]
\PYG{k}{define}\PYG{+w}{ }\PYG{n+nf}{WaitForDownClick}\PYG{p}{(}\PYG{p}{.}\PYG{p}{.}\PYG{p}{.}\PYG{p}{)}
\end{sphinxVerbatim}

\sphinxAtStartPar
\sphinxstylestrong{Example:}

\begin{sphinxVerbatim}[commandchars=\\\{\}]
\PYG{n+nv}{x}\PYG{+w}{ }\PYG{o}{\PYGZlt{}\PYGZhy{}}\PYG{+w}{ }\PYG{n+nf}{WaitForDownClick}\PYG{p}{(}\PYG{p}{)}
\PYG{n+nf}{Print}\PYG{p}{(}\PYG{l+s+s2}{\PYGZdq{}Click location: [\PYGZdq{}}\PYG{+w}{ }\PYG{o}{+}\PYG{+w}{ }\PYG{n+nf}{First}\PYG{p}{(}\PYG{n+nv}{x}\PYG{p}{)}\PYG{+w}{ }\PYG{o}{+}
\PYG{+w}{      }\PYG{l+s+s2}{\PYGZdq{}, \PYGZdq{}}\PYG{+w}{ }\PYG{o}{+}\PYG{+w}{ }\PYG{n+nf}{Second}\PYG{p}{(}\PYG{n+nv}{x}\PYG{p}{)}\PYG{+w}{ }\PYG{o}{+}\PYG{+w}{ }\PYG{l+s+s2}{\PYGZdq{}]\PYGZdq{}}\PYG{p}{)}
\end{sphinxVerbatim}

\sphinxAtStartPar
\sphinxstylestrong{See Also:}

\sphinxAtStartPar
\sphinxcode{\sphinxupquote{WaitForClickOnTarget()}}, \sphinxcode{\sphinxupquote{WaitForMouseButton()}}

\index{YesNoTrial@\spxentry{YesNoTrial}}\ignorespaces 

\subsection{YesNoTrial()}
\label{\detokenize{reference/utility:yesnotrial}}\label{\detokenize{reference/utility:index-44}}
\sphinxAtStartPar
\sphinxstylestrong{Description:}

\sphinxAtStartPar
These helper functions require gTextBox, gHeader, and gFooter to work.

\sphinxAtStartPar
\sphinxstylestrong{Usage:}

\begin{sphinxVerbatim}[commandchars=\\\{\}]
\PYG{k}{define}\PYG{+w}{ }\PYG{n+nf}{YesNoTrial}\PYG{p}{(}\PYG{n+nv}{text}\PYG{p}{)}
\end{sphinxVerbatim}

\index{ZeroPad@\spxentry{ZeroPad}}\ignorespaces 

\subsection{ZeroPad()}
\label{\detokenize{reference/utility:zeropad}}\label{\detokenize{reference/utility:index-45}}
\sphinxAtStartPar
\sphinxstyleemphasis{Pads the beginning of a number with 0s so the number is size long}

\sphinxAtStartPar
\sphinxstylestrong{Description:}

\sphinxAtStartPar
Takes a number and pads it with zeroes left of the   decimal point so that its length is equal to \textless{}size\textgreater{}. Argument must   be a positive integer and less than ten digits.  Returns a string.

\sphinxAtStartPar
\sphinxstylestrong{Usage:}

\begin{sphinxVerbatim}[commandchars=\\\{\}]
\PYG{k}{define}\PYG{+w}{ }\PYG{n+nf}{ZeroPad}\PYG{p}{(}\PYG{p}{.}\PYG{p}{.}\PYG{p}{.}\PYG{p}{)}
\end{sphinxVerbatim}

\sphinxAtStartPar
\sphinxstylestrong{Example:}

\begin{sphinxVerbatim}[commandchars=\\\{\}]
\PYG{n+nf}{Print}\PYG{p}{(}\PYG{n+nf}{ZeroPad}\PYG{p}{(}\PYG{l+m+mi}{33}\PYG{p}{,}\PYG{l+m+mi}{5}\PYG{p}{)}\PYG{p}{)}\PYG{+w}{     }\PYG{c+c1}{\PYGZsh{} \PYGZdq{}00033\PYGZdq{}}
\PYG{n+nf}{Print}\PYG{p}{(}\PYG{n+nf}{ZeroPad}\PYG{p}{(}\PYG{l+m+mi}{123456}\PYG{p}{,}\PYG{l+m+mi}{6}\PYG{p}{)}\PYG{p}{)}\PYG{+w}{ }\PYG{c+c1}{\PYGZsh{}\PYGZdq{}123456\PYGZdq{}}
\PYG{n+nf}{Print}\PYG{p}{(}\PYG{n+nf}{ZeroPad}\PYG{p}{(}\PYG{l+m+mi}{1}\PYG{p}{,}\PYG{l+m+mi}{8}\PYG{p}{)}\PYG{p}{)}\PYG{+w}{      }\PYG{c+c1}{\PYGZsh{}\PYGZdq{}00000001\PYGZdq{}}
\end{sphinxVerbatim}

\sphinxAtStartPar
\sphinxstylestrong{See Also:}

\sphinxAtStartPar
\sphinxcode{\sphinxupquote{Format()}}


\subsection{Functions Pending Documentation}
\label{\detokenize{reference/utility:functions-pending-documentation}}
\index{AppendDirList@\spxentry{AppendDirList}}\ignorespaces 

\subsection{AppendDirList()}
\label{\detokenize{reference/utility:appenddirlist}}\label{\detokenize{reference/utility:index-46}}
\sphinxAtStartPar
\sphinxstyleemphasis{Manages a list of directories for path navigation}

\sphinxAtStartPar
\sphinxstylestrong{Description:}

\sphinxAtStartPar
Appends a directory to a directory list, handling special directory names like “.” (current directory) and “..” (parent directory). When “..” is encountered, it removes the last directory from the list (going up one level). When “.” is encountered, the list remains unchanged. Regular directory names are simply appended. Useful for building and navigating directory paths dynamically.

\sphinxAtStartPar
\sphinxstylestrong{Usage:}

\begin{sphinxVerbatim}[commandchars=\\\{\}]
\PYG{k}{define}\PYG{+w}{ }\PYG{n+nf}{AppendDirList}\PYG{p}{(}\PYG{n+nv}{dirlist}\PYG{p}{,}\PYG{+w}{ }\PYG{n+nv}{dir}\PYG{p}{)}
\end{sphinxVerbatim}

\sphinxAtStartPar
\sphinxstylestrong{Example:}

\begin{sphinxVerbatim}[commandchars=\\\{\}]
\PYG{c+c1}{\PYGZsh{} Build a directory path dynamically}
\PYG{n+nv}{dirs}\PYG{+w}{ }\PYG{o}{\PYGZlt{}\PYGZhy{}}\PYG{+w}{ }\PYG{p}{[}\PYG{p}{]}
\PYG{n+nv}{dirs}\PYG{+w}{ }\PYG{o}{\PYGZlt{}\PYGZhy{}}\PYG{+w}{ }\PYG{n+nf}{AppendDirList}\PYG{p}{(}\PYG{n+nv}{dirs}\PYG{p}{,}\PYG{+w}{ }\PYG{l+s+s2}{\PYGZdq{}data\PYGZdq{}}\PYG{p}{)}\PYG{+w}{      }\PYG{c+c1}{\PYGZsh{} [\PYGZdq{}data\PYGZdq{}]}
\PYG{n+nv}{dirs}\PYG{+w}{ }\PYG{o}{\PYGZlt{}\PYGZhy{}}\PYG{+w}{ }\PYG{n+nf}{AppendDirList}\PYG{p}{(}\PYG{n+nv}{dirs}\PYG{p}{,}\PYG{+w}{ }\PYG{l+s+s2}{\PYGZdq{}exp1\PYGZdq{}}\PYG{p}{)}\PYG{+w}{      }\PYG{c+c1}{\PYGZsh{} [\PYGZdq{}data\PYGZdq{}, \PYGZdq{}exp1\PYGZdq{}]}
\PYG{n+nv}{dirs}\PYG{+w}{ }\PYG{o}{\PYGZlt{}\PYGZhy{}}\PYG{+w}{ }\PYG{n+nf}{AppendDirList}\PYG{p}{(}\PYG{n+nv}{dirs}\PYG{p}{,}\PYG{+w}{ }\PYG{l+s+s2}{\PYGZdq{}results\PYGZdq{}}\PYG{p}{)}\PYG{+w}{   }\PYG{c+c1}{\PYGZsh{} [\PYGZdq{}data\PYGZdq{}, \PYGZdq{}exp1\PYGZdq{}, \PYGZdq{}results\PYGZdq{}]}
\PYG{n+nv}{dirs}\PYG{+w}{ }\PYG{o}{\PYGZlt{}\PYGZhy{}}\PYG{+w}{ }\PYG{n+nf}{AppendDirList}\PYG{p}{(}\PYG{n+nv}{dirs}\PYG{p}{,}\PYG{+w}{ }\PYG{l+s+s2}{\PYGZdq{}..\PYGZdq{}}\PYG{p}{)}\PYG{+w}{        }\PYG{c+c1}{\PYGZsh{} [\PYGZdq{}data\PYGZdq{}, \PYGZdq{}exp1\PYGZdq{}] (go up)}
\PYG{n+nv}{dirs}\PYG{+w}{ }\PYG{o}{\PYGZlt{}\PYGZhy{}}\PYG{+w}{ }\PYG{n+nf}{AppendDirList}\PYG{p}{(}\PYG{n+nv}{dirs}\PYG{p}{,}\PYG{+w}{ }\PYG{l+s+s2}{\PYGZdq{}analysis\PYGZdq{}}\PYG{p}{)}\PYG{+w}{  }\PYG{c+c1}{\PYGZsh{} [\PYGZdq{}data\PYGZdq{}, \PYGZdq{}exp1\PYGZdq{}, \PYGZdq{}analysis\PYGZdq{}]}

\PYG{c+c1}{\PYGZsh{} Current directory is ignored}
\PYG{n+nv}{dirs}\PYG{+w}{ }\PYG{o}{\PYGZlt{}\PYGZhy{}}\PYG{+w}{ }\PYG{n+nf}{AppendDirList}\PYG{p}{(}\PYG{n+nv}{dirs}\PYG{p}{,}\PYG{+w}{ }\PYG{l+s+s2}{\PYGZdq{}.\PYGZdq{}}\PYG{p}{)}\PYG{+w}{         }\PYG{c+c1}{\PYGZsh{} [\PYGZdq{}data\PYGZdq{}, \PYGZdq{}exp1\PYGZdq{}, \PYGZdq{}analysis\PYGZdq{}] (unchanged)}
\end{sphinxVerbatim}

\sphinxAtStartPar
\sphinxstylestrong{See Also:}

\sphinxAtStartPar
\sphinxcode{\sphinxupquote{DirToText()}}, \sphinxcode{\sphinxupquote{GetDirectory()}}, \sphinxcode{\sphinxupquote{GetDirectoryListing()}}

\index{CreateParameters@\spxentry{CreateParameters}}\ignorespaces 

\subsection{CreateParameters()}
\label{\detokenize{reference/utility:createparameters}}\label{\detokenize{reference/utility:index-47}}
\sphinxAtStartPar
\sphinxstyleemphasis{Creates a parameter object from defaults and an optional parameter file}

\sphinxAtStartPar
\sphinxstylestrong{Description:}

\sphinxAtStartPar
Creates a parameter object by combining default values with parameters from a file. Supports both legacy CSV format (.par) and modern JSON format (.par.json). Can load parameters from local files or remote URLs (\sphinxurl{http://} or \sphinxurl{https://}). If the file doesn’t exist or a URL cannot be fetched, only the defaults are used. Parameters from the file override defaults.

\sphinxAtStartPar
The function auto\sphinxhyphen{}detects file format based on extension: .json files (and all URLs) use JSON format, while .par files use legacy CSV format.

\sphinxAtStartPar
\sphinxstylestrong{Usage:}

\begin{sphinxVerbatim}[commandchars=\\\{\}]
\PYG{k}{define}\PYG{+w}{ }\PYG{n+nf}{CreateParameters}\PYG{p}{(}\PYG{n+nv}{defaults}\PYG{p}{,}\PYG{+w}{ }\PYG{n+nv}{file}\PYG{p}{)}
\end{sphinxVerbatim}

\sphinxAtStartPar
\sphinxstylestrong{Example:}

\begin{sphinxVerbatim}[commandchars=\\\{\}]
\PYG{c+c1}{\PYGZsh{} Create parameters with defaults and local file}
\PYG{n+nv}{defaults}\PYG{+w}{ }\PYG{o}{\PYGZlt{}\PYGZhy{}}\PYG{+w}{ }\PYG{p}{[}\PYG{p}{[}\PYG{l+s+s2}{\PYGZdq{}trials\PYGZdq{}}\PYG{p}{,}\PYG{+w}{ }\PYG{l+m+mi}{20}\PYG{p}{]}\PYG{p}{,}\PYG{+w}{ }\PYG{p}{[}\PYG{l+s+s2}{\PYGZdq{}isi\PYGZdq{}}\PYG{p}{,}\PYG{+w}{ }\PYG{l+m+mi}{1000}\PYG{p}{]}\PYG{p}{,}\PYG{+w}{ }\PYG{p}{[}\PYG{l+s+s2}{\PYGZdq{}practice\PYGZdq{}}\PYG{p}{,}\PYG{+w}{ }\PYG{l+m+mi}{1}\PYG{p}{]}\PYG{p}{]}
\PYG{n+nv}{pars}\PYG{+w}{ }\PYG{o}{\PYGZlt{}\PYGZhy{}}\PYG{+w}{ }\PYG{n+nf}{CreateParameters}\PYG{p}{(}\PYG{n+nv}{defaults}\PYG{p}{,}\PYG{+w}{ }\PYG{l+s+s2}{\PYGZdq{}config.par.json\PYGZdq{}}\PYG{p}{)}

\PYG{c+c1}{\PYGZsh{} Now pars.trials, pars.isi, pars.practice are available}
\PYG{n+nf}{Print}\PYG{p}{(}\PYG{l+s+s2}{\PYGZdq{}Running \PYGZdq{}}\PYG{+w}{ }\PYG{o}{+}\PYG{+w}{ }\PYG{n+nv}{pars.trials}\PYG{+w}{ }\PYG{o}{+}\PYG{+w}{ }\PYG{l+s+s2}{\PYGZdq{} trials\PYGZdq{}}\PYG{p}{)}

\PYG{c+c1}{\PYGZsh{} Load from URL (uses JSON format automatically)}
\PYG{n+nv}{pars}\PYG{+w}{ }\PYG{o}{\PYGZlt{}\PYGZhy{}}\PYG{+w}{ }\PYG{n+nf}{CreateParameters}\PYG{p}{(}\PYG{n+nv}{defaults}\PYG{p}{,}\PYG{+w}{ }\PYG{l+s+s2}{\PYGZdq{}https://example.com/params.json\PYGZdq{}}\PYG{p}{)}

\PYG{c+c1}{\PYGZsh{} If file doesn\PYGZsq{}t exist, defaults are used}
\PYG{n+nv}{pars}\PYG{+w}{ }\PYG{o}{\PYGZlt{}\PYGZhy{}}\PYG{+w}{ }\PYG{n+nf}{CreateParameters}\PYG{p}{(}\PYG{n+nv}{defaults}\PYG{p}{,}\PYG{+w}{ }\PYG{l+s+s2}{\PYGZdq{}nonexistent.json\PYGZdq{}}\PYG{p}{)}
\PYG{c+c1}{\PYGZsh{} pars still has default values}
\end{sphinxVerbatim}

\sphinxAtStartPar
\sphinxstylestrong{See Also:}

\sphinxAtStartPar
\sphinxcode{\sphinxupquote{ReadJSONParameters()}}, \sphinxcode{\sphinxupquote{MakeParameterObject()}}, \sphinxcode{\sphinxupquote{ReadCSV()}}

\index{DirToText@\spxentry{DirToText}}\ignorespaces 

\subsection{DirToText()}
\label{\detokenize{reference/utility:dirtotext}}\label{\detokenize{reference/utility:index-48}}
\sphinxAtStartPar
\sphinxstyleemphasis{Converts directory and file lists into path strings}

\sphinxAtStartPar
\sphinxstylestrong{Description:}

\sphinxAtStartPar
Combines a list of directories and a list of filenames into complete path strings. Prepends directory names with backslashes and merges them with the filelist. This is a helper function for path manipulation when working with directory structures.

\sphinxAtStartPar
\sphinxstylestrong{Usage:}

\begin{sphinxVerbatim}[commandchars=\\\{\}]
\PYG{k}{define}\PYG{+w}{ }\PYG{n+nf}{DirToText}\PYG{p}{(}\PYG{n+nv}{dirlist}\PYG{p}{,}\PYG{+w}{ }\PYG{n+nv}{filelist}\PYG{p}{,}\PYG{+w}{ }\PYG{n+nv}{path}\PYG{p}{)}
\end{sphinxVerbatim}

